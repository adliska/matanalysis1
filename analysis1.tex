\documentclass[12pt]{article}

\usepackage[czech]{babel}
\usepackage[utf8]{inputenc}
\usepackage[T1]{fontenc}
\usepackage{lmodern}

\usepackage{csquotes}
\DeclareQuoteAlias{german}{czech}
\MakeOuterQuote{"}

\usepackage[pdftex]{graphicx}
\usepackage{url}
\usepackage[round]{natbib}
\usepackage{subfigure}
\usepackage{enumerate}
\usepackage{multicol}

\usepackage{tikz}
\usetikzlibrary{shapes,arrows}

\usepackage{amsmath}
\usepackage{amssymb}
\usepackage{amsthm}
\usepackage{amsfonts}
\usepackage{mathtools}
\usepackage{thmtools}

\usepackage{soulutf8}
\usepackage{regexpatch}

\usepackage{pgfplots}

% Patch for hyphens
\makeatletter
\regexpatchcmd*{\SOUL@eval}{\cO-}{\cA-}{}{}
\makeatother

\usepackage{hyperref}
\hypersetup{
    colorlinks,
    citecolor=black,
    filecolor=black,
    linkcolor=black,
    urlcolor=black
}

\setlength{\oddsidemargin}{0.25in}
\setlength{\textwidth}{6.5in}
\setlength{\topmargin}{0in}
\setlength{\textheight}{8.5in}
%\setlength{\parskip}{0.05in}
%\setlength{\parindent}{0pt}

\newtheorem{theorem}{Věta}[section]
\newtheorem{lemma}[theorem]{Lemma}
\newtheorem{corollary}[theorem]{Důsledek}
\newtheorem{observation}[theorem]{Pozorování}
\newtheorem{proposition}[theorem]{Tvrzení}
\newtheorem{metaproposition}[theorem]{Metatvrzení}


\theoremstyle{definition}
\newtheorem{definition}[theorem]{Definice}
\newtheorem{algorithm}[theorem]{Algoritmus}
%\theoremstyle{remark}
\newtheorem{remark}[theorem]{Poznámka}
\declaretheorem[name=Příklad,style=definition,qed=$\diamondsuit$,sibling=theorem]{example}
%\newtheorem{example}[theorem]{Příklad}

\newcommand*{\newterm}[1]{\ul{#1}}

\newcommand*{\R}{\mathbb{R}}
\newcommand*{\Rn}{\mathbb{R}^n}
\newcommand*{\Rstar}{\R^*}
\newcommand*{\C}{\mathbb{C}}
\newcommand*{\Cn}{\mathbb{C}^n}
\newcommand*{\Cnn}{\mathbb{C}^{n\times n}}
\newcommand*{\N}{\mathbb{N}}
\newcommand*{\Z}{\mathbb{Z}}
\newcommand*{\Q}{\mathbb{Q}}

\newcommand*{\e}{\varepsilon}

\newcommand*{\Necht}{Nechť }
\newcommand*{\necht}{nechť }
\newcommand*{\buno}{bez újmy na obecnosti }
\newcommand*{\tz}{tak, že }

\newcommand*{\ntoinfty}{n\to\infty}
\newcommand*{\xtoa}{x \to a}
\newcommand*{\limxtoa}{\lim_{\xtoa}}
\newcommand*{\limxo}{\lim_{x \to 0}}
\newcommand*{\limxa}{\lim_{x \to a}}
\newcommand*{\limyo}{\lim_{y \to 0}}
\newcommand*{\limho}{\lim_{h \to 0}}
\newcommand*{\limhop}{\lim_{h \to 0+}}
\newcommand*{\limhom}{\lim_{h \to 0-}}
\newcommand*{\limxinf}{\lim_{x \to \infty}}
\newcommand*{\limxminf}{\lim_{x \to -\infty}}
\newcommand*{\limninf}{\lim_{n \to \infty}}
\newcommand*{\limminf}{\lim_{m \to \infty}}
\newcommand*{\limsupn}{\limsup_{n\to\infty}}

\newcommand*{\ds}{\displaystyle}

\newcommand*{\tfan}{T_n^{f,a}}

\newcommand*{\inv}[1]{#1^{-1}}

\newcommand*{\fa}{\forall}
\newcommand*{\seq}[1]{\left\{#1\right\}}
\newcommand*{\rada}[1]{\sum_{n=1}^\infty #1}
\newcommand*{\qe}{\stackrel{?}{=}}

\DeclareMathOperator{\cotg}{cotg}
\DeclareMathOperator{\arccotg}{arccotg}
\DeclareMathOperator{\sgn}{sgn}
\DeclareMathOperator{\vnitrek}{int}

\usepackage{xcolor}
\newcommand\todo[1]{\textcolor{red}{\textbf{#1}}}

\newcommand{\thmfoot}{}

\begin{document}

\title{Matematická analýza I \\ 
    \vspace{1 mm} {\normalsize Zápisky z přednášek Stanislava 
        Hencla\footnote{\url{http://www.karlin.mff.cuni.cz/~hencl/}} \ 
        na MFF UK, zimní semestr, ak. rok 2007/2008}}

\author{Adam Li\v{s}ka\footnote{\url{http://www.adliska.com}}}

\date{\today}

\maketitle

\newpage
\tableofcontents

\section{Úvod}

\subsection{Výroková a predikátová logika}

Výroková a predikátová logika je věda o pravdivosti výroků. Výrok je tvrzení,
u kterého můžeme rozhodnout, zda-li je pravdivé nebo nikoliv.

\begin{definition} 
    \newterm{Výroková funkce} je výrok, do něhož dosazujeme proměnné.
\end{definition}

\begin{remark}[Příklady výrokových funkcí]
    \leavevmode
    \begin{itemize}
        \item $A(x): x < 3.$ $A(1)$ je pravda, $A(3)$ je lež.
        \item $B(x,y): x < y.$ $B(1,2)$ je pravda, $B(5,2)$ je lež.
    \end{itemize}
\end{remark}

\begin{remark}[Úmluva ohledně zápisu výroků]
    \leavevmode
    \begin{itemize}
        \item Zápisem $\forall x \in \R; x > 10: A(X)$ budeme rozumět výrok
            $\forall x \in \R: (x > 10 \implies A(X))$.
        \item Zápisem $\forall x \in \R, \exists y \in \R: B(x,y)$ budeme
            rozumět výrok $\forall x \in \R: (\exists y \in \R: B(x,y))$.
    \end{itemize}
\end{remark}

\begin{remark}[Pořadí kvantifikátorů]
    Na pořadí kvantifikátorů záleží, to jest zápisy 
    $$\forall x \in \R, \exists y \in \R: B(x,y)$$
    a 
    $$\exists x \in \R, \forall y \in \R: B(x,y)$$
    vyjadřují dva rozdílné výroky.
\end{remark}

\subsection{Základní metody důkazů}

\subsubsection{Důkaz sporem}
Naším cílem je dokázat výrok $A$. V důkazu sporem se prokáže, že výrok 
$\neg A$ vede ke sporu. Díky zákonu o vyloučení třetího tedy odvodíme, že 
výrok $A$ musí být pravdivý.

Mějme například následující výrok $A$:
$$\forall n \in \N: (\text{$n^2$ je liché} \implies \text{$n$ je liché}).$$
Nyní ukážeme, že výrok $\neg A$:
$$\exists n \in \N: (\text{$n^2$ je liché} \land \text{$n$ je sudé})$$
vede ke sporu. Tedy, pokud je $n$ sudé a $n^2$ liché, potom jejich součet,
$n + n^2$, je také lichý. To nicméně vede ke sporu, jelikož $n + n^2 = n(n+1)$
je součin dvou po sobě následujících čísel, z nichž jedno je liché a jedno
sudé. Součin sudého a lichého čísla je vždy sudé číslo.

\subsubsection{Nepřímý důkaz}

Naším cílem je dokázat implikaci typu $A \implies B$. Někdy je ovšem
jednodušší dokázat ekvivalentní výrok tvaru $\neg B \implies \neg A$.
V našem příkladě budeme tedy dokazovat výrok:
$$\forall n \in \N: (\text{$n$ je sudé} \implies \text{$n^2$ je sudé}).$$
Vyjádřeme $n$ jako $2k$. Potom $n^2 = 4k$, což je sudé číslo.

\subsubsection{Přímý důkaz}

Přímý důkaz implikace $A \implies B$ spočívá v nalezení řady 
výroků tvaru: 
$$A \implies C_1, C_1 \implies C_2, \dots, C_{n-1} \implies C_n, C_n \implies B.$$
V našem případě můžeme vyjádřit číslo $n$ jako součin $p_1 \dots p_k$. Potom
$n^2 = p_1^2 \dots p_k^2$. Pokud je číslo $n^2$ liché, potom žádný činitel
z $p_1^2 \dots p_k^2$ neobsahuje číslo $2$, a tedy žádný činitel z 
$p_1 \dots p_n$ neobsahuje číslo $2$ a tedy číslo $n = p_1 \dots p_k$
je liché.

\subsubsection{Matematická indukce}
Matematickou indukci využijeme v případě, kdy chceme dokázat platnost výroku
pro všechna přirozená čísla, či případně pro jinou, předem danou nekonečnou
posloupnost, např. pro všechna přirozená čísla větší než $5$.

V prvním kroce důkazu matematickou indukcí se ukáže, že tvrzení platí
pro nejmenší přirozené číslo $k$. V indukčním kroce se dokáže, že pokud
tvrzení platí pro $n = m$, pak platí i pro $n = m+1$. Dle principu matematické
indukce pak tvrzení platí pro každé přirozené číslo větší nebo rovno $k$.

Jako příklad dokážeme následující dvě věty.

\begin{theorem}[Bernoulliho nerovnost]
    \label{th:bernoulli}
    Nechť $x \in \R, x \geq -1$ a $n \in \N.$ Potom:
    $$(1 + x)^n \geq 1 + nx.$$
\end{theorem}

\begin{proof}
    \leavevmode
    \begin{enumerate}[I.]
        \item Pro $n = 1$ zřejmě platí: $(1 + x)^1  = 1 + 1\cdot x.$
        \item Předpokládejme, že tvrzení platí pro $n = m$ a pokusme se jej
            dokázat pro $n = m+1.$ Tedy:
            \begin{align*}
                (1 + x)^{m+1} &= (1+x)(1+x)^m \\
                              &\geq (1+x)(1+mx) \tag{dle indukčního předpokladu} \\
                              &= 1 + x + mx + mx^2 \\
                              &= 1 + (m+1)x + mx^2 \\
                              &\geq 1 + (m+1)x \tag{jelikož $mx^2 \geq 0$}
            \end{align*}
    \end{enumerate}
\end{proof}

\begin{theorem}[Vztah aritmetického a geometrického průměru]
    \label{th:agprumer}
    Aritmetický průměr nezáporných čísel je vždy větší nebo roven geometrickému 
    průměru. 
\end{theorem}

\begin{proof}
    Nejprve dokážeme, že toto
    tvrzení platí pro jedno číslo, $V(1)$. Poté ukážeme, že $V(n) \implies 
    V(2n)$. Nakonec ukážeme, že $V(n+1) \implies V(n)$. Tím je důkaz ukončen.

    Základní krok je jednoduchý:
    $$\frac{x_1}{1} = \sqrt[1]{x_1}$$

    Jednoduše se ukáže i platnost výroku $V(2)$:
    $$\frac{x_1 + x_2}{2} \geq \sqrt{x_1x_2} 
        \iff x_1 - 2\sqrt{x_1x_2} + x_2 \geq 0 
        \iff (\sqrt{x_1} - \sqrt{x_2})^2 \geq 0$$

    Nyní ukážeme platnost tvrzení $V(n) \implies V(2n)$:
    \begin{align*}
        \frac{x_1 + x_2 + \dots + x_{2n}}{2n} 
            &= \frac{\frac{x_1 + \dots + x_n}{n} + \frac{x_{n+1} + 
                \dots + x_{2n}}{n}}{2} \\
            &\geq \frac{\sqrt[n]{x_1 \dots x_n} + \sqrt[n]{x_{n+1} \dots 
                x_{2n}}}{2} \tag{indukční předpoklad $V(n)$} \\
            &\geq \sqrt{\sqrt[n]{x_1 \dots x_n} \sqrt[n]{x_{n+1} \dots x_{2n}}} 
                \tag{$V(2)$} \\
            &= \sqrt[2n]{x_1 \dots x_{2n}}
    \end{align*}

    Zbývá dokázat platnost tvrzení: $V(n+1) \implies V(n)$. Mějme čísla $x_1,
    \dots, x_n >0$. Položme $y_i = \frac{x_i}{\sqrt[n]{x_1 \dots x_n}}$ pro
    $i = 1, \dots, n$, a $y_{n+1} = 1$. Dle předpokladu:
    \begin{align*}
        \frac{y_1 + \dots + y_{n+1}}{n+1} 
            &\geq \sqrt[n+1]{y_1 \dots y_{n+1}} \\
            &=\sqrt[n+1]{\frac{x_1}{\sqrt[n]{x_1 \dots x_n}} \dots 
                \frac{x_n}{\sqrt[n]{x_1 \dots x_n}} \cdot 1} \\
            &= 1
    \end{align*}

    Potom:
    \begin{align*}
        \frac{\frac{x_1}{\sqrt[n]{x_1 \dots x_n}} + \dots + 
        \frac{x_n}{\sqrt[n]{x_1 \dots x_n}} + 1}{n+1} &\geq 1 \\
        \frac{x_1 + \dots + x_n}{\sqrt[n]{x_1 \dots x_n}} &\geq n \\
        \frac{x_1 + \dots + x_n}{n} &\geq \sqrt[n]{x_1 \dots x_n}
    \end{align*}
\end{proof}

\subsection{Množina reálných čísel}

Ze střední školy známe množiny:
\begin{itemize}
    \item $\N = \{1, 2, 3, \dots \}$
    \item $\Z = \{\dots, -2, -1, 0, 1, 2, \dots \}$
    \item $\Q = \{\frac{p}{q}, p \in \Z, q \in \N \}$
\end{itemize}

Tyto množiny nicméně neobsahují všechna čísla, se kterými pracujeme.

\begin{theorem}
    \leavevmode
    $\sqrt{2} \not \in \Q$
\end{theorem}

\begin{proof}
    Sporem. Nechť $\exists p \in \Z, q \in \N$ nesoudělná taková, že
    $$\sqrt{2} = \frac{p}{q}.$$
    Potom $2q^2 = p^2$, $p^2$ je sudé a $p$ musí být taktéž sudé: $\exists
    k \in \Z: p = 2k$ a $2q^2 = 4k^2$. Z toho vyplývá, že $q$ je také sudé, 
    čímž dostáváme spor s nesoudělností $p$ a $q$.
\end{proof}

\begin{remark}[Vlastnosti reálných čísel I.]
    Na množině $\R$ je dána binární relace $\leq \; \subset \R \times \R$, operace 
    sčítání ($+$), násobení ($\cdot$) a význačné prvky $0, 1$ tak, že 
    platí:
    \begin{enumerate}[i.]
        \item $\forall x,y,z \in \R: x + (y + z) = (x + y) + z$ (asociativita
            sčítání)
        \item $\forall x,y \in \R: x + y = y + x$ (komutativita sčítání)
        \item $\forall x \in \R: x + 0 = x$ (existence $0$)
        \item $\forall x \in \R, \exists (-x) \in \R: x + (-x) = 0$ (existence
            opačného prvku při sčítání)
        \item $\forall x,y,z \in \R: x\cdot(y\cdot z) = (x \cdot y) \cdot z$
            (asociativita násobení)
        \item $\forall x,y \in \R: x\cdot y = y \cdot x$ (komutativita 
            násobení)
        \item $\forall x \in \R: x \cdot 1 = x$ (existence $1$)
        \item $\forall x \in \R \setminus \{0\}: \exists x^{-1} \in \R: x \cdot
            x^{-1} = 1$ (existence opačného prvku při násobení)
        \item $\forall x,y,z \in \R: x \cdot (y + z) = xy + xz$ 
            (distributivita)
    \end{enumerate}
\end{remark}

\begin{remark}[Vlastnosti reálných čísel II.]
    \leavevmode
    \begin{enumerate}[i.]
        \item $\forall x,y \in \R: (x \leq y \land y \leq x) \implies x = y$
        \item $\forall x,y,z \in \R: (x \leq y \land y \leq z) \implies 
            x \leq z$
        \item $\forall x,y \in \R: (x \leq y) \lor (y \leq x)$
        \item $\forall x,y,z \in \R: (x \leq y) \implies x + z \leq y + z$
        \item $\forall x,y \in \R: (0 \leq x \land 0 \leq y) \implies
            0 \leq xy$
    \end{enumerate}
\end{remark}

\begin{definition}
    Množina $M \subset \R$ je \newterm{omezená shora (zdola)}, pokud
    existuje $x \in \R$ tak, že $\forall y \in M: y \leq x$ $(y \geq x)$.
    Číslo $x$ nazýváme \newterm{horní (dolní) závora} množiny $M$.
\end{definition}

\begin{definition}
    Nechť $M \subset \R$. Číslo $s \in \R$ nazveme \newterm{supremum} $M$
    (\newterm{nejmenší horní závora}) a značíme $\sup M$, pokud:
    \begin{enumerate}[i.]
        \item $\forall x \in M: x \leq s$ ($s$ je horní závora $M$)
        \item $\forall y < s \in \R: \exists x \in M: y < x$ ($s$ je 
            nejmenší horní závora)
    \end{enumerate}
\end{definition}

\begin{remark}[Vlastnosti reálných čísel III.]
    Nechť množina $M \subset \R$ je neprázdná shora omezená. Pak existuje 
    $\sup M \in \R.$
\end{remark}

Vlastnosti I., II. a III. určují jednoznačně množinu reálných čísel.

\begin{definition}
    \label{def:inf}
    Nechť $M \subset \R$. Číslo $i \in \R$ nazveme \newterm{infimem}, pokud:
    \begin{enumerate}[i.]
        \item $\forall x \in M: x \geq i$ ($i$ je dolní závora)
        \item $\forall y > i \in \R: \exists x \in M: x < y$ ($i$ je největší
            dolní závora)
    \end{enumerate}
\end{definition}

\begin{theorem}[O existenci infima]
    Nechť $M \subset \R$ je neprázdná zdola omezená. Pak existuje 
    $\inf M \in \R.$
\end{theorem}

\begin{proof}
    Definujme $-M = \{ -x; x \in M \}$. Tato množina je neprázdná shora omezená
    a tedy existuje $s = \sup(-M).$ Označíme $i = -s$ a ukážeme, že $i = \inf M$:
    \begin{enumerate}[i.]
        \item $x \in -M, x \leq s \implies \tilde{x} = -x, \tilde{x} \in M, 
            -\tilde{x} \leq -i, \tilde{x} \geq i$
        \item $\forall y \in \R, y < s: \exists x \in M: y < x \implies
            \tilde{x} = -x, \tilde{y} = -y, \forall \tilde{y} \in \R, \tilde{y} > -s = i:
            \exists \tilde{x} \in M: -\tilde{y} < -\tilde{x} \implies \tilde{y} > \tilde{x}$
    \end{enumerate}
\end{proof}

\begin{theorem}[Archimédova vlastnost]
    \label{th:archimedes}
    $\forall x \in \R: \exists n \in \N: x < n$
\end{theorem}

\begin{proof}
    Sporem. $\exists x \in \R: \forall n \in \N: x \geq n \implies \exists 
    s = \sup \N$, tedy $\forall n \in \N: n+1 \leq s$ a tedy $n \leq s - 1$.
    Číslo $s-1$ je také horní zavorou $\N$, což je spor s definicí $s$ jako 
    nejmenší horní zavory $\N$.
\end{proof}

\begin{theorem}[Hustota $\Q$ a $\R \setminus \Q$]
    \label{th:hustotaqrq}
    Nechť $a,b \in \R, a < b$. Pak existují $q \in \Q$ a $r \in \R$ takové, že
    $q \in (a,b), r \in (a,b).$
\end{theorem}

\begin{proof}
    Díky Archimédově vlastnosti (Věta~\ref{th:archimedes}) existuje 
    $n \in \N: n > \frac{1}{b-a}$ a tedy
    $b - a > \frac{1}{n}.$ Vezměme za $m$ nejmenší přirozené číslo větší než $na$.
    Potom $\frac{m}{n} = q \in (a,b)$. 
    
    Proč? Hledáme $m$ takové, že $a < \frac{m}{n} < b$, tj. $na < m 
    < nb$. Jelikož jsme vybrali $m$ jako nejmenší přirozené číslo větší než $na$,
    platí: $m-1 \leq na < m$. Z pravé části nerovnice přímo plyne $a < \frac{m}{n}$.
    Dále platí:
    \begin{align*}
        m &\leq na+1 \\
          &< n\left(b-\frac{1}{n}\right) +1 \tag{plyne z $n > \frac{1}{b-a}$} \\
          &=nb
    \end{align*}

    Tímto jsme dokázali větu o hustotě racionálních čísel v $\R$. Rozšíření na
    iracionální čísla je jednoduché: Mějme $q_1, q_2 \in \Q, q_1 < q_2 \in (a,b)$. 
    Položme $r = q_1 + \sqrt{2}
    \left(\frac{q_2-q_1}{2}\right)$. Zřejmě $r \in \R \setminus \Q$.
\end{proof}

\begin{theorem}[O existenci $n$-té odmocniny]
    Nechť $x \in [0; +\infty)$ a $n \in \N$. Pak existuje právě jedno $y \in 
        [0;+\infty)$ takové, že $y^n = x$.
\end{theorem}

\begin{proof}
    Definujme dvě množiny, $M_1$ a $M_2$ následovně:
    \begin{itemize}
        \item $M_1 = \left\{a \in [0;+\infty): a^n \leq x\right\}$. Tato množina
                je neprázdná a shora omezená, tedy existuje $y_1 = \sup M$.
        \item $M_2 = \left\{a \in [0;+\infty): a^n \geq x\right\}$. Tato množina
                je neprázdná a zdola omezená, tedy existuje $y_2 = \inf M$.
   \end{itemize}

   Pozorování: $y_1 \geq y_2$. Pokud by tomu tak nebylo, potom existuje $q \in \R:
   y_1 < q < y_2$ a buď $q^n \leq x$ nebo $x \leq q^n$. Obě možnosti vedou ke 
   sporu s definicemi suprema či infima.

   Tvrdím: $y_1^n \leq x$ a $y_2^n \geq x$. Tvrzení dokážeme sporem. Vezměme
   libovolné $k \in \N$ takové, že $k > \frac{ny_1^{n-1}}{y_1^n - x}$. Z
   definice suprema víme, že pro $y_1 - \frac{1}{k}$ existuje $a \in M_1: 
   a > y_1 - \frac{1}{k}$. Potom dostáváme spor:
   \begin{align*}
       y_1^n - x &\leq y_1^n - a^n \tag{$a \in M_1$ a tedy $a^n \leq x$} \\
                 &= (y_1 -a)(y_1^{n-1} + y_1^{n-2}a + \dots + y_1a^{n-2} + a^{n-1}) \\
                 &\leq (y_1-a)ny_1^{n-1} \tag{jelikož $y_1 \geq a$}\\
                 &< \frac{1}{k}ny_1^{n-1} \tag{vzpomeňme $a > y_1 - \frac{1}{k}$}\\
                 &< y_1^n-x
   \end{align*}

   Podobně lze ukázat, že $y_2^n \geq x$, čímž dostáváme dvě sady nerovností:
   \begin{itemize}
       \item $y_1 \geq y_2$, a
       \item $y_1^n \leq x \leq y_2^n$.
   \end{itemize}
   Zřejmě $x = y_1^n = y_2^n$.
\end{proof}

\begin{definition}
    Pro $x \in \R$ definujeme \newterm{absolutní hodnotu} $|x|$ následovně:
    $$|x| = \begin{cases}
        x &\text{pokud $x \geq 0,$} \\
        -x &\text{pokud $x < 0$.}
    \end{cases}$$
\end{definition}

\begin{theorem}[Vlastnosti absolutní hodnoty]
    \label{th:triangleineq}
    \leavevmode
    \begin{enumerate}[i.]
        \item $\fa x \in \R: |x| \geq 0,$
        \item $\fa x \in \R: |x| = 0 \iff x = 0,$
        \item $\fa x \in \R: x \leq |x|,$
        \item $\fa x,y \in \R: |xy| = |x||y|,$
        \item $\fa x,y \in \R: ||x| - |y|| \leq |x \pm y| \leq |x| + 
            |y|$ (rozšířená \newterm{trojúhelníková nerovnost}).
    \end{enumerate}
\end{theorem}

\begin{proof}
    První čtyři vlastnosti jsou zřejmé: Postačí jednoduché rozepsání případů 
    ($x \geq 0$, $x < 0$, $y \geq 0$, $y < 0$) či jejich kombinací.

    Dokažme nyní poslední vlastnost. Z předchozích bodů vyplývá:
    $$-2|x||y| \leq 2xy \leq 2|x||y|.$$
    Přičtěme v nerovnostech výraz $|x|^2 + |y|^2 = x^2 + y^2$:
    $$|x|^2 - 2|x||y| + |y|^2 \leq x^2 + 2xy + y^2 \leq |x|^2 + 2|x||y| + |y|^2$$
    Tyto nerovnosti můžeme dále upravit:
    $$ (|x| - |y|)^2 \leq (x + y)^2 \leq (|x|+|y|)^2$$
    Jelikož platí:
    $$|a| \leq |b| \iff a^2 \leq b^2,$$
    dostáváme:
    $$ ||x| - |y|| \leq |x + y| \leq ||x| + |y|| = |x| + |y|.$$

    Pokud dále dosadíme $-y$ za $y$, dostaneme:
    $$||x| - |y|| \leq |x - y| \leq |x| + |y|$$
\end{proof}

\section{Posloupnosti}

\subsection{Úvod}
\begin{definition}
    Nechť pro $\forall n \in \N$ máme dáno $a_n \in \R$. Pak 
    $\left\{a_n\right\}_{n=1}^{\infty} = \left\{a_1, a_2, a_3, \dots \right\}$
    nazveme \newterm{posloupnost reálných čísel}. Číslo $a_n$ nazveme 
    $n$-tým prvkem posloupnosti.
\end{definition}

\begin{remark}[Příklady posloupností]
    \leavevmode
    \begin{multicols}{2}
        \begin{itemize}
            \item $\left\{\frac{1}{n}\right\}_{n=1}^\infty$
                
                Tedy $1, \frac{1}{2}, \frac{1}{3},$ atd.

            \item $\left\{2^n\right\}_{n=1}^\infty$

            \item $\left\{p_n\right\}_{n=1}^\infty$, kde $p_n$ je $n$-té prvočíslo.

            \item $a_1=1, a_{n+1} = 1 + a_n^2.$ 
                
                Tato posloupnost je zadána rekursivně.
        \end{itemize}
    \end{multicols}
\end{remark}

\begin{definition}
    Posloupnost $\seq{a_n}$ je \newterm{omezená}, pokud je množina členů 
    posloupnosti $\seq{a_n}_{n=1}^\infty \subset \R$ omezená množina. 
    Analogicky definujeme omezenost shora a omezenost zdola.
\end{definition}

\begin{definition}
    Řekneme, že posloupnost $\seq{a_n}_{n\in\N}$ je:
    \begin{itemize}
        \item \newterm{neklesající}, pokud $\forall n \in \N: a_n \leq a_{n+1},$
        \item \newterm{nerostoucí}, pokud $\forall n \in \N: a_n \geq a_{n+1},$
        \item \newterm{rostoucí}, pokud $\forall n \in \N: a_n < a_{n+1},$
        \item \newterm{klesající}, pokud $\forall n \in \N: a_n > a_{n+1}.$
    \end{itemize}
\end{definition}

\subsection{Vlastní limita posloupnosti}

\begin{definition}
    Nechť $A \in \R$ a $\seq{a_n}_{n=1}^\infty$ je posloupnost. Řekneme, že $A$ je 
    \newterm{vlastní limitou} posloupnosti $\seq{a_n}$, jestliže:
    $$\fa \e > 0 \; \exists n_0 \in \N \; \fa n \geq n_0,n\in \N: |a_n - A| < \e$$
    Značíme: $$\lim_{n\to\infty}a_n = A.$$
\end{definition}

\begin{remark}[Příklady limit]
    \label{remark:limitexamples}
    \leavevmode
    \begin{itemize}
        \item
            Mějme posloupnost $\seq{\frac{1}{n}}_{n=1}^\infty$. 
            Tato posloupnost zřejmě směřuje k nule, formálně: $$\lim_{n\to\infty}
            \frac{1}{n} = 0.$$
            Dle definice limity musí platit:
            $$\fa \e > 0 \; \exists n_0 \in \N \; \fa n \geq n_0,n\in \N: 
            |\frac{1}{n} - 0| < \e$$
            Pro dané $\e$ volíme $n_0 > \frac{1}{\e}$.
            Pro všechna $n \geq n_0 > \frac{1}{\e}$ platí: $\frac{1}{n} < \e$, a tedy: 
            $|\frac{1}{n} - 0| = \frac{1}{n} < \e$.
        \item
            $\lim_{n\to\infty}\sqrt[n]{n} = 1$

            Sporem: Nechť daná limita neexistuje, tedy: 
            $$\exists \e > 0 \; \fa n_0 \in \N \; 
            \exists n \geq n_0: |\sqrt[n]{n} - 1| \geq \e.$$ 
            Jelikož $\fa n \in \N: \sqrt[n]{n} \geq 1$, 
            vyplývá, že $\sqrt[n]{n} \geq 1 + \e$. Potom ovšem:
            \begin{align*}
                n &\geq (1 + \e)^n \\
                  &\geq 1 + n\e + \frac{n(n-1)}{2}\e^2 \tag{první tři členy 
                    binomického rozvoje} \\
                  &\geq n\e(1 + \frac{n-1}{2}\e)
            \end{align*}
            Po vydělení obou stran číslem $n$ dostáváme:
            $$1 \geq \e(1 + \frac{n-1}{2}\e)$$
            což očividně pro příliš velká $n \in \N$ nemůže platit.
        \item
            $\lim_{n\to\infty} (-1)^n$ neexistuje. 
            
            Sporem: Předpokládejme, že
            daná limita $A$ existuje, tj.: 
            $$\fa \e > 0 \; \exists n_0 \in \N \; \fa n \geq n_0,n\in \N: 
            |(-1)^n - A| < \e$$
            Ukážeme, že existuje protipříklad. Nechť $\e = \frac{1}{4}$. Potom
            dostáváme následující spor:
            \begin{align*}
                2 &= |(-1)^n - (-1)^{n+1}| \\
                  &\leq |(-1)^n - A| + |A - (-1)^{n+1}| \tag{trojúhelníková
                    nerovnost} \\
                  &\leq \frac{1}{4} + \frac{1}{4} \tag{z definice limity pro
                $\fa n \geq n_0$} \\
                  &= \frac{1}{2}
            \end{align*}
    \end{itemize}
\end{remark}

\begin{theorem}[jednoznačnost vlastní limity]
    \label{th:jednoznacnostlimity}
    Každá posloupnost má nejvýše jednu limitu.
\end{theorem}

\begin{proof}
    Nechť má posloupnost $\seq{a_n}$ dvě různé limity, $A_1$ a $A_2$. Bez újmy
    na obecnosti: $A_1 < A_2$. Zvolíme
    $0 < \e < \frac{A_2-A_1}{2}$. Z definice víme, že existují $n_1, n_2 \in
    \N$ takové, že: $\fa n \geq n_1: |a_n - A_1| < \e$ a $\fa n \geq n_2: |a_n
    -A_2| < \e$. Vezměme $n_0 \coloneqq \max(n_1,n_2)$. Potom dostáváme 
    následující spor:
    \begin{align*}
        \fa n \geq n_0: A_2 - A_1 &= |A_2 - A_1| \\
                                  &\leq |A_2 - a_n| + |a_n - A_1| 
                                     \tag{trojúhelníková nerovnost}\\
                                 &< 2\e \tag{z definice limity} \\
                                 &< A_2 - A_1 \tag{díky volbě $\e < \frac{A_2 - A_1}{2}$}
    \end{align*}
\end{proof}

\begin{theorem}
    \label{th:seqomez}
    Nechť posloupnost $\seq{a_n}$ má vlastní limitu $A \in \R$. Pak je množina
    $\seq{a_n}$ omezená.
\end{theorem}

\begin{proof}
    Zvolme $\e = 1$. Dle definice limity:
    $$\exists n_0 \in \N \; \fa n \geq n_0,n \in \N: |a_n - A| < 1.$$
    Pro všechna $n \geq n_0$ potom platí: 
    $$|a_n| = |a_n - A + A| \leq |a_n - A| + |A| < 1 + |A|.$$
    Nyní zvolme $$K \coloneqq \max{\{|A|+1, |a_1|, |a_2|, \dots, |a_{n_0}|\}}.$$
    Zřejmě: $\fa n \in \N: |a_n| \leq K.$
\end{proof}

\begin{definition}
    Řekneme, že posloupnost $\seq{b_k}_{k \in \N}$ je \newterm{vybraná} z posloupnosti
    $\seq{a_n}_{n \in \N}$, jestliže existuje rostoucí posloupnost přirozených
    čísel $\seq{n_k}_{k=1}^\infty$ tak, že $b_k = a_{n_k}$.
\end{definition}

\begin{remark}[Příklad vybrané posloupnosti]
    \leavevmode
    \begin{itemize}
        \item Posloupnost $\seq{\frac{1}{2^n}}$ je vybraná z posloupnosti 
            $\seq{\frac{1}{n}}$.
    \end{itemize}
\end{remark}

\begin{theorem}[O limitě vybrané posloupnosti]
    \label{th:vybranalimita}
    Nechť $\lim_{n \to \infty} a_n = A \in \R$ a nechť $\seq{b_k}$ je vybraná z
    $\seq{a_n}$. Potom $\lim_{n\to\infty} b_k = A$.
\end{theorem}

\begin{proof}
    Z definice limity víme, že:
    $$\fa \e > 0 \; \exists n_0 \in \N \; \fa n \geq n_0,n\in \N: |a_n - A| < \e$$
    K tomuto $\e$ volíme $k_0 \coloneqq n_0$. Potom $\fa k \geq k_0, k \in \N$ platí
    $n_k \geq k \geq n_0$ a tedy $|b_k - A| = |a_{n_k} -A| < \e$.
\end{proof}

\begin{remark}
    Předchozí implikace neplatí v opačném směru. 
    Uvažujte například posloupnost $\seq{(-1)^n}$ a její
    možné vybrané posloupnosti.
\end{remark}

\begin{theorem}[Aritmetika limit]
    \label{th:voal}
    Nechť $\lim_{n\to\infty}a_n = A \in \R$ a $\lim_{n\to\infty}b_n = B \in \R$.
    Pak platí:
    \begin{enumerate}[i.]
        \item $\lim_{n \to \infty} a_n + b_n = A + B$
        \item $\lim_{n \to \infty} a_nb_n = AB$
        \item pokud $\fa n \in \N: b_n \neq 0$ a $B \neq 0$, pak
            $\lim_{n \to \infty} \frac{a_n}{b_n} = \frac{A}{B}.$
    \end{enumerate}
\end{theorem}

\begin{proof}
    \leavevmode
    \begin{enumerate}[i.]
        \item Z definice limity dostáváme:
            $$\fa \e > 0 \; \exists n_1 \in \N \; \fa n \geq n_1,n\in \N: 
            |a_n - A| < \e,$$
            $$\fa \e > 0 \; \exists n_2 \in \N \; \fa n \geq n_2,n\in \N: 
            |b_n - B| < \e.$$
            Zvolme $n_0 \coloneqq \max\{n_1,n_2\}.$ Potom:
            \begin{align*}
                \fa n \geq n_0, n\in \N: 
                    |a_n + b_n - (A+B)| &\leq |a_n - A| + |b_n - B| \\
                                        &< \e + \e = 2\e
            \end{align*}
        \item Mějme $n_0 \coloneqq \max\{n_1,n_2\}$ jako v předchozím bodě. Potom:
            \begin{align*}
                \fa n \geq n_0, n\in \N:
                    |a_nb_n - AB| &\leq |a_nb_n - a_nB| + |a_nB - AB| \\
                                  &\leq |a_n||b_n - B| + |B||a_n - A| \\
                                  &< |a_n|\e + |B|\e
            \end{align*}
            Z Věty~\ref{th:seqomez} víme, že posloupnost $\seq{a_n}$ je omezená,
            tj. $\exists K \; \fa n \in \N: |a_n| \leq K$, a tedy:
            $$|a_n|\e + |B|\e \leq \e(K + |B|).$$

        \item Mějme $n_1$ a $n_2$ jako v předchozích bodech. Navíc, pro $\tilde{\e}
            = \frac{|B|}{2}$:
            $$\exists n_3 \in \N \; \fa n \geq n_3, n \in \N: |b_n - B| < \frac{|B|}{2}.$$
            Dle rozšířené trojúhelníkové nerovnosti (Věta~\ref{th:triangleineq}) 
            dále platí:
            $$|b_n - B| \geq ||b_n| - |B|| \geq |b_n| - |B|.$$
            a tedy:
            $$\fa n \geq n_3, n \in \N: |b_n| > \frac{|B|}{2}.$$
            Zvolme $n_0 \coloneqq \max\{n_1,n_2, n_3\}$, potom
            \begin{align*}
                \fa n \geq n_0, n\in\N: 
                    \left|\frac{a_n}{b_n} - \frac{A}{B}\right| 
                    &= \left|\frac{a_nB - AB + AB -Ab_n}{b_nB}\right| \\
                    &\leq \left|\frac{a_nB - AB}{b_nB}\right|+ \left|\frac{AB -Ab_n}{b_nB}\right| \\
                    &\leq \frac{|B||a_n - A|}{|b_n||B|} + \frac{|A||B - b_n|}{|b_n||B|} \\
                    &< \frac{\e}{\frac{|B|}{2}} + \frac{|A|\e}{\frac{|B|}{2}|B|} 
                    \tag{jelikož $|b_n| > \frac{|B|}{2}$}\\
                    &= \e\left(\frac{2}{|B|}+\frac{2|A|}{|B|^2}\right)
            \end{align*}
    \end{enumerate}
\end{proof}

\begin{theorem}[Limita a uspořádání]
    \label{th:limitaausporadani}
    Nechť $\lim_{n\to\infty} a_n = A \in \R, \lim_{n\to\infty}b_n = B \in \R.$
    \begin{enumerate}[i.]
        \item Jestliže $A < B$, pak $\exists n_0 \in \N \; \fa n \geq n_0: 
            a_n < b_n.$
        \item Jestliže existuje $n_0 \in \N$ takové, že pro každé $n \geq n_0$ 
            platí $a_n \geq b_n$, pak $A \geq B.$
    \end{enumerate}
\end{theorem}

\begin{proof}
    \leavevmode
    \begin{enumerate}[i.]
        \item
            Zvolme $0<\e < \frac{B-A}{2}.$ Dle definice limity:
            $$\exists n_1 \in \N \; \fa n \geq n_1, n \in \N: |a_n - A| < \e$$
            $$\exists n_2 \in \N \; \fa n \geq n_2, n \in \N: |b_n - B| < \e$$
            Položmě $n_0 \coloneqq \max\{n_1,n_2\}$. Potom:
            \begin{align*}
                \fa n \geq n_0: a_n &< A+\e \\
                                                      &< B-\e \tag{$0<\e < \frac{B-A}{2}$} \\
                                                      &< b_n
            \end{align*}

        \item Sporem: Nechť A < B. Potom dle předchozího bodu:
            $$\exists \widetilde{n_0} \in \N \; \fa n \geq \widetilde{n_0}: 
            a_n < b_n,$$
            což je ve sporu s předpoklady.
    \end{enumerate}
\end{proof}

\begin{theorem}[O dvou strážnících]
    \label{th:dvastraznici}
    Nechť $\seq{a_n}, \seq{b_n}, \seq{c_n}$ jsou posloupnosti splňující:
    \begin{enumerate}[i.]
        \item $\exists n_0 \in \N \; \fa n \geq n_0: a_n \leq c_n \leq b_n,$
        \item $\lim a_n = \lim b_n = A \in \R.$
    \end{enumerate}
    Pak $\lim c_n = A.$
\end{theorem}

\begin{proof} 
    Zvolme $\e > 0.$ Dle definice limity:
    $$\exists n_1 \in \N \; \fa n \geq n_1, n \in \N: |a_n - A| < \e$$
    $$\exists n_2 \in \N \; \fa n \geq n_2, n \in \N: |b_n - A| < \e$$
    Položme $n_3 \coloneqq \max\{n_0,n_1,n_2\}$. Potom:
    $$\fa n \geq n_3: A - \e < a_n \leq c_n \leq b_n < A + \e,$$
    tedy $\fa n \geq n_3: |c_n - A| < \e$, a proto $\lim c_n = A$.
\end{proof}

\begin{remark}[Příklad využití věty o dvou strážnících]
    Pomocí předchozí věty dokážeme následující tvrzení:
    \begin{quote}  
        \Necht $a>0.$ Pak $\lim_{n\to\infty} \sqrt[n]{a} = 1.$
    \end{quote}
    Podle hodnoty $a$ rozdělíme důkaz do tří částí:
    \begin{itemize}
        \item[$(a = 1)$] Triviální.
        \item[$(a > 1)$] Zřejmě: $$\exists n_0 \in \N, n_0 \geq a, \fa n \geq n_0: 
            a \leq n.$$ Potom:
            $$\fa n \geq n_0: 1 \leq \sqrt[n]{a} \leq \sqrt[n]{n}.$$
            Jelikož $\lim 1 = 1$ a $\lim \sqrt[n]{n} = 1$ 
            (Poznámka~\ref{remark:limitexamples}), potom dle věty o dvou 
            strážnících i $\lim \sqrt[n]{a} = 1$.
        \item[$(0 < a < 1)$] 
            S pomocí věty o aritmetice limit (Věta~\ref{th:voal}) převedeme 
            problém na již vyřešený případ $a > 1:$
            \begin{align*}
                \lim_{n \to \infty} \sqrt[n]{a} &= \lim_{n \to \infty} 
                \frac{1}{\sqrt[n]{\frac{1}{a}}} \\
                &= \frac{\lim_{n\to\infty} 1}{\lim_{n\to\infty} 
            \sqrt[n]{\frac{1}{a}}} \tag{aritmetika limit} \\
                &= \frac{1}{1} = 1 \tag{$\frac{1}{a} > 0$}
            \end{align*}
    \end{itemize}
    
\end{remark}

\begin{theorem}[O limitě součinu omezené a mizející posloupnosti]
    \label{th:mizejici}
    Nechť $\lim a_n = 0$ a $\seq{b_n}$ je omezená. Potom:
    $$\lim_{n\to\infty}a_nb_n = 0.$$
\end{theorem}

\begin{proof}
    Posloupnost $\seq{b_n}$ je omezená, tedy $\exists K: \fa n \in \N: |b_n| 
    \leq K.$ Potom:
    $$ 0 \leq |a_nb_n| = |a_n||b_n| \leq K|a_n|$$
    a s pomocí dvou strážníků (Věta~\ref{th:dvastraznici}) je 
    $\lim_{\ntoinfty} a_nb_n = 0.$
\end{proof}

\begin{remark}
    Předchozí větu můžeme využít např. při důkazu 
    $\lim_{n\to\infty} {\frac{1}{n}\sin n} = 0.$
\end{remark}

\subsection{Nevlastní limita posloupnosti}

\begin{definition}
    \label{df:nevlastnilimitaposl}
    Řekneme, že posloupnost $\seq{a_n}_{n \in \N}$ má \newterm{nevlastní limitu}
    $+\infty$ (respektive $-\infty$), pokud:
    $$\fa K \in \R \; \exists n_0 \in \N \; \fa n \geq n_0, n\in \N: a_n > K$$
    $$(\fa K \in \R \; \exists n_0 \in \N \; \fa n \geq n_0, n\in \N: a_n < K)$$
\end{definition}

\begin{remark}[Příklady nevlastních limit]
    \leavevmode
    \begin{itemize}
        \item $\lim_{n \to \infty} n^2 = +\infty$

            Ke $K \in \R$ zvol $n_0 \in \N$ takové, že $n_0 > \sqrt{K}.$ Pak
            $\fa n \geq n_0: n\geq n_0 \geq \sqrt{K},$ a tedy: $n^2 > K.$

        \item $\lim_{n \to \infty} -\sqrt{n} = -\infty$

            Ke $K \in \R$ zvol $n_0 \in \N$ takové, že $n_0 > K^2.$ Pak
            $\fa n \geq n_0: \sqrt{n} > -K,$ a tedy $-\sqrt{n} < K$.
    \end{itemize}
\end{remark}

\begin{definition}
    Nechť $\lim a_n = A.$ Pokud $A \in \R,$ říkáme, že posloupnost $\seq{a_n}$ 
    \newterm{konverguje}. Pokud $A = \pm \infty,$ říkáme, že posloupnost
    \newterm{diverguje}.
\end{definition}

\begin{metaproposition}
    Věty \ref{th:jednoznacnostlimity}, \ref{th:vybranalimita}, 
    \ref{th:limitaausporadani} a \ref{th:dvastraznici} platí i v případě,
    že uvažujeme nevlastní limity.
\end{metaproposition}

\begin{proof}
    Důkazy zmíněných vět je třeba rozepsat pro jednotlivé případy: 
    vlastní limita, nevlastní limita, kombinace vlastní a nevlastní 
    limity, atd. 
\end{proof}

\begin{definition}
    \newterm{Rozšířená reálná osa} je množina $\R^* \coloneqq \R \cup 
    \left\{+\infty\right\}\cup \left\{-\infty\right\}$ s následujícími vlastnostmi:

    \begin{center}
        \begin{tabular}{lr}
            Uspořádání: &$\fa a \in \R: -\infty < a < +\infty$ \\
            Absolutní hodnota: &$|-\infty| = |+\infty| = +\infty$ \\
            Sčítání: &$\fa a \in \R^* \setminus \{+\infty\}: -\infty + a = -\infty$ \\
                     &$\fa a \in \R^* \setminus \{-\infty\}: +\infty + a = +\infty$ \\
            Násobení: &$\fa a \in \R^*, a>0: a\cdot (\pm\infty) = \pm \infty$ \\
                      &$\fa a \in \R^*, a<0: a\cdot (\pm\infty) = \mp \infty$ \\
            Dělení: &$\fa a \in \R: \frac{a}{\pm\infty} = 0$ \\
        \end{tabular}
    \end{center}

    Výrazy $-\infty + \infty, 0 \cdot (\pm\infty), \frac{\pm\infty}{\pm\infty},
    \frac{\text{cokoli}}{0}$ nejsou definovány.
\end{definition}
\begin{definition}[Rozšíření definice suprema a infima]
    \leavevmode
    \begin{itemize}
        \item Pokud množina $M$ není shora omezená, potom $\sup M = +\infty.$
        \item Pokud množina $M$ není zdola omezená, potom $\inf M = -\infty.$
        \item Pokud $M = \emptyset$, potom $\sup M = -\infty$ a 
            $\inf M = +\infty.$
    \end{itemize}
\end{definition}

\begin{remark}
    Všimněte si, že při $M = \emptyset$ je $\inf M > \sup M.$
\end{remark}

\begin{theorem}[aritmetika limit podruhé]
    \Necht $\lim_{\ntoinfty} a_n = A\in\Rstar$ a $\lim_{\ntoinfty} b_n = B \in \Rstar.$
    Pak platí:
    \begin{enumerate}[i.]
        \item $\lim_{\ntoinfty} a_n + b_n = A + B,$ pokud je výraz $A+B$ definován,
        \item $\lim_{\ntoinfty} a_nb_n = AB,$ pokud je výraz $AB$ definován, a
        \item pokud $B \neq 0$ a $\fa n \in \N: b_n \neq 0$, pak
            $\lim_{\ntoinfty} \frac{a_n}{b_n} = \frac{A}{B},$ pokud je výraz 
            $\frac{A}{B}$ definován.
    \end{enumerate}
\end{theorem}

\begin{proof}
    Tato věta je rozšířením původní věty o aritmetice limit (Věty~\ref{th:voal}), 
    ve které jsme uvažovali pouze vlastní limity. 

    Jako příklad podívejme na důkaz bodu $(i)$ a případ $A = +\infty, B \in \R$.

    K $\e = 1$:
    $$\exists n_1 \in \N, \fa n \geq n_1, n \in \N: |b_n - B| < 1,$$ 
    a tedy $\fa n \geq n_1, n \in \N: b_n > B - 1.$ Dále, ke $K \in \R:$
    $$\exists n_2 \in \N, \fa n \geq n_2, n\in\N: a_n > K - B + 1.$$
    Zvolme $n_0 \coloneqq \max\{n_1, n_2\},$ potom:
    $$\fa n \geq n_0, n\in\N: a_n + b_n > K -B +1 +B -1 = K.$$
\end{proof}

\begin{theorem}[Limita typu $\frac{A}{0}$]
    \Necht $\lim_{\ntoinfty} a_n = A \in \Rstar, A > 0, \lim_{\ntoinfty} b_n = 0$
    a $\exists n_0 \in \N, \fa n \geq n_0, n \in \N: b_n > 0.$ Potom 
    $\lim_{\ntoinfty} \frac{a_n}{b_n} = \infty.$
\end{theorem}

\begin{remark}
    \leavevmode
    \begin{itemize}
        \item $\lim_{\ntoinfty} \frac{(-1)^n}{n} = 0$
        \item $\lim_{\ntoinfty} \frac{1}{\frac{(-1)^n}{n}}$ neexistuje,
            jelikož porušuje podmínku $\exists n_0 \in \N, \fa n \geq 
            n_0, n \in \N: b_n > 0.$
    \end{itemize}
\end{remark}

\begin{proof}
    Uvažujme případ, kdy $A \in \R$, tedy $\lim a_n$ je vlastní.
    Pro $\e = \frac{A}{2}$ platí:
    $$\exists n_1 \in \N, \fa n \geq n_1, n\in\N: |a_n - A| < \frac{A}{2},$$
    a tedy $\fa n \geq n_1, n \in \N: a_n > \frac{A}{2}.$ Zvolme $K > 0$ pevné,
    potom k $\e = \frac{\frac{A}{2}}{K}:$
    $$\exists n_2 \in \N, \fa n \geq n_2, n\in\N: |b_n| < \frac{\frac{A}{2}}{K}.$$
    Zvolme $n_3 = \max \{n_0, n_1, n_2\}.$ Potom:
    $$\fa n \geq n_3, n\in\N: \frac{a_n}{b_n} > 
    \frac{\frac{A}{2}}{\frac{\frac{A}{2}}{K}} = K.$$
\end{proof}

\subsection{Monotónní posloupnosti}

\begin{theorem}[O limitě monotónní posloupnosti]
    \label{th:monotonniposl}
    Každá monotónní posloupnost má limitu.
\end{theorem}

\begin{proof}
    Nechť je \buno posloupnost $\seq{a_n}$ neklesající. Položme
    $$A = \sup \{a_n, n\in \N\}.$$ Tvrdím, že $\lim a_n = A.$
    \begin{enumerate}[i.]
        \item \Necht $A \in \R$ a $\e > 0$. Z definice suprema:
            $$\exists n_0 \in \N: a_{n_0} > A - \e.$$
            Potom $\fa n \geq n_0, n\in\N$ platí: 
            \begin{align*}
                A &\geq a_n \tag{supremum} \\
                  &\geq a_{n_0} \tag{motononie} \\
                  &> A - \e
            \end{align*}
            a tedy $\fa n \geq n_0, n\in\N: |a_n - A| < \e$
        \item \Necht $A = +\infty$ a $K \in \R$. Z definice suprema:
            $$\exists n_0 \in \N: a_{n_0} > K.$$
            Potom $\fa n \geq n_0, n\in\N$ platí: 
            \begin{align*} 
                a_n &\geq a_{n_0} \tag{monotonie} \\
                    &>K.
            \end{align*}
    \end{enumerate}
\end{proof}

\begin{remark}[Příklad využití Věty~\ref{th:monotonniposl}]
    Určeme limitu (pokud existuje) rekursivně zadané posloupnosti $\seq{x_n}:$
    \begin{align*}
        x_1 &= 2 \\
        x_{n+1} &= \frac{x_n^2 + 2}{2x_n}
    \end{align*}

    Dokažme nejprve, že $\fa n \in \N: x_n > 0;$ toto pozorování se nám bude
    hodit později. Využijeme indukci:
    \begin{itemize}
        \item Pro $n=1$ platí $x_1 = 2 > 0.$
        \item Nechť $x_n > 0.$ Potom $\frac{x_n^2 + 2}{2x_n}$ je zřejmě také
            větší než $0,$ jelikož jak čitatel, tak jmenovatel jsou větší 
            než $0.$
    \end{itemize}

    Nyní dokažme, že daná posloupnost $\seq{x_n}$ je nerostoucí, tj. $\fa n 
    \in N: x_{n+1} \leq x_n.$ Pro $x_n > 0$ je tato nerovnost ekvivalentní:
    \begin{align*}
        \frac{x_n^2 + 2}{2x_n} &\leq x_n \\
        x_n^2 + 2 &\leq 2x_n^2 \\
        2 &\leq x_n^2 \\
        \sqrt{2} &\leq x_n
    \end{align*}
    Je třeba tedy dokázat tvrzení: $\fa n \in \N: x_n \geq \sqrt{2}.$ Využijme
    znovu indukci:
    \begin{itemize}
        \item Pro $n=1$ platí: $2 \geq \sqrt{2}.$
        \item Nechť $x_n \geq \sqrt{2}$. Potom:
            \begin{align*}
                x_{n+1} &= \frac{x_n^2 + 2}{2x_n} \\
                &= \frac{x_n^2 +2}{2} \cdot \frac{1}{x_n} \\
                &\geq \sqrt{2x_n^2} \cdot \frac{1}{x_n} 
                \tag{vztah aritmetického a geometrického průměru, Věta~\ref{th:agprumer}} \\
                &= \sqrt{2}
            \end{align*}
    \end{itemize}

    Potud jsme o posloupnosti $\seq{x_n}$ dokázali, že je:
    \begin{itemize}
        \item nerostoucí -- a podle věty o limitě monotónní posloupnosti 
            (Věta~\ref{th:monotonniposl}) má tedy limitu,
        \item zdola omezená -- a tedy její limita je vlastní.
    \end{itemize}
    Označme tuto limitu $A$ (tedy: $\lim_{\ntoinfty} x_n = A \in \R$). Potom platí:
    \begin{align*}
        A &= \lim_{\ntoinfty} x_{n} \\
          &= \lim_{\ntoinfty} x_{n+1} \tag{Věta~\ref{th:vybranalimita} a $b_k = x_{k+1}$}\\
          &= \lim_{\ntoinfty} \frac{x_n^2 + 2}{2x_n} \\
          &= \frac{\lim x_n \cdot \lim x_n + \lim 2}{2\lim x_n} 
        \tag{Věta~\ref{th:voal}} \\
        &= \frac{A\cdot A + 2}{2A}
    \end{align*}
    Vyřešením této rovnice získáme $A = \sqrt{2}.$
\end{remark}

\begin{definition}
    \Necht $\seq{a_n}_{n\in\N}$ je posloupnost a označme 
    $$b_k = \sup\{a_n, n\geq k\},$$
    $$c_k = \inf\{a_n, n\geq k\}.$$
    Je-li $\seq{a_n}$ shora (zdola) neomezená, pak klademe $\lim_{k \to\infty}
    b_k = \infty \; (\lim_{k \to \infty} c_k = -\infty).$

    Potom:
    \begin{itemize}
        \item Číslo $\lim_{k\to\infty} b_k$ nazýváme \newterm{limes superior} posloupnosti
    $\seq{a_n}$ a značíme $\lim\sup_{\ntoinfty} a_n.$
        \item Číslo $\lim_{k\to\infty} c_k$ nazýváme \newterm{limes inferior} posloupnosti
    $\seq{a_n}$ a značíme $\lim\inf_{\ntoinfty} a_n.$
    \end{itemize}
\end{definition}

\begin{remark}
    Nechť $\seq{a_n}$ je libovolná posloupnost. Potom $\limsup a_n$ a 
    $\liminf a_n$ existují, jelikož $\seq{b_k}$ a $\seq{c_k}$ jsou monotónní 
    posloupnosti, které dle věty o limitě monotónní posloupnosti 
    (Věta~\ref{th:monotonniposl}) mají limitu.
\end{remark}

\begin{theorem}[vztah limity, limes superior a limes inferior]
    \label{th:limitalimsupliminf}
    $$\lim a_n = A \in \Rstar \iff \limsup_{\ntoinfty} a_n = 
    \liminf_{\ntoinfty} a_n = A \in \Rstar$$
\end{theorem}

\begin{proof}
    \leavevmode
    \begin{itemize}
        \item[$\impliedby$] Nechť jsou $b_k$ a $c_k$ definovány jako v předchozí 
            definici. Potom:
            $$\fa k \in \N: c_k \leq a_k \leq b_k,$$
            Jelikož $\lim c_k = \lim b_k = A \in \Rstar,$ s použitím 
            Věty~\ref{th:dvastraznici} o dvou strážnících je i 
            $\lim a_k = A.$
        \item[$\implies$] Nechť $A \in \R.$ Z definice limity víme:
            $$\exists n_0 \in \N, \fa n \geq n_0, n \in \N: A - \e < a_n < A + \e.$$
            Zřejmě také platí:
                $$\fa n \geq n_0, n\in\N: A -\e \leq c_n \leq a_n \leq b_n \leq A + \e,$$
            a proto $\lim b_n = \lim c_n = A.$

            \Necht naopak $A = +\infty.$ Potom je posloupnost $\seq{a_n}$ shora 
            neomezená a $\limsup a_n = +\infty.$
            Zvolme $K \in \R.$ Z definice limity:
            $$\exists n_0 \in \N, \fa n \geq n_0, n\in\N: a_n > K.$$
            Potom $c_{n_0} \geq K.$ Jelikož posloupnost $\seq{c_n}$ je neklesající,
            platí:
            $$\fa n \geq n_0, n \in \N: c_n \geq K.$$ 
            Zřejmě:
            $$ \lim c_n = +\infty.$$
            Analogicky pro $A = -\infty.$
    \end{itemize}
\end{proof}

\begin{theorem}[Bolzano-Weierstrass]
    \label{th:bolzanoweierstrass}
    Z každé omezené posloupnosti lze vybrat konvergentní podposloupnost.
\end{theorem}

\begin{proof}
    Mějme posloupnost $\seq{a_n}.$ Jelikož je omezená, platí:
    $$\exists K, L \in \R: \fa n \in \N: K \leq a_n \leq L.$$
    Rozpůlme interval $[K,L]$ na dva nové intervaly: $[K, \frac{K+L}{2}],
    [\frac{K+L}{2}, L]$ (bod $\frac{K+L}{2}$ leží v obou intervalech). 
    Potom alespoň jeden z nových intervalů obsahuje nekonečně mnoho 
    členů posloupnosti $\seq{a_n}.$ Tento interval označíme $[K_1, L_1]$
    a znovu jej rozpůlíme na dva podintervaly. Ten, ve kterém se nachází
    nekonečně mnoho členů posloupnosti $\seq{a_n}$, označíme $[K_2, L_2].$

    Tento postup opakujeme a získáme tak posloupnost intervalů $[K_k, L_k],$ 
    pro něž platí:
    \begin{enumerate}[i.]
        \item $\fa k \in \N: [K_k, L_k] \supset [K_{k+1}, L_{k+1}]$
        \item $\fa k \in \N: L_k - K_k = (L-K)/2^k,$ a tedy velikost intervalů
            konverguje k nule.
        \item $\fa k \in \N: [K_k, L_k]$ obsahuje nekonečně mnoho členů
            posloupnosti $\seq{a_n}.$
    \end{enumerate}
    Můžeme proto vybrat rostoucí posloupnost přirozených čísel $n_k$ takovou, že
    $\fa k \in \N: a_{n_k} \in [K_k, L_k].$ 

    Díky vlastnosti $(i)$ 
    posloupnosti intervalů $[K_k, L_k]:$
    $$\exists x \in \R, \fa k \in \N: x \in [K_k, L_k].$$
    Tvrdím, že posloupnost $\seq{a_{n_k}}$ konverguje k $x.$ Zvolme $\e > 0$ a
    $k_0 \in \N$ takové, že 
    $$\fa k \geq k_0, k \in \N: L_k - K_k < \e.$$
    Potom
    $$\fa k \geq k_0, k \in \N: |a_{n_k} - x| < \e,$$
    jelikož jak $a_{n_k},$ tak $x$ náleží do $[K_k, L_k].$
\end{proof}

\begin{theorem}[Bolzano-Cauchyho podmínka]
    \label{th:bolzanocauchy}
    Posloupnost $\seq{a_n}_{n \in \N}$ má vlastní limitu, právě když
    splňuje Bolzano-Cauchyho podmínku, tedy:
    $$\fa \e > 0 \; \exists n_0 \in \N \; \fa m,n \in \N, n \geq n_0, m \geq n_0:
    |a_n - a_m| < \e.$$
\end{theorem}

\begin{proof}
    \leavevmode
    \begin{itemize}
        \item[$\implies$] $\lim a_n = A \in \R.$ Z definice limity:
            $$\fa \e > 0 \; \exists n_0 \in \N, \fa n \geq n_0, n\in \N:
            |a_n - A| < \frac{\e}{2}.$$
            Pro $m,n \geq n_0$ platí:
            $$|a_n - a_m| \leq |a_n -A | + |A - a_m| < \frac{\e}{2} + \frac{\e}{2}
            = \e.$$
        \item[$\impliedby$] Definujme posloupnosti:
            $$b_n = \sup\{a_n, a_{n+1}, ...\},$$
            $$c_n = \inf\{a_n, a_{n+1}, ...\}.$$
            Posloupnost $\seq{b_n}$ klesá k $\limsup a_n$; posloupnost $\seq{c_n}$
            stoupá k $\liminf a_n.$ Dále $\fa n \in \N: b_n \geq c_n.$
            V následujícím ukážeme, že $\liminf a_n = \limsup a_n,$ z čehož za
            použití věty o vztahu limity, limes superior a limes inferior 
            (Věta~\ref{th:limitalimsupliminf}) plyne, že posloupnost 
            $\seq{a_n}$ konverguje.

            Cauchyho podmínka říká, že
            $$\fa \e > 0 \; \exists n_0 \in \N \; \fa m,n \in \N, n \geq n_0, 
            m \geq n_0: |a_n - a_m| < \e.$$
            Zvolíme $m = n_0.$ Potom $\fa n \geq n_0, n\in\N:$
                $$a_{n_0} - \e < a_n < a_{n_0} + \e$$
                $$a_{n_0} - \e \leq c_n \leq a_n \leq b_n \leq a_{n_0} + \e$$
                $$a_{n_0} - \e \leq \liminf a_n \leq a_n \leq \limsup a_n \leq a_{n_0} + \e$$
            a tedy:
            $$\fa \e> 0: |\limsup a_n - \liminf a_n| \leq 2\e,$$
            z čehož vyplývá, že $\limsup a_n = \liminf a_n$ a posloupnost 
            $\seq{a_n}$ konverguje.

    \end{itemize}
\end{proof}

\section{Funkce jedné reálné proměnné}

\subsection{Základní definice}

\begin{definition}
    \newterm{Funkcí jedné reálné proměnné} rozumíme zobrazení
    $$f: M \rightarrow \R,$$
    kde $M \subset \R.$
\end{definition}

\begin{definition}
    Řekneme, že funkce $f: M \rightarrow \R, M \subset \R$ je:
    \begin{itemize} 
        \item \newterm{rostoucí}, pokud $\fa x,y \in M, x < y: f(x) < f(y),$
        \item \newterm{klesající}, pokud $\fa x,y \in M, x < y: f(x) > f(y),$
        \item \newterm{nerostoucí}, pokud $\fa x,y \in M, x < y: f(x) \geq f(y),$
        \item \newterm{neklesající}, pokud $\fa x,y \in M, x < y: f(x) \leq f(y).$
    \end{itemize}
\end{definition}

\begin{definition}
    Řekneme, že funkce $f: M \rightarrow \R, M \subset \R$ je:
    \begin{itemize}
        \item \newterm{sudá}, pokud $\fa x \in M: (-x \in M) \; \& \; (f(x) = f(-x)),$
        \item \newterm{lichá}, pokud $\fa x \in M: (-x \in M) \; \& \; (f(x) =- f(-x)),$
        \item \newterm{periodická}, pokud $\exists p > 0, \fa x \in M: (x+p \in M) \; \& \; 
            (x-p \in M) \; \& \; (f(x) = f(x+p)).$
    \end{itemize}
\end{definition}

\begin{definition}
    Řekneme, že funkce $f: M \rightarrow \R, M \subset \R$ je \newterm{omezená}
    (omezená shora, omezená zdola), jestliže $f(M)$ je omezená (omezená
    shora, omezená zdola) podmnožina $\R.$
\end{definition}

\begin{definition}
    \Necht $\delta > 0$ a $a \in \R.$ \newterm{Prstencové okolí} bodu je:
    \begin{align*}
        P(a, \delta) &= (a - \delta, a+ \delta) \setminus \{a\}, \\
        P(+\infty, \delta) &= (\frac{1}{\delta}, +\infty),\\
        P(-\infty, \delta) &= (-\frac{1}{\delta}, -\infty).
    \end{align*}
    \newterm{Pravé a levé prstencové okolí} bodu $a$ je:
    \begin{align*}
        P_+(a, \delta) &= (a, a+\delta), \\
        P_-(a, \delta) &= (a-\delta, a).
    \end{align*}
    \newterm{Okolí bodu} je: 
    \begin{align*}
        U(a, \delta) &= (a - \delta, a+ \delta), \\
        U(+\infty, \delta) &= (\frac{1}{\delta}, +\infty),\\
        U(-\infty, \delta) &= (-\frac{1}{\delta}, -\infty).
    \end{align*}
    \newterm{Pravé a levé okolí} bodu $a$ je:
    \begin{align*}
        U_+(a, \delta) &= [a, a+\delta), \\
        U_-(a, \delta) &= (a-\delta, a].
    \end{align*}
\end{definition}

\begin{definition}
    \Necht $f:M \rightarrow \R, M \subset \R.$ Řekneme, že $f$ má v bodě $a \in 
    \Rstar$ \newterm{limitu} $A \in \Rstar,$ jestliže platí:
    $$\fa \e > 0 \; \exists \delta > 0 \; \fa x \in P(a,\delta): f(x) \in U(A,\e).$$
    Značíme:
    $$\lim_{x\to a} f(x) = A.$$
\end{definition}

\begin{remark}[Poznámky k definici limity]
    \leavevmode
    \begin{itemize}
        \item Funkce $f$ nemusí být v bodě $a \in \Rstar$ definována, aby v něm
            měla limitu. Z definice limity vyplývá, že pokud $\lim_{x\to a} f(x)$
            existuje, tak je funkce $f$ definována na nějakém prstencovém okolí
            bodu $a.$

            Navíc, je-li $f$ v bodě $a$ definována, na hodnotě $f(a)$ nezáleží.
        \item Pokud $\lim_{\xtoa} f(x)$ existuje a je rovna $A$, tak potom je
            buď \newterm{vlastní} ($A \in \R$) nebo \newterm{nevlastní}
            ($A = \pm \infty$).

        \item $\lim_{\xtoa} f(x)$ je nazývá \newterm{limitou ve vlastním bodě},
            pokud $a \in \R$, nebo \newterm{limitou v nevlastním bodě},
            pokud $a = \pm \infty.$
    \end{itemize}
\end{remark}

\begin{definition}
    \Necht $f:M \rightarrow \R, M \subset \R.$ Řekneme, že $f$ má v bodě $a \in 
    \Rstar$ \newterm{limitu zprava (zleva)} rovnou $A \in \Rstar,$ jestliže platí:
    $$\fa \e > 0 \; \exists \delta > 0 \; \fa x \in P_+(a,\delta): 
    f(x) \in U(A,\e)$$ 
    $$(\fa \e > 0 \; \exists \delta > 0 \; \fa x \in P_-(a,\delta): 
    f(x) \in U(A,\e)).$$
    Značíme:
    $$\lim_{x\to a+} f(x) = A$$
    $$(\lim_{x\to a-} f(x) = A).$$
\end{definition}

\begin{observation}[Vztah limity a jednostranných limit]
    \label{obs:jednostrannelimity}
    \Necht $f:M \rightarrow \R, M \subset \R, a \in \Rstar, A \in \Rstar.$ 
    Potom:
    $$\lim_{\xtoa} f(x) = A \iff \lim_{\xtoa +} f(x) = \lim_{\xtoa -} f(x) = A.$$
\end{observation}

\begin{remark}[Příklady limit]
    \leavevmode
    \begin{itemize}
        \item $f(x) = x$ 
            
            Její limita v bodě $a \in \Rstar:$
            $$\lim_{\xtoa} f(x) = a.$$
            K $\e > 0$ volme $\delta = \e.$ Potom $f(P(a,\delta)) 
            \subseteq U(a,\e).$

        \item $f(x) = k$ 
            
            $\fa a \in \Rstar: \lim_{\xtoa} f(x) = k.$

        \item $f(x) = \sgn(x):$
            \begin{center}
                \begin{tikzpicture}
                    \begin{axis}[
                            axis lines=middle,
                            xlabel=$x$,
                            ylabel={$\sgn(x)$},
                            xmin=-3, xmax=3,
                            ymin=-1.5, ymax=1.5,
                            xtick=\empty,
                            ytick={0, 1},
                            extra y ticks={-1},
                            extra y tick style={
                                tick label style={anchor=west, xshift=3pt},
                            },
                            function line/.style={
                                black,
                                thick,
                                samples=2,
                            },
                            single dot/.style={
                                black,
                                mark=*,
                            },
                            empty point/.style={
                                only marks,
                                mark=*,
                                mark options={fill=white, draw=black},
                            },
                        ]
                        \addplot[function line, domain=\pgfkeysvalueof{/pgfplots/xmin}:0] {-1};
                        \addplot[function line, domain=0:\pgfkeysvalueof{/pgfplots/xmax}] {1};
                        \addplot[single dot] coordinates {(0, 0)};
                        \addplot[empty point] coordinates {(0, -1) (0, 1)};
                    \end{axis}
                \end{tikzpicture}
            \end{center}
            Z grafu je zřejmé, že jednostranné limity se sobě nerovnají:
            $$\lim_{x \to 0 +} \sgn(x) = 1,$$ 
            $$\lim_{x \to 0 -} \sgn(x) = -1,$$ 
            a tedy dle pozorování o vztahu limity a jednostranných limit
            (Pozorování~\ref{obs:jednostrannelimity})
            $\lim_{x \to 0} \sgn(x)$ neexistuje. 
        
        \item Dirichletova funkce:
            $$D(x) = \begin{cases}
                1, & \text{pokud $x \in \Q,$} \\
                0, & \text{pokud $x \in \R \setminus \Q.$}
            \end{cases}$$

            Tato funkce nemá limitu nikde, jelikož dle věty o hustotě $\Q$ 
            a $\R \setminus \Q$ (Věta~\ref{th:hustotaqrq})
            každé prstencové okolí bodu 
            $a\in \Rstar$ obsahuje alespoň jedno racionální a iracionální číslo.
            
        \item Riemannova funkce:
            $$R(x) = \begin{cases}
                \frac{1}{q}, & \text{pokud $x \in \Q,$ tj. $x = \frac{p}{q},$ kde
                $p \in \Z, q\in\N,$ a $p,q$ jsou nesoudělná,}\\
                0, & \text{pokud $x \in \R \setminus \Q.$}
            \end{cases}$$

            Jako domácí cvičení dokažte:
            $$\fa a \in \R: \lim_{\xtoa} R(x) = 0.$$
    \end{itemize}
\end{remark}

\begin{definition}
    \Necht $f:M \rightarrow \R, M \subset \R, a \in M.$ Řekneme, že $f$ je v 
    bodě \newterm{spojitá (spojitá zleva, spojitá zprava)}, jestliže:
    $$\lim_{x \to a} f(x) = f(a) \;\; (\lim_{x \to a+} f(x) = f(a),\;
    \lim_{x \to a-} f(x) = f(a))$$
\end{definition}

\begin{remark}[Příklady spojitých a nespojitých funkcí]
    \leavevmode
    \begin{itemize}
        \item $f(x) = x$

            Spojitá na $\R.$

        \item $f(x) = \sgn(x)$

            Spojitá na $\R \setminus \{0\}.$

        \item $f(x) = D(x),$ Dirichletova funkce

            Není spojitá v žádném bodě $\R.$

        \item $f(x) = R(x),$ Riemannova funkce

            Spojitá v $\R \setminus \Q.$
    \end{itemize}
\end{remark}

\subsection{Věty o limitách}

\begin{theorem}[Heine]
    \label{th:heine}
    Nechť $A \in \Rstar, f: M \rightarrow \R$ a $f$ je definována na prstencovém
    okolí bodu $a \in \Rstar.$ Následující tvrzení jsou ekvivalentní:
    \begin{enumerate}[i.]
        \item $\lim_{x\to a} f(x) = A$
        \item pro každou posloupnost $\seq{x_n}_{n \in \N}$ takovou, že:
            $$\fa n \in \N \; x_n \in M, x_n \neq a, \text{a zároveň} 
            \lim_{n \to \infty} x_n = a$$
            platí:
            $$\lim_{n\to\infty} f(x_n) = A.$$
    \end{enumerate}
\end{theorem}

\begin{proof}
    \leavevmode
    \begin{itemize}
        \item[$\implies$] Z definice limity:
            $$\fa \e > 0 \; \exists \delta > 0 \; \fa x \in P(a,\delta): 
            f(x) \in U(A,\e).$$
            Nechť máme dále posloupnost $\seq{x_n},$ jež splňuje podmínky
            bodu $(ii)$. Jelikož $\lim x_n = a$ a $\fa n \in \N: x_n \neq a$, 
            tak k $\delta > 0:$
            $$\exists n_0 \in \N, \fa n \geq n_0, n\in\N: x_n \in P(a,\delta)$$
            a tedy 
            $$\fa n \geq n_0, n \in \N: f(x_n) \in U(A, \e).$$ 
            Potom
            $$\lim_{\ntoinfty} f(x_n) = A.$$
        \item[$\impliedby$]
            Implikaci dokážeme nepřímo, tj. dokážeme tvrzení 
            $\neg (i.) \implies \neg (ii.),$
            tedy že z tvrzení, že limita funkce $f$ neexistuje nebo není
            rovna $A,$ vyplývá existence 
            alespoň jedné posloupnosti $\seq{x_n}$, která splňuje zadaná 
            kritéria a zároveň $\lim f(x_n) \neq A.$

            Pokud $\lim_{\xtoa} f(x)$ neexistuje nebo není rovna $A,$ potom
            $$\exists \e > 0 \; \fa \delta > 0 \; \exists x \in P(a,\delta):
            f(x) \not \in U(A,\e).$$
            Pro $\delta_n = \frac{1}{n}, n=1,2,3,\dots$ vezmeme takové $x$ a
            označíme ho $x_n.$ 
            
            Zřejmě platí, že $\lim_{\ntoinfty} x_n = a.$ Jelikož dané elementy
            $x_n$ vybíráme z prstencového okolí $a,$ platí, že $\fa n \in \N:
            x_n \neq a.$ Z definice této posloupnosti navíc vyplývá, že 
            $\fa n \in \N: f(x_n) \not \in U(A,\e),$ takže $\lim_{\ntoinfty}
            f(x_n) \neq A.$ Tím je implikace splněna.
    \end{itemize}
\end{proof}

\begin{theorem}[o jednoznačnosti limity]
    Funkce $f$ má v daném bodě nejvýše jednu limitu.
\end{theorem}

\begin{proof}
    Sporem. Nechť $A_1$ a $A_2$ jsou dvě různé limity funkce $f$ daném bodě 
    $a \in \Rstar.$  Mějme dále posloupnost $\seq{x_n}$, $\lim_{\ntoinfty} x_n = a,$
    $\fa n \in \N: x_n \neq a.$ Potom dle Heineho (Věta~\ref{th:heine})
    $\lim_{\ntoinfty} f(x_n) = A_1$ a zároveň $\lim_{\ntoinfty} f(x_n) = A_2.$
    Dostáváme tím spor s větou o jednoznačnosti limity posloupnosti 
    (Věta~\ref{th:jednoznacnostlimity}).
\end{proof}

\begin{theorem}[limita a omezenost]
    \Necht $f$ má vlastní limitu v bodě $a\in\Rstar.$ Pak existuje $\delta > 0$
    tak, že $f$ je na $P(a,\delta)$ omezená.
\end{theorem}

\begin{proof}
    \Necht $\lim_{\xtoa} f(x) = A \in \R.$ Z definice limity vyplývá, že:
    $$\fa \e > 0 \; \exists \delta > 0: f(P(a, \delta)) \subseteq U(A, \e).$$
    Jelikož je limita vlastní, platí dále:
    $$U(A, \e) = (A-\e, A + \e).$$

    Zvolme $\e = 1$. Platí:
    $$\exists \delta > 0 \; \fa x \in P(a,\delta): f(x) \in U(A,1) = (A - 1, A+1),$$
    a tedy $f(x)$ je omezená na $P(a, \delta).$
\end{proof}

\begin{theorem}[o aritmetice limit funkcí]
    \label{th:voalf}
    \Necht $a \in \Rstar,$ $\lim_{\xtoa} f(x) = A \in \Rstar,$ a $\lim_{\xtoa}
    g(x) = B \in \Rstar.$ Pak platí:
    \begin{enumerate}[i.]
        \item $\lim_{\xtoa} (f(x) + g(x)) = A+B,$ pokud je výraz $A+B$ definován;
        \item $\lim_{\xtoa} f(x)g(x) = AB,$ pokud je výraz $AB$ definován;
        \item $\lim_{\xtoa} \frac{f(x)}{g(x)} = \frac{A}{B},$ pokud je výraz 
            $\frac{A}{B}$ definován.
    \end{enumerate}
\end{theorem}

\begin{proof}
    Dokážeme pouze pro bod $(i)$; ostatní případy se řeší analogicky.

    Zvolme libovolnou posloupnost $\seq{x_n},$ splňující
    $$\lim_{\ntoinfty} x_n = a, \fa n \in \N: x_n \neq a.$$
    Potom dle Heineho (Věta~\ref{th:heine}):
    $$\lim_{\ntoinfty} f(x_n) = A,$$
    $$\lim_{\ntoinfty} g(x_n) = B.$$
    a dle věty o aritmetice limit posloupností (Věta~\ref{th:voal}):
    $$\lim_{\ntoinfty} (f(x_n) + g(x_n)) = A + B.$$
    Protože posloupnost $\seq{x_n}$ je libovolná, dle Heineho:
    $$\lim_{\xtoa} (f(x) + g(x)) = A+B.$$
\end{proof}

\begin{corollary}
    \Necht jsou funkce $f$ a $g$ spojité v bodě $a \in \R.$ Pak jsou i 
    funkce $f + g$, $f\cdot g$ spojité v bodě $a.$ Pokud je navíc $g(a) \neq 0$,
    pak je i funkce $\frac{f}{g}$ spojitá v $a.$
\end{corollary}

\begin{theorem}[Limita a uspořádání]
    \label{th:limitaausporadanifce}
    \Necht $a \in \Rstar.$
    \begin{enumerate}[i.]
        \item \Necht $\limxtoa f(x) > \limxtoa g(x).$ Pak existuje prstencové
            okolí $P(a,\delta)$ tak, že:
            $$\fa x \in P(a,\delta): f(x) > g(x).$$

        \item \Necht existuje prstencové okolí $P(a, \delta)$ tak, že:
            $$\fa x \in P(a, \delta): f(x) \leq g(x).$$
            \Necht existují $\limxtoa f(x)$ a $\limxtoa g(x).$ Potom platí:
            $$\limxtoa f(x) \leq \limxtoa g(x).$$

        \item \Necht na nějakém prstencovém okolí $P(a, \delta)$ platí $f(x)
            \leq h(x) \leq g(x).$ \Necht $\limxtoa f(x) = \limxtoa g(x) = A 
            \in \Rstar.$
            Pak existuje $\limxtoa h(x)$ a všechny tři limity se rovnají.
    \end{enumerate}
\end{theorem}

\begin{proof}
    \leavevmode
    \begin{enumerate}[i.]
        \item \Necht $\limxtoa f(x) = A,$ $\limxtoa g(x) = B, A > B.$ Zvolme
            $0 < \e < \frac{A-B}{2}.$ Dle definice limity:
            $$\exists \delta_1: f(P(a,\delta_1)) \subseteq U(A, \e),$$
            $$\exists \delta_2: g(P(a,\delta_2)) \subseteq U(B, \e).$$
            Zvolme $\delta_0 = \min\{\delta_1, \delta_2\}.$ Zřejmě:
            $$\fa x \in P(a,\delta_0): f(x) > g(x).$$

        \item Sporem. Důkaz je analogický k bodu ($ii$) v důkazu věty o 
            limitě a uspořádání posloupností (Věta~\ref{th:limitaausporadani}).

        \item Pro $\e > 0$ existují $\delta_1, \delta_2$ jako v bodě $(i)$. Pro
            $\delta_0 = \min\{\delta, \delta_1, \delta_2\}$ platí:
            $$h(P(a, \delta_0)) \subseteq U(A, \e),$$
            a tedy $\limxtoa h(x) = A.$

    \end{enumerate}
\end{proof}

\begin{definition}
    Mějme funkce $f: M \rightarrow \R, M \subset \R$ a $g: N \rightarrow \R, 
    N \subset \R.$ Pokud $g(N) \subseteq M$, potom funkci $h: N \rightarrow \R,
    h(x)=f(g(x))$ nazveme \newterm{složenou funkcí}. 
    
    Složenou funkci $h$ značíme: $h = f \circ g.$
    Funkci $f$ se říká \newterm{vnější funkce}, funkci $g$ \newterm{vnitřní funkce}.
\end{definition}

\begin{remark}[Vztah limit vnější, vnitřní a složené funkce]
    Nechť:
    $$\lim_{\xtoa} g(x) = A,$$
    $$\lim_{x \to A} f(x) = B.$$
    Platí obecně:
    $$\lim_{\xtoa} f(g(x)) = B?$$
    Neplatí! Uvažujme následující dvě funkce:
    $$g(x) = 3 \; \fa x \in N,$$
    $$f(x) = \begin{cases}
        1 &\text{pro $x = 3 $} \\
        0 &\text{pro $x \neq 3$}
    \end{cases}$$
    Zřejmě:
    $$\lim_{x \to 0} g(x) = \underbrace{3}_{A},$$
    $$\lim_{x \to A=3} f(x) = \underbrace{0}_{B}.$$
    Limita složené funkce:
    $$\lim_{x \to 0} f(g(x)) = \lim_{x \to 0} f(3) = \lim_{x \to 0} 1 = 1 \neq B.$$

    Schéma z počátku poznámky nicméně platí při splnění dodatečných 
    podmínek, které jsou popsány v následující větě.
\end{remark}

\begin{theorem}[Limita složené funkce]
    \label{th:slozenafunkce}
    Nechť funkce $f$ a $g$ splňují:
    \begin{enumerate}[i.]
        \item $\lim_{x \to a} g(x) = A,$
        \item $\lim_{y \to A} f(y) = B.$
    \end{enumerate}
    Je-li navíc splněna alespoň jedna z podmínek:
    \begin{itemize}
        \item[(P1)] $f$ je spojitá v $A,$
        \item[(P2)] $\exists \eta > 0 \; \fa x \in P(a,\eta): g(x) \neq A,$
    \end{itemize}
    pak platí $\lim_{x \to a} f(g(x)) = B.$
\end{theorem}

\begin{remark}
    Funkce $f$ a $g$ z předchozí poznámky nesplňovaly podmínky (P1) a (P2).
    Funkce $f$ nebyla spojitá v $A = 3$ a pro funkci $g$ neexistovalo
    prstencové okolí bodu $a = 0,$ ve kterém nenabývala své limity $A = 3.$
\end{remark}

\begin{proof}
    \leavevmode
    \begin{itemize}
        \item[(P1)] Díky existenci limity a spojitosti funkce $f$ v bodě $A$ 
            platí, že ke zvolenému $\e > 0$:
            $$\exists \varphi > 0: f(U(A,\varphi)) \subseteq U(B,\e).$$
            Dále, k danému $\varphi$:
            $$\exists \chi > 0: g(P(a, \chi)) \subseteq U(A,\varphi).$$
            Nakonec:
            $$f(g(P(a, \chi))) \subseteq f(U(A, \varphi)) \subseteq U(B,\e),$$
            a tedy $\lim_{x \to a} f(g(x)) = B.$
        \item[(P2)]Díky existenci limity funkce $f$ v bodě $A$ 
            platí, že ke zvolenému $\e > 0$:
            $$\exists \varphi > 0: f(P(A,\varphi)) \subseteq U(B,\e).$$
            Dále, k danému $\varphi$:
            $$\exists \chi > 0: g(P(a, \chi)) \subseteq U(A,\varphi).$$
            Pro $\psi = \min(\chi, \eta)$ díky podmínce (P2) dále platí:
            $$g(P(a, \psi)) \subseteq P(A,\varphi).$$
            Nakonec:
            $$f(g(P(a, \psi))) \subseteq f(P(A, \varphi)) \subseteq U(B,\e),$$
            a tedy $\lim_{x \to a} f(g(x)) = B.$
    \end{itemize}
\end{proof}

\begin{theorem}[limita monotónní funkce]
    \label{th:limitamonotonnifce}
    Nechť funkce $f$ je monotónní na intervalu $(a,b), \; a,b \in \Rstar.$
    Potom existuje $\lim_{x \to a+} f(x)$ i $\lim_{x \to b-} f(x).$
\end{theorem}

\begin{proof}
    Větu dokážeme pro $f$ neklesající a pro $\lim_{x \to a+}.$ 
    Ostatní případy se dokáží analogicky.

    Definujme množinu $M = f((a,b)) = \{f(x), x \in (a,b)\}$, položme
    $A \coloneqq \inf M$ a zvolme $\e > 0.$ Z vlastností infima 
    (Definice~\ref{def:inf}) víme, že:
    $$\exists y_0 = f(x_0) \in M: A \leq y_0 < A + \e.$$
    Potom díky monotonii funkce $f$:
    $$\fa x, \; a < x < x_0: f(x) \in U(A,\e).$$
    Nyní zvolíme $\delta > 0$ takové, aby $P_+(a,\delta) \subseteq (a,x_0).$
    Potom:
    $$\fa x \in P_+(a,\delta): f(x) \in U(A,\e),$$
    a tedy $\lim_{x \to a+} f(x) = A.$
\end{proof}

\subsection{Funkce spojité na intervalu}

\begin{definition}
    Množina $M \subseteq \R$ je \newterm{interval}, pokud 
    $\exists a,b \in \Rstar$ tak, že: 
    $$M = \{x \in \R: a \prec x \prec b \},$$ 
    kde relace $\prec$ je buď $\leq$ nebo $<.$

    Body $a,b$ nazýváme \newterm{krajními body} intervalu; ostatní body 
    intervalu $M$ nazýváme \newterm{vnitřními body}.
\end{definition}

\begin{observation}
    \label{obs:intervalconvex}
    Množina $M \subseteq \R$ je interval, právě když
    $$\fa x,y,z\in \R: x\leq z\leq y, x \in M, y\in M \implies z \in M,$$
    tj. právě když je \newterm{konvexní podmnožinou} $\R$.
\end{observation}

\begin{proof}
    \leavevmode
    \begin{itemize}
        \item[$\implies$] Zřejmé: Každý interval je konvexní množina.
        \item[$\impliedby$] Nechť $M \subseteq \R$ je konvexní množina.
            Označme $a \coloneqq \inf M$, $b \coloneqq \sup M.$
            Potom
            $$(a,b) \subseteq M \subseteq [a,b].$$
            Proč? Pokud $x \in (a,b),$ potom z definice suprema a infima
            $\exists \alpha, \beta \in M: \alpha < x < \beta.$ Díky konvexivitě
            je i $x \in M.$ Pokud naopak $x \in M,$ z definice suprema a infima
            vyplývá $a \leq x$ a $x \leq b$, a tedy $x \in [a,b].$

            Množina $M$ se tedy od $(a,b)$ liší jen eventuálním přidáním
            jednoho nebo obou bodů $a,b$, a je tedy intervalem.
    \end{itemize}
\end{proof}

\begin{definition}
    \Necht $f$ je funkce a $J$ interval. Řekneme, že $f$ je 
    \newterm{spojitá na intervalu} $J$, 
    jestliže je spojitá ve všech vnitřních bodech $J$. Je-li počáteční bod $J$
    prvkem $J$, tak požadujeme i spojitost zprava v tomto bodě, a je-li koncový
    bod $J$ prvkem $J$, tak požadujeme i spojitost zleva v tomto bodě.
\end{definition}

\begin{theorem}[Darboux]
    \label{th:darboux}
    \Necht $f$ je spojitá na $[a,b]$ a platí $f(a) < f(b).$ Pak pro každé $y \in 
    (f(a),f(b))$ existuje $x \in (a,b)$ tak, že $f(x) = y.$
\end{theorem}

\begin{proof}
    Definujme množinu $M \coloneqq \{x \in [a,b], f(x) < y \}.$ Označme dále 
    $x_0 = \sup M.$ Tvrdím, že $f(x_0) = y.$ Toto tvrzení nyní dokážeme
    sporem s vlastnostmi suprema.

    Nechť platí $f(x_0) < y.$ Zvolme $\e = \frac{y - f(x_0)}{2} > 0.$ Jelikož
    dle předpokladů je $f$ spojitá v $x_0$, existuje $\delta > 0 \; 
    \fa x \in U(x_0, \delta): f(x) \in U(f(x_0), \e),$ neboli $f(x) < y.$ 
    Zde nicméně dostáváme spor s definicí suprema: $x_0$ nemůže býti 
    supremem množiny $M$, neboť existují $x > x_0,$ pro které také 
    platí $f(x) < y.$

    Nechť naopak platí $f(x_0) > y.$ Zvolme $\e = \frac{f(x_0) - y}{2} > 0.$
    Jelikož je $f$ spojitá, existuje $\delta > 0 \; \fa x \in U(x_0, \delta):
    f(x) \in U(f(x_0), \e),$ neboli $f(x) > y.$ Potom ale $\fa x \in (x_0-\delta,
    x_0): f(x) > y$ a tedy $x_0 - \delta$ je také horní závora množiny $M$ 
    a dostáváme se tak do sporu s druhou vlastností suprema.
\end{proof}

\begin{theorem}[Zobrazení intervalu spojitou funkcí]
    Nechť $J$ je interval. \Necht funkce $f: J \rightarrow \R$ je spojitá.
    Pak je množina $f(J)$ také interval.
\end{theorem}

\begin{proof}
    Nechť $x,y \in f(J), z \in \R$ a $x \leq z \leq y.$ Potom $x = f(\alpha)$
    a $y = f(\beta)$ pro $\alpha, \beta \in J.$ Nechť bez újmy na obecnosti 
    $\alpha \leq \beta.$

    Protože zúžená funkce $f: [\alpha, \beta] \rightarrow \R$ je spojitá
    a $f(\alpha) \leq z \leq f(\beta),$ podle Darbouxovy věty 
    (Věta~\ref{th:darboux}) máme
    i $z = f(\gamma)$ pro nějaké $\gamma \in [\alpha,\beta].$ Množina $f(J)$ je 
    tedy konvexní a dle Pozorování~\ref{obs:intervalconvex} je tedy interval.
\end{proof}

\begin{definition}
    Nechť $f: M \rightarrow \R, M \subseteq \R.$ Řekneme, že funkce $f$ nabývá v
    bodě $a\in M$
    \begin{center}
        \begin{tabular}{lr}
            \newterm{maxima} na $M$, jestliže $\fa x \in M:$ &$f(x) \leq f(a),$ \\
            \newterm{minima} na $M$, jestliže $\fa x \in M:$ &$f(x) \geq f(a),$ \\
            \newterm{ostrého maxima} na $M$, jestliže $\fa x \in M, x \neq a:$ &$f(x) < f(a),$ \\
            \newterm{ostrého minima} na $M$, jestliže $\fa x \in M, x \neq a:$ &$f(x) > f(a),$ \\
        \end{tabular}
    \end{center}
    \newterm{lokálního maxima (ostrého lokálního maxima, ostrého lokálního minima,
    lokálního minima)}, jestliže existuje $\delta > 0$ tak, že $f$ nabývá na $M
    \cap U(a,\delta)$ svého maxima (ostrého maxima, ostrého minima, minima).
\end{definition}

\begin{observation}[Heineho věta pro spojitost]
    \label{obs:heinespojitost}
    Následující tvrzení jsou ekvivalentní:
    \begin{enumerate}[i.]
        \item $f$ je spojitá v $a$, tj. $\lim_{\xtoa} f(x) = f(a);$
        \item pro každou posloupnost $\seq{x_n}_{n \in \N}$ takovou, že:
            $$\fa n \in \N \; x_n \in M, x_n \neq a, \text{a zároveň} 
            \lim_{n \to \infty} x_n = a$$
            platí:
            $$\lim_{n\to\infty} f(x_n) = f(a).$$
    \end{enumerate}
\end{observation}

\begin{theorem}[spojitost funkce a nabývání extrémů]
    \label{th:spojitaextremy}
    Nechť $f$ je spojitá funkce na intervalu $[a,b]$. Pak funkce $f$ nabývá na 
    $[a,b]$ svého maxima a minima.
\end{theorem}

\begin{proof}
    Označme $A \coloneqq \sup \{f(x), x \in [a,b]\}.$ Z vlastností suprema
    existuje posloupnost $\seq{x_n}$ taková, že $\lim_{\ntoinfty} f(x_n) = A$, 
    např. $f(x_1) > A -1, f(x_2) > A - \frac{1}{2}, f(x_3) > A - \frac{1}{3},$ atd.

    Jelikož $\fa n \in \N: x_n \in [a, b]$, posloupnost $\seq{x_n}$ je omezená, 
    a tedy dle Bolzano-Weierstrassovy věty (Věta~\ref{th:bolzanoweierstrass})
    existuje vybraná konvergentní podposloupnost $\seq{x_{n_k}}$:
    $$\lim_{k \to \infty} x_{n_k} = z \in [a,b].$$
    Jelikož je funkce $f$ v bodě $z$ spojitá, platí dle Heineho věty pro
    spojitost (Pozorování~\ref{obs:heinespojitost}):
    $$\lim_{k \to \infty} f(x_{n_k}) = f(z).$$
    Zároveň ale víme, že 
    $$\lim_{\ntoinfty} f(x_n) = A$$
    Dle věty o limitě vybrané posloupnosti (Věta~\ref{th:vybranalimita}) je i
    $$\lim_{k \to \infty} f(x_{n_k}) = A$$
    a dle věty o jednoznačnosti limity posloupnosti 
    (Věta~\ref{th:jednoznacnostlimity}) platí $f(z) = A,$ 
    a tedy funkce $f$ nabývá svého maxima $A$ v bodě $z \in [a,b].$

    Důkaz minima je analogický.
\end{proof}

\begin{theorem}[spojitost funkce a omezenost]
    \Necht $f$ je spojitá funkce na intervalu $[a,b].$ Pak je funkce $f$ na 
    $[a,b]$ omezená.
\end{theorem}

\begin{proof}
    Dle věty o spojitosti funkce a nabývání extrémů (Věta~\ref{th:spojitaextremy})
    nabývá funkce $f$ na $[a,b]$ svého maxima ($A$) i minima ($B$). 
    Potom $\fa x \in [a,b]: B \leq f(x) \leq A$ a funkce $f$ je tedy na intervalu
    $[a,b]$ omezená.
\end{proof}

\begin{definition}
    Nechť $f$ je funkce a $J$ interval. Řekneme, že $f$ je \newterm{prostá} na $J$,
    pokud $\fa x,y \in J: x \neq y \implies f(x) \neq f(y).$

    Pro prostou funkci $f: J \rightarrow \R$ definujeme \newterm{inversní funkci}
    $f^{-1}: f(J) \rightarrow \R$ předpisem: $f^{-1}(y) = x \iff f(x) = y.$
\end{definition}

\begin{theorem}[o inversní funkci]
    \label{th:inversnifce}
    \Necht $f$ je spojitá a rostoucí (klesající) funkce na intervalu $J$. Potom
    je funkce $f^{-1}$ spojitá a rostoucí (klesající) na itervalu $f(J).$
\end{theorem}

\begin{proof}
    Nechť je $f$ například rostoucí. Nejprve dokážeme sporem, že i $\inv{f}$
    je rostoucí. Nechť existují $y_1,y_2 \in f(J), y_1 < y_2$ tak, že $\inv{f}(y_1) 
    > \inv{f}(y_2).$ Potom ovšem dostáváme spor:
    $y_1 = f(\inv{f}(y_1)) > f(\inv{f}(y_2)) = y_2.$

    Dokažme nyní spojitost $\inv{f}.$ Zvolme $y_0 \in f(J), y_0 = f(x_0), \e > 0$ a 
    uvažujme nejprve možnost, kdy $y_0$ je vnitřní bod $f(J)$ (a tedy $x_0$ 
    je vnitřní bod $J$). Potom existují $x_1,x_2$ tak,
    že $x_1 < x_0 < x_2$ a $(x_1,x_2) \subseteq U(x_0, \e).$ Zvolme
    $\delta > 0$ tak, aby $U(y_0,\delta) \subseteq (f(x_1), f(x_2)).$ Potom:
    $$\inv{f}(U(y_0,\e)) \subseteq \inv{f}(f(x_1),f(x_2)) = (x_1,x_2) \subseteq
    U(x_0,\e) = U(\inv{f}(y_0),\e).$$

    Uvažujme dále možnost, že je $J$ uzavřený interval a $x_0$ je jeho levý 
    krajní bod. Zvolme $x_1 \in U_+(x_0, \e)$ a $\delta = f(x_1) - y_0.$ Potom
    díky monotonii funkce $\inv{f}$ platí pro všechny $y \in U_+(y_0, \delta):$
    $$\inv{f}(y) \in U_+(x_0, \e) \subseteq U(x_0, \e).$$
    Pravý krajní bod se řeší analogicky.
\end{proof}

\subsection{Elementární funkce}

\subsubsection{Exponenciála a logaritmus}

\begin{theorem}[Zavedení exponenciály]
    \label{th:exp}
    Existuje právě jedna funkce $\exp : \R \rightarrow \R$ splňující následující
    dvě podmínky:
    \begin{enumerate}[i.]
        \item $\exp(x+y) = \exp(x)\exp(y),$
        \item $\fa x \in \R: \exp(x) \geq 1+x$
    \end{enumerate}
\end{theorem}

\begin{proof}
    Předpokládejme nejprve, že daná funkce existuje, a ukažme si, že v tom případě
    je definována jednoznačně: Postupně odvodíme několik jejích vlastností, až se
    tak dostaneme k jejímu jednoznačnému vyjadření. Poté dokážeme i její 
    existenci\footnote{Důkaz této věty jsem zpracoval na základě 
    poznámek Petra Baudiše z přednášek Luboše Picka: \url{http://math.or.cz/analyza}.}.

    \begin{enumerate}[A.]
        \item Jednoznačnost.

            \begin{enumerate}[1.]
                \item $\fa n \in \N \; \fa x \in \R: \exp(nx) = \exp(x)^n.$

                    Důkaz indukcí:
                    \begin{enumerate}[I.]
                        \item $\exp(1x) = \exp(x)$
                        \item $\exp((n+1)x) = \exp(nx + x) = \exp(nx)\exp(x) = (\exp(x))^n
                            \exp(x) = (\exp(x))^{n+1}$
                    \end{enumerate}

                \item $\exp(0) = 1.$
                    
                    Rozepišme nejprve výraz $\exp(0)$:
                    $$\exp(0) = \exp(0 + 0) \stackrel{i.}{=} \exp(0)\exp(0) = (\exp(0))^2.$$
                    $\exp(0)$ tedy může být buď $0$ nebo $1$. První možnost je 
                    ve sporu s podmínkou $(ii)$, a proto $\exp(0) = 1.$

                \item $\fa x \in \R: \exp(-x) = \frac{1}{\exp(x)} \land \exp(x) \neq 0.$

                    $1 = \exp(0) = \exp(x-x) = \exp(x)\exp(-x).$

                \item $\lim_{x \to \infty} \exp(x) = +\infty.$

                    Plyne z podmínky $(ii)$ a Věty~\ref{th:limitaausporadanifce} (limita a 
                    uspořádání).

                \item $\lim_{x \to -\infty} \exp(x) = 0.$

                    Vyplývá z předchozích dvou bodů.

                \item $\fa x \in \R: \exp(x) > 0.$

                    $\exp(x) = \exp(\frac{x}{2} + \frac{x}{2}) \stackrel{i.)}{=}
                                    \exp(\frac{x}{2})\exp(\frac{x}{2}) \geq 0.$

                \item $\fa x > 0: \exp(x) > 1.$

                    Vyplývá z podmínky $(ii).$

                \item Exponenciála je rostoucí funkce na $\R.$

                    $\fa x, y \in \R, x < y: 1 \stackrel{7.}{<} \exp(y - x) = 
                    \frac{\exp(y)}{\exp(x)}$. Jelikož se jedná o kladnou funkci, platí
                    $\exp(y) > \exp(x).$

                \item $\lim_{x \to 0} \frac{\exp(x) -1}{x} = 1.$

                    $$\frac{1}{\exp(x)} \stackrel{3.}{=} \exp(-x) \stackrel{ii.}{\geq} 1-x,$$          
                    a tedy: $$1 + x \stackrel{(ii.)}{\leq} \exp(x) \leq \frac{1}{1-x}.$$ 
                    Další úpravou získáváme:
                    $$x \leq \exp(x) -1  \leq \frac{1}{1-x} - 1 = \frac{x}{1-x}$$
                    a po vydělení výrazem $x$:
                    $$1 \leq \frac{\exp(x) - 1}{x} \leq \frac{1}{1-x}.$$
                    Výslednou limitu získáme díky Větě~\ref{th:limitaausporadanifce} 
                    (limita a uspořádání).

                \item $\exp(x)$ je spojitá na $\R.$
                    \begin{align*}
                        \lim_{x \to x_0} (\exp(x) - \exp(x_0)) &= \lim_{x \to x_0} 
                        (\exp((x-x_0) + x_0) - \exp(x_0)) \\
                        &= \lim_{x\to x_0} (\exp(x - x_0)\exp(x_0) - \exp(x_0)) \\
                        &= \lim_{x \to x_0} \exp(x_0) \underbrace{\frac{\exp(x-x_0) - 1}{x - x_0}}_{\to 1} 
                        \underbrace{(x - x_0)}_{\to 0} \\
                        &= 0
                    \end{align*}

                \item $\fa x \in \R: \exp(x) = \lim_{\ntoinfty} 
                    \left(1 + \frac{x}{n}\right)^n$

                    Pokud se nám podaří dokázat tuto rovnost, dokážeme
                    díky jednoznačnosti limity posloupnosti (Věta~\ref{th:jednoznacnostlimity})
                    i jednoznačnost definice exponenciály.

                    Dle podmínky $(ii)$ platí:
                    $$\exp\left(-\frac{x}{n}\right) \geq 1 - \frac{x}{n}$$
                    Zvolme $k > |x|.$ Pro $n \geq k, n \in \N$ je i pravá strana 
                    nerovnice kladná a při umocnění obou stran na $n$-tou dostáváme:
                    $$\exp\left(-\frac{x}{n}\right)^n \stackrel{1.}{=} 
                    \exp\left(-n\frac{x}{n}\right) = \exp(-x) 
                        \stackrel{3.}{=} \frac{1}{\exp(x)}
                        \geq \left(1-\frac{x}{n}\right)^n$$
                    a tedy:
                    $$\exp{x} \leq \frac{1}{\left(1 - \frac{x}{n}\right)^n}$$
                    Můžeme tedy psát:
                    $$\left(1 + \frac{x}{n}\right)^n \leq \exp\left(\frac{x}{n}\right)^n
                    = \exp(x) \leq \frac{1}{\left(1 - \frac{x}{n}\right)^n}$$
                    a po vydělení výrazem $\left(1 + \frac{x}{n}\right)^n$:
                    $$1 \leq \frac{\exp(x)}{\left(1 + \frac{x}{n}\right)^n} 
                    \leq \frac{1}{\left(1 - \frac{x^2}{n^2}\right)^n}$$
                    Dle Bernoulliho (Věta~\ref{th:bernoulli}) dále platí:
                    $$\frac{1}{\left(1 - \frac{x^2}{n^2}\right)^n} \leq
                    \frac{1}{\left(1 - \frac{x^2}{n}\right)}$$
                    a proto:
                    $$1 \leq \frac{\exp(x)}{\left(1 + \frac{x}{n}\right)^n} 
                    \leq \underbrace{\frac{1}{\left(1 - \frac{x^2}{n}\right)}}_{\to 1 \text{ při } n \to \infty}$$
                    Nyní již stačí využít strážníků (Věta~\ref{th:limitaausporadani})
                    k dokázání limity z počátku.             
            \end{enumerate}

        \item Existence. 

            V následujících bodech ukážeme, že posloupnost:
                $$a_n = \left(1 + \frac{x}{n}\right)^n$$
            je pro dostatečně velká $n$ neklesající a omezená pro všechna $x \in \R,$
            což zaručí, že tato posloupnost má limitu a že tato limita je vlastní
            (viz také Větu~\ref{th:monotonniposl}). 
            Tím dokončíme formální zavedení exponenciály.

            \begin{enumerate}[I.]
                \item Monotonie.

                    Podobně jako výše zvolme $k > |x|.$ Pro $n \geq k, n \in \N$
                    platí:
                    $$\sqrt[n+1]{\left(1 + \frac{x}{n}\right)^n} \leq 
                    \frac{n\cdot(1+\frac{x}{n})+1}{n+1} = 1 + \frac{x}{n+1}$$

                    Zde jsme využili vztahu geometrického a aritmetického průměru
                    (Věta~\ref{th:agprumer}) pro $z_1 = z_2 = \dots = z_n = 1 + 
                    \frac{x}{n}$ a $z_{n+1} = 1.$ Pokud umocnímě obě strany 
                    nerovnosti na $(n+1)$-tou, dostáváme:                    
                    $$\left(1 + \frac{x}{n}\right)^n \leq 
                    \left(1 + \frac{x}{n+1}\right)^{n+1},$$
                    a tedy platí $\fa n \geq k: a_n \leq a_{n+1}.$ Posloupnost
                    $\seq{a_n}$ je tedy neklesající.

                \item Omezenost.

                    Nyní ukážeme, že posloupnost $\seq{a_n}$ má omezenou 
                    podposloupnost $\seq{a_{n_k}}.$ Z toho pak vyplývá, že 
                    i posloupnost $\seq{a_n}$ je omezená\footnote{Zde
                    využíváme jednoduchého pozorování, které tvrdí, že
                    posloupnost, která je monotónní a která má omezenou
                    podposloupnost, je omezená. Nechť $\seq{a_n}$ je 
                    neklesající posloupnost a nechť pro její podposloupnost 
                    $\seq{a_{n_k}}$ platí: $\fa k \in \N: |a_{n_k}| \leq L.$ 
                    Nechť, pro spor, posloupnost $\seq{a_n}$ není omezená. 
                    Potom $\fa K \; \exists n_0: a_{n_0} > K.$ Díky 
                    monotonii dále platí: $\fa n \geq n_0: a_n > K.$ 
                    Zvolme $K = L.$ Potom $\fa k \geq n_0: n_k \geq k \geq n_0: 
                    a_{n_k} > L,$ čímž dostáváme spor s omezeností
                    podposloupnosti $\seq{a_{n_k}}.$}.

                    Zvolme $k > |x|.$ Dokážeme, že
                    $$\fa n \in \N: \left(1 + \frac{x}{nk}\right)^{nk} 
                    \leq \left(1 - \frac{x}{k}\right)^{-k},$$
                    a tedy, že $\fa n\in \N: a_{nk} \leq
                    \left(1 - \frac{x}{k}\right)^{-k}.$
                    
                    Pro zvolené $k$ a pro $\fa n \in \N$ platí:
                    \begin{align*}
                        \left(1 + \frac{x}{nk}\right)^{-n} 
                            &= \left(\frac{nk+x}{nk}\right)^{-n} \\
                            &= \left(\frac{nk}{nk + x}\right)^n \\
                            &= \left(1 - \frac{x}{nk + x}\right)^n \\
                            &\geq 1 - \frac{nx}{nk + x} \tag{Bernoulliho nerovnost,
                                Věta~\ref{th:bernoulli}} \\
                                &\geq 1 - \frac{x}{k} \tag{$\frac{x}{k} \geq \frac{nx}{nk+x}$} \\
                                &> 0.
                    \end{align*}
                    Po umocnění obou stran nerovnice na $k$-tou dostáváme:
                    $$\left(1 + \frac{x}{nk}\right)^{-nk}
                    \geq \left(1 - \frac{x}{k}\right)^k$$
                    a po úpravě:
                    $$\left(1 + \frac{x}{nk}\right)^{nk} 
                    \leq \left(1 - \frac{x}{k}\right)^{-k}$$
            \end{enumerate}
    \end{enumerate}
\end{proof}

\begin{definition}
    Funkci inversní k exponenciále $\exp$ nazveme \newterm{logaritmus} $\log.$
\end{definition}

\begin{theorem}[Vlastnosti logaritmu]
    Funkce $\log$ splňuje:
    \begin{enumerate}[a)]
        \item $\log: (0, \infty) \rightarrow \R$ je spojitá rostoucí
            funkce.
        \item $\fa x,y > 0: \log(xy) = \log(x) + \log(y).$
        \item $\lim_{x \to 1} \frac{\log(x)}{x - 1} = 1.$
    \end{enumerate}
\end{theorem}

\begin{proof}
    V důkazech budeme využívat vlastnosti exponenciály dokázané v předchozí
    větě (Věta~\ref{th:exp}).
    \begin{enumerate}[a)]
        \item Exponenenciála $\exp: \R \rightarrow (0, \infty)$ je rostoucí
            a spojitá funkce. Podle věty o inversní 
            funkci (Věta~\ref{th:inversnifce}) je tedy i logaritmus jako
            její inversní funkce spojitá a rostoucí funkce.
            
        \item $\log(x) = A, x = \exp(A)$ a $\log(y) = B, y = \exp(B).$
            Potom:
            $$xy = \exp(A)\exp(B) = \exp(A + B)$$
            a tedy
            $$\log(xy) = A + B = \log(x) + \log(y).$$

        \item Definujme funkce
            $$f(y) = \frac{\exp(y) - 1}{y}, \; g(x) = \log(x)$$ 
            pro jejichž limity platí: 
            $$\lim_{y \to 0} f(y) = 1, \; \lim_{x \to 1} g(x) = 0\footnote{Víme, 
                že $\exp(0) = 1$ a tedy $\log(1) = 0.$ Dále víme, že
                logaritmus je spojitá funkce.}.$$
            Potom pro limitu složené funkce $f(g(x))$ v bodě $1$ platí:
            $$\lim_{x \to 1} f(g(x)) = 1.$$
            Zde jsme využili větu o limitě složené funkce 
            (Věta~\ref{th:slozenafunkce}) a její podmínky (P2), 
            tj. že funkce $\log$ nenabývá v prstencovém okolí bodu $1$ hodnoty $0.$
            Můžeme tedy psát:
            $$1 = \lim_{x \to 1} f(g(x)) = \lim_{x \to 1} \frac{\exp(\log(x)) - 1}
            {\log(x)} = \lim_{x \to 1} \frac{x - 1}{\log(x)}.$$
    \end{enumerate}
\end{proof}

\begin{definition}
    Nechť $a > 0$ a $b \in \R.$ Pak definujeme \newterm{obecnou mocninu} jako:
    $$a^b \coloneqq \exp(b\log(a)).$$
    Je-li navíc $b > 0,$ pak definujeme \newterm{logaritmus při základu $b$}
    následovně:
    $$\log_b(a) \coloneqq \frac{\log(a)}{\log(b)}.$$
\end{definition}

\begin{remark}[Korektnost definice obecné mocniny]
    Pro $x > 0$ platí:
    \begin{align*}
        x^n &= \exp(n\log(x)) \tag{nová definice} \\
            &= \exp(\log(x^n)) \tag{matematickou indukcí} \\
            &= x^n \tag{stará definice}
    \end{align*}
\end{remark}

\begin{remark}[Logaritmus při základu $10$]
    Na případu $b = 10$ ukážeme, že $b^{\log_b(x)} = x,$ a tedy že definice 
    logaritmu při základu $b$ je korektní:
    $$10^{\log_{10}(x)} = 10^{\frac{\log(x)}{\log(10)}} = 
    (e^{\log(10)})^{\frac{\log(x)}{\log(10)}} = e^{\log(x)} = x.$$
\end{remark}

\begin{remark}[Odmocnina jako obecná mocnina]
    $$\sqrt[n] x = \begin{cases}
        \exp\left(\frac{1}{n}\log(x)\right) &x>0, \\
        0 &x = 0.
    \end{cases}$$
\end{remark}

\subsubsection{Goniometrické funkce}
\begin{theorem}
    \label{th:goniom}
    Existuje právě jedna funkce $\sin:\R \rightarrow \R$ a právě jedna
    funkce $\cos:\R \rightarrow \R$ splňující:
    \begin{enumerate}[a)]
        \item \begin{tabular}[t]{ll}
                $\fa x,y \in \R:$
                &$\sin(x +y) = \sin x \cos y + \cos x \sin y$ \\
                &$\cos(x + y) = \cos x \cos y - \sin x \sin y$ \\
                &$\cos(-x) = \cos x,$ \\
                &$sin(-x) = -\sin x,$
            \end{tabular}
        \item $\exists \pi > 0$ tak, že $\sin$ je rostoucí na $[0, \frac{\pi}{2}]$
            a $\sin(\frac{\pi}{2}) = 1,$
        \item $$\lim_{x \to 0} \frac{\sin x}{x} = 1.$$
    \end{enumerate}
\end{theorem}

\begin{remark}[Další vlastnosti goniometrických funkcí]
    Ze "základních" vlastností goniometrických funkcí uvedených v předchozí 
    větě lze odvodit jejich další vlastnosti: periodicita, intervaly monotónnosti, 
    ostatní součtové vzorce, atd. V následujícím textu budeme předpokládat,
    že tyto vlastnosti známe ze střední školy, a dokazovat je nebudeme.
\end{remark}

\begin{remark}
    Dokažme tuto limitu:
    $$\lim_{x \to 0} \frac{1 - \cos x}{x^2}.$$
    Nejprve daný výraz v několika krocích upravíme:
    $$\lim_{x \to 0} \frac{1 - \cos x}{x^2} = \lim_{x \to 0}\frac{1 - \cos x}{x^2}
    \frac{1 + \cos x}{1 + \cos x} = \lim_{x \to 0} \frac{1 - \cos^2 x}{x^2}
    \frac{1}{1 + \cos x} = \lim_{x \to 0} \frac{\sin^2 x}{x^2} \frac{1}{1 + \cos x}.$$

    Jelikož
    $$\lim_{x \to 0} \frac{\sin^2 x}{x^2} = \left(\lim_{x\to 0} 
    \frac{\sin x}{x}\right)^2 = 1$$
    a kosinus je spojitá funkce (jak dokážeme později), můžeme využít věty o
    aritmetice limit (Věta~\ref{th:voalf}) a psát:
    $$\lim_{x \to 0} \frac{1 - \cos x}{x^2} = 
    \lim_{x \to 0} \frac{\sin^2 x}{x^2} \frac{1}{1 + \cos x}
    = \left(\lim_{x\to 0}  \frac{\sin x}{x}\right)^2 \lim_{x \to 0}
    \frac{1}{1 + \cos x} = \frac{1}{2}.$$
\end{remark}

\begin{definition}
    Pro $x \in \R \setminus \{\frac{\pi}{2} + k\pi, k \in \Z\}$ a
    $y \in \R \setminus \{k\pi, k \in \Z\}$ definujeme funkce \newterm{tangens}
    a \newterm{kotangens} předpisem:
    $$\tan x = \frac{\sin x}{\cos x} \text{ a } \cotg y = \frac{\cos y}{\sin y}.$$
\end{definition}

\begin{theorem}[spojitost sinu a kosinu]
    Funkce $\sin, \cos, \tan$ a $\cotg$ jsou spojité na svém definičním oboru.
\end{theorem}

\begin{proof}
    Začněme sinem. Nechť $a \in \R.$ Potom:
        \begin{align*}
            \lim_{x \to a} (\sin x - \sin a) 
            &= \lim_{x \to a} 2\cdot\sin\left(\frac{x - a}{2}\right)
                \cos\left(\frac{x-a}{2}\right) \tag{goniometrický vzorec} \\
            &= \lim_{x \to a} 2 \cdot 
                \underbrace{\frac{\sin\left(\frac{x - a}{2}\right)}{\frac{x - a}{2}}}_{\to 1}
                \underbrace{\frac{x - a}{2}}_{\to 0} 
                \underbrace{\cos\left(\frac{x-a}{2}\right)}_{\leq 1}
                \tag{viz níže} \\
            &= 0.
        \end{align*}
    Ve druhém kroku jsme využili věty o limitě složené funkce (Věta~\ref{th:slozenafunkce})
    pro:
    $$f(y) = \frac{\sin y}{y} \text{ a } g(x) = \frac{x-a}{2}$$
    a to za podmínky (P2).

    Nyní dokážeme spojitost pro kosinus. Pokud do vzorce:
    $$\sin(x + y) = \sin x \cos y + \cos x \sin y$$
    dosadíme $y = \frac{\pi}{2},$ získáme následující vztah mezi sinem 
    a kosinem:
    $$\cos x = \sin(x + \frac{\pi}{2}).$$ 
    Dle věty o limitě složené funkce (Věta~\ref{th:slozenafunkce})
    je tedy kosinus spojitý, neboť jak sinus, tak $x \rightarrow x + \frac{\pi}{2}$
    jsou spojité funkce.

    Dle věty o aritmetice limit funkcí (Věta~\ref{th:voalf}) (limita podílu) 
    jsou i $\tan$ a $\cotg$ spojité funkce.
\end{proof}

\begin{definition}
    Nechť
    \begin{align*}
        \sin^* x &= \sin x \text{ pro } x \in [-\frac{\pi}{2}, \frac{\pi}{2}], \\
        \cos^* x &= \cos x \text{ pro } x \in [0, \pi], \\
        \tan^* x &= \tan x \text{ pro } x \in (-\frac{\pi}{2}, \frac{\pi}{2}) \text{ a} \\
        \cotg^* x &= \cotg x \text{ pro } x \in (0, \pi).
    \end{align*}
    Definujeme $\arcsin$ (resp. $\arccos$, $\arctan$, $\arccotg$) jako inversní
    funkce k funkci $\sin^*$ (resp. $\cos^*, \tan^*, \arctan^*$).
\end{definition}

\begin{remark}
    $\arcsin(\sin x) = x$ pouze pro $x \in [-\frac{\pi}{2}, \frac{\pi}{2}].$
\end{remark}

\begin{remark}[Příklady limit]
    \leavevmode
    \begin{itemize}
        \item $\lim_{x \to 0} \frac{\arcsin x}{x}.$

            Definujme následující dvě funkce:
            $$f(y) = \begin{cases}
                \frac{\sin y}{y} &y \neq 0 \\
                1 &y = 0
            \end{cases}, \; g(x) = \arcsin x.$$
            Potom limita jejich složené funkce v bodě $0$ se rovná převrácené
            hodnotě hledané limity:
            $$\lim_{x \to 0} f(g(x)) = \limxo \frac{\sin(\arcsin x)}{\arcsin x} 
            = \limxo \frac{x}{\arcsin x}$$
            Víme dále, že:
            $$\limxo g(x) = 0 \text{ a } \limxo f(y) = 1.$$
            Za použití věty o limitě složené funkce (Věta~\ref{th:slozenafunkce})
            a podmínky (P1) (funkce $f$ je v bodě $0$ spojitá):
            $$\limxo f(g(x)) = 1,$$
            a tedy:
            $$\lim_{x \to 0} \frac{\arcsin x}{x} = 1.$$

        \item $\lim_{\ntoinfty} n\cdot\sin(\frac{1}{n}).$

            $$\lim_{\ntoinfty} n\cdot\sin(\frac{1}{n}) = \lim_{\ntoinfty}
            \frac{\sin(\frac{1}{n})}{\frac{1}{n}} = 1.$$
    \end{itemize}
\end{remark}

\subsection{Derivace funkce}

\begin{definition}
    Nechť $f$ je reálná funkce a $a \in \R.$ Pak \newterm{derivací $f$ v bodě $a$}
    budeme rozumět:
    $$f'(a) = \limho \frac{f(a+h) - f(a)}{h};$$
    \newterm{derivací $f$ v bodě $a$ zprava} budeme rozumět:
    $$f_+'(a) = \limhop \frac{f(a+h) - f(a)}{h};$$
    \newterm{derivací $f$ v bodě $a$ zleva} budeme rozumět:
    $$f_-'(a) = \limhom \frac{f(a+h) - f(a)}{h}.$$
\end{definition}

\begin{remark}
    $$f'(a) = \limho \frac{f(a+h) - f(a)}{h} = \limxa \frac{f(x) - f(a)}{x-a}.$$
\end{remark}

\begin{remark}[Příklady limit]
    \leavevmode
    \begin{itemize}
        \item $f(x) = x^n, a \in \R, n \in \N.$
            \begin{align*}
                f'(a) &= \limho \frac{(a+h)^n - a^n}{h}  \\
                &= \limho \frac{a^n + \binom{n}{1}a^{n-1}h
                + \binom{n}{2}a^{n-2}h^2 + \dots + \binom{n}{n}h^n - a^n}{h} \\
                &=\binom{n}{1}a^{n-1} \\
                &=na^{n-1}.
            \end{align*}

        \item $f(x) = \sgn(x).$

            Vyjádřeme nejdříve jednostranné limity:
            $$f_+'(0) = \limhop \frac{f(h) - f(0)}{h} = \limhop \frac{1-0}{h} = 
            +\infty,$$
            $$f_-'(0) = \limhom \frac{f(h) - f(0)}{h} = \limhom \frac{-1-0}{h} = 
            +\infty.$$
            Díky vztahu limity a jednostranných limit 
            (Pozorování~\ref{obs:jednostrannelimity}) můžeme psát:
            $$\sgn' 0 = +\infty.$$

        \item $f(x) = |x|.$
            $$f_+'(0) = \limhop \frac{f(h) - f(0)}{h} = \limhop \frac{h}{h} = 1,$$
            $$f_-'(0) = \limhom \frac{f(h) - f(0)}{h} = \limhom \frac{-h}{h} = -1$$
            Derivace $f(x)$ v bodě $0$ neexistuje.

        \item $f(x) = \exp x.$

            Vyjádřeme derivaci funkce $\exp$ v bodě $a \in \R:$
            $$\limho \frac{\exp(a+h) - \exp a}{h} 
            = \limho \frac{\exp a \exp h - \exp a}{h} 
            = \limho \frac{\exp a \cdot (\exp h - 1)}{h}.$$
            Jelikož $\limho \exp a = \exp a$ a 
            $\limho \frac{\exp h - 1}{h} = 1,$ získáme za použití
            věty o aritmetice limit funkcí (Věta~\ref{th:voalf}) následující rovnost:
            $$f'(a) = \limho \exp a \cdot \limho \frac{\exp h -1}{h} = \exp a.$$

        \item $f(x) = \log x.$
            $$\limho \frac{\log(a+h) - \log(a)}{h} 
            = \limho \frac{\log\left(1 + \frac{h}{a}\right)}{\frac{h}{a}}\cdot \frac{1}{a}
            = \frac{1}{a}.$$
   \end{itemize}
\end{remark}

\begin{theorem}[vztah derivace a spojitosti]
    \label{th:derivacespojitost}
    \Necht má funkce $f$ v bodě $a \in \R$ vlastní derivaci $f'(a) \in \R.$ Pak
    je $f$ v bodě $a$ spojitá.
\end{theorem}

\begin{proof}
    Chceme dokázat, že $\limxa (f(x) - f(a)) = 0:$
    $$\limxa (f(x) - f(a)) = \limxa \frac{f(x) - f(a)}{x-a}(x-a) =
    \limxa \frac{f(x) - f(a)}{x-a} \limxa (x-a) = f'(a) \cdot 0 = 0.$$
\end{proof}

\begin{remark}
    \label{rm:jednostrannaderivacespojitost}
    Podobná věta platí i pro jednostranné limity. Platí-li $f'_+(a) \in \R,$
    je funkce $f$ spojitá v bodě $a$ zprava.
\end{remark}

\begin{theorem}
    Nechť $f'(a)$ a $g'(a)$ existují. Potom:
    \begin{enumerate}[i.]
        \item $(f+g)'(a) = f'(a) + g'(a),$ pokud má pravá strana smysl.
        \item Nechť je $g$ spojitá v $a,$ pak $(fg)'(a) = f'(a)g(a) + f(a)g'(a),$
            pokud má pravá strana smysl.
        \item \Necht je $g$ spojitá v $a$ a $g(a) \neq 0,$ pak 
            $\left(\frac{f}{g}\right)'(a) = \frac{f'(a)g(a) - f(a)g'(a)}{g^2(a)},$
            pokud má pravá strana smysl.
    \end{enumerate}
\end{theorem}

\begin{proof}
    \leavevmode
    \begin{enumerate}[i.]
        \item 
            \begin{align*}
                (f+g)'(a) &= \\
                          &= \limho \frac{f(a+h) + g(a + h) - f(a) - g(a)}{h}  \\
                &= \limho \frac{f(a+h) - f(a)}{h} + \limho \frac{g(a+h) - g(a)}{h} \\
                &= f'(a) + g'(a).
            \end{align*}

        \item \begin{align*}
                (fg)'(a) &= \\
                &=\limho \frac{f(a+h)g(a+h) - f(a)g(a)}{h} \\
                &=\limho \frac{f(a+h)g(a+h) - f(a)g(a+h) + f(a)g(a+h) - f(a)g(a)}{h} \\
                &=\limho \frac{f(a+h)-f(a)}{h}  \limho g(a+h)  + 
                \limho f(a) \limho \frac{g(a+h)-g(a)}{h} \\
                &= f'(a)g(a) -f(a)g'(a).
            \end{align*}
                Spojitosti funkce $g$ jsme využili při vyjádření limity: 
            $\limho g(a+h) = g(a).$

        \item Funkce $g$ spojitá v $a$ a $g(a) \neq 0,$ tedy $\exists \delta >0 \;
            \fa h \in U(a,\delta): g(a+h) \neq 0.$ Dále:
            \begin{align*}
                \left(\frac{f}{g}\right)'(a) &= \\
                &= \limho \frac{\frac{f(a+h)}{g(a+h)}-\frac{f(a)}{g(a)}}{h} \\
                &= \limho \frac{f(a+h)g(a) - f(a)g(a) + f(a)g(a) -f(a)g(a+h)}{g(a+h)g(a)h} \\
                &= \limho \frac{1}{g(a+h)g(a)}  
                \left( \limho g(a) \limho \frac{f(a+h) - f(a)}{h}
                +\limho (-f(a)) \limho \frac{g(a+h) - g(a)}{h}\right) \\
                &= \frac{1}{g^2(a)} \left(g(a)f'(a) - f(a)g'(a)\right).
            \end{align*}
    \end{enumerate}
\end{proof}

\begin{theorem}[derivace složené funkce]
    \Necht $f$ má derivaci v bodě $y_0, g$ má derivaci v $x_0$ a je v $x_0$ spojitá
    a $y_0 = g(x_0).$ Pak:
    $$(f \circ g)'(x_0) = f'(y_0)g'(x_0) = f'(g(x_0))g'(x_0),$$
    je-li výraz vpravo definován.
\end{theorem}

\begin{example}
    Určete derivaci funkce $e^{x^2}$ v bodě $a$.

    Funkce $e^{x^2}$ je složená funkce $f(g(x)):$
    $$f(y) = e^y, g(x) = x^2, f'(y) = e^y, g'(x) = 2x$$
    Potom:
    $$(e^{x^2})' = e^{x^2} \cdot 2x.$$
\end{example}

\begin{proof}
    Potřebujeme upravit následující limitu:
    $$\limho \frac{f(g(x+h)) - f(g(x))}{h}.$$
    Z existence derivace funkce $f$ v bodě $y_0$ víme, že tato funkce je 
    definována na jistém okolí toho bodu: $U(y_0, \e).$ Funkce $g$ je spojitá
    v $x_0$ a $g(x_0) = y_0.$ Z toho vyplývá, že $\exists \delta >0:
    g(U(x_0, \delta)) \subseteq U(y_0, \e)$ a tedy že složená funkce
    $f \circ g$ je definovaná na $U(x_0, \delta).$
    
    Zabývejme se nejprve případem, kdy derivace funkce $f$ v bodě $y_0$ je vlastní,
    tj. $f'(y_0) \in \R.$ Definujme následující funkci:
    $$F(y) \coloneqq 
        \begin{cases}
            \frac{f(y) - f(y_0)}{y-y_0} &y \neq y_0 \\
            f'(y_0) &y=y_0.
        \end{cases}
    $$
    Funkce $F$ je spojitá v bodě $y_0.$ Díky spojitosti funkce $g$ platí 
    $\lim_{x \to x_0} g(x) = g(x_0) = y_0.$ Dle věty o limitě 
    složené funkce (Věta~\ref{th:slozenafunkce}), podmínka (P1) platí:
    \begin{align*}
        f'(y_0) &= \lim_{y \to y_0} F(y) \\
                &= \lim_{x \to x_0} F(g(x)) \\
                &= \lim_{x \to x_0} \frac{f(g(x)) - f(g(x_0))}{x-x_0} \\
                &= \lim_{x \to x_0} \frac{f(g(x)) - f(g(x_0))}{g(x)-g(x_0)} \cdot \frac{g(x) - g(x_0)}{x-x_0} \\
                &= \lim_{x\to x_0} F(g(x)) \frac{g(x) - g(x_0)}{x-x_0} \\
                &= \lim_{x\to x_0} F(g(x)) \lim_{x \to x_0} \frac{g(x)-g(x_0)}{x-x_0} \\
                &= f'(y_0)g'(x_0).
    \end{align*}

    Uvažujme nyní případ, kdy derivace funkce $f$ v bodě $y_0$ je nevlastní, 
    tj. $f'(y_0) = \pm \infty.$ V tomto případě musí platit, že 
    $g'(x_0) \neq 0,$ jinak by nebyl výraz $f'(y_0)g'(x_0)$ definován.
    Z toho vyplývá, že:
    $$\exists \delta > 0 \; \fa x \in P(x_0, \delta): \frac{g(x) - g(x_0)}{x-x_0}
    \neq 0,$$
    a tedy, že na tomto okolí $g(x) \neq g(x_0).$ Definujme dále funkci 
    $F(y)$ pro $y \neq y_0$:
    $$F(y) = \frac{f(y) - f(y_0)}{y - y_0}.$$
    Potom můžeme psát:
    \begin{align*}
        \lim_{x \to x_0} \frac{f(g(x)) - f(g(x_0))}{x-x_0} 
        &= \lim_{x \to x_0} F(g(x)) \cdot \frac{g(x)-g(x_0)}{x-x_0} \\
        &= f'(y_0)g'(x_0) \tag{viz níže} 
    \end{align*}
    V posledním kroce jsme vuyžili věty o limitě složené funkce 
    (Věta~\ref{th:slozenafunkce}) a podmínky (P2): $g(x)$ nenabývá své limity
    na okolí $x_0.$
\end{proof}

\begin{theorem}[derivace inversní funkce]
    \label{th:derivaceinv}
    \Necht $f$ je na intervalu $(a,b)$ spojitá a rostoucí (klesající). \Necht
    $f$ má v bodě $x_0 \in (a,b)$ derivaci $f'(x_0)$ vlastní a různou od nuly.
    Potom má funkce $f^{-1}$ derivaci v bodě $y_0 = f(x_0)$ a platí:
    $$(f^{-1})'(y_0) = \frac{1}{f'(f^{-1}(y_0))}.$$
\end{theorem}

\begin{proof}
    Definujme funkci $F(x):$
    $$F(x) = 
        \begin{cases}
            \frac{f(x) - f(x_0)}{x-x_0} &x \neq x_0 \\
            f'(x_0) &x=x_0.
        \end{cases}
    $$
    Tato funkce je spojitá v bodě $x_0$. Definujme dále funkci $g(y):$
    $$g(y) = f^{-1}(y).$$
    Funkce inversní k $f$ je spojitá (Věta~\ref{th:inversnifce}), a tedy:
    $$\lim_{y \to y_0} g(y) = \lim_{y \to y_0} f^{-1}(y) = f^{-1}(y_0).$$
    Potom dle věty o limitě složené funkce (Věta~\ref{th:slozenafunkce}, (P1)):
    $$\lim_{y\to y_0} F(g(y)) = f'(x_0).$$
    Zároveň ovšem:
    \begin{align*}
        \lim_{y\to y_0} F(g(y)) 
        &= \lim_{y\to y_0} \frac{f(f^{-1}(y)) - f(f^{-1}(y_0))}{f^{-1}(y) - f^{-1}(y_0)} \\
        &= \lim_{y\to y_0} \frac{y - y_0}{f^{-1}(y) - f^{-1}(y_0)}
    \end{align*}
    Nakonec tedy:
    $$(f^{-1})'(y_0) = \lim_{y\to y_0}\frac{f^{-1}(y) - f^{-1}(y_0)}{y - y_0} = \frac{1}{f'(x_0)}$$
\end{proof}

\begin{remark}[Derivace elementárních funkcí]
    \label{rm:derivaceelfce}
    \leavevmode
    \begin{itemize}
        \item $(k)' = 0,$ $k$ je konstanta.
        \item $(x^n)' = nx^{n-1}, n \in \N, x \in \R.$
        \item $(e^x)' = e^x, x \in \R.$
        \item $x^a, x> 0, a \in \R:$
            $$(x^a)' = (e^{a\log x})' = e^{a \log x} \cdot (a \log x)' = 
            x^a \cdot \frac{a}{x} = ax^{a-1}.$$
        \item $\sin x:$
            \begin{align*}
                (\sin x)' &= \limho \frac{\sin(x+h) - \sin(x)}{h} \\
                          &= \limho \frac{2\sin(\frac{x+h-x}{2})\cos(\frac{x+h+x}{2})}{h} \\
                          &= \limho \frac{\sin(\frac{h}{2})}{\frac{h}{2}} \cdot \limho \cos(x + \frac{h}{2}) \\
                          &= 1 \cdot \cos x = \cos x.
            \end{align*}
        \item $\cos x:$
            \begin{align*}
                (\cos x)' &= \limho \frac{\cos(x+h) - \cos x}{h} \\
                          &= \limho \frac{-2\sin(x+\frac{h}{2})\sin(\frac{h}{2})}{h} \\
                          &= - \limho \frac{\sin(\frac{h}{2})}{\frac{h}{2}} \cdot \limho \sin(x + \frac{h}{2}) \\
                          &= -\sin x.
            \end{align*}
        \item $\tan x:$
            \begin{align*}
                (\tan x)' &= \left(\frac{\sin x}{\cos x}\right)' \\
                         &= \frac{\sin' x \cos x - \sin x \cos' x}{\cos^2 x} \\
                         &= \frac{\cos^2 x + \sin^2 x}{\cos^2 x} \\
                         &= \frac{1}{\cos^2 x}.
            \end{align*}

        \item $(\cotg x)' = - \frac{1}{\sin^2 x},$ analogicky.
        \item $\arcsin x, x \in (-1, 1):$
            \begin{align*}
                (\arcsin x)' &= \frac{1}{\sin' y} \tag{pro $y = \arcsin x$, Věta~\ref{th:derivaceinv}} \\
                             &= \frac{1}{\cos y} \\
                             &= \frac{1}{\sqrt{1 - x^2}}. \tag{$\cos^2 y = 1 - \sin^2 y = 1 - x^2$}
            \end{align*}
        \item $(\arccos x)' = -\frac{1}{\sqrt{1 - x^2}}.$
        \item $(\arctan x)' = \frac{1}{1 + x^2}.$
        \item $(\arccotg x)' = - \frac{1}{1 + x^2}$
    \end{itemize}
\end{remark}

\begin{theorem}[Fermatova]
    \label{th:fermat}
    \Necht $a \in \R$ je bod lokálního extrému funkce $f$ na $M.$ Pak $f'(a)$ 
    neexistuje nebo $f'(a) = 0.$
\end{theorem}

\begin{proof}
    Sporem. \Necht v bodě $a$ existuje nenulová derivace, tj. 
    $f'(a) \neq 0.$ \Necht je \buno $f'(a) > 0.$ Potom:
    $$\exists \delta > 0 \; \fa x \in P(a, \delta): \frac{f(x) - f(a)}{x-a} > 0.$$
    Zvolme $\e = \frac{f'(a)}{2}.$ Potom:
    $$\exists \delta > 0 \; \fa x \in P(a, \delta): 
        |\frac{f(x) - f(a)}{x-a} - f'(a)| < \frac{f'(a)}{2}.$$
    Po úpravě:
    $$\fa x \in P(a, \delta): 0 < \frac{f'(a)}{2} < \frac{f(x) - f(a)}{x-a}.$$
    Potom pro $x \in P_+(a, \delta): x - a> 0,$ a tedy $f(x) - f(a) > 0.$ Naopak,
    pro $x \in P_-(a, \delta): x - a < 0,$ a tedy $f(x) - f(a) < 0.$
    V tom případě ovšem bod $a$ není ani lokálním maximem, ani lokálním minimem.
\end{proof}

\begin{remark}
    V typické úloze máme spojitou funkci na intervalu $[a,b]$ a naším úkolem je
    nalézt její maxima a minima:
    \begin{enumerate}[a.]
        \item Dle věty o spojitosti funkce a nabývání extrémů 
            (Věta~\ref{th:spojitaextremy}) víme, že tato funkce nabývá na daném
            intervalu $[a,b]$ svých extrémů.
        \item Dále dle Fermatovy věty (Věta~\ref{th:fermat}) víme, v jakých 
            bodech se tyto extrémy mohou nacházet, tj. víme, kde hledat:
            $$\{x \in [a,b], f'(x) = 0 \} 
            \cup \{x \in [a,b], f'(x) \text{ neexistuje}\}
            \cup \{a,b\}$$
    \end{enumerate}
\end{remark}

\begin{theorem}[Rolleova]
    \label{th:rolle}
    \Necht $f$ je spojitá na intervalu $[a,b],$ $f'(x)$ existuje pro každé
    $x \in (a,b)$ a $f(a) = f(b).$ Pak existuje $\xi \in (a,b): f'(\xi) = 0.$
\end{theorem}

\begin{proof}
    Pokud pro $\fa x \in [a,b]: f(x) = f(a),$ potom $\fa x \in (a,b): f'(x) = 0.$ 

    \Necht existuje $x \in (a,b)$ \tz $f(x) \neq f(a).$ \Necht \buno $f(x) > f(a).$
    Spojitá funkce na $[a,b]$ nabývá extrémů, a tedy dle Věty~\ref{th:spojitaextremy}:
    $$\exists \xi \in [a,b]: f(\xi) = \max_{x \in [a,b]} f(x).$$
    Jelikož víme, že $\exists x \in (a,b)$ \tz $f(x) > f(a),$ vyplývá, že $\xi \neq a$
    a $\xi \neq b.$ Dále z předpokladů víme, že $f'(x)$ existuje pro všechna 
    $x \in (a,b).$ Z toho nutně vyplývá, že $f'(\xi) = 0.$
\end{proof}

\begin{theorem}[Lagrangeova věta o střední hodnotě]
    \label{th:lagrangemean}
    \Necht je funkce $f$ spojitá na intervalu $[a,b]$ a má derivaci v každém bodě
    intervalu $(a,b).$ Pak existuje $\xi \in (a,b)$ \tz
    $$f'(\xi) = \frac{f(b) - f(a)}{b -a}.$$
\end{theorem}

\begin{proof}
    Definujme následující funkci:
    $$F(x) \coloneqq f(x) - f(a) - \frac{f(b) - f(a)}{b - a}(x-a).$$
    Tato funkce je spojitá na $[a,b]$ a zároveň má derivaci na $(a,b).$ Dále
    $F(a) = F(b) = 0.$ Funkce $F(x)$ tedy na intervalu $[a,b]$ splňuje podmínky
    Rolleovy věty (Věta~\ref{th:rolle}), ze které vyplývá, že existuje $\xi \in
    (a,b)$ \tz $F'(\xi) = 0.$ Potom:
    $$0 = F'(\xi) = f'(\xi) - 0 - \frac{f(b) - f(a)}{b-a} \cdot 1,$$
    a tedy:
    $$f'(\xi) = \frac{f(b) - f(a)}{b-a}.$$
\end{proof}

\begin{theorem}[Cauchyho věta o střední hodnotě]
    \label{th:cauchymean}
    \Necht $f,g$ jsou spojité funkce na intervalu $[a,b]$ takové, že $f$ má v každém
    bodě $(a,b)$ derivaci a $g$ má v každém bodě $(a,b)$ vlastní derivaci různou 
    od nuly. Pak existuje $\xi \in (a,b)$ \tz
    $$\frac{f'(\xi)}{g'(\xi)} = \frac{f(b) - f(a)}{g(b) - g(a)}.$$
\end{theorem}

\begin{proof}
    Z předpokladů vyplývá, že $g(b) \neq g(a), $ protože jinak by dle Rolleovy 
    věty (Věta~\ref{th:rolle}) existovalo $x \in (a,b): g'(x) = 0.$
    Definujme podobně jako výše následující funkci:
    $$H(x) \coloneqq (f(b) - f(a))(g(x)-g(a)) - (f(x) - f(a))(g(b) - g(a)).$$
    Funkce $H(x)$ je spojitá na $[a,b]$ a má na intervalu $(a,b)$ derivaci.
    Dále $H(a) = H(b) = 0$ a tedy dle Rolleovy věty (Věta~\ref{th:rolle})
    $$\exists \xi \in (a,b): H'(\xi) = 0.$$
    Potom:
    $$0 = H'(\xi) = (f(b) - f(a))(g'(\xi) - 0) - (f'(x)-0)(g(b) - g(a))$$
    a tedy:
    $$\frac{f'(\xi)}{g'(\xi)} = \frac{f(b) - f(a)}{g(b) - g(a)}.$$
\end{proof}

\begin{theorem}[l'Hospitalovo pravidlo]
    \label{th:lhospital}
    \leavevmode
    \begin{enumerate}[(i)]
        \item \Necht $a \in \Rstar, \lim_{x \to a+} f(x) = \lim_{x \to a+} g(x) = 0$
            a \necht existuje $\lim_{x \to a+} \frac{f'(x)}{g'(x)}.$ Pak
            $$ \lim_{x \to a+} \frac{f(x)}{g(x)} = \lim_{x \to a+} \frac{f'(x)}{g'(x)}.$$
        \item \Necht $a \in \Rstar, \lim_{x \to a+} |g(x)| = \infty$
            a \necht existuje $\lim_{x \to a+} \frac{f'(x)}{g'(x)}.$ Pak
            $$ \lim_{x \to a+} \frac{f(x)}{g(x)} = \lim_{x \to a+} \frac{f'(x)}{g'(x)}.$$
   \end{enumerate}
\end{theorem}

\begin{remark}[Příklad využití l'Hospitalova pravidla]
    Vypočtěte následující limitu:
    $$\limxo \frac{\sin x - x}{x^3}.$$
    Limita je typu $\frac{0}{0}, $ je zde tedy šance na použití bodu (i) z 
    l'Hospitalova pravidla. Vyjádřeme limitu derivací:
    $$\limxo \frac{(\sin x - x)'}{(x^3)'} = \limxo \frac{\cos x - 1}{3x^2}.$$
    Dostali jsme znovu limitu typu $\frac{0}{0}.$ Pokusme se znovu zderivovat jak
    čitatel, tak jmenovatel:
    \begin{align*}
        \limxo \frac{(\cos x - 1)'}{(3x^2)'} &= \limxo \frac{-\sin x}{6x} \\
                                             &= \limxo \frac{-1}{6} \cdot \limxo \frac{\sin x}{x} \\
                                             &= -\frac{1}{6} \tag{$\limxo \frac{\sin x}{x} = 1$, Věta~\ref{th:goniom}}
    \end{align*}
    Díky l'Hospitalovu pravidlu dostáváme:
    $$\limxo \frac{\sin x - x}{x^3} = \limxo \frac{\cos x - 1}{3x^2} = \limxo \frac{-\sin x}{6x}  = - \frac{1}{6}.$$
    Mimochodem, pro vyjádření poslední limity jsme mohli místo vlastností sina
    použít další iteraci l'Hospitalova pravidla; výsledek bychom dostali stejný.
\end{remark}

\begin{proof}
    \leavevmode
    \begin{enumerate}[(i)]
        \item 
            \begin{itemize}
                \item $a \in \R, \lim_{x \to a+} \frac{f'(x)}{g'(x)} = A \in \Rstar.$

                    Jelikož limita je definována pouze pro reálné funkce, 
                    vyplývá z existence $\lim_{x \to a+} \frac{f'(x)}{g'(x)},$
                    že:
                    $$\exists \delta > 0 \; \fa x \in P_+(a, \delta): f'(x) \in \R
                    \text{ a } g'(x) \in \R \setminus \{0\}.$$

                    Dodefinujme funkce $f$ a $g$ \tz $f(a) = g(a) = 0.$ Vezměme
                    nějaké $x \in P_+(a, \delta).$ Funkce $f$ a $g$ jsou
                    spojité na $[a,x], $ neboť mají vlastní derivaci na 
                    intervalu $(a,x]$ a v bodě $a$ jsme spojitost dodefinovali.
                    Tím jsou splněny předpoklady Cauchyho věty o střední
                    hodnotě (Věta~\ref{th:cauchymean}) a platí, že 
                    $$\fa x \in P_+(a, \delta) \; \exists \xi(x) \in (a,x):
                    \frac{f'(\xi(x))}{g'(\xi(x))} = \frac{f(x) - f(a)}{g(x) - g(a)}
                    \stackrel{\text{při }a=0}{=} \frac{f(x)}{g(x)}$$

                    Zvolme $\e > 0.$ Z existence
                    $\lim_{x \to a+} \frac{f'(x)}{g'(x)}$ víme, že 
                    $$\exists \delta_1 < \delta \; \fa x \in P_+(a, \delta_1):
                    \frac{f'(x)}{g'(x)} \in U(A, \e).$$

                    Nyní, $\fa x \in P_+(a, \delta_1)$ platí, že $a < \xi(x) < x,$
                    a tedy $\frac{f'(\xi(x))}{g'(\xi(x))} \in U(A, \e)$, a tím
                    pádem i $\frac{f(x)}{g(x)} \in U(A, \e).$ Proto:
                    $$\lim_{x \to a+} \frac{f(x)}{g(x)} = A.$$

                \item $a = - \infty, \lim_{x \to a+} \frac{f'(x)}{g'(x)} = A \in \Rstar.$

                    Jelikož platí:
                    $$\lim_{x \to -\infty} h(x) = C \iff \lim_{x \to 0+} h(-\frac{1}{x}) = C,$$
                    můžeme tento případ převést na předchozí pomocí substituce.

                    Definujme následující dvě funkce:
                    $$F(y) = f(-\frac{1}{y}), \; G(y) = g(-\frac{1}{y}).$$
                    Jejich derivace jsou rovny:
                    $$F'(y) = f'(-\frac{1}{y})(\frac{1}{y^2}), 
                    \; G'(y) = g'(-\frac{1}{y})(\frac{1}{y^2}).$$
                    Potom:
                    $$\lim_{x \to -\infty}\frac{f(x)}{g(x)} = 
                    \lim_{y \to 0+} \frac{F(y)}{G(y)} =
                    \lim_{y \to 0+} \frac{F'(y)}{G'(y)} = 
                    \lim_{y \to 0+} \frac{f'(-\frac{1}{y})(\frac{1}{y^2})}{g'(-\frac{1}{y})(\frac{1}{y^2})} =
                    \lim_{x \to -\infty} \frac{f'(x)}{g'(x)}.$$
            \end{itemize}

        \item Tuto variantu si dokážeme pouze pro případ 
            $a \in \R, \lim_{x \to a+} \frac{f'(x)}{g'(x)} = A \in \R, 
            \lim_{x \to a+} g(x) = \infty.$

            Zvolme $0<\e<1.$ Z předpokladů víme, že:
            $$\exists \delta_1 > 0 \; \fa y \in P_+(a, \delta_1): 
            |\frac{f'(y)}{g'(y)} - A| < \e.$$
            Zvolme pevné $y \in P_+(a, \delta_1).$ Z $\lim_{x \to 0+} g(x) = \infty$
            dále vyplývá, že:
            $$\exists \delta_2 < \delta_1 \; \fa x \in P_+(a, \delta_2):
            |\frac{g(y)}{g(x)}| < \e \land |\frac{f(y)}{g(x)}| < \e.$$

            Nechť $x \in P_+(a, \delta_2): a < x < y.$ Potom na $[x,y]$ splňují
            funkce $f,g$ předpoklady Cauchyho věty o střední hodnotě 
            (Věta~\ref{th:cauchymean}), a tedy:
            $$\exists \xi \in (x,y): \frac{f(y) - f(x)}{g(y)-g(x)} = 
            \frac{f'(\xi)}{g'(\xi)}.$$
            Potom:
            $$f(y) - f(x) = \frac{f'(\xi)}{g'(\xi)} g(y) - 
            \frac{f'(\xi)}{g'(\xi)} g(x).$$
            Po vydělení obou stran rovnice výrazem $\frac{1}{g(x)}:$
            $$\frac{f(y)}{g(x)} - \frac{f(x)}{g(x)} = 
            \frac{f'(\xi)}{g'(\xi)}\frac{g(y)}{g(x)} - 
            \frac{f'(\xi)}{g'(\xi)}\frac{g(x)}{g(x)}.$$
            Po úpravách:
            $$\frac{f(x)}{g(x)}  =\frac{f'(\xi)}{g'(\xi)} + 
            \frac{f(y)}{g(x)} -\frac{f'(\xi)}{g'(\xi)}\frac{g(y)}{g(x)}.$$
            Potom:
            \begin{align*}
                |\frac{f(x)}{g(x)} - A| &= |\frac{f'(\xi)}{g'(\xi)} + 
                \frac{f(y)}{g(x)} -\frac{f'(\xi)}{g'(\xi)}\frac{g(y)}{g(x)} - A| \\
                &\leq |\frac{f'(\xi)}{g'(\xi)} - A| + 
                |\frac{f'(\xi)}{g'(\xi)}\frac{g(y)}{g(x)}| +
                |\frac{f(y)}{g(x)}| \tag{trojúhelníková nerovnost} \\
                &< \e + (|A| + \e)\e + \e.
            \end{align*}
            A tedy:
            $$\lim_{x \to a+} \frac{f(x)}{g(x)} = A.$$
    \end{enumerate}
\end{proof}

\begin{theorem}[derivace a limita derivace]
    \label{th:derlimder}
    \Necht je funkce $f$ spojitá zprava v $a$ a \necht existuje $\lim_{x \to a+}
    f'(x) = A \in \Rstar.$ Pak $f'_+(a) = A.$
\end{theorem}

\begin{proof}
    Vyjádřeme $f'_+(a)$ pomocí definice:
    \begin{align*}
        f'_+(a) &= \lim_{x \to a+} \frac{f(x) - f(a)}{x-a} \\
                &\stackrel{?}{=} \lim_{x \to a+} \frac{f'(x)}{1} 
                    \tag{L'Hospitalovo pravidlo, nutno ověřit podmínky} \\
                &= A.
    \end{align*}
    Nyní je třeba ověřit, že použití L'Hospitalova pravidla 
    (Věta~\ref{th:lhospital}) bylo korektní. Jde jednoduše nahlédnout, že
    se jedná o případ (i): $\frac{0}{0}.$ 
\end{proof}

\begin{definition}
    \Necht $J$ je interval. Množinu všech vnitřních bodů $J$ nazýváme 
    \newterm{vnitřek} J a značíme $\vnitrek J.$
\end{definition}

\begin{theorem}[o vztahu derivace a monotonie]
    \label{th:derivacemonotonie}
    \Necht $J \subseteq \R$ je interval a $f$ je spojitá na $J$ a v každém
    vnitřním bodě $J$ má derivaci.
    \begin{enumerate}[(i)]
        \item Je-li $f'(x) > 0$ na $\vnitrek J$, pak je $f$ rostoucí na $J.$
        \item Je-li $f'(x) < 0$ na $\vnitrek J$, pak je $f$ klesající na $J.$
        \item Je-li $f'(x) \geq 0$ na $\vnitrek J$, pak je $f$ neklesající na $J.$
        \item Je-li $f'(x) \leq 0$ na $\vnitrek J$, pak je $f$ nerostoucí na $J.$
    \end{enumerate}
\end{theorem}

\begin{proof}
    Ukážeme si pouze první případ; ostatní se dokazují analogicky.

    Mějme body $a,b \in J, a < b.$ Funkce $f$ je spojitá na $[a,b]$ a má derivaci
    v každém vnitřním bodu intervalu $(a,b).$ Dle Lagrangeovy věty o střední
    hodnotě (Věta~\ref{th:lagrangemean}) platí, že
    $$\exists \xi \in (a,b): f'(\xi) = \frac{f(b) - f(a)}{b-a}.$$ Jelikož
    dle předpokladů $f'(\xi) > 0$ a zároveň $b-a>0$, musí platit i $f(b) - f(a) > 0.$
    Funkce $f$ je tedy na intervalu $J$ rostoucí.
\end{proof}

\begin{remark}
    Implikace v předchozí větě neplatí v opačném směru: Uvažujte například
    funkci $f(x) = x^3$ a případ (i).
\end{remark}

\begin{definition}
    \Necht $n \in \N, a \in \R$ a \necht $f$ má vlastní $n$-tou derivaci na okolí
    bodu $a.$ Pak \newterm{$(n+1)$-ní derivací} funkce $f$ v bodě $a$ budeme rozumět:
    $$f^{(n+1)}(a) = \lim_{h \to 0} \frac{f^{(n)}(a + h) - f^{(n)}(a)}{h}.$$
\end{definition}

\subsection{Konvexní a konkávní funkce}

\begin{definition}
    \Necht $f$ má vlastní derivaci v bodě $a \in \R.$ Označme
    $$T_a = \{[x,y], x \in \R, y = f(a) + f'(a)(x-a)\}.$$
    Řekneme, že bod $[x, f(x)], x \in D_f$ leží \newterm{nad (pod) tečnou} $T_a,$
    jestliže platí:
    $$f(x) > f(a) + f'(a)(x-a) (f(x) < f(a) + f'(a)(x-a)).$$
\end{definition}

\begin{definition}
    Funkce $f$ má v bodě $a$ \newterm{inflexi} ($a$ je \newterm{inflexní bod}),
    jestliže $f'(a) \in \R$ a existuje $\Delta > 0$ \tz
    \begin{enumerate}[(i)]
        \item $\fa x \in (a-\Delta, a): [x,f(x)]$ leží nad tečnou,
        \item $\fa x \in (a, a+ \Delta): [x,f(x)]$ leží pod tečnou,
    \end{enumerate}
    nebo
    \begin{enumerate}[(i)]
        \item $\fa x \in (a-\Delta, a): [x,f(x)]$ leží pod tečnou,
        \item $\fa x \in (a, a+ \Delta): [x,f(x)]$ leží nad tečnou.
    \end{enumerate}
\end{definition}

\begin{theorem}[nutná podmínka pro inflexi]
    \Necht $f''(a) \neq 0.$ Pak $a$ není inflexní bod funkce $f.$
\end{theorem}

\begin{proof}
    \Necht je \buno $f''(a) > 0,$ a tedy
    $$\lim_{x \to a} \frac{f'(x) - f'(a)}{x-a} > 0.$$
    Ukážeme, že body $[x,f(x)]$ leží nad tečnou $T_a$
    jak v levém, tak v pravém okolí bodu $a.$

    Díky předpokladu $f''(a) > 0$ existuje $\delta > 0$ \tz:
    $$\fa x \in P_+(a,\delta): f'(x) > f'(a) \land
    \fa x \in P_-(a, \delta): f'(x) < f'(a).$$

    Zvolme libovolné $y \in P_+(a,\delta).$ Funkce $f$ je spojitá na $[a,y]$ a
    má vlastní derivace ve všech bodech intervalu $(a,y).$ Potom dle
    Langrageovy věty o střední hodnotě (Věta~\ref{th:lagrangemean}):
    $$\exists \xi_1 \in (a,y): f'(a) < f'(\xi_1) = \frac{f(y) - f(a)}{y-a}.$$
    Pro všechna $y \in P_+(a,\delta)$ tedy platí:
    $$f(y) > f(a) + f'(a)(y-a);$$
    jinými slovy, leží nad tečnou $T_a.$

    Zvolme nyní libovolné $z \in P_-(a, \delta).$ Analogicky ukážeme, že
    $$\exists \xi_2 \in (z, a): f'(a) > f'(\xi_2) = \frac{f(a) - f(z)}{a-z},$$
    a tedy, že pro všechna $z \in P_-(a, \delta):$
    $$f(z)>f(a) + f'(a)(z-a)$$
\end{proof}

\begin{theorem}[postačující podmínka pro inflexi]
    \label{th:inflexepostac}
    \Necht existuje $f'(a) \in \R$ a \necht existuje $\delta > 0$ \tz:
    $$\fa x \in P_+(a, \delta): f''(x) > 0 
    \land \fa x \in P_-(a, \delta): f''(x) < 0.$$
    Pak $z$ je inflexní bod $f.$
\end{theorem}

\begin{proof}
    Jelikož $f''(a) > 0$ na $P_+(a, \delta),$ je dle věty o vztahu derivace
    a monotonie (Věta~\ref{th:derivacemonotonie}) $f'(x)$ rostoucí na 
    $P_+(a, \delta),$ a tedy $\fa x \in P_+(a, \delta): f'(x) > f'(a).$

    Zvolme $x \in P_+(a,\delta).$ Dle Lagrangeovy věty o střední hodnotě 
    (Věta~\ref{th:lagrangemean}) platí, že:
    $$\exists \xi_1 \in (a,x): f'(a) < f'(\xi_1) = \frac{f(x) - f(a)}{x-a}.$$
    Z tohoto vztahu vyplývá, že funkce je nad tečnou $T_a$ v pravém okolí
    bodu $a.$
    
    Analogicky ukážeme, že v levém okolí bodu $a$ je funkce pod tečnou $T_a$,
    čímž jsou splněny podmínky pro inflexi.
\end{proof}

\begin{definition}
    Funkci $f$ na intervalu $I$ nazveme \newterm{konvexní (konkávní)}, jestliže:
    $$\fa x_1,x_2,x_3 \in I, x_1 < x_2 < x_3 \implies
    \frac{f(x_2) - f(x_1)}{x_2 - x_1} \leq \frac{f(x_3) - f(x_1)}{x_3 - x_1}$$
    $$\left(\fa x_1,x_2,x_3 \in I, x_1 < x_2 < x_3 \implies
    \frac{f(x_2) - f(x_1)}{x_2 - x_1} \geq \frac{f(x_3) - f(x_1)}{x_3 - x_1}\right).$$
    Funkci nazveme \newterm{ryze konvexní (ryze konkávní)}, je-li příslušná
    nerovnost ostrá.
\end{definition}

\begin{remark}[ekvivalentní definice konvexity]
    Funkce $f$ je na intervalu $J$ konvexní, pokud:
    $$\fa \alpha \in (0,1) \; \fa x,y \in J: f(\alpha x + (1-\alpha)y) \leq
    \alpha f(x) + (1-\alpha)f(y).$$
\end{remark}

\begin{lemma}
    \label{lm:konvexita}
    \Necht je funkce $f$ na intervalu $I$ konvexní, pak:
    $$\fa x_1,x_2,x_3 \in I, x_1 < x_2 < x_3 \implies
    \frac{f(x_2) - f(x_1)}{x_2 - x_1} \leq \frac{f(x_3) - f(x_1)}{x_3 - x_1}
    \leq \frac{f(x_3) - f(x_2)}{x_3 - x_2}.$$
\end{lemma}

\begin{proof}
    Platnost první nerovnosti je dána již z definice konvexnosti. Nás proto
    zajímá druhá nerovnost, tj. chceme dokázat, že platí:
    $$\fa x_1,x_2,x_3 \in I, x_1 < x_2 < x_3 \implies
    \frac{f(x_3) - f(x_1)}{x_3 - x_1}
    \leq \frac{f(x_3) - f(x_2)}{x_3 - x_2}.$$
    Platí:
    \begin{align*}
        \frac{f(x_3) - f(x_2)}{x_3 - x_2} 
        &= \frac{f(x_3) - f(x_1) + f(x_1) - f(x_2)}{x_3 - x_2} \\
        &=    \frac{f(x_3) - f(x_1)}{x_3 - x_2} - \frac{f(x_2) - f(x_1)}{x_2 - x_1} \frac{x_2 - x_1}{x_3 - x_2} \\
        &\geq \frac{f(x_3) - f(x_1)}{x_3 - x_2} - \frac{f(x_3) - f(x_1)}{x_3 - x_1} \frac{x_2 - x_1}{x_3 - x_2} \tag {z definice konvexity} \\
        &\geq \frac{f(x_3) - f(x_1)}{x_3 - x_2} - \frac{f(x_3) - f(x_1)}{x_3 - x_2} \frac{x_2 - x_1}{x_3 - x_1} \\
        &\geq \frac{f(x_3) - f(x_1)}{x_3 - x_2} \left(1 - \frac{x_2 - x_1}{x_3 - x_1}\right) \\
        &\geq \frac{f(x_3) - f(x_1)}{x_3 - x_2} \left(\frac{x_3 - x_1 - x_2 + x_1}{x_3 - x_1}\right) \\
        &= \frac{f(x_3) - f(x_1)}{x_3 - x_1}
    \end{align*}
\end{proof}

\begin{lemma}
    \Necht je funkce $f$ na intervalu $I$ ryze konvexní, pak:
    $$\fa x_1,x_2,x_3 \in I, x_1 < x_2 < x_3 \implies
    \frac{f(x_2) - f(x_1)}{x_2 - x_1} < \frac{f(x_3) - f(x_1)}{x_3 - x_1}
    < \frac{f(x_3) - f(x_2)}{x_3 - x_2}.$$
\end{lemma}

\begin{theorem}[konvexita a jednostranné derivace]
    \label{th:konvexitajednostrannederivace}
    \Necht je funkce $f$ na intervalu $J$ konvexní a $a \in \vnitrek J.$
    Pak $f'_+(a) \in \R$ a $f'_-(a) \in \R.$
\end{theorem}

\begin{proof}
    Omezíme se na důkaz existence $f'_+(a).$ Případ jednostranné derivace zleva
    se dokazuje analogicky.

    Naším úkolem je dokázat, že limita:
    $$\lim_{x \to a+} \frac{f(x) - f(a)}{x-a}$$
    existuje a že je reálná. Z konvexity funkce $f$ na $J$ vyplývá, že
    existuje $\delta > 0$ \tz funkce 
    $$H(x) = \frac{f(x) - f(a)}{x-a}$$
    je na $(a, a+\delta)$ neklesající a tedy dle věty o limitě monotónní funkce
    (Věta~\ref{th:limitamonotonnifce}) má limitu $A$.

    Navíc, nechť $y \in J, y<a.$ Potom dle lemmatu~\ref{lm:konvexita} platí:
    $$\fa x \in (a, a+\delta): \frac{f(a) - f(y)}{a-y} \leq \frac{f(x) - f(a)}{x-a}.$$
    Funkce $H(x)$ je tedy na $(a, a+\delta)$ omezená zdola a $A \in \R.$
\end{proof}

\begin{theorem}
    \Necht $f$ je konvexní na otevřeném intervalu $J.$ Pak je $f$ spojitá na $J.$
\end{theorem}

\begin{proof}
    Uvažujme libovolný bod $a \in \vnitrek J.$ Funkce $f$ je konvexní a tedy 
    dle věty o konvexitě a jednostranných derivacích 
    (Věta~\ref{th:konvexitajednostrannederivace}) existují jednostranné
    derivace funkce $f$ v bodě $a.$ Z věty o vztahu derivace a 
    spojistosti a následující poznámky (Věta~\ref{th:derivacespojitost}, 
    Poznámka~\ref{rm:jednostrannaderivacespojitost}) dále vyplývá, že funkce 
    funkce $f$ je v bodě $a$ spojitá zleva i zprava, a tedy spojitá.
\end{proof}

\begin{theorem}[vztah druhé derivace a konvexity (konkávity)]
    \Necht $f$ má na intervalu $(a,b)$ spojitou první derivaci.
    \begin{enumerate}[(i)]
        \item Jestliže $\fa x \in (a,b): f''(x) \geq 0,$ pak $f$ je konvexní na $(a,b).$
        \item Jestliže $\fa x \in (a,b): f''(x) \leq 0,$ pak $f$ je konkávní na $(a,b).$
        \item Jestliže $\fa x \in (a,b): f''(x) > 0,$ pak $f$ je ryze konvexní na $(a,b).$
        \item Jestliže $\fa x \in (a,b): f''(x) < 0,$ pak $f$ je ryze konkávní na $(a,b).$
    \end{enumerate}
\end{theorem}

\begin{proof}
    Ukážeme si pouze případ (i); ostatní se dokazují analogicky. 
    
    Nechť $x_1,x_2,x_3 \in (a,b), x_1 < x_2 < x_3.$ Dle Lagrangeovy věty
    o střední hodnotě (Věta~\ref{th:lagrangemean}):
    $$\exists \xi_1, \xi_2: f'(\xi_1) = \frac{f(x_2)-f(x_1)}{x_2-x_1} \land
    f'(\xi_2) = \frac{f(x_3)-f(x_2)}{x_3-x_2}.$$
    Dále, jelikož $\fa x \in (a,b): f''(x) \geq 0,$ první derivace funkce $f$ je
    dle věty o vztahu derivace a monotonie (Věta~\ref{th:derivacemonotonie})
    neklesající funkce. Protože $\xi_1 < \xi_2$, platí $f'(\xi_1) \leq f'(\xi_2),$
    a tedy:
    $$\frac{f(x_2)-f(x_1)}{x_2-x_1} \leq \frac{f(x_3)-f(x_2)}{x_3-x_2},$$
    po úpravě:
    $$f(x_3) \geq (f(x_2)-f(x_1))\frac{x_3-x_2}{x_2-x_1} + f(x_2).$$
    Odečtěme od obou stran nerovnice výraz $f(x_1):$
    $$f(x_3) -f(x_1) \geq (f(x_2)-f(x_1))\frac{x_3-x_2}{x_2-x_1} + f(x_2)-f(x_1)$$
    a následně je vynásobme výrazem $\frac{1}{x_3-x_1}$ a upravme:
    \begin{align*}
        \frac{f(x_3) -f(x_1)}{x_3 -x_1} 
        &\geq (f(x_2)-f(x_1)) \left(\frac{x_3-x_2}{(x_2-x_1)(x_3-x_1)} + \frac{1}{x_3-x_1}\right) \\
        &= (f(x_2)-f(x_1)) \frac{x_3-x_2+x_2-x_1}{(x_2-x_1)(x_3-x_1)} \\
        &= \frac{f(x_2)-f(x_1)}{x_2-x_1}.
    \end{align*}
    Tím jsme ukázali, že funkce $f$ je konvexní $(a,b).$
\end{proof}

\subsection{Průběh funkce}

\begin{definition}
    Řekneme, že funkce $ax + b, a,b \in \R,$ je asymptotou funkce $f$ v $+\infty$ 
    ($-\infty$), jestliže:
    $$\lim_{x \to \infty} (f(x)-(ax+b)) = 0$$
    $$(\lim_{x \to -\infty} (f(x)-(ax+b)) = 0)$$
\end{definition}

\begin{theorem}[tvar asymptoty]
    Funkce $f$ má v $\infty$ asymptotu $ax+b,$ právě když
    $$\lim_{x\to\infty}\frac{f(x)}{x} = a\in\R \; \text {a} \; 
    \lim_{x\to\infty}(f(x) - ax) = b \in \R.$$
\end{theorem}

\begin{proof}
    \leavevmode
    \begin{itemize}
        \item[$\implies$] Použijeme větu o aritmetice limit funkcí 
            (Věta~\ref{th:voalf}) a rozepíšeme obě limity. Pro koeficient $a$:
            $$\limxinf \frac{f(x)}{x} 
            \qe \limxinf \frac{f(x) - (ax + b)}{x} + \limxinf \frac{ax+b}{x}
            \qe \frac{0}{\infty} + a = a.$$
            Rozepsání je platné, jen pokud pravé strany mají smysl (proto ty 
            otazníky nad rovnítky). V tomto případě smysl mají a použití věty o 
            aritmetice limit funkcí je tedy korektní.

            Podobně pro koeficient $b:$
            $$\limxinf (f(x) - ax) 
            \qe \limxinf (f(x) - ax + b) + \limxinf b 
            \qe 0 + b = b.$$
        \item[$\impliedby$]
            $$\limxinf (f(x) - (ax + b)) 
            \qe \limxinf (f(x) - ax) - \lim b
            \qe b - b = 0.$$
    \end{itemize}
\end{proof}

\begin{remark}
    \label{rm:prubehfce}
    Při vyšetření průběhu funkce provádíme následující kroky:
    \begin{enumerate}
        \item Určíme definiční obor a obor spojitosti funkce.
        \item Zjistíme průsečíky se souřadnými osami.
        \item Zjistíme symetrii funkce: lichost, sudost, periodicita.
        \item Dopočítáme limity v "krajních bodech definičního oboru."
        \item Spočteme první derivaci, určíme intervaly monotonie a 
            nalezneme lokální a globální extrémy.
        \item Spočteme druhou derivaci a určíme intervaly, kde je $f$ konvexní
            nebo konkávní. Určíme inflexní body.
        \item Vypočteme asymptoty funkce.
        \item Načrtneme graf funkce a určíme obor hodnot.
    \end{enumerate}
\end{remark}

\begin{example}
    Vyšetřete průběh funkce:
    $$f(x) = \sqrt[3]{(x+2)^2} - \sqrt[3]{(x-2)^2}.$$
    \begin{enumerate}
        \item Určíme definiční obor a obor spojitosti funkce. 
            
            $D_f = \R, $ funkce je spojitá na $\R.$

        \item Zjistíme průsečíky se souřadnými osami. 
            
            Položme $x=0:$ 
            $$f(0) = \sqrt[3]{(0+2)^2} - \sqrt[3]{(0 - 2)^2} = 0,$$
            a $y=0:$
            \begin{align*}
                0 &= \sqrt[3]{(x+2)^2} - \sqrt[3]{(x-2)^2} \\
                \sqrt[3]{(x+2)^2} &=  \sqrt[3]{(x-2)^2} \\
                x^2 + 4x + 4 &= x^2 -4x + 4 \\
                x&=0.
            \end{align*}
            Jediný průsečík s osami $x$ a $y$ je bod $P[0,0].$
        
        \item Zjistíme symetrii funkce: lichost, sudost, periodicita.

            Pro určení parity funkce vyjádříme $f(-x):$
            \begin{align*}
                f(-x) 
                &= \sqrt[3]{(-x+2)^2} - \sqrt[3]{(-x-2)^2} \\
                &= \sqrt[3]{(-1)^2(x-2)^2} - \sqrt[3]{(-1)^2(x+2)^2} \\
                &= \sqrt[3]{(x-2)^2} - \sqrt[3]{(x+2)^2} \\
                &= -1 (\sqrt[3]{(x+2)^2} - \sqrt[3]{(x-2)^2}) \\
                &= -f(x).
            \end{align*}
            Funkce $f$ je lichá.
        \item Dopočítáme limity v "krajních bodech definičního oboru."

            \begin{align*}
                \limxinf f(x) 
                &= \limxinf \sqrt[3]{(x+2)^2} - \sqrt[3]{(x - 2)^2} \\
                &= \limxinf \sqrt[3]{(x+2)^2} - \sqrt[3]{(x - 2)^2} 
                \frac{\sqrt[3]{(x+2)^4} + \sqrt[3]{(x+2)^2(x - 2)^2} + \sqrt[3]{(x - 2)^4}}{\sqrt[3]{(x+2)^4} + \sqrt[3]{(x+2)^2(x - 2)^2} + \sqrt[3]{(x - 2)^4}} \\
                &= \limxinf \frac{(x+2)^2 - (x-2)^2}{\sqrt[3]{(x+2)^4} + \sqrt[3]{(x+2)^2(x - 2)^2} + \sqrt[3]{(x - 2)^4}} \tag{vzorec pro $a^3 - b^3$} \\
                &\sim \limxinf \frac{x}{x^{\frac{4}{3}}} 
                = \limxinf \frac{1}{x^{\frac{1}{3}}} = 0.
            \end{align*}

        \item Spočteme první derivaci, určíme intervaly monotonie a 
            nalezneme lokální a globální extrémy.
            
            \begin{enumerate}
                \item Derivace.
                    \begin{align*}
                        f'(x) 
                        &= \frac{2}{3}(x+2)^{\frac{2}{3}-1} \cdot 1 
                            - \frac{2}{3}(x-2)^{\frac{2}{3}-1}\cdot 1 \\
                        &= \frac{2}{3}\left(\frac{1}{\sqrt[3]{x+2}} - \frac{1}{\sqrt[3]{x-2}}\right)
                    \end{align*}
                    Tento výraz platí pro $\fa x \neq \pm2.$ Pro určení derivace v bodech
                    $x = \pm 2$ se pokusíme využít věty o derivaci a limitě derivace
                    (Věta~\ref{th:derlimder}):
                    $$\lim_{x \to 2+} f'(x) = -\infty \implies f'_+(2) = -\infty,$$
                    $$\lim_{x \to 2-} f'(x) = +\infty \implies f'_-(2) = +\infty.$$
                    Díky lichosti funkce platí:
                    $$f'_+(-2) = -f'_+(2) = +\infty,$$
                    $$f'_-(-2) = -f'_-(2) = -\infty.$$
                \item Intervaly monotonie.

                    Určíme intervaly, kde $f'(x) > 0,$ $f'(x) < 0$ a $f'(x) =0:$
                    \begin{align*}
                        f'(x) > 0 
                        &\iff \frac{1}{\sqrt[3]{x+2}} > \frac{1}{\sqrt[3]{x-2}} \\
                        &\iff \frac{1}{x+2} > \frac{1}{x-2} \\
                        &\iff x \in (-2,2).
                    \end{align*}
                    Podobně
                    $$f'(x) < 0 \iff x \in (-\infty,-2) \lor x \in (2,+\infty).$$
                    Dále vyplývá, že $\fa x \in D_f: f'(x) \neq 0.$

                \item Globální extrémy.
                    
                    Dle Fermatovy věta (Věta~\ref{th:fermat}) hledáme globální 
                    extrémy v bodech, kde $f'(x) = 0$ nebo $f'(x)$ neexistuje.
                    V našem případě: $x \in \{-2,2\}.$ 
                    
                    Dle intervalů monotonie určíme,
                    že $f$ má globální minimum v bodě $-2$ a to $-\sqrt[3]{4^2}$
                    a globální maximum v bodě $2$ a to $\sqrt[3]{4^2}.$

                \item První hrubý náčrt grafu funkce.

                    Víme, že v $\limxminf f(x) = 0,$ poté funkce klesá až do bodu
                    $x = -2,$ poté roste do bodu $x=2$ a pak zase klesá, až k nule
                    ($\limxinf f(x) = 0$).
                        \begin{center}
                            \begin{tikzpicture}
                                \begin{axis}[
                                        axis lines=middle,
                                        xlabel=$x$,
                                        ylabel={$f(x)$},
                                        xmin=-4, xmax=4,
                                        ymin=-1.5, ymax=1.5,
                                        xtick={-2, 2},
                                        ytick=\empty,
                                        function line/.style={
                                            black,
                                            thick,
                                            samples=2,
                                        },
                                        single dot/.style={
                                            black,
                                            mark=*,
                                        },
                                        empty point/.style={
                                            only marks,
                                            mark=*,
                                            mark options={fill=white, draw=black},
                                        },
                                    ]
                                    \addplot[function line, domain=\pgfkeysvalueof{/pgfplots/xmin}:-2] {-1-0.1*x};
                                    \addplot[function line, domain=-2:2] {0.4*x};
                                    \addplot[function line, domain=2:\pgfkeysvalueof{/pgfplots/xmax}] {1-0.1*x};
                                    \addplot[single dot] coordinates {(0, 0)};
                                \end{axis}
                            \end{tikzpicture}
                        \end{center}
        \end{enumerate}

        \item Spočteme druhou derivaci a určíme intervaly, kde je $f$ konvexní
            nebo konkávní. Určíme inflexní body.
            \begin{enumerate}
                \item Druhá derivace.

                    $\fa x \in D_f \setminus \{\pm 2\}$ platí:
                    $$f''(x) = -\frac{2}{9}\left((x+2)^{-\frac{4}{3}} - (x-2)^{-\frac{4}{3}}\right).$$
                \item Konvexita, konkávita.

                    Musíme určit intervaly, ve kterých $f''(x) > 0$ nebo
                    $f''(x) < 0.$ Platí:
                    $$f''(x) > 0 \iff (x+2)^{-\frac{4}{3}} < (x-2)^{-\frac{4}{3}},$$
                    $$f''(x) < 0 \iff (x+2)^{-\frac{4}{3}} > (x-2)^{-\frac{4}{3}},$$
                    a tedy funkce $f$ je:
                    \begin{itemize}
                        \item konvexní na $(0,2)$ a $(2, +\infty),$
                        \item konkávní na $(-\infty, -2)$ a $(-2,0).$
                    \end{itemize}
                \item Inflexní bod.

                    Z výrazu pro druhou derivaci víme, že $f''(x) = 0 \iff x =0.$
                    Zároveň jsme již určili, že $f''(x) < 0$ na $(-2,0)$ a 
                    $f''(x) > 0$ na $(0,2).$ Tím jsme splinili postačující podmínku
                    pro inflexi (Věta~\ref{th:inflexepostac}), a proto se v 
                    $x=0$ nachází inflexní bod.
            \end{enumerate}
        \item Vypočteme asymptoty funkce.

            Jelikož $\limxinf f(x) = \limxminf f(x) = 0,$ asymptoty splývají s
            osou $x.$

        \item Načrtneme graf funkce a určíme obor hodnot.
            \begin{center}
                \begin{tikzpicture}
                    \begin{axis}[
                            axis lines=middle,
                            xlabel=$x$,
                            ylabel={$f(x)$},
                            function line/.style={
                                black,
                                thick,
                                samples=200,
                            },
                        ]
                        \addplot[function line, domain=-3:3] {((x+2)^2)^(1/3) - ((x-2)^2)^(1/3)};
                    \end{axis}
                \end{tikzpicture}
            \end{center}

            $H_f = [-\sqrt[3]{4^2},\sqrt[3]{4^2}].$
    \end{enumerate}
\end{example}

\subsection{Taylorův polynom}

\begin{definition}
    \Necht $f$ je reálná funkce, $a \in \R$ a existuje vlastní $n$-tá derivace
    $f$ v bodě $a.$ Pak polynom
    $$\tfan(x) = f(a) + f'(a)(x-a) + \dots + \frac{1}{n!}f^{(n)}(x-a)^n$$
    nazýváme \newterm{Taylorovým polynomem řádu $n$ funkce $f$ v bodě $a$}.
\end{definition}

\begin{remark}[vlastnosti Taylorova polynomu]
    \leavevmode
    \begin{itemize}
        \item $\deg \tfan(x) \leq n.$
        \item Derivace Taylorova polynomu:
            \begin{align*}
                (\tfan)'(x) 
                &= 0 + f'(a) \cdot 1 + \dots + \frac{1}{n!} \cdot f^{(n)}(a)\cdot n \cdot (x-a)^{n-1} \\
                &= T^{f',a}_{n-1}(x).
            \end{align*}
    \end{itemize}
\end{remark}

\begin{lemma}
    \label{lm:taylorqp0}
    \Necht $Q$ je polynom, $a \in \R, \deg Q \leq n$ 
    a $\limxa \frac{Q(x)}{(x-a)^n} = 0.$ Pak $Q \equiv 0.$
\end{lemma}

\begin{proof}
    Lemma dokážeme indukcí. V základním kroce ($n=1$) je polynom $Q$ lineární. 
    Dále:
    \begin{align*}
        Q(a)
        &= \limxa \left(\frac{Q(x)}{x-a}(x-a)\right) \\
        &\qe \limxa \frac{Q(x)}{x-a} \cdot \limxa (x-a)^n 
            \tag{aritmetika limit funkcí, Věta~\ref{th:voalf}}\\
        &= 0 \cdot 0 = 0. \tag{předpoklad}
    \end{align*}
    Polynom $Q$ tedy můžeme vyjádřit jako $Q(x) = c(x-a).$ Nakonec
    odvodíme, že koeficient $c$ je roven nule, jelikož:
    $$ 0 = \limxa \frac{Q(x)}{x-a} = \limxa \frac{c(x-a)}{x-a} = \limxa c = c.$$

    V indukčním kroce ($n-1 \rightarrow n$) platí podobně jako výše, že $a$ je
    kořenem polynomu $Q,$ a proto jej můžeme vyjádřit jako:
    $$Q(x) = (x-a)R(x),$$
    kde $R$ je polynom a $\deg R \leq n-1.$ Dle indukčního předpokladu je 
    $R \equiv 0$, a tedy i $Q \equiv 0.$
\end{proof}

\begin{theorem}[o nejlepší aproximaci Taylorovým polynomem]
    \Necht $a \in \R, f^{(n)}(a) \in \R$ a $P$ je polynom stupně nejvýše $n.$ Pak
    $$\limxa \frac{f(x) - P(x)}{(x-a)^n} = 0 \iff P = \tfan. $$
\end{theorem}

\begin{proof}
    \leavevmode
    \begin{itemize}
        \item[$\impliedby$]
            Využijeme matematickou indukci. Pro případ $n=1:$
            \begin{align*}
                \limxa \frac{f(x) - T^{f,a}_1}{x-a} 
                &= \limxa \frac{f(x) - f(a) - f'(a)(x-a)}{x-a} \\
                &= \limxa \left(\frac{f(a)-f(a)}{x-a} - f'(a)\right) \\
                &= f'(a) - f'(a) = 0.
            \end{align*}
            Indukční krok:
            \begin{align*}
                \limxa \frac{f(a) - \tfan(x)}{(x-a)^n}
                &= \limxa \frac{f'(x) - (\tfan)'(x)}{n(x-a)^{n-1}}
                    \tag{L'Hospital (Věta~\ref{th:lhospital}),
                    $\frac{0}{0}$} \\
                &= \frac{1}{n} \limxa \frac{f'(x) - T^{f',a}_{n-1}(x)}{(x-a)^{n-1}} \\
                &= \frac{1}{n}\cdot0 = 0\tag{indukční předpoklad}
            \end{align*}

        \item[$\implies$]
            Rozepišme limitu na levé straně implikace za použití věty 
            o limitě limit funkcí (Věta~\ref{th:voalf}) jako součet dvou limit:
            $$\limxa \frac{P(x) - \tfan(x)}{(x-a)^n} 
            \qe \underbrace{\limxa \frac{P(x) - f(x)}{(x-a)^n}}_{A} 
                + \underbrace{\limxa \frac{f(x) - \tfan(x)}{(x-a)^n}}_{B}$$
            Výraz $A$ je roven nule dle předpokladů. Výraz $B$ je dle předchozího
            bodu taktéž roven nule, a proto
            $$\limxa \frac{P(x) - \tfan(x)}{(x-a)^n}  = 0$$
            Díky lemmatu~\ref{lm:taylorqp0} platí $P = \tfan.$
    \end{itemize}
\end{proof}

\begin{theorem}[Taylor, či obecný tvar zbytku]
    \Necht funkce $f$ má vlastní $(n+1)$-ní derivaci v intervalu $[a,x]$ a \necht
    $\phi$ je spojitá funkce v $[a,x]$ a má vlastní derivaci v $(a,x),$ která je
    v každém bodě tohoto intervalu různá od nuly. Pak existuje $\xi \in (a,x)$ \tz
    $$f(x) - \tfan(x) = \frac{1}{n!}\frac{\phi(x)-\phi(a)}{\phi'(\xi)}f^{(n+1)}(\xi)(x-\xi)^n.$$
    Speciálně existuje $\xi_1 \in (a,x)$ \tz
    $$f(x) - \tfan(x) = \frac{1}{(n+1)!}f^{(n+1)}(\xi_1)(x-a)^{n+1} \; 
    \text{(Lagrangeův tvar zbytku)}$$
    a existuje $\xi_2 \in (a,x)$ \tz
    $$f(x) - \tfan(x) = \frac{1}{n!}f^{(n+1)}(\xi_2)(x-\xi_2)^n(x-a). \;
    \text{(Cauchyho tvar zbytku)}$$
\end{theorem}

\begin{proof}
    Rozdíl $f(x) - \tfan(x)$ nazýváme zbytek. Pro $t \in [a,x]$ definujme 
    následující funkci:
    $$F(t) = f(x) - T_n^{f,t}(x) = f(x) - \left(f(t) + f'(t)(x-t) + \dots + \frac{f^{(n)}(t)}{n!}(x-t)^n\right)$$
    Tato funkce je:
    \begin{itemize}
        \item spojitá na $[a,x],$
        \item má vlastní derivaci na $(a,x),$
        \item $F(x) = 0,$
        \item $F(a) = f(x) - \tfan(x).$ 
    \end{itemize}
    Dle Cauchyho věty o střední hodnotě (Věta~\ref{th:cauchymean}):
    $$\exists \xi \in (a,x): \frac{F'(\xi)}{\phi'(\xi)} = \frac{F(x) - F(a)}{\phi(x) - \phi(a)} = \frac{0 - (f(x) - \tfan(x))}{\phi(x) - \phi(a)},$$
    z čehož po úpravě dostáváme:
    $$f(x) - \tfan(x) = -\frac{F'(\xi)}{\phi'(\xi)}(\phi(x)-\phi(a)).$$
    Vyjádřeme si nyní derivaci funkce $F:$
    \begin{align*}
        F'(t)
        &= 0 - (f'(t) + f''(t)(x-t) + f'(t)(-1) + \dots + \frac{f^{(n+1)}(t)}{n!}(x-t)^n + \frac{f^{(n)}}{n!}n(x-t)^{n-1}(-1) \\
        &= -\frac{f^{(n+1)}(t)}{n!}(x-t)^n.
    \end{align*}
    Tento výraz nyní můžeme dosadit do vzorce zbytku, který jsme vyjádřili výše,
    a získáme výraz pro obecný tvar zbytku:
    \begin{align*}
        f(x) - \tfan(x) 
        &= -\frac{-\frac{f^{(n+1)}(\xi)}{n!}(x-\xi)^n}{\phi'(\xi)}(\phi(x)-\phi(a)) \\
        &= \frac{1}{n!}\frac{\phi(x)-\phi(a)}{\phi'(\xi)}f^{(n+1)}(\xi)(x-\xi)^n.
    \end{align*}

    Pro Lagrangeův tvar zbytku volíme $\phi(t) = (x-t)^{n+1}.$ Potom:
    \begin{align*}
        \phi(x) - \phi(a) &= 0 - (x-a)^{n+1}, \\
        \phi'(t) &= (n+1)(x-t)^n(-1), 
    \end{align*}
    a po dosazení do obecného tvaru zbytku pro $\xi_1 \in (a,x):$
    \begin{align*}
        f(x) - \tfan(x) 
        &= \frac{1}{n!}\frac{-(x-a)^{n+1}}{(n+1)(x-\xi_1)^n(-1)}f^{(n+1)}(\xi_1)(x-\xi_1)^n \\
        &=  \frac{1}{(n+1)!}f^{(n+1)}(\xi_1)(x-a)^{n+1} 
    \end{align*}
    
    Podobně pro Cauchyho tvar zbytku volíme $\phi(t) = t.$ Potom:
    \begin{align*}
        \phi(x) - \phi(a) &= x-a, \\
        \phi'(t) &= 1, 
    \end{align*}
    a po dosazení do obecného tvaru zbytku pro $\xi_2 \in (a,x):$
    \begin{align*}
        f(x) - \tfan(x) 
        &= \frac{1}{n!}\frac{x-a}{1}f^{(n+1)}(\xi_2)(x-\xi_2)^n \\
        &= \frac{1}{n!}(x-a)f^{(n+1)}(\xi_2)(x-\xi_2)^n 
    \end{align*}

\end{proof}

\begin{remark}
    Taylorova věta platí i pro interval $(x,a).$ 
\end{remark}

\begin{example}[Taylorův polynom pro exponenciálu]
    Z poznámky~\ref{rm:derivaceelfce} víme, že
    $$(e^x)' = (e^x)'' = \dots = (e^x)^{(n)} = e^x.$$
    Jelikož při $x=0$ jsou všechny derivace rovny $1,$ platí:
    $$T^{exp, 0}_n = \sum_{j=0}^n \frac{x^j}{j!}.$$
    Zvolme pevné $x.$ Potom pro dané $n$ existuje $\xi_n \in (0, x)$ (nebo $(x,0),$
    pokud je $x < 0$) \tz pro Lagrangeův tvar zbytku platí:
    $$e^x - T^{exp, 0}_n =e^x - \sum_{j=0}^n \frac{x^j}{j!} 
    = \frac{1}{(n+1)!}e^{\xi_n}x^{n+1}$$
    Jelikož $e^{\xi_n} \leq e^{|x|}$ a $x^{n+1} \leq |x|^{n+1},$ platí:
    $$|e^x - T^{exp, 0}_n| \leq \frac{e^{|x|}|x|^{n+1}}{(n+1)!}.$$
    Za použití věty o dvou strážnících (Věta~\ref{th:dvastraznici}) a 
    $\limninf \frac{x^n}{n!} = 0$ dostáváme:
    $$\limninf e^x - T^{exp, 0}_n = 0$$
    a tedy:
    $$\fa x \in \R: e^x = \sum_{j=0}^\infty \frac{x^j}{j!}.$$
    Speciálně pro $e:$
    $$e = e^1 = \sum_{j=0}^\infty \frac{1^j}{j!}= 1 + \frac{1}{1} + \frac{1}{2!}
    + \dots + \frac{1}{n!} + \dots$$
\end{example}

\begin{example}
    Spočtěte hodnotu $e$ s chybou $0.001.$

    Vyjádřeme $e$ jako součet dvou sum:
    $$e = \sum_{j=0}^\infty \frac{1}{j!} = \sum_{j=0}^n \frac{1}{j!} + \sum_{j=n+1}^\infty \frac{1}{j!}.$$
    Chceme, aby:
    $$\sum_{j=n+1}^\infty \frac{1}{j!} < 0.001,$$
    a naším úkolem je zjistit hodnotu $n$, tj. zjistit, kolik členů Taylorova
    polynomu musíme spočítat pro zajištění dané přesnosti. Platí:
    \begin{align*}
        \sum_{j=n+1}^\infty \frac{1}{j!}
        &= \sum_{j=0}^\infty \frac{1}{(n+1+j)!} \\
        &\leq \frac{1}{(n+1)!}\sum_{j=0}^\infty \frac{1}{(n+1)^j}
            \tag{$(n+1+j)! \geq (n+1)!(n+1)^j$} \\
        &= \frac{1}{(n+1)!} \frac{1}{1 - \frac{1}{n+1}} 
            \tag{součtový vzorec geometrické řady}\\
        &= \frac{1}{(n+1)!} \frac{n+1}{n} \\
        &= \frac{1}{n\cdot n!}
    \end{align*}
    Již při $n = 6$ je $\frac{1}{n\cdot n!} < 0.001,$ a nám tedy stačí 
    spočítat prvních 6+1 členů Taylorova polynomu ($n$ se počítá od nuly):
    $$e \approx 1 + \frac{1}{1} + \frac{1}{2} + \frac{1}{6} + \frac{1}{24} + 
    \frac{1}{120} + \frac{1}{720}.$$
\end{example}

\begin{example}
    Dokažte, že číslo $e$ je iracionální.

    Sporem. \Necht $e = \frac{p}{q},$ kde $p,q \in \N.$ Z předchozího příkladu víme,
    že:
    $$\sum_{n=0}^q \frac{1}{n!} < e = \sum_{n=0}^q \frac{1}{n!} 
    + \sum_{n=q+1}^\infty \frac{1}{q!} < \sum_{n=0}^q\frac{1}{n!} + \frac{1}{q\cdot q!}$$
    Vynásobme nerovnici výrazem $(q\cdot q!)$ a získáme:
    $$\underbrace{(q\cdot q!)\sum_{n=0}^q \frac{1}{n!}}_{A} 
    < pq!
    < \underbrace{(q\cdot q!) \sum_{n=0}^q \frac{1}{n!}}_{A} + 1.$$
    Jelikož $A \in \N$ i $pq! \in \N,$ dostáváme spor, jelikož 
    $pq!$ by muselo být přirozené číslo mezi $A$ a $A+1.$
\end{example}

\begin{example}[Taylorův polynom pro funkci $\sin$]
    \Necht $a = 0.$ Potom platí:
    \begin{align*}
        &\sin'(x) = \cos(x), \; \sin'(a) = 1, \\
        &\sin''(x) = -\sin(x), \; \sin''(a) = 0, \\
        &\sin'''(x) = -\cos(x), \; \sin'''(a) = -1, \\
        &\sin^{(4)}(x) = \sin(x), \; \sin^{(4)}(a) = 0. \\
        &\sin^{(5)}(x) = \cos(x), \; \sin^{(5)}(a) = 1, \; \text{atd.}
    \end{align*}
    Potom:
    $$T^{\sin,0}_n(x) = (0) + \left(\frac{1}{1!}x\right) 
    + \left(\frac{1}{2!}0x^2 \right) 
    + \left(\frac{1}{3!}(-1)x^3\right) 
    + \left(\frac{1}{4!}0x^4 \right) 
    + \left(\frac{1}{5!}1x^5 \right) 
    + \dots$$
    Dále:
    \begin{align*}
        \left|\sin(x) - T^{\sin,0}_n\right| 
        &= \left|\frac{1}{n+1}!f^{(n+1)}(\xi)x^{n+1}\right| \\
        &\leq \left|\frac{x^{n+1}}{(n+1)!}\right|
    \end{align*}
    a tím pádem $\limninf (\sin(x) - T^{\sin,0}_n) = 0.$
    Můžeme tedy psát:
    $$\sin(x) = \sum_{n=0}^{\infty} (-1)^n\frac{x^{2n+1}}{(2n+1)!}.$$
    Podobně pro $\cos(x):$
    $$\cos(x) = \sum_{n=0}^{\infty} (-1)^n\frac{x^{2n}}{(2n)!}.$$
\end{example}

\section{Řady}

\subsection{Úvod}

\begin{definition}
    \label{df:rady}
    \Necht $\seq{a_n}_{n\in\N}$ je posloupnost. Číslo $s_m = a_1 + a_2 + \dots
    + a_m$ nazveme \newterm{$m$-tým částečným součtem} řady $\sum_{n=1}^\infty a_n.$
    
    \newterm{Součtem} nekonečné řady $\sum_{n=1}^\infty a_n$ nazveme limitu
    posloupnosti $\seq{s_m}_{m\in\N},$ pokud tato limita existuje.

    Je-li $\lim_{m\to\infty} s_m$ konečná, pak řekneme, že řada je 
    \newterm{konvergentní}. Je-li tato limita nekonečná nebo neexistuje,
    pak řekneme, že řada je \newterm{divergentní}. Tuto limitu budeme
    značit $\rada{a_n}.$
\end{definition}

\begin{example}
    \label{ex:priklady_rad}
    \leavevmode
    \begin{itemize}
        \item $\rada{(-1)^n}$
            
            Diverguje, neboť $s_{2k+1}=-1$ a $s_{k} = 0.$

        \item $\rada{\frac{1}{n(n+1)}}$
            
            Platí:
            \begin{align*}
                \rada{\frac{1}{n(n+1)}}
                &= \rada{\frac{n + 1 - n}{n(n+1)}} \\
                &= \rada{\left(\frac{1}{n} - \frac{1}{n+1}\right)} \\
                &= \underbrace{1 - \frac{1}{2}}_{a_1}
                   + \underbrace{\frac{1}{2} - \frac{1}{3}}_{a_2} 
                   + \frac{1}{3} - \frac{1}{4} 
                   + \frac{1}{4} - \frac{1}{5} + \dots
            \end{align*}
            a tedy:
            $$s_m = 1 - \frac{1}{m}.$$
            Potom 
            $$\rada{\frac{1}{n(n+1)}} = 1.$$

        \item Geometrická řada: $\rada{q^{n-1}} = 1 + q + q^2 + q^3 + \dots$

            Pokud $q = 1,$ potom triviálně:
            $$s_m = m.$$
            Pro $q \neq 1$ můžeme využít vzorečku $a^n -1 = (a-1)(a^{n-1} 
            + a^{n-2} + \dots + 1)$ a psát:
            $$s_m = 1 + q + q^2 + \dots + q^{m-1} = \frac{q^m - 1}{q-1}.$$
            Pro limitu $s_m$ při $m \to \infty$ platí:
            $$\lim_{m\to\infty} s_m = \begin{cases}
                \infty &q\geq1 \\
                \frac{1}{1-q} &|q|<1 \\
                \text{neexistuje} &q\leq-1
            \end{cases}$$
            a řada $\rada{q^{n-1}}$ konverguje při $|q| < 1.$
            
        \item $\rada{\frac{1}{n^2}} = \frac{\pi^2}{6}.$

            Tento výsledek si odvodíme, až když budeme brát Fourierovy řady.
    \end{itemize}
\end{example}

\begin{theorem}[nutná podmínka pro konvergenci řad]
    \label{th:radykonvergencenutna}
    Jestliže je $\rada{a_n}$ konvergentní, pak $\limninf a_n = 0.$
\end{theorem}

\begin{proof}
    Řada $\rada{a_n}$ konverguje, a tedy $s = \lim_{m\to\infty} s_m \in \R.$ Potom
    $$\limminf s_{m+1} = \limminf s_m = s.$$
    Dále:
    \begin{align*}
        0 
        &= s-s \\
        &= \limminf s_{m+1} - \limminf s_m \\
        &= \limminf (s_{m+1} - s_m) \tag{aritmetika limit, Věta~\ref{th:voal}}  \\
        &= \limminf a_{m+1} = \limminf a_m.
    \end{align*}
\end{proof}

\begin{example}
    Geometrická řada $\rada{q^{n-1}}$ konverguje, právě když $|q| < 1.$ Zároveň
    platí:
    $$|q| < 1 \implies \limninf q^{n-1} = 0.$$
\end{example}

\begin{remark}
    \label{rm:konvergence_harmonicke_rady}
    Implikaci v předešlé větě nelze obrátit. Uvažujme například harmonickou
    řadu: 
    $$\rada{\frac{1}{n}}.$$
    Pro částečné součty $m$ a $2m$ členů platí:
    \begin{align*}
        s_m &=1 + \frac{1}{2} + \frac{1}{3} + \dots + \frac{1}{m} \\
        s_{2m} &= 1 + \frac{1}{2} + \frac{1}{3} + \dots + \frac{1}{m} + \frac{1}{m+1} + \frac{1}{m+2} + \dots + \frac{1}{2m} \\
        s_{2m} - s_m &= \frac{1}{m+1} + \frac{1}{m+2} + \dots + \frac{1}{2m} \\
                     &\geq \frac{1}{2m} + \frac{1}{2m} + \dots + \frac{1}{2m} \\
                     &= m \frac{1}{2m} = \frac{1}{2},
    \end{align*}
    a tedy:
    $$\fa m \in \N: s_{2m}-s_m \geq \frac{1}{2}.$$
    Posloupnost $\seq{s_m}_{m\in\N}$ tím nesplňuje Bolzano-Cauchyho
    podmínku (Věta~\ref{th:bolzanocauchy}) a nemá vlastní limitu.
\end{remark}

\begin{theorem}[linearita konvergentních řad]
    \label{th:linearitakonvrad}
    \leavevmode
    \begin{enumerate}[(i)]
        \item \Necht $\alpha \in \R \setminus \{0\},$ pak
            $$\rada{a_n} \text{ konverguje} \iff \rada{\alpha a_n} \text{ konverguje.}$$

        \item \Necht $\rada{a_n}$ konverguje a $\rada{b_n}$ konverguje,
            pak $$\rada{(a_n + b_n)} \text{ konverguje.}$$
    \end{enumerate}
\end{theorem}

\begin{proof}
    Jednoduchý, s využitím věty o aritmetice limit (Věta~\ref{th:voal}).
\end{proof}

\subsection{Řady s nezápornými členy}

\begin{observation}
    Nechť $\fa n \in \N: a_n \geq 0.$ Potom
    $\rada{a_n} \in \R$ nebo $\rada{a_n} = +\infty.$
\end{observation}
    
\begin{proof}
    Díky nezápornosti členů $a_n$ je posloupnost částečných součtů 
    $\seq{s_m}$ neklesající a tedy dle věty o limitě 
    monotónní posloupnosti (Věta~\ref{th:monotonniposl}) má limitu.
\end{proof}

\begin{theorem}[srovnávací kritérium]
    \label{th:srovnavacikrit}
    \Necht $\rada{a_n}$ a $\rada{b_n}$ jsou řady s nezápornými členy a nechť
    existuje $n_0 \in \N$ \tz pro všechna $n\in\N,n\geq n_0$ platí $a_n \leq b_n.$
    Pak
    \begin{enumerate}[(i)]
        \item $\displaystyle \rada{b_n} \text{ konverguje} \implies \rada{a_n} \text{ konverguje,}$
        \item $\displaystyle \rada{a_n} \text{ diverguje} \implies \rada{b_n} \text{ diverguje.}$
    \end{enumerate}
\end{theorem}

\begin{proof}
    \leavevmode
    \begin{enumerate}[(i)]
        \item Označme částečné součty:
            \begin{align*}
                s_m &= a_1 + a_2 + \dots + a_m \\
                \sigma_m &= b_1 + b_2 + \dots + b_m.
            \end{align*}
            Jelikož $\rada{b_n}$ konverguje, označme $\limminf \sigma_m = \sigma.$
            Pro všechna $m \geq n_0$ platí:
            \begin{align*}
                s_m &= a_1 + a_2 + \dots + a_{n_0} + a_{n_0 + 1} + \dots + a_m \\
                    &\leq a_1 + a_2 + \dots + a_{n_0} + b_{n_0 + 1} + \dots + b_m \\
                    &\leq a_1 + a_2 + \dots + a_{n_0} + \sigma_m \\
                    &\leq a_1 + a_2 + \dots + a_{n_0} + \sigma.
            \end{align*}
            Posloupnost $\seq{s_m}_{m \in \N}$ je tedy neklesající posloupnost omezená shora
            číslem $(a_1 + \dots + a_{n_0} + \sigma) \in \R$ a dle věty o monotónní
            posloupnosti (Věta~\ref{th:monotonniposl}) má vlastní limitu.
        \item Ekvivalentní s bodem (i): $(A \implies B) \implies (\lnot B \implies \lnot A).$
    \end{enumerate}
\end{proof}

\begin{theorem}[limitní srovnávací kritérium]
    \Necht $\rada{a_n}$ a $\rada{b_n}$ jsou řady s nezápornými členy a nechť
    $$\limninf \frac{a_n}{b_n} = K \in \Rstar.$$
    \begin{enumerate}[(i)]
        \item Jestliže $K \in (0, \infty),$ pak $\displaystyle \rada{b_n} 
            \text{ konverguje} \iff \rada{a_n} \text{ konverguje.}$
         \item Jestliže $K = 0,$ pak $\displaystyle \rada{b_n} 
            \text{ konverguje} \implies \rada{a_n} \text{ konverguje.}$
         \item Jestliže $K = \infty,$ pak $\displaystyle \rada{a_n} 
            \text{ konverguje} \implies \rada{b_n} \text{ konverguje.}$
    \end{enumerate}
\end{theorem}

\begin{proof}
    Jelikož obě řady jsou řady s nezápornými členy, platí, že $K \geq 0.$
    \begin{enumerate}[(i)]
        \item Z definice limity vyplývá, že pro $\e = \frac{K}{2}:$
            $$\exists n_0 \in \N \; \fa n \geq n_0, n\in\N: \left|\frac{a_n}{b_n}
            -K\right| < \frac{K}{2},$$
            a tedy $\fa n \geq n_0, n\in\N:$ 
            $$\frac{K}{2} < \frac{a_n}{b_n} < \frac{3}{2}K,$$
            $$\frac{K}{2}b_n < a_n < \frac{3}{2}Kb_n.$$
            Potom: 
            \begin{align*}
                \rada{a_n} \text{ konverguje} 
                &\implies \rada{\frac{K}{2}b_n} \text{ konverguje}
                    \tag{srovnávací kritérium, Věta~\ref{th:srovnavacikrit}} \\
                &\implies \rada{b_n} \text{ konverguje}
                    \tag{linearita konvergentních řad, 
                    Věta~\ref{th:linearitakonvrad}}
            \end{align*}
            Opačný směr řešíme podobně: 
            \begin{align*}
                \rada{b_n} \text{ konverguje} 
                &\implies \rada{\frac{3}{2}Kb_n} \text{ konverguje}
                    \tag{linearita konvergentních řad} \\
                &\implies \rada{a_n} \text{ konverguje}
                   \tag{srovnávací kritérium}
            \end{align*}

        \item Zvolme $\e = 1.$ Potom
            $$\exists n_0 \in \N\; \fa v \geq n_0, n\in\N: \left|\frac{a_n}{b_n}\right| < 1$$
            a tedy
            $$\fa n \geq n_0, n\in\N: a_n < b_n.$$
            Potom, pokud $\rada{b_n}$ konverguje, tak podle srovnávacího kritéria
            (Věta~\ref{th:srovnavacikrit}) konverguje i $\rada{a_n}.$

        \item Zvolme $L = 1.$ Potom dle definice nevlastní limity 
            (Definice~\ref{df:nevlastnilimitaposl}):
            $$\exists n_0 \in \N\; \fa v \geq n_0, n\in\N: \frac{a_n}{b_n} > L = 1,$$
            a tedy
            $$\fa n \geq n_0, n\in\N: a_n > b_n.$$
            Pokud konverguje $\rada{a_n},$ pak dle srovnávacího kritéria 
            (Věta~\ref{th:srovnavacikrit}) konverguje i $\rada{b_n}.$
    \end{enumerate}
\end{proof}

\begin{example}
    Určete, zda-li následující řady konvergují:
    \begin{multicols}{2}
        \begin{enumerate}[(i)]
            \item $\displaystyle \rada{\frac{n-\sqrt{n}}{n^2 + 3n}},$
            \item $\displaystyle \rada{\frac{n^5}{3^n}}.$
        \end{enumerate}
    \end{multicols}
    Řešení:
    \begin{enumerate}[(i)]
        \item
            Pokud si z této řady vezmeme jen nejdůležitější členy, vidíme, že je podobná
            řadě $\rada{\frac{n}{n^2}} = \rada{\frac{1}{n}},$ o které víme, že diverguje 
            (Poznámka~\ref{rm:konvergence_harmonicke_rady}). Označme 
            $a_n = \frac{n-\sqrt{n}}{n^2 + 3n}$ a $b_n = \frac{1}{n}$ a pokusme se použít
            limitní srovnávací kritérium:
            \begin{align*}
                \limninf \frac{a_n}{b_n}
                &=\limninf \frac{\frac{n-\sqrt{n}}{n^2 + 3n}}{\frac{1}{n}} \\
                &= \limninf \frac{n - \sqrt{n}}{n + 3} \\
                &= \limninf \frac{1 - \frac{1}{\sqrt{n}}}{1 + \frac{3}{n}} \\
                &=1
            \end{align*}
            Dle limitního srovnávacího kritéria, bodu (i) diverguje i
            $\rada{\frac{n-\sqrt{n}}{n^2 + 3n}}.$
        \item Označme podobně $a_n = \frac{n^5}{3^n}$ a $b_n = \frac{1}{2^n}.$
            Řada $\rada{b_n}$ konverguje, jelikož se jedná o geometrickou řadu
            a $q = \frac{1}{2} < 1$ (Poznámka~\ref{ex:priklady_rad}).
            Dále
            $$\limninf \frac{a_n}{b_n} = \frac{\frac{n^5}{3^n}}{\frac{1}{2^n}}
            = \frac{n^5}{\left(\frac{3}{2}\right)^n} = 0.$$
            Dle limitního srovnávacího kritéria, bodu (ii) konverguje i řada 
            $\rada{n^5}{3^n}.$
    \end{enumerate}
\end{example}

\begin{theorem}[Cauchyho odmocninové kritérium]
    \Necht $\rada{a_n}$ je řada s nezápornými členy. 
    \begin{enumerate}[(i)]
        \item $\displaystyle \exists q \in (0,1) \; \exists n_0 \in \N \; 
            \fa n \in \N, n\geq n_0: \sqrt[n]{a_n} \leq q 
            \implies \rada{a_n} \text{ konverguje,}$
        \item $\displaystyle \limsup_{n\to\infty} \sqrt[n]{a_n} < 1 
            \implies \rada{a_n} \text{ konverguje,}$
        \item $\displaystyle \lim_{n\to\infty} \sqrt[n]{a_n} < 1 
            \implies \rada{a_n} \text{ konverguje,}$
        \item $\displaystyle \limsup_{n\to\infty} \sqrt[n]{a_n} > 1 
            \implies \rada{a_n} \text{ diverguje,}$
        \item $\displaystyle \lim_{n\to\infty} \sqrt[n]{a_n} > 1 
            \implies \rada{a_n} \text{ diverguje.}$


    \end{enumerate}
\end{theorem}

\begin{proof}
    \leavevmode
    \begin{enumerate}[(i)]
        \item Označme $b_n = q^n.$ Jelikož $q \in (0,1),$ geometrická řada
            $\rada{b_n}$ konverguje (Příklad~\ref{ex:priklady_rad}). 
            Dále, jelikož dále $\fa n \geq n_0, n\in\N: a_n \leq b_n,$
            dle srovnávacího kritéria (Věta~\ref{th:srovnavacikrit}) konverguje
            i řada $\rada{a_n}.$

        \item Označme $\limsup \sqrt[n]{a} = A.$ Zvolme $\e = \frac{1-A}{2}.$
            Potom $A + \e < 1.$ Dle definice limity:
            $$\exists n_0 \in \N, \fa n\in\N,n \geq n_0: 
            \sup\{\sqrt[k]{a_k},k\geq n\} \leq A + \e,$$
            a tedy
            $$\fa n\in\N, n \geq n_0: \sqrt[n]{a_n} \leq A + \e.$$ 
            Označme $q = A + \e.$ Potom dle předchozího bodu (i) důkazu 
            řada $\rada{a_n}$ konverguje.

        \item Jelikož existuje $\limninf \sqrt[n]{a_n},$ existuje dle
            věty o vztahu limity a limes superior
            (Věta~\ref{th:limitalimsupliminf}) i limes superior a tyto dvě
            limity se rovnají. Dle bodu
            (ii) řada $\rada{a_n}$ konverguje.

        \item Jelikož 
            $$\limsup_{n\to\infty} \sqrt[n]{a_n} > 1,$$
            existuje vybraná posloupnost $\seq{a_{n_k}}_{k=1}^\infty$ \tz
            $$\fa k  \in \N: \sqrt[n_k]{a_{n_k}},$$
            a tedy $\fa k \in \N: a_{n_k} > 1.$ Tím není splněna nutná
            podmínka pro konvergenci řad (Věta~\ref{th:radykonvergencenutna})
            a řada $\rada{a_n}$ diverguje.

        \item Vyplývá z předchozího bodu (iv).
    \end{enumerate}
\end{proof}

\begin{example}
    Určete, zda-li následující řada konverguje:
    $$\rada{\frac{n^5}{3^n}}.$$

    Pokusíme zjistit $\limninf \sqrt[n]{a_n}$ a případně použít Cauchyho 
    odmocninové kritérium. Tedy:
    $$\limninf \sqrt[n]{a_n}
    = \limninf \sqrt[n]{\frac{n^5}{3^n}}
    = \limninf \frac{\sqrt[n]{n^5}}{3}
    = \frac{1}{3}.$$
    Jelikož $\limninf \sqrt[n]{a_n} < 1$, tak dle Cauchyho odmocninového
    kritéria řada $\rada{a_n}$ konverguje.
\end{example}

\begin{theorem}[d'Alambertovo podílové kritérium]
    \Necht $\rada{a_n}$ je řada s kladnými členy.
    \begin{enumerate}[(i)]
        \item $\displaystyle \exists q \in (0,1) \; \exists n_0 \in \N \;
            \fa n \in \N,n\geq n_0: \frac{a_{n+1}}{a_n} < q \implies
            \rada{a_n} \text{ konverguje}.$
        \item $\displaystyle \limsupn \frac{a_{n+1}}{a_n} < 1 \implies
            \rada{a_n} \text{ konverguje}.$
        \item $\ds \limninf \frac{a_{n+1}}{a_n} < 1 \implies
            \rada{a_n} \text{ konverguje}.$
        \item $\ds \limninf \frac{a_{n+1}}{a_n} > 1 \implies
            \rada{a_n} \text{ diverguje}.$
    \end{enumerate}
\end{theorem}

\begin{proof}
    \leavevmode
    \begin{enumerate}[(i)]
        \item Pro $n \geq n_0$ platí:
            $$a_{n+1} < qa_n < q^2a_{n-1} < \dots < q^{n-n_0+1}a_{n_0}.$$
            Geometrická řada 
            $$\rada{q^n\underbrace{a_{n_0}q^{1-n_0}}_{\text{konst.}}}$$
            konverguje (Příklad~\ref{ex:priklady_rad}) a tedy dle srovnávacího
            kritéria (Věta~\ref{th:srovnavacikrit}) konverguje i řada $\rada{a_n}.$
        \item Označme $\limsupn \frac{a_{n+1}}{a_n} = A < 1$ a zvolme $\e = \frac{1-A}{2}.$
            Dle definice limes superior existuje $n_0 \in \N$ \tz:
            $$\fa n \in \N, n\geq n_0: \sup\left\{\frac{a_{k+1}}{a_k}, k \geq n\right\} < A + \e.$$
            Označme $q = A + \e.$ Potom platí:
            $$\fa n \in \N, n\geq n_0: \frac{a_{n+1}}{a_n} < q.$$
            Dle předchozího bodu (i) řada $\rada{a_n}$ konverguje.
        \item Jelikož existuje $\limninf \frac{a_{n+1}}{a_n},$ existuje i
            $\limsupn \frac{a_{n+1}}{a_n}.$ Zbytek dle předchozího bodu (ii).
        \item Dle definice limity:
            $$\exists n_0 \in \N \; \fa n \in \N, n\geq n_0: \frac{a_{n+1}}{a_n} > 1.$$
            Posloupnost je tedy počínaje indexem $n_0$ rostoucí a tím pádem nesplňuje
            nutnou podmínku konvergence (Věta~\ref{th:radykonvergencenutna}),
            jelikož její limita nemůže být nulová.
    \end{enumerate}
\end{proof}

\begin{example}
    Určete, zda-li následující řady konvergují:
    \begin{multicols}{2}
        \begin{enumerate}[(i)]
            \item $\ds \rada{\frac{n^5}{3^n}},$
            \item $\ds \rada{\frac{c^n}{n!}}, c > 0.$
        \end{enumerate}
    \end{multicols}

    Řešení:
    \begin{enumerate}[(i)]
        \item Označme $a_n = \frac{n^5}{3^n}.$ 
            Spočtěme limitu podílu dvou následujících členů posloupnosti 
            $\seq{a_n}:$
            $$\limninf \frac{a_{n+1}}{a_n} 
            = \limninf \frac{\frac{(n+1)^5}{3^{n+1}}}{\frac{n^5}{3^n}}
            = \limninf \frac{3^n(n+1)^5}{3^{n+1}n^5}
            = \limninf \frac{1}{3} \limninf \frac{(n+1)^5}{n^5} 
            = \frac{1}{3}\cdot 1 = \frac{1}{3} < 1.$$
            Jelikož je tato limita menší než 1, můžeme za použití d'Alambertova
            podílového kritéria říci, že řada $\rada{\frac{n^5}{3^n}}$ konverguje.

        \item Analogicky:
            $$\limninf \frac{a_{n+1}}{a_n}
            = \limninf \frac{\frac{c^{n+1}}{(n+1)!}}{\frac{c^n}{n!}} 
            = \limninf \frac{c}{n+1} = 0 < 1.$$
            Řada $\rada{\frac{c^n}{n!}}, c > 0$ konverguje.
    \end{enumerate}
\end{example}

\begin{remark}
    d'Alambertovo podílové kritérium nám nepomůže v případě, že
    $$\limninf \frac{a_{n+1}}{a_n} = 1.$$ 
    Uvažujme například následující dvě řady:
    $$\left(\rada{\frac{1}{n}}\right) \text{ a } \left(\rada{\frac{1}{n^2}}\right).$$
    Pro obě posloupnosti platí, že
    $$\limninf \frac{a_{n+1}}{a_n} = 1,$$
    nicméně harmonická řada $\rada{\frac{1}{n}}$ diverguje 
    (Poznámka~\ref{rm:konvergence_harmonicke_rady}), kdežto řada
    $\rada{\frac{1}{n^2}}$ konverguje (viz dále).
\end{remark}

\begin{theorem}[Raabeho kritérium]
    \Necht $\rada{a_n}$ je řada s kladnými členy.
    \begin{enumerate}[(i)]
        \item $\ds \limninf n(\frac{a_n}{a_{n+1}} -1) > 1 \implies 
            \rada{a_n} \text{ konverguje},$
        \item $\ds \limninf n(\frac{a_n}{a_{n+1}} -1) < 1 \implies
            \rada{a_n} \text{ diverguje}.$
    \end{enumerate}
\end{theorem}

\begin{proof}
    Bez důkazu.
\end{proof}

\begin{theorem}[kondenzační kritérium]
    \Necht $\rada{a_n}$ je řada s nezápornými členy splňující $a_{n+1} \leq a_n$
    pro všechna $n \in \N.$ Pak
    $$\rada{a_n} \text{ konverguje} \iff \rada{2^na_{2^n}} \text{ konverguje}.$$
\end{theorem}

\begin{proof}
    \leavevmode
    \begin{itemize}
        \item[$\implies$]
            Označme:
            $$s_n = a_1 + a_2 + \dots + a_n.$$
            Potom:
            $$s_{2^n} - s_{2^{n-1}} = \underbrace{a_{2^n} + a_{2^n -1} 
                + a_{2^n-2} + \dots + a_{2^{n-1}+1}}_{2^{n-1} \text{ členů}}$$
            Jelikož $\fa n \in \N: a_{n+1} \leq a_n,$ platí:
            $$ 2^{n-1}a_{2^{n-1}} \geq s_{2^n} - s_{2^{n-1}} \geq 2^{n-1}a_{2^n}.$$
            Tuto sadu nerovností sečteme pro $n=1,\dots,k:$
            \begin{align*}
                \sum_{n=1}^k 2^{n-1}a_{2^{n-1}}
                &\geq \sum_{n=1}^k (s_{2^n}-s_{2^{n-1}}) \\
                &= (s_{2^k} - s_{2^{k-1}}) + (s_{2^{k-1}} - s_{2^{k-2}})
                    + \dots + (s_2 - s_1) \\
                &= s_{2^k} - s_1 \\
                &\geq \sum_{n=1}^k 2^{n-1}a_{2^n}.
            \end{align*}

            Nechť řada $\rada{a_n}$ konverguje. Potom existuje $\lim_{k \to \infty}
            (s_{2^k} - s_1),$ řada $\rada{(s_{2^n} - s_{2^{n-1}})}$ konverguje
            a dle srovnávacího kritéria (Věta~\ref{th:srovnavacikrit}) konverguje
            i řada $\rada{2^{n-1}a_{2^n}}$ a dle linearity konvergentních řad
            (Věta~\ref{th:linearitakonvrad})

        \item[$\impliedby$] Nechť řada $\rada{2^na_{2^n}}$ konverguje. Potom
            konverguje i řada $\rada{2^{n-1}a_{2^{n-1}}}.$

            Označme $\rada{2^{n-1}a_{2^{n-1}}} = A \in \R.$ Potom dle předchozího bodu:
            $$\fa k \in \N: s_{2^k} \leq A + s_1.$$
            Posloupnost $\seq{s_{2^k}}_{k \in \N}$ je rostoucí a omezená shora 
            číslem $A + s_1 \in \R$, 
            a proto má dle věty o limitě monotónní posloupnosti 
            (Věta~\ref{th:monotonniposl}) konečnou limitu. Řada $\rada{a_n}$ 
            tedy konverguje.
    \end{itemize}
\end{proof}

\begin{example}
    Dokažte následující dvě tvrzení:
    \begin{enumerate}[(i)]
        \item Řada $\rada{\frac{1}{n^\alpha}}$ konverguje, právě když 
            $\alpha > 1.$
        \item Řada $\sum_{n=2}^\infty \frac{1}{n\log^\alpha n}$ konverguje,
            právě když $\alpha > 1.$
    \end{enumerate}

    Řešení:
    \begin{enumerate}[(i)]
        \item
            Pro $\alpha \leq 0$ platí, že $\limninf \frac{1}{n^\alpha} \neq 0,$ řada
            $\rada{\frac{1}{n^\alpha}}$ nesplňuje nutnou podmínku konvergence 
            (Věta~\ref{th:radykonvergencenutna}), a proto diverguje.

            \Necht $\alpha > 0.$ Platí, že $\limninf \frac{1}{n^\alpha} = 0.$ Dále, dle
            kondenzačního kritéria platí:
            $$\rada{\frac{1}{n^\alpha}} \text{ konverguje}
            \iff \rada{2^n\left(\frac{1}{2^n}\right)^\alpha} 
            = \rada{\left(2^{1-\alpha}\right)^n} \text{ konverguje}.$$
            Řada $\rada{\left(2^{1-\alpha}\right)^n}$ je geometrická řada s 
            $q = 2^{1-\alpha},$ která konverguje (Příklad~\ref{ex:priklady_rad}), 
            právě když $q<1,$ tj. $\alpha > 1.$

        \item Uvažujme případ $\alpha > 0.$ Posloupnost 
            $\seq{\frac{1}{n\log^\alpha n}}$ klesá k nule a řada 
            $\sum_{n=2}^\infty \frac{1}{n\log^\alpha n}$ konverguje, právě když 
            konverguje řada:
            $$\sum_{n=2}^\infty 2^n \frac{1}{2^n\log^\alpha 2^n} 
            = \sum_{n=2}^\infty \frac{1}{\log^\alpha 2^n} 
            = \frac{1}{\log^\alpha 2} \sum_{n=2}^\infty \frac{1}{n^\alpha} 
            $$
            Znovu jsme získali geometrickou řadu a ta konverguje, když 
            $\frac{1}{n^\alpha} < 1,$ tj. když $\alpha > 1.$
    \end{enumerate}
\end{example}

\subsection{Neabsolutní konvergence řad}

\begin{definition}
    \Necht pro řadu $\rada{a_n}$ platí, že $\rada{|a_n|}$ konverguje. Pak říkáme,
    že $\rada{a_n}$ \newterm{konverguje absolutně}.
\end{definition}

\begin{theorem}[Bolzano-Cauchyho podmínka pro konvergenci řad]
    \label{th:bolzano_cauchy_rady}
    Řada $\rada{a_n}$ konverguje, právě tehdy, když je splněna následující
    podmínka
    $$\fa \e > 0 \; \exists n_0 \in \N \; \fa m,n \in \N, m\geq n_0, n\geq n_0:
    \left|\sum_{j=n+1}^m a_j \right| < \e.$$
\end{theorem}

\begin{proof}
    Řada $\rada{a_n}$ konverguje, právě když existuje vlastní limita posloupnosti
    částečných součtů $\limminf s_m.$ Dle Bolzano-Cauchyho podmínky pro
    posloupnosti (Věta~\ref{th:bolzanocauchy}) tato limita existuje,
    právě když:
    $$\fa \e > 0 \; \exists n_0 \in \N \; \fa m,n \in \N, m \geq n_0, 
    n \geq n_0: |s_m - s_n| = \left|\sum_{j=n+1}^m a_j \right|  < \e.$$
\end{proof}

\begin{theorem}[vztah konvergence a absolutní konvergence]
    Nechť řada $\rada{a_n}$ konverguje absolutně, pak řada $\rada{a_n}$
    konverguje.
\end{theorem}

\begin{proof}
    Dle Bolzano-Cauchyho podmínky pro konvergenci řad 
    (Věta~\ref{th:bolzano_cauchy_rady}) řada $\rada{|a_n|}$ konverguje, 
    právě když:
    $$\fa \e > 0 \; \exists n_0 \in \N \; \fa m,n \in \N, m\geq n_0, n\geq n_0:
        \left|\sum_{j=n}^m |a_j| \right| < \e.$$
    Dále platí (trojúhelníková nerovnost, Věta~\ref{th:triangleineq}):
    $$ \left|\sum_{j=n}^m a_j \right| \leq \left|\sum_{j=n}^m |a_j| \right| < \e,$$
    a tedy Bolzano-Cauchyho podmínka pro konvergenci řad je splněna i pro 
    řadu $\rada{a_n}$ a ta tedy konverguje.
\end{proof}

\begin{lemma}[Abelova parciální sumace]
    \label{lm:abelova_parcialni_sumace}
    Nechť $a_1,\dots,a_n, b_1, \dots, b_n \in \R.$ Označme $s_k = \sum_{i=1}^k a_i.$
    Pak platí
    $$\sum_{i=1}^n a_ib_i = \sum_{i=1}^{n-1} s_i(b_i-b_{i+1}) + s_nb_n.$$
    Jestliže navíc $b_1 \geq b_2 \geq \dots \geq b_n \geq 0,$ pak
    $$\left|\sum_{i=1}^n a_ib_i\right|\leq b_1 \max |s_i|.$$
\end{lemma}

\begin{proof}
    Platí:
    \begin{align*}
        \sum_{i=1}^n a_ib_i 
        &= a_1b_1 + a_2b_2 + a_3b_3 + \dots + a_nb_n \\
        &= \underbrace{s_1}_{a_1}b_1 +
            \underbrace{(s_2-s_1)}_{a_2}b_2 + 
            (s_3-s_2)b_3 + \dots + (s_n - s_{n-1})b_n \\
        &= s_1(b_1-b_2) + s_2(b_2-b_3) + \dots + s_{n-1}(b_{n-1} -b_n) + s_nb_n \\
        &= \sum_{i=1}^{n-1} s_i(b_i-b_{i+1}) + s_nb_n.
    \end{align*}

    Jestliže navíc $b_1 \geq b_2 \geq \dots \geq b_n \geq 0,$ platí:
    \begin{align*}
        \left|\sum_{i=1}^n a_ib_i \right|
        &= \left|\sum_{i=1}^{n-1} s_i(b_i - b_{i+1}) + s_nb_n\right| \\
        &\leq \sum_{i=1}^{n-1}|s_i|(b_i-b_{i+1}) + |s_n|b_n \tag{$b_i - b_{i+1} \geq 0, b_n \geq 0$} \\
        &\leq \max_{i=1,\dots,n} |s_i| \left(\sum_{i=1}^{n-1}(b_i - b_{i+1}) + b_n\right) \\
        &= \max_{i=1,\dots,n}|s_i| \cdot b_1.
    \end{align*}
\end{proof}

\begin{theorem}[Abel-Dirichletovo kritérium]
    \label{th:abel_dirichletovo_kriterium}
    \Necht $\seq{a_n}_{n\in\N}$ je posloupnost reálných čísel a $\seq{b_n}_{n\in\N}$
    je nerostoucí posloupnost nezáporných čísel. Jestliže je některá z následujících
    podmínek splněna, pak je řada $\rada{a_nb_n}$ konvergentní.
    \begin{itemize}
        \item[(A)] $\ds \rada{a_n}$ je konvergentní.
        \item[(D)] $\ds \limninf b_n = 0$ a $\ds \rada{a_n}$ má omezené částečné
            součty, tedy:
            $$\exists K > 0 \; \fa m \in \N: |s_m| = \left|\sum_{i=1}^m a_i\right|
            < K.$$
    \end{itemize}
\end{theorem}

\begin{proof}
    \leavevmode
    \begin{itemize}
        \item[(A)] Jelikož řada $\rada{a_n}$ konverguje, platí, dle Bolzano-Cauchyho
            podmínky pro konvergenci řad (Věta~\ref{th:bolzano_cauchy_rady}), že
            pro pevné $\e > 0$ existuje $n_0 \in \N$ \tz
            $$\fa m,n \in \N, m,n \geq n_0: |a_m + a_{m-1} + \dots + a_{n+1}| < \e.$$
            Nyní bychom chtěli z předpokladů věty dokázat Bolzano-Cauchyho podmínku 
            i pro řadu $\rada{a_nb_n}.$ Pro dané $\e$ zvolíme stejné $n_0$
            jako výše. Potom za použití Abelovy parciální sumace 
            (Lemma~\ref{lm:abelova_parcialni_sumace}) pro $t_k=\sum_{i=n+1}^ka_i$
            platí:
            \begin{align*}
                |a_mb_m + a_{m-1}b_{m-1} + \dots + a_{n+1}b_{n+1}|
                &\leq \max_{i=n+1, \dots, m}|t_i| \cdot b_{n+1} \\
                &\leq \max_{i=n+1, \dots, m}|t_i| \cdot b_1 
                    \tag{$\seq{b_n}$ je nerostoucí} \\
                &\leq \e \cdot b_1.
                    \tag{Bolzano-Cauchyho podmínka pro $\seq{a_n}$}
            \end{align*}
            
        \item[(D)] 
            Dle definice limity pro dané $\e> 0$ existuje $n_0 \in \N$ \tz
            $$\fa n \in \N,
            n\geq n_0: |b_n| < \e.$$ Potom znovu za použití Abelovy parciální sumace 
            (Lemma~\ref{lm:abelova_parcialni_sumace}) pro $t_k=\sum_{i=n+1}^ka_i$
            a $m,n\geq n_0$ platí:
            \begin{align*}
                |a_mb_m + a_{m-1}b_{m-1} + \dots + a_{n+1}b_{n+1}|
                &\leq \max_{i=n+1, \dots, m}|t_i| \cdot b_{n+1} \\
                &= \max_{i=n+1, \dots, m}|s_i -s_n| \cdot b_{n+1} \\
                &\leq \max_{i=n+1, \dots, m}(|s_i| + |s_n|) \cdot b_{n+1}
                    \tag{trojúhelníková nerovnost, Věta~\ref{th:triangleineq}} \\
                &\leq 2K\e.
                    \tag{$\seq{a_n}$ má omezené částečné součty}
            \end{align*}
    \end{itemize}
\end{proof}

\begin{remark}
    Řada $\rada{a_nb_n}$ konverguje za podmínky (A), i pokud neplatí,
    že $\fa n \in \N: b_n \geq 0.$ \Necht je posloupnost $\seq{b_n}$
    nerostoucí a nechť $\limninf b_n = b < 0.$ Potom:
    $$\rada{a_nb_n} = \rada{a_n(b_n-b)} + \rada{a_nb}.$$
    Jelikož $\rada{a_n}$ konverguje a $\fa n \in \N: (b_n - b) \geq 0,$ 
    řada $\rada{a_n(b_n-b)}$ konverguje dle podmínky (A). Řada $\rada{a_nb}$
    konverguje dle linearity konvergence řad, a konečně řada $\rada{a_nb_n}$
    konverguje, jelikož se rovná součtu dvou konvergentních řad.
\end{remark}

\begin{example}
    Určete, zda-li následující řady konvergují:
    \begin{multicols}{3}
        \begin{enumerate}[(i)]
            \item $\ds \rada{\frac{(-1)^n}{n}},$
            \item $\ds \rada{\frac{\sin (n)}{\log(n+1)}},$
            \item $\ds \rada{\frac{\cos(n)}{\sqrt{n}}\arctan(n)}.$
        \end{enumerate}
    \end{multicols}

    Řešení:
    \begin{enumerate}[(i)]
        \item Posloupnost $\seq{(-1)^n}_{n\in\N}$ má omezené částečné součty a 
            posloupnost $\seq{\frac{1}{n}}_{n\in\N}$ je nerostoucí. Dle
            Abel-Dirichletova kritéria (D) řada konverguje.

        \item Podobně, posloupnost $\seq{\sin(n)}_{n\in\N}$ má omezené částečné 
            součty (zatím bez důkazu) a posloupnost 
            $\seq{\frac{1}{\log(n+1)}}_{n\in\N}$ je nerostoucí. Dle 
            Abel-Dirichletova kritéria (D) řada konverguje.

        \item Posloupnost $\seq{\cos(n)}_{n\in\N}$ má omezené částečné 
            součty a posloupnost $\seq{\frac{1}{\sqrt{n}}}_{n\in\N}$ je nerostoucí.
            Řada $\rada{\frac{\cos(n)}{\sqrt{n}}}$ tedy konverguje a dle linearity
            konvergence řad konverguje i $\rada{-\frac{\cos(n)}{\sqrt{n}}}.$ Dále,
            posloupnost $\seq{-\arctan(n)}$ je nerostoucí a klesá k $-\frac{\pi}{2}.$
            Dle podmínky (A) Abel-Dirichletova kritéria řada
            $\rada{\frac{\cos(n)}{\sqrt{n}}\arctan(n)}$ konverguje.


    \end{enumerate}
\end{example}

\begin{theorem}[Leibnizovo kritérium]
    \Necht $\seq{a_n}_{n\in\N}$ je nerostoucí posloupnost nezáporných čísel.
    Pak
    $$\rada{(-1)^na_n} \text{ konverguje} \iff \limninf a_n = 0.$$
\end{theorem}

\begin{proof}
    Posloupnost $\seq{(-1)^n}$ má omezené součty a $\limninf a_n = 0.$ Potom
    dle Abel-Dirichletova kritéria (Věta~\ref{th:abel_dirichletovo_kriterium})
    je řada $\rada{(-1)^na_n}$ konvergentní.
\end{proof}

\subsection{Přerovnání řad}

\begin{definition}
    \Necht $\rada{a_n}$ je řada a $p: \N \rightarrow \N$ je bijekce. Řadu 
    $\rada{a_{p(n)}}$ nazýváme \newterm{přerovnáním řady} $\rada{a_n}.$
\end{definition}

\begin{theorem}[přerovnání absolutně konvergentní řady]
    Nechť $\rada{a_n}$ je absolutně konvergentní řada a $\rada{a_{p(n)}}$ je její
    přerovnání. Pak $\rada{a_{p(n)}}$ je absolutně konvergentní a má stejný součet
    jako $\rada{a_n}.$
\end{theorem}

\begin{proof}
    Jelikož $\rada{|a_n|}$ konverguje, pro $\e > 0$ existuje dle Bolzano-Cauchyho
    podmínky pro konvergenci řad (Věta~\ref{th:bolzano_cauchy_rady}) 
    $n_0 \in \N$ \tz
    $$\fa m,n \in \N, m,n \geq n_0: \left|\sum_{n+1}^m |a_i|\right| < \e.$$
    
    Nyní chceme ukázat, že Bolzano-Cauchyho podmínka platí i pro 
    přerovnanou řadu $\rada{|a_{p(n)}|}.$ Pro již dané $\e > 0$ a zjištěné 
    $n_0 \in \N$ zvolme
    $\widetilde{n_0} = \max_{i=1, \dots, n_0} p(i).$ Nechť $m,n\geq 
    \widetilde{n_0}.$ Pak:
    $$\left|\sum_{n+1}^m a_{p(i)}\right| 
    \leq \sum_{n+1}^m |a_{p(i)}| 
    \leq \sum_{n_0+1}^\infty |a_i| 
    \leq \e.$$
\end{proof}

\begin{theorem}[Riemann]
    Neabsolutně konvergentní řadu lze přerovnat k libovolnému součtu z $\Rstar.$
    Neboli: \Necht $\rada{a_n}$ konverguje, $\rada{|a_n|} = \infty$ a nechť 
    $A \in \Rstar.$ Pak existuje bijekce $p: \N \rightarrow \N$ \tz
    $$\rada{a_{p(n)}} = A.$$
\end{theorem}

\begin{proof}
    Bez důkazu.
\end{proof}

\subsection{Součin řad}

\begin{definition}
    \Necht $\rada{a_n}$ a $\rada{b_n}$ jsou řady. \newterm{Cauchyovským součinem}
    těchto řad nazveme řadu
    $$\sum_{k=2}^\infty\left(\sum_{i=1}^{k-1}a_{k-i}b_i\right).$$
\end{definition}

\begin{remark}
    Trochu jiný, ale ekvivalentní pohled: 
    Cauchyovským součinem řad $\rada{a_n}$ a $\rada{b_n}$ 
    je řada $\sum_{n=2}^\infty{c_n},$ kde
    $$c_n = \sum_{k=1}^{n-1}a_kb_{n-k}.$$
    Potom:
    \begin{align*}
        &c_1 \text{ není definováno} \\
        &c_2 = a_1b_1 \\
        &c_3 = a_1b_2 + a_2b_1 \\
        &c_4 = a_1b_3 + a_2b_2 + a_3b_1 \\
        &\text{atd.}
    \end{align*}
\end{remark}

\begin{theorem}[o součinu řad]
    Nechť $\rada{a_n}$ a $\rada{b_n}$ konvergují absolutně. Pak:
    $$\sum_{k=2}^\infty\left(\sum_{i=1}^{k-1}a_{k-i}b_i\right) 
    = \left(\rada{a_n}\right)\cdot \left(\rada{b_n}\right).$$
\end{theorem}

\begin{proof}
    Řady $\rada{a_n}$ a $\rada{b_n}$ absolutně konvergují, a proto existuje 
    $K \in \R$ \tz $\rada{|a_n|} < K$ a $\rada{|b_n|} < K.$
    Označme částečné součty řad $\rada{a_n},$ $\rada{b_n}$ a jejich součinu
    postupně $s_n$, $\sigma_n$ a $S_n.$ Platí
    $$\limninf s_n = s = \rada{a_n}, \; \limninf \sigma_n = \sigma = \rada{b_n}.$$
    Dále, z aritmetiky limit posloupností (Věta~\ref{th:voal}) platí, že k 
    pevnému $\e > 0$ existuje $n_1 \in \N$ \tz:
    $$\fa n \in \N, n \geq n_1: |s\sigma - s_n\sigma_n| < \e.$$
    Pro již dané $\e$ dále dle Bolzano-Cauchyho podmínky pro konvergenci řad
    (Věta~\ref{th:bolzano_cauchy_rady}) existuje $n_2 \in \N$ \tz:
    $$\sum_{n=n_2}^\infty |a_n| < \e, \sum_{n=n_2}^\infty |b_n| < \e.$$
    Označme $n_0 = \max(n_1,n_2).$ Potom:
    \begin{align*}
        |S_n - s_{n_0}\sigma_{n_0}| 
        &\leq \sum_{\substack{i,j=1\\i \geq n_0 \lor j\geq n_0}}^\infty |a_i||b_i| \\
        &\leq \left(\sum_{i=n_0}^\infty|a_i|\right)
              \cdot
              \left(\sum_{j=1}^\infty|b_j|\right) 
              +
              \left(\sum_{i=n_0}^\infty|b_i|\right)
              \cdot
              \left(\sum_{j=1}^\infty|a_j|\right) \\
        &\leq \e\cdot K + K \cdot \e,
    \end{align*}
    a tedy:
    $$|S_n - s\sigma| \leq (2K + 1) \e.$$

\end{proof}


\end{document}
