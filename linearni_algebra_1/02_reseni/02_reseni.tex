\section{Řešení soustav: Gaussova eliminační metoda}

Pro řešení soustav lineárních rovnic se používají tzv. elementární
ekvivalentní (řádkové) úpravy.

\begin{definition}
    \newterm{Elementární úpravou} matice $\mat{A}$ vznikne matice $\mat{A'}$ 
    (značíme $\mat{A} \sim \mat{A'}$), a to buď:
    \begin{enumerate}
        \item vynásobením $i$-tého řádku číslem $t \neq 0$:
            \begin{equation*}
                  a'_{kl} = \begin{cases}
                      t \cdot a_{il}, & \text{pokud $k=i$}.\\
                      a_{kl}, & \text{jinak}.
                        \end{cases}
            \end{equation*}
        \item přičtením $j$-tého řádku k $i$-tému:
            \begin{equation*}
                  a'_{kl} = \begin{cases}
                      a_{il} + a_{jl}, & \text{pokud $k=i$}.\\
                      a_{kl}, & \text{jinak}.
                        \end{cases}
            \end{equation*}
    \end{enumerate}
\end{definition}

\begin{remark}
    Z těchto dvou úprav se dají odvodit i úpravy:
    \begin{itemize}
        %\setcounter{enumi}{2}
        \item přičtení $t$-násobku $j$-tého řádku k $i$-tému,
        \item záměna $i$-tého a $j$-tého řádku.
    \end{itemize}
\end{remark}

Nyní ukážeme, že výše zmíněné elementární úpravy nemění množinu řešení 
soustavy.

\begin{theorem}
    Nechť $\mat{A}\vec{x}=\vec{b}$ a $\mat{A'}\vec{x}=\vec{b'}$ jsou
    soustavy takové, že $(\mat{A}|\vec{b}) \sim (\mat{A'}|\vec{b'})$. 
        Potom obě soustavy mají shodné množiny řešení.
\end{theorem}

\begin{proof}
    Stačí dokázat pro úpravy 1 a 2 a pro $i$-tý řádek (jelikož ostatní řádky
    se nemění).
    \begin{enumerate}
        \item Vynásobení $i$-tého řádku číslem $t \neq 0$. 
            
            Předpokládejme nejprve, že $\vx$ je 
            řešením $\mat{A}\vec{x}=\vec{b}$.
            \begin{align*}
                a'_{i1}x_1 + a'_{i2}x_2 + \dots + a'_{in}x_n &= 
                t \cdot a_{i1}x_1 + t \cdot a_{i2}x_2 + \dots 
                + t \cdot a_{in}x_n \tag{definice 1. úpravy} \\
                &= t \cdot (a_{i1}x_1 + a_{i2}x_2 + \dots + a_{in}x_n) 
                \tag{distributivita násobení ku sčítání zleva} \\
                &= t \cdot b_i \tag{předpoklad $\mat{A}\vec{x}=\vec{b}$ 
                $i$-té rovnice} \\
                    &= b'_i \tag{definice 1. úpravy}
            \end{align*}

            Obráceně, předpokládejme nyní, že $\vec{x}$ je řešením
            $\mat{A'}\vec{x} = \vec{b'}$.
            \begin{align*}
                a_{i1}x_1 + \dots + a_{in}x_n &= 
                \frac{t}{t}(a_{i1}x_1 + \dots + a_{in}x_n) \\
                &= \frac{1}{t}(ta_{i1}x_1 + \dots + ta_{in}x_n) \\
                &= \frac{1}{t}(a'_{i1}x_1 + \dots + a'_{in}x_n) \\
                &= \frac{1}{t}b'_i 
                = \frac{1}{t}tb_i 
                = b_i
            \end{align*}
        \item Přičtení $i$-tého řádku k $j$-tému.

            Předpokládejme, že $\vec{x}$ je řešením $\mat{A}\vec{x}=\vec{b}$.
            \begin{align*}
                a'_{i1}x_1 + a'_{i2}x_2 + \dots + a'_{in}x_n &= 
                (a_{i1} + a_{j1})x_1 + \dots + (a_{in} + a_{jn})x_n\\
                &= (a_{i1}x_1 + \dots + a_{in}x_n) + (a_{j1}x_1 + \dots +
                a_{jn}x_n) \\
                &= b_i + b_j = b'_i
            \end{align*}

            Předpokládejme, že $\vec{x}$ je řešením $\mat{A'}\vec{x} = \vec{b'}$.
            \begin{align*}
                a_{i1}x_1 + \dots + a_{in}x_n &= 
                                              (a_{i1}x_1 + \dots +
                a_{in}x_n) + (a_{j1}x_1 + \dots + a_{jn}x_n) 
                - (a_{j1}x_1 + \dots + a_{jn}x_n) \\
                &= (a_{i1} + a_{j1})x_1 + \dots + (a_{in} + a_{jn})x_n -
                (a_{j1}x_1 + \dots + a_{jn}x_n) \\
                &= a'_{i1}x_1 + \dots + a'_{in}x_n - (a_{j1}x_1 + \dots +
                a_{jn}x_n) \\
                &= b'_i - b_j = (b_i + b_j) - b_j = b_i
            \end{align*}
    \end{enumerate}
\end{proof}

Postup řešení soustavy lineárních rovnic:
\begin{enumerate}
    \item Sestavíme rozšířenou matici soustavy.
    \item Tuto matici elementárními úpravami převedeme na odstupňovaný tvar.
    \item Pomocí zpětné substituce popíšeme všechna řešení.
\end{enumerate}

\begin{definition}
    Říkáme, že matice $\mat{A} \in \mathbb{R}^{m \times n}$ je v
    \newterm{odstupňovaném tvaru}, pokud nenulové řádky jsou \emph{ostře}
    uspořádány podle počtu počátečních nul a nulové řádky jsou 
    až za nenulovými.\footnote{Ostré uspořádání zajistí, že v matici nejsou 
    žádné dva nenulové řádky o stejném počtu počátečních nul.} Formálně: 
    $\exists r \in \{0; \dots; m\}$ takové, že:
    \begin{enumerate}
        \item označíme-li pro $i \in \{1; \dots; r\}$: 
            $$j(i) := min\{j|a_{ij} \neq 0\},$$ 
            tak platí: $j(1) < j(2) < \dots < j(r) \le n$
        \item $\forall i > r, \forall j: a_{ij} = 0$
    \end{enumerate}
    Prvkům $a_{i,j(i)}$ pro $i = 1, \dots, r$ se říká \newterm{pivoty}.
\end{definition}

\begin{algorithm}
    Algoritmus \newterm{Gaussovy eliminace} pro úpravu dané matice 
    $\mat{A} \in \mathbb{R}^{m \times n}$ na odstupňovaný tvar 
    elementárními řádkovými úpravami:
    \begin{enumerate}
        \item Setřídíme řádky vzestupně podle počátečních nul.
        \item Pokud mají dva nenulové řádky stejně počátečních nul, tj. $j(i) =
            j(i+1)$, potom k~$(i+1)$-tému řádku přičteme vhodný násobek
            $i$-tého řádku:
            $$-\frac{a_{i+1,j(i)}}{a_{i,j(i)}}$$
        \item Kroky 1 a 2 opakujeme, dokud některé řádky mají steně mnoho
            počátečních nul. 
        \item Matice $\mat{A}$ je v odstupňovaném tvaru.
    \end{enumerate}
\end{algorithm}

\begin{remark}[Složitost a konečnost Gaussovy eliminace]
    Algoritmus Gaussovy eliminace je konečný, jelikož v každém kroku vzroste počet nul o
    jednu. Nejvýše tedy můžeme dosáhnout $m \cdot n$ iterací. Celková složitost
    algoritmu je $\bigo{mn(m\log{m} + n)}$. 
    
    Tato složitost lze zlepšit, 
    pokud ušetříme třídění: pro $i=1,...,m$ hledáme první sloupec $j(i)$ takový,
    že $a_{i,j(i)} \neq 0$ nebo $a_{k,j(i)} \neq 0$ pro nějaké $k > i$. V prvním
    případně eliminujeme prvky pod $a_{i,j(i)}$; v druhém nejprve zaměníme
    $i$-tý a $j$-tý řádek a teprve potom eliminujeme. Složitost v tomto případě
    je $\bigo{n^2m}$.
        \end{remark}

\begin{observation}
    \label{obs:pivotvb}
    Nechť $(\mat{A'}|\vec{b'})$ je matice soustavy v odstupňovaném tvaru.
    Pokud poslední sloupec $\vec{b'}$ obsahuje pivot, potom soustava nemá
    řešení.
\end{observation}

\begin{definition}
    Pro matici soustavy $(\mat{A'}|\vec{b'})$ v odstupňovaném tvaru nazveme
    proměnné, jež odpovídají sloupcům s pivoty v $\mat{A'}$, \newterm{bázovými
    proměnnými}.\footnote{Bázové proměnné jsou tedy proměnné $x_{j(i)}$ pro
    $i=1,\dots,r$ (nenulové řádky).} Ostatní proměnné se nazývají
    \newterm{volné}.
\end{definition}

\begin{theorem}[Věta o jednoznačnosti řešení]
    \label{th:jednoznacnostreseni}
    Nechť $(\mat{A'}|\vec{b'})$ je rozšířená matice soustavy v odstupňovaném
    tvaru, kde sloupec $\vec{b'}$ neobsahuje pivot. Potom libovolné hodnoty
    volných proměnných lze doplnit jednoznačně hodnotami bázových proměnných
    na řešení celé soustavy $\mat{A'}\vec{x} = \vec{b'}$.
\end{theorem}

\begin{proof}
    Větu dokážeme indukcí pro $i=r,r-1,r-2,\dots,1$.

    Nechť $x_{j(i)}$ je $i$-tá bázová proměnná a hodnoty následujících
    bázových proměnných a všech volných proměnných jsou dány. Potom $i$-tá
    rovnice soustavy zní:
    \begin{equation}
        \label{eq:jednoznacnostreseni}
        0x_1 + 0x_2 + \dots + 0x_{j(i) -1} + a'_{i,j(i)}x_{j(i)} +
        a'_{i,j(i)+1}x_{j(i)+1} + \dots + a'_{in}x_n = b'_i
    \end{equation}
    Nechť $\alpha = a'_{i,j(i)+1}x_{j(i)+1} + \dots + a'_{in}x_n$. Hodnotu
    tohoto výrazu známe (viz předpoklady). Z~rovnice 
    \ref{eq:jednoznacnostreseni} se potom stává lineární rovnice s jednou
    neznámou a nenulovým koeficientem a ta má jednoznačné řešení:
    \begin{equation*}
        a'_{i,j(i)}x_{j(i)} + \alpha = b'_i
    \end{equation*}
\end{proof}

\begin{corollary}
    Každé řešení soustavy lze získat zpětnou substitucí.
\end{corollary}

\begin{proof}
    Nechť $\vec{x}=\rowvec{x_1,\dots,x_n}^\top$ je libovolné řešení. Vezmeme z
    $\vec{x}$ hodnoty volných proměnných a zpětnou substitucí dopočítáme
    bázové proměnné. Díky jednoznačnosti musíme dostat zpět $\vec{x}$.
\end{proof}

\begin{theorem}
    Pro každou matici $\mat{A}$ platí, že sloupce s pivoty libovolné
    matice v odstupňovaném tvaru, kterou lze z $\mat{A}$ získat
    elementárními úpravami, jsou určeny jednoznačně.
\end{theorem}

\begin{proof}
    Sporem. Předpokládejme, že $\mat{A'},\mat{A''} \sim \mat{A}$ mají pivoty
    v různých sloupcích. Sestrojme rozšířené matice soustav $(\mat{A}|0)$,
    $(\mat{A'}|0)$ a $(\mat{A''}|0)$. Nechť dále bez újmy na obecnosti $x_k$
    je proměnná, která je v $(\mat{A'}|0)$ bázická a zároveň v
    $(\mat{A''}|0)$ volná, a všechny následující proměnné mají v obou
    soustavách stejný charakter.

    Zafixujeme hodnoty volných proměnných $x_j$ pro $j > k$, ale potom $x_k$
    je určena jednoznačně v $(\mat{A'}|0)$ a zároveň může mít libovolnou
    hodnotu v $(\mat{A''}|0)$. Spor, jelikož obě soustavy mají mít stejné
    množiny řešení.
\end{proof}

\begin{definition}
    \newterm{Hodnost matice} $\mat{A}$ je rovna počtu pivotů v libovolné 
    matici $\mat{A'}$ v odstupňovaném tvaru, kterou lze z $\mat{A}$ získat
    elementárními úpravami. Značí se $\rank(\mat{A})$.
\end{definition}

\begin{theorem}
    Soustava $\mat{A}\vec{x} = \vec{b}$ má alespoň jedno řešení, právě když
    platí $\rank(\mat{A}) = \rank(\mat{A}|\vec{b})$.
\end{theorem}

\begin{proof}
    Nechť má soustava alespoň jedno řešení a nechť $(\mat{A}|\vec{b}) \sim
    (\mat{A'}|\vec{b'})$. Potom
    \begin{align*}
        \rank(\mat{A}|\vec{b}) &= 
            \rank(\mat{A'}|\vec{b'}) \tag{dle definice hodnosti} \\
            &= \rank(\mat{A'}) \tag{dle Pozorování \ref{obs:pivotvb}} \\
            &= \rank(\mat{A}) \tag{dle definice hodnosti}
    \end{align*}

    Opačně, nechť platí $\rank(\mat{A}) = \rank(\mat{A}|\vec{b})$. Potom
    \begin{align*}
        \rank(\mat{A'}|\vec{b'}) &=
            \rank(\mat{A}|\vec{b}) \tag{dle definice hodnosti} \\
            &= \rank(\mat{A}) \tag{předpoklad} \\
            &= \rank(\mat{A'}). \tag{dle definice hodnosti}
    \end{align*}
    Jelikož $\rank(\mat{A'}|\vec{b'}) = \rank(\mat{A'})$, soustava
    $(\mat{A'}|\vec{b'})$ nemá pivot v posledním sloupci. Díky Větě
    \ref{th:jednoznacnostreseni} má tedy alespoň jedno řešení.
\end{proof}

\begin{definition}
    \newterm{Homogenní soustava lineárních rovnic} je soustava tvaru 
        $\mat{A}\vec{x} = \mathbf{0}$.
\end{definition}

\begin{observation}
    \label{obs:resenihomogennisoustavy}
    Jsou-li $\vec{\bar{x}}$ a $\vec{x'}$ řešeními soustavy $\mat{A}\vec{x} =
    \vec{b}$, pak $\vec{\bar{x}} - \vec{x'}$ je řešením $\mat{A}\vec{x} =
    \mathbf{0}$.
\end{observation}

\begin{proof}
    Podívejme se na $i$-tý řádek soustavy $\mat{A}\vec{x} = \mathbf{0}$ po
    dosazení $\vec{\bar{x}} - \vec{x'}$ za $\vec{x}$:
    \begin{align*}
        a_{i1}(\overline{x}_1 - x'_1) + \dots + a_{in}(\bar{x}_n - x'_n) &= 
            (a_{i1}\bar{x}_1 + \dots + a_{in}\bar{x}_n) - (a_{i1}x'_1 + \dots
            + a_{in}x'_n) \\
            &= b_i - b_i = 0
    \end{align*}
\end{proof}

\begin{observation}
    Jsou-li $\vec{\bar{x}}$ řešením $\mat{A}\vec{x} = \vec{b}$ a $\vec{x'}$
    řešením $\mat{A}\vec{x} = \mathbf{0}$, pak $\vec{\bar{x}} + \vec{x'}$ je
    řešením $\mat{A}\vec{x} = \vec{b}$.
\end{observation}

\begin{proof}
    Viz důkaz Pozorování \ref{obs:resenihomogennisoustavy}.
\end{proof}

\begin{theorem}
    Nechť $\mat{A} \in \mathbb{R}^{m \times n}$ je matice hodnosti $r < n$.
    Pak existují řešení $\vec{h^1},\vec{h^2},\dots,\vec{h^{n-r}}$ soustavy 
    $\mat{A}\vec{x} = \mathbf{0}$ taková, že každé řešení homogenní 
    soustavy $\mat{A}\vec{x} = \mathbf{0}$ 
    lze vyjádřit ve tvaru $\vec{x} = p_1\vec{h^1} + p_2\vec{h^2} + \dots +
    p_{n-r}\vec{h^{n-r}}$, kde $p_1,\dots,p_{n-r}$ jsou vhodná reálná čísla.
\end{theorem}

\begin{proof}
    Nechť $\mat{A'} \sim \mat{A}$ je v odstupňovaném
    tvaru. Všechny proměnné lze vyjádřit pomocí volných proměnných
    (indukcí podobně jako ve Větě \ref{th:jednoznacnostreseni}). Tyto volné
    proměnné použijeme jako parametry $p_1,\dots,p_{n-r}$. 
    Řešení $\vec{h^1},\dots,\vec{h^{n-r}}$ jsou vektory koeficientů volných
    proměnných.

    Mějmě například matici
    \[ \mat{A} = \begin{pmatrix}
            1 &2 &3 &4\\
            0 &1 &2 &3\\ 
            0 &0 &0 &0\\
            0 &0 &0 &0\\
    \end{pmatrix}  \]
    Proměnné $x_1$ a $x_2$ jsou bázové; proměnné $x_3$ a $x_4$ jsou volné.
    Vyjádříme všechny proměnné pomocí volných proměnných:
    \begin{align*}
        x_4 &= x_4 \\
        x_3 &= x_3 \\
        x_2 &=-2x_3 - 3x_4 \\
        x_1 &= -2x_2 - 3x_3 - 4x_4 = 4x_3 + 6x_4 - 3x_3 - 4x_4 = x_3 + 2x_4
    \end{align*}

    Řešení $\vec{h^1}$, odpovídající volné proměnné $x_3$, a řešení
    $\vec{h^2}$, odpovídající volné proměnné $x_4$, potom vyjádříme pomocí
    koeficientů volných proměnných:
    \begin{align*}
        \vec{h^1} = \colvec{1\\-2\\1\\0}; \vec{h^2} = \colvec{2\\-3\\0\\1} 
    \end{align*}

\end{proof}

\begin{corollary}
    Každé řešení řešitelné soustavy $\mat{A}\vec{x} = \vec{b}$ lze vyjádřit
    ve tvaru:
    $$ \vec{x} = \vec{x^0} + p_1\vec{h}^1 + \dots + p_{n-r}\vec{h}^{n-r},$$
    kde $r = \rank(\mat{A})$, $\vec{h}^1,\dots,\vec{h}^{n-r}$ jsou řešeními homogenní
    soustavy $\mat{A}\vec{x} = \mathbf{0}$, $p_1,\dots,p_n$ jsou vhodná
    reálná čísla a $\vec{x}^0$ je libovolné řešení $\mat{A}\vec{x} =
    \vec{b}$.
\end{corollary}

\begin{proof}
    Nechť $\bar{\vec{x}}$ je libovolné řešení soustavy $\mat{A}\vec{x} =
    \vec{b}$. Potom $\vec{\bar{x}} - \vec{x}^0$ je řešení $\mat{A}\vec{x} =
    \mathbf{0}$ a lze vyjádřit jako
    $$(\bar{\vec{x}} - \vec{x}^0) = \sum_{i=1}^{n-r}{p_i\vec{h}^i}.$$
\end{proof}

\begin{definition}
    Matice v \newterm{redukovaném odstupňovaném tvaru} má všechny pivoty 
    rovny 1 a ve sloupcích nad pivoty pouze nuly.
\end{definition}
