\section{Vektorové prostory}

\begin{definition}
    Nechť $(k, +, \cdot)$ je těleso. Množinu $V$ spolu s binární operací $+$ na 
    $V$ a zobrazením $\cdot: k \times V \rightarrow V$ se nazývá 
    \newterm{vektorový prostor} $(V, +, \cdot)$ nad $k$, pokud platí následující
    axiomy:
    \begin{enumerate}
        \item[(SA)] $\forall \vu,\vv,\vw \in V: (\vu+\vv)+\vw = \vu+(\vv+\vw)$ 
            (sčítání je asociativní)
        \item[(SK)] $\forall \vu,\vv \in V: \vu+\vv = \vv+\vu$ (sčítání je 
            komutativní)
        \item[(S0)] $\exists \vzero \in V,  \forall \vu \in V: \vu+\vzero = \vu$ 
            (existence nulového prvku)
        \item[(SI)] $\forall \vu \in V, \exists -\vu \in V: \vu+(-\vu) = \vzero$ 
            (existence opačného prvku)
        \item[(NA)] $\forall a,b \in k, \forall \vu \in V a \cdot (b \cdot \vu) = 
            (a \cdot b)\cdot \vu$ (násobení vektoru je asociativní)
        \item[(N1)] $\forall \vn \in V: 1 \cdot \vn = \vn$, kde $1$ je jednotkový 
            prvek tělesa $k$ (invariance vektoru při násobení jednotkovým prvkem
            tělesa)
        \item[(D1)] $\forall a,b \in k, \forall \vu \in V: (a + b)\vu = a\vu + 
            b\vu$ (distributivita násobení vektoru vzhledem ke sčítání prvků 
            tělesa)
        \item[(D2)] $\forall a \in k, \forall \vu,\vv \in V: a(\vu+\vv) = a\vu 
            + a\vv$ (distributivita násobení vektoru vzhledem ke sčítání vektorů)
    \end{enumerate}
\end{definition}

\begin{remark}
    Ingredience vektorového prostoru:
    \begin{enumerate}[i.]
        \item Těleso $k$ s operacemi $+$ a $\cdot$. Jeho prvky se nazývají 
            skaláry.
        \item Prostor $V$ s operací $+$\footnote{Operace $+$ na $V$ je odlišná
            od operace $+$ na $k$. Obě se nicméně značí obvykle stejně.}. 
            Jeho prvky se nazývají vektory.
        \item Operace $\cdot$ ``mezi $k$ a $V$''.
    \end{enumerate}
\end{remark}

\begin{remark}[Příklady vektorových prostorů]
    \leavevmode
    \begin{enumerate}[i.]
        \item $V = \{0\}.$ Triviální vektorový prostor nad libovolným tělesem 
            $k$.
        \item $K^n.$ Aritmetický vektorový prostor dimenze $n$. 
            Vektory jsou uspořádané $n$-tice prvků z $K$. Operace $+$ a $\cdot$
            se provadějí po složkách. Z každého tělesa lze vybudovat vektorový 
            prostor téže velikosti $K^1$.
        \item Matice typu $m \times n$ nad $K$.
        \item Polynomy omezeného stupně.
        \item Spojité funkce, diferencovatelné funkce v $\mathbb{R}$.
        \item Systém podmožin nějaké množiny $X$ jako prostor nad $\mathbb{Z}_2$.
    \end{enumerate}
\end{remark}

\begin{proposition}
    Prvky $\vzero$ a $-\vec{a}$ jsou určeny jednoznačně.
\end{proposition}

\begin{proof}
    Důkaz je stejný jako pro skaláry.
\end{proof}

\begin{proposition}
    $\forall \vu \in V, \forall a \in K: 0 \cdot \vu = a \cdot \vzero = \vzero$
\end{proposition}

\begin{proof}
    $$0\vu = 0\vu + 0 = 0\vu + 0\vu - 0\vu = (0 + 0)\vu - 0\vu = 0\vu - 0\vu = 
    \vzero$$
    $$a\vzero = a\vzero + \vzero = a\vzero + a\vzero - a\vzero = a(\vzero + \vzero) 
    - a\vzero = a\vzero - a\vzero = \vzero$$
\end{proof}

\begin{proposition}
    Pokud $a\cdot\vu = \vzero$, potom $a = 0$ nebo $\vu = \vzero$.
\end{proposition}

\begin{proof}
    Sporem. Nechť $a \neq 0$ a $\vu \neq \vzero$.
    $$\vzero \neq \vu = 1 \cdot \vu = \inv{a} \cdot a \cdot \vu = 
    \inv{a} \cdot \vzero = \vzero$$
\end{proof}

\begin{definition}
    Nechť $(V, +, \cdot)$ je vektorový prostor nad tělesem $K$ a $U$ je 
    neprázdná podmnožinu $V$ taková, že:
    \begin{enumerate}[i.]
        \item $\forall \vu,\vv \in U: \vu + \vv \in U$
        \item $\forall \vu \in V, \forall a \in K: a\cdot \vu \in U.$
    \end{enumerate}
    Potom $(U, +, \cdot)$ nazýváme \newterm{podprostorem} $V$.
\end{definition}

\begin{remark}
    Množinu $U$ splňující výše uvedené podmínky nazýváme \newterm{uzavřenou} 
    na operace $+$ a $\cdot$.
\end{remark}

\begin{observation}
    Podprostor $(U, +, \cdot)$ vektorového prostoru $V$ je vektorový 
    prostor.
\end{observation}

\begin{proof}
    Je třeba ověřit všech 8 axiomů z definice vektorového prostoru. Existence
    nulového a opačného prvku plyne z uzavřenosti $U$ vůči operaci $\cdot$ 
    ($0 \cdot \va = \vzero, -1 \cdot \va = -\va$).
\end{proof}

\begin{remark}[Příklady podprostorů $\mathbb{R}^3$]
    \leavevmode
    \begin{itemize}
        \item
            rovina $\pi$ procházející počátkem
        \item přímka $p$ procházející počátkem
        \item bod $\{0\}$
    \end{itemize}
\end{remark}

\begin{proposition}
    Nechť $(U_i, i \in I)$ je systém podprostorů nějakého vektorového prostoru
    $V$. Potom průnik těchto podprostorů, t.j. $\cap_{i \in I} U_i$, je
    podprostorem $V$.
\end{proposition}

\begin{proof}
    Je třeba ukázat uzavřenost $W$ na $+$ a $\cdot$. 
    
    Označme $W \coloneqq \cap_{i \in I} U_i$. Potom:
    \begin{itemize}
        \item uzavřenost na $+$: $u,v \in W \Rightarrow \vu,\vv \in 
            \cap_{i \in I} U_i \Rightarrow \forall i \in I: \vu,\vv \in U_i 
            \Rightarrow \forall i \in I: \vu+\vv \in U_i \Rightarrow \vu+\vv \in 
            \cap_{i \in I} U_i = W$
        \item uzavřenost na $\cdot$: $a \in K, \vu \in W \Rightarrow \forall i
            \in I: \vu \in U_i \Rightarrow \forall i \in I: a \cdot \vu \in U_i
            \Rightarrow a \cdot \vu \in \cap_{i \in I} U_i = W$
    \end{itemize}
\end{proof}

\begin{definition}
    Nechť $V$ je vektorový prostopr nad $K$ a $X$ je podmnožina $V$. Potom 
    $\obal(X)$ značí \newterm{podprostor generovaný $X$} (či \newterm{lineární 
    obal} množiny X), což je průnik všech podprostorů $V$, které obsahují $X$. 
    Formálně: $$\obal(X) \coloneqq 
    \cap \{U | X \subseteq U, U \text{ podprostor } V\}.$$ 
\end{definition}

\begin{proposition}
    Nechť $V$ je vektorový prostor nad $K$ a $X \subseteq V$. Potom $\obal(X)$
    obsahuje všechny lineární kombinace vektorů z $X$, neboli $$\obal(X) = 
    \{\vu | \vu = \sum_{i=1}^{n}a_i \vec{x_i}, n \geq 0, 
    \forall i = 1, \dots, n: a_i \in K, \vec{x_i} \in X \}$$
\end{proposition}

\begin{proof}
    Definujme:
    $$W_1 \coloneqq \cap_{X \subseteq U \subseteq V} U$$
    $$W_2 \coloneqq \left\{ \sum_{i=1}^{n} a_i \vec{x_i}, a_i \in K, 
    \vec{x_i} \in X \right\}$$

    Nejprve ukážeme, že množina $W_2$ je podprostorem $V$, t.j. že je 
    uzavřená na $+$ a $\cdot$. 
    \begin{itemize}
        \item Uzavřenost na $+$. Nechť $\vu,\vv \in W_2$. Potom:
            $$ \vu = \sum_{i=1}^{k}a_i \vec{x_i}, a_i \in K, \vec{x_i} \in X$$
            $$ \vv = \sum_{i=1}^{l}{a'}_i \vec{{x'}_i}, {a'}_i \in K, 
            \vec{{x'}_i} \in X$$

            Označme $\{\vec{y_1}, \dots, \vec{y_n}\} = \{\vec{x_1}, \dots, 
            \vec{x_k}\} \cup \{\vec{{x'}_1}, \dots, \vec{{x'}_l}\}$. 
            Po přeznačení a doplnění koeficientů lze vyjádřit:
            $$ \vu = \sum_{i=1}^{n} b_i \vec{y_i}$$
            $$ \vv = \sum_{i =1}^{n} {b'}_i \vec{y_i}$$
            Potom
            $$ \vu + \vv = \sum_{i=1}^{n} b_i \vec{y_i} + \sum_{i=1}^{n} 
            {b'}_i \vec{y_i} = \sum_{i=1}^{n} (b_i + {b'}_i)\vec{y_i}$$
            a $\vu + \vv \in W_2$ z definice $W_2$.

        \item Uzavřenost na $\cdot$. Nechť $\vu \in W_2, c \in K$.
            $$c \cdot \vu = c \cdot \sum_{i=1}^{k} a_i \vec{x_i} = 
            \sum_{i = 1}^{k} \underbrace{(ca_i)}_{\in K} 
            \underbrace{\vec{x_i}}_{\in X} \in W_2$$
    \end{itemize}

    Nyní $W_1 \subseteq W_2$, protože $W_2$ lze vzít za nějaké $U_i$, přes
    které děláme průniky, jelikož obsahuje $X$ a je podprostorem $V$. Dále také
    $W_2 \subseteq W_1$, protože každé $U_i$ musí být uzavřené na $+$ a 
    $\cdot$, a tedy $\forall i: W_2 \subseteq U_i \implies W_2 \subseteq \cap U_i = W_1$. 
    Z tohoto plyne $W_1 = W_2$.
\end{proof}

\begin{definition}[Prostory určené maticí]
    Nechť $\mat{A}$ je matice typu $m \times n$ nad tělesem $K$.
    \begin{itemize}
        \item \newterm{Sloupcový prostor} $S(\mat{A})$ je podprostor 
            $\mathbb{K}^m$ generovaný sloupci matice $\mat{A}$. Formálně:
            $S(\mat{A}) = \{\vu \in \mathbb{K}^m, \vu = \mat{A}\vx,
            \vx \in \mathbb{K}^n \}.$
        \item \newterm{Řádkový prostor} $R(\mat{A})$ je podprostor 
            $\mathbb{K}^n$ generovaný řádky matice $\mat{A}$. Formálně:
            $R(\mat{A}) = \{\vv \in \mathbb{K}^n, \vv = \transmat{A}\vy, 
            \vy \in \mathbb{K}^m \}$.
        \item \newterm{Jádro matice} $Ker(\mat{A})$ je podprostor
            $\mathbb{K}^n$ tvořený všemi řešeními homogenní soustavy
            $\mat{A}\vec{x} = \vec{0}$.
    \end{itemize}
\end{definition}

\begin{observation}
    Elementární úpravy nemění $R(A)$ ani $Ker(A)$.
\end{observation}

\begin{observation}
    Nechť $\vx \in Ker(A)$ a $\vv \in R(A)$. Potom $\vv^{\top}\vx = 0$.
\end{observation}

\begin{proof}
    $\vv^{\top}\vx = (\mat{A}^{\top}\vy)^{\top}\vx = \vy^{\top}\mat{A}\vx = 
    \vy^{\top} \cdot \vzero = 0$
\end{proof}
