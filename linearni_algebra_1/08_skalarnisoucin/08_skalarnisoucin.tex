\section{Skalární součin}

\begin{definition}
    Nechť $V$ je vektorový prostor nad $\mathbb{C}$. Zobrazení, které
    dvojici vektorů $\vec{u}, \vec{v} \in V$ přiřadí 
    $\langle\vec{u}|\vec{v}\rangle \in \mathbb{C}$, se nazývá 
    \newterm{skalární součin}, pokud splňuje následující axiomy:
    \begin{itemize}
        \item[(N)] $\forall \vec{u} \in V: \langle \vec{u} | \vec{u}
            \rangle = 0 \iff \vec{u} = \vzero$
        \item[(L1)] $\forall \vec{u},\vec{v}, \vec{w} \in V:
            \langle \vec{u} + \vec{v} | \vec{w} \rangle = 
            \scp{\vec{u}}{\vec{w}} + \scp{\vec{v}}{\vec{w}}$
        \item[(L2)] $\forall \vec{u}, \vec{v} \in V, \forall a \in
            \mathbb{C}: \scp{a\cdot\vec{u}}{\vec{v}} = a\cdot
            \scp{\vec{u}}{\vec{v}}$
        \item[(KS)] $\forall \vec{u}, \vec{v} \in V: 
            \scp{\vec{u}}{\vec{v}} = \overline{\scp{\vec{v}}{\vec{u}}}$
        \item[(P)] $\forall \vec{u} \in V: \scp{\vec{u}}{\vec{u}} 
            \geq 0$ 
    \end{itemize}
\end{definition}

\begin{remark}[Poznámka k axiomu (P)]
    Jelikož $\forall \vec{u} \in V: \scp{\vec{u}}{\vec{u}} \geq 0$, 
    vyplývá tedy, že $\scp{\vec{u}}{\vec{u}} \in \mathbb{R}$.
\end{remark}

\begin{remark}[Příklady skalárních součinů]
    \leavevmode
    \begin{itemize}
        \item Skalární součin pro aritmetické vektorové prostory:
            \begin{itemize}
                \item $V = \mathbb{C}^n: \scp{\vec{u}}{\vec{v}} = 
                    \sum_{i=1}^n u_i \overline{v_i}$
                \item $V = \mathbb{R}^n: \scp{\vec{u}}{\vec{v}} = 
                    \sum_{i=1}^n u_i v_i$
            \end{itemize}
        \item Skalární součin na $\mathbb{R}^n$ definovaný pomocí
            regulární matice: $\scp{\vec{u}}{\vec{v}} = \vec{u}^\top
            \cdot \transmat{A} \cdot \mat{A} \cdot \vec{v}$
        \item Skalární součin na prostoru reálných spojitých funkcí
            integrovatelných na intervalu $(a;b)$: 
            $\scp{f(x)}{g(x)} \coloneqq \int_a^b f(x)g(x) \,dx$
    \end{itemize}
\end{remark}

\begin{observation}
    $\scp{\vx}{\vzero} = \scp{\vzero}{\vx} = 0$
\end{observation}

\begin{proof}
    $\scp{\vx}{\vzero} = \scp{\vx}{0 \cdot \vx} = 0 \cdot \scp{\vx}{\vx}
    = \scp{0\cdot \vx}{\vx} = \scp{\vzero}{\vx}$
\end{proof}

\begin{definition}
    \label{def:normaskalarnisoucin}
    Nechť $V$ je vektorový prostor se skalárním součinem, potom
    \newterm{norma} odvozená od skalárního součinu je zobrazení
    $\|\bullet\|: V \rightarrow \mathbb{R}$ dané předpisem:
    $$\|\vec{u}\| \coloneqq \sqrt{\scp{\vec{u}}{\vec{u}}}.$$
\end{definition}

\begin{remark}[Geometrická interpretace normy a skalárního součinu
    v $\mathbb{R}^n$]
    \leavevmode
    \begin{itemize}
        \item $\|\vec{u}\|$ určuje délku vektoru $\vec{u}$
        \item $\|\vec{u} - \vec{v}\|$ určuje vzdálenost vektorů
            $\vec{u}$ a $\vec{v}$
        \item $\scp{\vec{u}}{\vec{v}}$ určuje úhel mezi vektory
            $\vec{u}$ a $\vec{v}$
    \end{itemize}
\end{remark}

\begin{observation}
    Pro standardní skalární součin a jím určenou normu na 
    $\mathbb{R}^n$ platí: 
    $$ \scp{\vec{u}}{\vec{v}} = \|\vec{u}\|\cdot\|\vec{v}\|\cdot \cos 
    \varphi,$$
    kde $\varphi$ je úhel sevřený vektory $\vec{u}$ a $\vec{v}$.
\end{observation}

\begin{proof}
    Vektory $\vec{u}$, $\vec{v}$ a $\vec{u}-\vec{v}$ tvoří trojúhelník.
    Podle kosinové věty: $$\|\vec{u} - \vec{v}\|^2 = \|\vec{u}\|^2
    + \|\vec{v}\|^2 - 2 \cdot \|\vec{u}\| \cdot \|\vec{v}\| \cdot 
    \cos \varphi,$$
    tedy:
    $$
        \scp{\vec{u}-\vec{v}}{\vec{u}-\vec{v}} = \scp{\vec{u}}{\vec{u}} +
        \scp{\vec{v}}{\vec{v}} - 2 \cdot \|\vec{u}\| \cdot \|\vec{v}\| \cdot 
        \cos \varphi.$$

    Podle axiomů skalárního součinu ovšem také platí:
    \begin{align*}
        \scp{\vec{u}-\vec{v}}{\vec{u}-\vec{v}} &= \\
        &= \scp{\vec{u}}{\vec{u}-\vec{v}} - \scp{\vec{v}}{\vec{u}-\vec{v}} \\
        &= \scp{\vec{u} - \vec{v}}{\vec{u}} - \scp{\vec{u} - 
            \vec{v}}{\vec{v}} \\
        &= \scp{\vec{u}}{\vec{u}} - \scp{\vec{v}}{\vec{u}} - 
            \scp{\vec{u}}{\vec{v}} + \scp{\vec{v}}{\vec{v}} \\
        &= \scp{\vec{u}}{\vec{u}} - 2\cdot \scp{\vec{u}}{\vec{v}} 
            + \scp{\vec{v}}{\vec{v}}
    \end{align*}

    Odečteme-li tento výsledek od rovnice kosinové věty, dostáváme (po 
    úpravách):
    $$ \scp{\vec{u}}{\vec{v}} = \|\vec{u}\|\cdot\|\vec{v}\|\cdot \cos 
    \varphi $$
\end{proof}

\begin{theorem}[Cauchy-Schwarzova nerovnost]
    Nechť $V$ je vektorový prostor nad $\mathbb{C}$ se skalárním součinem
    a s normou určenou tímto součinem. Potom platí:
    $$ \forall u,v \in V: |\scp{\vec{u}}{\vec{v}}| \leq \|\vec{u}\| \cdot
    \|\vec{v}\|. $$
\end{theorem}

\begin{proof}
    Je-li $\vec{u} = \vzero$ nebo $\vec{v} = \vzero$, nerovnost platí. 
    
    Určitě $\forall a \in \mathbb{C}: \|\vec{u} + a\cdot\vec{v}\|^2 \geq 0$.
    Tedy:
    \begin{align*}
        0 \leq \|\vec{u} + a\vec{v}\|^2 &= \\
        &= \scp{\vec{u} + a\vec{v}}{\vec{u} + a\vec{v}} \\
        &= \scp{\vec{u}}{\vec{u} + a\vec{v}} + \scp{a\vec{v}}{\vec{u} 
            + a\vec{v}} \\
        &= \overline{\scp{\vu + a\vv}{\vu}} + a
            \overline{\scp{\vu + a \vv}{\vv}} \\
        &= \scp{\vu}{\vu} + \overline{a \scp{\vv}{\vu}} + 
            a \overline{\scp{\vu}{\vv}} + a  \overline{a}
            \scp{\vv}{\vv} \\
        &= \overline{a}\scp{\vec{u}}{\vec{v}} + \scp{\vec{u}}{\vec{u}} +
            a\scp{\vec{v}}{\vec{u}} + a\overline{a}\scp{\vec{v}}{\vec{v}}
    \end{align*}

    Dosadíme 
    $a \coloneqq -\frac{\scp{\vec{u}}{\vec{v}}}{\scp{\vec{v}}{\vec{v}}}$,
    čímž se zbavíme prvního a posledního členu. Zbývá:
    \begin{align*}
        0 &\leq \scp{\vec{u}}{\vec{u}} 
        -\frac{\scp{\vec{u}}{\vec{v}}\scp{\vec{v}}{\vec{u}}}
        {\scp{\vec{v}}{\vec{v}}} \\
        \scp{\vec{u}}{\vec{v}}\scp{\vec{v}}{\vec{u}} &\leq
        \scp{\vec{u}}{\vec{u}}\scp{\vec{v}}{\vec{v}}
    \end{align*}

    Upravíme levou stranu nerovnosti:
    $$\scp{\vec{u}}{\vec{v}}\scp{\vec{v}}{\vec{u}} = 
        \scp{\vec{u}}{\vec{v}}\overline{\scp{\vec{u}}{\vec{v}}} =
        |\scp{\vec{u}}{\vec{v}}|^2,$$
    a tedy:
    \begin{align*}
        |\scp{\vec{u}}{\vec{v}}|^2 &\leq
        \scp{\vec{u}}{\vec{u}}\scp{\vec{v}}{\vec{v}} \\
        |\scp{\vec{u}}{\vec{v}}| &\leq \|\vec{u}\|\|\vec{v}\|
    \end{align*}
\end{proof}

\begin{corollary}[Vztah mezi aritmetickým a kvadratickým průměrem]
    Nechť $u_i \in \mathbb{R}$. Potom:
    $$\frac{1}{n}\sum_{i=1}^n u_i \leq \sqrt{\frac{1}{n}\sum_{i=1}^{n} u_i^2}$$
\end{corollary}

\begin{proof}
    Položme 
    \begin{align*} 
        \vec{v} &= \rowvec{1,1,\dots,1}^\top \\
        \vec{u} &= (\text{seřazená čísla})^\top.
    \end{align*}
    Potom 
    $$\scp{\vec{u}}{\vec{v}} = \sum_{i=1}^n u_i, 
    \|\vec{u}\| = \sqrt{\sum_{i=1}^n u_i^2}, \text{ a }
    \|\vec{v}\| = \sqrt{n}.$$
\end{proof}

\begin{corollary}[Trojúhelníková nerovnost]
    Norma odvozená od skalárního součinu splňuje trojúhelníkovou nerovnost:
    $$\|\vec{u} + \vec{v}\| \leq \|\vec{u}\| + \|\vec{v}\|$$
\end{corollary}

\begin{proof}
    \begin{align*}
        \|\vec{u} + \vec{v}\| &= \sqrt{\scp{\vec{u} + \vec{v}}{\vec{u} + 
            \vec{v}}} \\
            &= \sqrt{\scp{\vu}{\vu + \vv} + \scp{\vv}{\vu + \vv}} \\
            &= \sqrt{\scp{\vu}{\vu} + \scp{\vu}{\vv} + \scp{\vv}{\vu} 
                + \scp{\vv}{\vv}} \\
            &= \sqrt{\scp{\vu}{\vu} + \scp{\vu}{\vv} + 
                \overline{\scp{\vu}{\vv}} + \scp{\vv}{\vv}} \\
            &\leq \sqrt{\scp{\vu}{\vu} + 2|\scp{\vu}{\vv}| + \scp{\vv}{\vv}}
                \tag{V $\mathbb{C}$: $a + \overline{a} \leq 2 \cdot |a|$} \\
            &\leq \sqrt{\scp{\vu}{\vu} + 2\|\vu\|\|\vv\| + \scp{\vv}{\vv}} \\
            &= \sqrt{\|\vu\|^2 + 2\|\vu\|\|\vv\| + \|\vv\|^2} \\
            &= \sqrt{\left(\|\vu\| + \|\vv\|\right)^2} = \|\vu\| + \|\vv\|
    \end{align*}
\end{proof}

\begin{definition}
    Obecně \newterm{norma} na vektorovém prostoru $V$ je zobrazení 
    $\|\bullet\|: V \rightarrow \mathbb{R}$, které splňuje následující
    podmínky:
    \begin{enumerate}[i.]
        \item $\|\vec{0}\| = 0$
        \item $\|\vu\| \geq 0$
        \item $\|a \cdot \vu\| = a \cdot \|\vu\|$
        \item $\|\vu + \vv\| \leq \|\vu\| + \|\vv\|$
    \end{enumerate}
\end{definition}

\begin{remark}[Příklady norm]
    \leavevmode
    \begin{itemize}
        \item Norma odvozená od skalárního součinu (viz definici 
            \ref{def:normaskalarnisoucin}).
        \item $L_p$ norma, definovaná: $$ \|\vu\|_p = \sqrt[p]{\sum_{i=1}^n
            |u_i|^p}.$$
            Obvyklá Eukleidovská norma ($\sqrt{\sum_{i=1}^n |u_i|^2}$) je tedy
            speciální případ $L_p$ normy s $p = 2$.
    \end{itemize}
\end{remark}
