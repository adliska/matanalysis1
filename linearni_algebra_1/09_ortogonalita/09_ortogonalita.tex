\section{Ortogonalita}

\begin{definition}
    Vektory $\vu$ a $\vv$ z prostoru se skalárním součinem se nazývají vzájemně
    \newterm{kolmé}, pokud platí: $$\scp{\vu}{\vv} = 0.$$ 
    Značíme $\vu \perp \vv$.
\end{definition}

\begin{observation}
    Každý systém vzájemně kolmých netriviálních vektorů je lineárně nezávislý.
\end{observation}

\begin{proof}
    Sporem. Nechť $\vec{u_0} = \sum_{i=1}^n a_i\vec{u_i}$. Potom:
    $$ 0 \neq \scp{\vec{u_0}}{\vec{u_0}} 
        = \scp{\vec{u_0}}{\sum_{i=1}^n a_i\vec{u_i}}
        = \sum_{i=1}^n\overline{a_i}\underbrace{\scp{\vec{u_0}}{\vec{u_i}}}_{0 \text{ pro } \forall i} = 0$$
\end{proof}

\begin{observation}
    Nechť $V$ je vektorový prostor se skalárním součinem. Potom:
    $$\forall \vv \in V: \vv \perp \vzero.$$
\end{observation}

\begin{definition}
    Nechť $V$ je vektorový prostor se skalárním součinem a $Z$ jeho báze 
    taková, že:
    \begin{itemize}
        \item $\forall \vec{v} \in Z: \|\vec{v}\| = 1$
        \item $\forall \vv \neq \vec{v'} \in Z: \vv \perp \vec{v'}$
    \end{itemize}
    Potom se $Z$ nazývá \newterm{ortonormální báze} prostoru $V$.
\end{definition}

\begin{proposition}
    Nechť $Z = \{\vec{v_1}, \dots, \vec{v_n}\}$ je ortonomální báze prostoru 
    $V$. Potom $\forall \vu \in V$ platí:
    $$\vu = \scp{\vu}{\vec{v_1}}\vvone + \scp{\vu}{\vec{v_2}}\vvtwo + 
        \dots + \scp{\vu}{\vec{v_n}}\vvn$$
\end{proposition}

\begin{proof}
    $\vu = \sum_{i=1}^n a_i\vec{v_i}$, a tedy: 
    $[\vu]_Z = \rowvec{a_1,\dots,a_n}^\top$. Potom:
    $$\scp{\vu}{\vec{v_1}} = \scp{\sum_{i=1}^n a_i\vec{v_i}}{\vec{v_1}}
        = \sum_{i=1}^n a_i \scp{\vec{v_i}}{\vec{v_1}} = a_i,$$
    neboť:
    $$ \scp{\vec{v_i}}{\vec{v_1}} = \begin{cases}
            0 \text{ pro } i \neq 1, \\
            1 \text{ pro } i = 1
        \end{cases}$$
\end{proof}

\begin{definition}
    Nechť $W$ je prostor se skalárním součinem, $V$ je podprostor $W$, a
    $Z = \{\vvone, \dots, \vvn\}$ je nějaká ortonormální báze $V$.

    Zobrazení $p_Z: W \rightarrow V$ definované předpisem:
    $$p_Z(\vu) = \sum_{i=1}^n \scp{\vu}{\vvi}\vvi$$
    se nazývá \newterm{ortogonální projekce} prostoru $W$ na $V$.
\end{definition}

\begin{observation}
    Ortogonální projekce je lineární zobrazení.
\end{observation}

\begin{lemma}
    \label{lemma:gso}
    Nechť $p_Z$ je ortogonální projekce prostoru $W$ na $V$. Potom 
    $\forall \vu \in W$ platí:
    $$(\vu - p_Z(\vu)) \perp \vvi \text{ pro } \forall \vvi \in Z.$$
\end{lemma}

\begin{proof}
    $$\scp{\vu - p_Z(\vu)}{\vvi} = \scp{\vu - \sum_{j=1}^n \scp{\vu}{\vvj}
    \vvj}{\vvi} = \scp{\vu}{\vvi} - \sum_{i=n}^n \scp{\vu}{\vvj}
    \scp{\vvj}{\vvi} = \scp{\vu}{\vvi} - \scp{\vu}{\vvi} = 0$$
    jelikož
    $$\scp{\vvj}{\vvi} = \begin{cases}
        1 \text{ pro } i=j, \\
        0 \text{ pro } i \neq j
    \end{cases}$$
\end{proof}

\begin{observation}
    Projekce $p_Z(\vu)$ je nejbližší vektor k vektoru $\vu$, který leží v 
    prostoru $V$.
\end{observation}

\begin{definition}
    V zápise $\vu = \scp{\vu}{\vvone}\vvone + \dots + \scp{\vu}{\vvn}\vvn$
    se koeficienty $\scp{\vu}{\vvi}$ nazývají 
    \newterm{Fourierovy koeficienty}.
\end{definition}

\begin{algorithm}
    \newterm{Gram-Schmidtova ortonormalizace} je postup, který převede 
    libovolnou
    bázi $\{\vec{u_1}, \dots, \vec{u_n}\}$ na ortonormální bázi $\{\vvone,
    \dots, \vvn\}$.

    Postup. Pro $i$ od $1$ do $n$ opakuj:
    \begin{enumerate}
        \item $\vec{w_i} = \vec{u_i} - \sum_{j=1}^{i-1} \scp{\vui}{\vvj}\vvj$,
            neboli odečtení projekce na doposud spočtený lineární obal 
            $\obal(\vvone, \dots, \vec{v_{i-1}})$.
        \item $\vvi = \frac{1}{\|\vec{w_i}} \vec{w_i}$
    \end{enumerate}
\end{algorithm}

\begin{remark}
    Dokažme korektnost Gram-Schmidtovy ortonormalizace.
    \begin{enumerate}
        \item $\vvi \perp \vvj \forall j < i$, dle Lemmatu \ref{lemma:gso}
        \item $\|\vvi\| = \|\frac{1}{\|\vec{w_i}\|}\vec{w_i}\| 
            = \frac{1}{\|\vec{w_i}\|}\|\vec{w_i}\| = 1$
        \item Zůstáváme ve stejném prostoru, neboli: $$\obal(\vvone, \dots, 
        \vec{v_{i-1}}, \vui, \vec{u_{i+1}}, \dots, \vun) 
        = \obal(\vvone, \dots, \vvi, \vec{u_{i+1}}, \dots, \vun),$$ dle 
        Lemmatu \ref{lemma:ovymene}.
    \end{enumerate}
\end{remark}

\begin{definition}
    Nechť $V$ je množina vektorů ve vektorovém prostoru $W$ se skalárním
    součinem. \newterm{Ortogonální doplněk} $V$ je množina $V^\bot$ 
    definována:
    $$V^\bot \coloneqq \{\vu \in W, \forall \vv \in V: \vu \perp \vv \}$$
\end{definition}

\begin{remark}[Příklady ortogonálních doplňků v $\mathrm{R}^3$]
    Doplněk přímky je kolmá rovina. Doplňek roviny je kolmá přímka.
\end{remark}

\begin{observation}
    Pokud $U \subseteq V$, tak potom $U^\bot \supseteq V^\bot$.
\end{observation}

\begin{proof}
    $\vu \in V^\bot \iff \vu \perp \vv \text{ pro } \forall \vv \in V \implies
    \vu \perp \vv \text{ pro } \forall \vv \in U \iff \vu \in U^\bot$
\end{proof}

\begin{theorem}
    Nechť $V$ je podprostorem prostoru $W$ se skalárním součinem. Potom platí:
    \begin{enumerate}[i.]
        \item $V^\bot$ je podprostorem $W$
        \item $V \cap V^\bot = \{\vec{0}\}$
    \end{enumerate}
    Pokud je navíc $W$ konečné dimenze, tak platí:
    \begin{enumerate}[i.]
        \setcounter{enumi}{2}
        \item $\dm V + \dm V^\bot = \dm W$
        \item $(V^\bot)^\bot = V$
    \end{enumerate}
\end{theorem}

\begin{proof}
    \leavevmode
    \begin{enumerate}
        \item[i.] Je třeba ověřit uzavřenost na sčítání a násobení:
            \begin{itemize}
                \item $\scp{\vu + \vv}{\vw} = \scp{\vu}{\vw} + 
                    \scp{\vv}{\vw} = 0 + 0 = 0$
                \item $\scp{a\vu}{\vw} = a\cdot \scp{\vu}{\vw}
                    = a\cdot0 = 0$
            \end{itemize}
        \item[ii.] Pokud $\vu \in V \cap V^\bot$, potom $\scp{\vu}{\vu} = 0$
            a tedy $\vu = \vec{0}$.
        \item[iii.-iv.] Sestrojíme ortonormální bázi $V$ a doplníme ji
            na ortonormální bázi $W$. To, co jsme přidali, je ortonormální
            báze $V^\bot$.
    \end{enumerate}
\end{proof}
