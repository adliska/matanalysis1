\documentclass[12pt]{article}

\usepackage[czech]{babel}
\usepackage[utf8]{inputenc}
\usepackage[T1]{fontenc}
\usepackage{lmodern}

\usepackage[pdftex]{graphicx}
\usepackage{url}
\usepackage[round]{natbib}
\usepackage{subfigure}
\usepackage{enumerate}
\usepackage{multicol}

\usepackage{amsmath}
\usepackage{amsthm}
\usepackage{amsfonts}
\usepackage{mathtools}

\usepackage{soulutf8}
\usepackage{regexpatch}
% Patch for hyphens
\makeatletter
\regexpatchcmd*{\SOUL@eval}
{\cO-}
{\cA-}
{}{}
\makeatother

\usepackage{hyperref}
\hypersetup{
    colorlinks,
    citecolor=black,
    filecolor=black,
    linkcolor=black,
    urlcolor=black
}

\setlength{\oddsidemargin}{0.25in}
\setlength{\textwidth}{6.5in}
\setlength{\topmargin}{0in}
\setlength{\textheight}{8.5in}
%\setlength{\parskip}{0.05in}
%\setlength{\parindent}{0pt}

\newtheorem{theorem}{Věta}[section]
\newtheorem{lemma}[theorem]{Lemma}
\newtheorem{corollary}[theorem]{Důsledek}
\newtheorem{observation}[theorem]{Pozorování}
\newtheorem{proposition}[theorem]{Tvrzení}
\newtheorem{metaproposition}[theorem]{Metatvrzení}

\theoremstyle{definition}
\newtheorem{definition}[theorem]{Definice}
\newtheorem{algorithm}[theorem]{Algoritmus}
%\theoremstyle{remark}
\newtheorem{remark}[theorem]{Poznámka}

\renewcommand{\vec}[1]{\mathbf{#1}}
\newcommand*{\mat}[1]{\mathbf{#1}}
\newcommand*{\invmat}[1]{\mat{#1^{-1}}}
\newcommand*{\identmat}[1]{\mat{I_{#1}}}
\newcommand*{\transmat}[1]{\mat{#1}^{\top}}
\newcommand*\colvec[1]{\begin{pmatrix}#1\end{pmatrix}}
\newcommand*{\rowvec}[1]{\left( #1\right)}
\newcommand*{\mattype}[3]{\mat{#1} \in \mathbb{R}^{#2 \times #3}}
\newcommand*{\bigo}[1]{\mathcal{O}(#1)}
\newcommand*{\rank}{\mathop{\mathrm{rank}}}
\newcommand*{\obal}{\mathop{\mathrm{span}}}
\newcommand*{\dm}{\mathop{\mathrm{dim}}}
\newcommand*{\inv}[1]{#1^{-1}}
\newcommand*{\scp}[2]{\langle #1 | #2 \rangle}

\newcommand*{\newterm}[1]{\ul{#1}}
%\DeclareMathOperator{\rank}{rank}

\newcommand*{\va}{\vec{a}}
\newcommand*{\vn}{\vec{n}}
\newcommand*{\vu}{\vec{u}}
\newcommand*{\vv}{\vec{v}}
\newcommand*{\vw}{\vec{w}}
\newcommand*{\vx}{\vec{x}}
\newcommand*{\vy}{\vec{y}}
\newcommand*{\vz}{\vec{z}}
\newcommand*{\vzero}{\vec{0}}
\newcommand*{\vvone}{\vec{v_1}}
\newcommand*{\vuone}{\vec{u_1}}
\newcommand*{\vvtwo}{\vec{v_2}}
\newcommand*{\vvthree}{\vec{v_3}}
\newcommand*{\vvd}{\vec{v_d}}
\newcommand*{\vvi}{\vec{v_i}}
\newcommand*{\vvj}{\vec{v_j}}
\newcommand*{\vvk}{\vec{v_k}}
\newcommand*{\vvn}{\vec{v_n}}
\newcommand*{\vui}{\vec{u_i}}
\newcommand*{\vun}{\vec{u_n}}

\newcommand*{\mA}{\mat{A}}


\begin{document}

\title{Lineární Algebra I. \\ 
\vspace{1 mm} {\normalsize Zápisky z přednášek Jiřího 
    Fialy\footnote{\url{http://kam.mff.cuni.cz/~fiala}} \ na MFF UK, 
    zimní semestr, ak. rok 2007/2008}}

\author{Adam Li\v{s}ka\footnote{\url{http://www.adliska.com}}}

\date{\today}


\maketitle

\newpage
\tableofcontents

\section{Soustavy lineárních rovnic}

\begin{definition}
    \newterm{Reálný $n$\mbox{-}složkový vektor} $\vec{b}$ je uspořádaná $n$-tice
    reálných čísel:
    $$\vec{b} = \colvec{b_1\\b_2\\\vdots\\b_n}.$$
    Značíme $\vec{b} \in \mathbb{R}^n$. Všechny vektory jsou sloupcové. Pro
    řádkový zápis použijeme transposici:
    $$ \rowvec{b_1,b_2,\dots,b_n}^\top = \colvec{b_1\\b_2\\\vdots\\b_n};
    \colvec{b_1\\b_2\\\vdots\\b_n}^\top = \rowvec{b_1,b_2,\dots,b_n}$$
    Podobně uspořádaná $n$-tice neznámých hodnot $\vec{x} =
    \rowvec{x_1,\dots,x_n}^\top$ se nazývá \newterm{$n$-složkový vektor
    neznámých}.
\end{definition}

\begin{definition}
    \newterm{Reálná matice} $\mat{A}$ řádu $m \times n$ je soubor $m \cdot n$ reálných
    čísel uspořádaných do útvaru o $m$ řádcích a $n$ sloupcích:
    \[ \mat{A} = \begin{pmatrix}
            a_{11} &\dots  &a_{1n}\\
            \vdots &\ddots &\vdots\\
            a_{m1} &\dots  &a_{mn}
    \end{pmatrix}  \]
    Píšeme $\mat{A} \in \mathbb{R}^{m \times n}$, prvky matice značíme
    versálkami s dolními indexy:
    $$ a_{ij} = (\mat{A})_{ij}$$
    \newterm{Čtvercová matice} má stejný počet řádků a sloupců.
\end{definition}

\begin{definition}
    Nechť $\mat{A} \in \mathbb{R}^{m \times n}$,  $\vec{b} \in \mathbb{R}^m$
    a $x = \rowvec{x_1,\dots,x_n}^\top$ je vektor neznámých. Potom 
    \newterm{soustavou $m$ lineárních rovnic o $n$ neznámých} rozumíme zápis:
    $$ \mat{A}\vec{x} = \vec{b}.$$
    Tutéž soutavu lze zapsat v rozvinutém tvaru jako:
    \begin{alignat*}{5}
        a_{11}x_1 &+a_{12}x_2 &+\dots &+a_{1n}x_n &=b_1  \\
        a_{21}x_1 &+a_{22}x_2 &+\dots &+a_{2n}x_n &=b_2  \\
        \vdots \\
        a_{m1}x_1 &+a_{m2}x_2 &+\dots &+a_{mn}x_n &=b_m 
    \end{alignat*}
    Matice $\mat{A}$ se nazývá \newterm{matice soustavy}, vektor $\vec{b}$ se 
    nazývá \newterm{vektor pravých stran}.
    Matice $(\mat{A}|\vec{b})$ je \newterm{rozšířená matice soustavy}.
\end{definition}

\begin{definition}
    Reálný vektor $x \in \mathbb{R}^n$ se nazývá \newterm{řešením soustavy}
    $\mat{A}\vec{x} = \vec{b}$, pokud splňuje všech $m$ rovnic soustavy. To
    jest, $\forall i: a_{i1}x_1 + a_{i2}x_2 + \dots + a_{in}x_n = b_i$.
\end{definition}

\section{Řešení soustav: Gaussova eliminační metoda}

Pro řešení soustav lineárních rovnic se používají tzv. elementární
ekvivalentní (řádkové) úpravy.

\begin{definition}
    \newterm{Elementární úpravou} matice $\mat{A}$ vznikne matice $\mat{A'}$ 
    (značíme $\mat{A} \sim \mat{A'}$), a to buď:
    \begin{enumerate}
        \item vynásobením $i$-tého řádku číslem $t \neq 0$:
            \begin{equation*}
                  a'_{kl} = \begin{cases}
                      t \cdot a_{il}, & \text{pokud $k=i$}.\\
                      a_{kl}, & \text{jinak}.
                        \end{cases}
            \end{equation*}
        \item přičtením $j$-tého řádku k $i$-tému:
            \begin{equation*}
                  a'_{kl} = \begin{cases}
                      a_{il} + a_{jl}, & \text{pokud $k=i$}.\\
                      a_{kl}, & \text{jinak}.
                        \end{cases}
            \end{equation*}
    \end{enumerate}
\end{definition}

\begin{remark}
    Z těchto dvou úprav se dají odvodit i úpravy:
    \begin{itemize}
        %\setcounter{enumi}{2}
        \item přičtení $t$-násobku $j$-tého řádku k $i$-tému,
        \item záměna $i$-tého a $j$-tého řádku.
    \end{itemize}
\end{remark}

Nyní ukážeme, že výše zmíněné elementární úpravy nemění množinu řešení 
soustavy.

\begin{theorem}
    Nechť $\mat{A}\vec{x}=\vec{b}$ a $\mat{A'}\vec{x}=\vec{b'}$ jsou
    soustavy takové, že $(\mat{A}|\vec{b}) \sim (\mat{A'}|\vec{b'})$. 
        Potom obě soustavy mají shodné množiny řešení.
\end{theorem}

\begin{proof}
    Stačí dokázat pro úpravy 1 a 2 a pro $i$-tý řádek (jelikož ostatní řádky
    se nemění).
    \begin{enumerate}
        \item Vynásobení $i$-tého řádku číslem $t \neq 0$. 
            
            Předpokládejme nejprve, že $\vx$ je 
            řešením $\mat{A}\vec{x}=\vec{b}$.
            \begin{align*}
                a'_{i1}x_1 + a'_{i2}x_2 + \dots + a'_{in}x_n &= 
                t \cdot a_{i1}x_1 + t \cdot a_{i2}x_2 + \dots 
                + t \cdot a_{in}x_n \tag{definice 1. úpravy} \\
                &= t \cdot (a_{i1}x_1 + a_{i2}x_2 + \dots + a_{in}x_n) 
                \tag{distributivita násobení ku sčítání zleva} \\
                &= t \cdot b_i \tag{předpoklad $\mat{A}\vec{x}=\vec{b}$ 
                $i$-té rovnice} \\
                    &= b'_i \tag{definice 1. úpravy}
            \end{align*}

            Obráceně, předpokládejme nyní, že $\vec{x}$ je řešením
            $\mat{A'}\vec{x} = \vec{b'}$.
            \begin{align*}
                a_{i1}x_1 + \dots + a_{in}x_n &= 
                \frac{t}{t}(a_{i1}x_1 + \dots + a_{in}x_n) \\
                &= \frac{1}{t}(ta_{i1}x_1 + \dots + ta_{in}x_n) \\
                &= \frac{1}{t}(a'_{i1}x_1 + \dots + a'_{in}x_n) \\
                &= \frac{1}{t}b'_i 
                = \frac{1}{t}tb_i 
                = b_i
            \end{align*}
        \item Přičtení $i$-tého řádku k $j$-tému.

            Předpokládejme, že $\vec{x}$ je řešením $\mat{A}\vec{x}=\vec{b}$.
            \begin{align*}
                a'_{i1}x_1 + a'_{i2}x_2 + \dots + a'_{in}x_n &= 
                (a_{i1} + a_{j1})x_1 + \dots + (a_{in} + a_{jn})x_n\\
                &= (a_{i1}x_1 + \dots + a_{in}x_n) + (a_{j1}x_1 + \dots +
                a_{jn}x_n) \\
                &= b_i + b_j = b'_i
            \end{align*}

            Předpokládejme, že $\vec{x}$ je řešením $\mat{A'}\vec{x} = \vec{b'}$.
            \begin{align*}
                a_{i1}x_1 + \dots + a_{in}x_n &= 
                                              (a_{i1}x_1 + \dots +
                a_{in}x_n) + (a_{j1}x_1 + \dots + a_{jn}x_n) 
                - (a_{j1}x_1 + \dots + a_{jn}x_n) \\
                &= (a_{i1} + a_{j1})x_1 + \dots + (a_{in} + a_{jn})x_n -
                (a_{j1}x_1 + \dots + a_{jn}x_n) \\
                &= a'_{i1}x_1 + \dots + a'_{in}x_n - (a_{j1}x_1 + \dots +
                a_{jn}x_n) \\
                &= b'_i - b_j = (b_i + b_j) - b_j = b_i
            \end{align*}
    \end{enumerate}
\end{proof}

Postup řešení soustavy lineárních rovnic:
\begin{enumerate}
    \item Sestavíme rozšířenou matici soustavy.
    \item Tuto matici elementárními úpravami převedeme na odstupňovaný tvar.
    \item Pomocí zpětné substituce popíšeme všechna řešení.
\end{enumerate}

\begin{definition}
    Říkáme, že matice $\mat{A} \in \mathbb{R}^{m \times n}$ je v
    \newterm{odstupňovaném tvaru}, pokud nenulové řádky jsou \emph{ostře}
    uspořádány podle počtu počátečních nul a nulové řádky jsou 
    až za nenulovými.\footnote{Ostré uspořádání zajistí, že v matici nejsou 
    žádné dva nenulové řádky o stejném počtu počátečních nul.} Formálně: 
    $\exists r \in \{0; \dots; m\}$ takové, že:
    \begin{enumerate}
        \item označíme-li pro $i \in \{1; \dots; r\}$: 
            $$j(i) := min\{j|a_{ij} \neq 0\},$$ 
            tak platí: $j(1) < j(2) < \dots < j(r) \le n$
        \item $\forall i > r, \forall j: a_{ij} = 0$
    \end{enumerate}
    Prvkům $a_{i,j(i)}$ pro $i = 1, \dots, r$ se říká \newterm{pivoty}.
\end{definition}

\begin{algorithm}
    Algoritmus \newterm{Gaussovy eliminace} pro úpravu dané matice 
    $\mat{A} \in \mathbb{R}^{m \times n}$ na odstupňovaný tvar 
    elementárními řádkovými úpravami:
    \begin{enumerate}
        \item Setřídíme řádky vzestupně podle počátečních nul.
        \item Pokud mají dva nenulové řádky stejně počátečních nul, tj. $j(i) =
            j(i+1)$, potom k~$(i+1)$-tému řádku přičteme vhodný násobek
            $i$-tého řádku:
            $$-\frac{a_{i+1,j(i)}}{a_{i,j(i)}}$$
        \item Kroky 1 a 2 opakujeme, dokud některé řádky mají steně mnoho
            počátečních nul. 
        \item Matice $\mat{A}$ je v odstupňovaném tvaru.
    \end{enumerate}
\end{algorithm}

\begin{remark}[Složitost a konečnost Gaussovy eliminace]
    Algoritmus Gaussovy eliminace je konečný, jelikož v každém kroku vzroste počet nul o
    jednu. Nejvýše tedy můžeme dosáhnout $m \cdot n$ iterací. Celková složitost
    algoritmu je $\bigo{mn(m\log{m} + n)}$. 
    
    Tato složitost lze zlepšit, 
    pokud ušetříme třídění: pro $i=1,...,m$ hledáme první sloupec $j(i)$ takový,
    že $a_{i,j(i)} \neq 0$ nebo $a_{k,j(i)} \neq 0$ pro nějaké $k > i$. V prvním
    případně eliminujeme prvky pod $a_{i,j(i)}$; v druhém nejprve zaměníme
    $i$-tý a $j$-tý řádek a teprve potom eliminujeme. Složitost v tomto případě
    je $\bigo{n^2m}$.
        \end{remark}

\begin{observation}
    \label{obs:pivotvb}
    Nechť $(\mat{A'}|\vec{b'})$ je matice soustavy v odstupňovaném tvaru.
    Pokud poslední sloupec $\vec{b'}$ obsahuje pivot, potom soustava nemá
    řešení.
\end{observation}

\begin{definition}
    Pro matici soustavy $(\mat{A'}|\vec{b'})$ v odstupňovaném tvaru nazveme
    proměnné, jež odpovídají sloupcům s pivoty v $\mat{A'}$, \newterm{bázovými
    proměnnými}.\footnote{Bázové proměnné jsou tedy proměnné $x_{j(i)}$ pro
    $i=1,\dots,r$ (nenulové řádky).} Ostatní proměnné se nazývají
    \newterm{volné}.
\end{definition}

\begin{theorem}[Věta o jednoznačnosti řešení]
    \label{th:jednoznacnostreseni}
    Nechť $(\mat{A'}|\vec{b'})$ je rozšířená matice soustavy v odstupňovaném
    tvaru, kde sloupec $\vec{b'}$ neobsahuje pivot. Potom libovolné hodnoty
    volných proměnných lze doplnit jednoznačně hodnotami bázových proměnných
    na řešení celé soustavy $\mat{A'}\vec{x} = \vec{b'}$.
\end{theorem}

\begin{proof}
    Větu dokážeme indukcí pro $i=r,r-1,r-2,\dots,1$.

    Nechť $x_{j(i)}$ je $i$-tá bázová proměnná a hodnoty následujících
    bázových proměnných a všech volných proměnných jsou dány. Potom $i$-tá
    rovnice soustavy zní:
    \begin{equation}
        \label{eq:jednoznacnostreseni}
        0x_1 + 0x_2 + \dots + 0x_{j(i) -1} + a'_{i,j(i)}x_{j(i)} +
        a'_{i,j(i)+1}x_{j(i)+1} + \dots + a'_{in}x_n = b'_i
    \end{equation}
    Nechť $\alpha = a'_{i,j(i)+1}x_{j(i)+1} + \dots + a'_{in}x_n$. Hodnotu
    tohoto výrazu známe (viz předpoklady). Z~rovnice 
    \ref{eq:jednoznacnostreseni} se potom stává lineární rovnice s jednou
    neznámou a nenulovým koeficientem a ta má jednoznačné řešení:
    \begin{equation*}
        a'_{i,j(i)}x_{j(i)} + \alpha = b'_i
    \end{equation*}
\end{proof}

\begin{corollary}
    Každé řešení soustavy lze získat zpětnou substitucí.
\end{corollary}

\begin{proof}
    Nechť $\vec{x}=\rowvec{x_1,\dots,x_n}^\top$ je libovolné řešení. Vezmeme z
    $\vec{x}$ hodnoty volných proměnných a zpětnou substitucí dopočítáme
    bázové proměnné. Díky jednoznačnosti musíme dostat zpět $\vec{x}$.
\end{proof}

\begin{theorem}
    Pro každou matici $\mat{A}$ platí, že sloupce s pivoty libovolné
    matice v odstupňovaném tvaru, kterou lze z $\mat{A}$ získat
    elementárními úpravami, jsou určeny jednoznačně.
\end{theorem}

\begin{proof}
    Sporem. Předpokládejme, že $\mat{A'},\mat{A''} \sim \mat{A}$ mají pivoty
    v různých sloupcích. Sestrojme rozšířené matice soustav $(\mat{A}|0)$,
    $(\mat{A'}|0)$ a $(\mat{A''}|0)$. Nechť dále bez újmy na obecnosti $x_k$
    je proměnná, která je v $(\mat{A'}|0)$ bázická a zároveň v
    $(\mat{A''}|0)$ volná, a všechny následující proměnné mají v obou
    soustavách stejný charakter.

    Zafixujeme hodnoty volných proměnných $x_j$ pro $j > k$, ale potom $x_k$
    je určena jednoznačně v $(\mat{A'}|0)$ a zároveň může mít libovolnou
    hodnotu v $(\mat{A''}|0)$. Spor, jelikož obě soustavy mají mít stejné
    množiny řešení.
\end{proof}

\begin{definition}
    \newterm{Hodnost matice} $\mat{A}$ je rovna počtu pivotů v libovolné 
    matici $\mat{A'}$ v odstupňovaném tvaru, kterou lze z $\mat{A}$ získat
    elementárními úpravami. Značí se $\rank(\mat{A})$.
\end{definition}

\begin{theorem}
    Soustava $\mat{A}\vec{x} = \vec{b}$ má alespoň jedno řešení, právě když
    platí $\rank(\mat{A}) = \rank(\mat{A}|\vec{b})$.
\end{theorem}

\begin{proof}
    Nechť má soustava alespoň jedno řešení a nechť $(\mat{A}|\vec{b}) \sim
    (\mat{A'}|\vec{b'})$. Potom
    \begin{align*}
        \rank(\mat{A}|\vec{b}) &= 
            \rank(\mat{A'}|\vec{b'}) \tag{dle definice hodnosti} \\
            &= \rank(\mat{A'}) \tag{dle Pozorování \ref{obs:pivotvb}} \\
            &= \rank(\mat{A}) \tag{dle definice hodnosti}
    \end{align*}

    Opačně, nechť platí $\rank(\mat{A}) = \rank(\mat{A}|\vec{b})$. Potom
    \begin{align*}
        \rank(\mat{A'}|\vec{b'}) &=
            \rank(\mat{A}|\vec{b}) \tag{dle definice hodnosti} \\
            &= \rank(\mat{A}) \tag{předpoklad} \\
            &= \rank(\mat{A'}). \tag{dle definice hodnosti}
    \end{align*}
    Jelikož $\rank(\mat{A'}|\vec{b'}) = \rank(\mat{A'})$, soustava
    $(\mat{A'}|\vec{b'})$ nemá pivot v posledním sloupci. Díky Větě
    \ref{th:jednoznacnostreseni} má tedy alespoň jedno řešení.
\end{proof}

\begin{definition}
    \newterm{Homogenní soustava lineárních rovnic} je soustava tvaru 
        $\mat{A}\vec{x} = \mathbf{0}$.
\end{definition}

\begin{observation}
    \label{obs:resenihomogennisoustavy}
    Jsou-li $\vec{\bar{x}}$ a $\vec{x'}$ řešeními soustavy $\mat{A}\vec{x} =
    \vec{b}$, pak $\vec{\bar{x}} - \vec{x'}$ je řešením $\mat{A}\vec{x} =
    \mathbf{0}$.
\end{observation}

\begin{proof}
    Podívejme se na $i$-tý řádek soustavy $\mat{A}\vec{x} = \mathbf{0}$ po
    dosazení $\vec{\bar{x}} - \vec{x'}$ za $\vec{x}$:
    \begin{align*}
        a_{i1}(\overline{x}_1 - x'_1) + \dots + a_{in}(\bar{x}_n - x'_n) &= 
            (a_{i1}\bar{x}_1 + \dots + a_{in}\bar{x}_n) - (a_{i1}x'_1 + \dots
            + a_{in}x'_n) \\
            &= b_i - b_i = 0
    \end{align*}
\end{proof}

\begin{observation}
    Jsou-li $\vec{\bar{x}}$ řešením $\mat{A}\vec{x} = \vec{b}$ a $\vec{x'}$
    řešením $\mat{A}\vec{x} = \mathbf{0}$, pak $\vec{\bar{x}} + \vec{x'}$ je
    řešením $\mat{A}\vec{x} = \vec{b}$.
\end{observation}

\begin{proof}
    Viz důkaz Pozorování \ref{obs:resenihomogennisoustavy}.
\end{proof}

\begin{theorem}
    Nechť $\mat{A} \in \mathbb{R}^{m \times n}$ je matice hodnosti $r < n$.
    Pak existují řešení $\vec{h^1},\vec{h^2},\dots,\vec{h^{n-r}}$ soustavy 
    $\mat{A}\vec{x} = \mathbf{0}$ taková, že každé řešení homogenní 
    soustavy $\mat{A}\vec{x} = \mathbf{0}$ 
    lze vyjádřit ve tvaru $\vec{x} = p_1\vec{h^1} + p_2\vec{h^2} + \dots +
    p_{n-r}\vec{h^{n-r}}$, kde $p_1,\dots,p_{n-r}$ jsou vhodná reálná čísla.
\end{theorem}

\begin{proof}
    Nechť $\mat{A'} \sim \mat{A}$ je v odstupňovaném
    tvaru. Všechny proměnné lze vyjádřit pomocí volných proměnných
    (indukcí podobně jako ve Větě \ref{th:jednoznacnostreseni}). Tyto volné
    proměnné použijeme jako parametry $p_1,\dots,p_{n-r}$. 
    Řešení $\vec{h^1},\dots,\vec{h^{n-r}}$ jsou vektory koeficientů volných
    proměnných.

    Mějmě například matici
    \[ \mat{A} = \begin{pmatrix}
            1 &2 &3 &4\\
            0 &1 &2 &3\\ 
            0 &0 &0 &0\\
            0 &0 &0 &0\\
    \end{pmatrix}  \]
    Proměnné $x_1$ a $x_2$ jsou bázové; proměnné $x_3$ a $x_4$ jsou volné.
    Vyjádříme všechny proměnné pomocí volných proměnných:
    \begin{align*}
        x_4 &= x_4 \\
        x_3 &= x_3 \\
        x_2 &=-2x_3 - 3x_4 \\
        x_1 &= -2x_2 - 3x_3 - 4x_4 = 4x_3 + 6x_4 - 3x_3 - 4x_4 = x_3 + 2x_4
    \end{align*}

    Řešení $\vec{h^1}$, odpovídající volné proměnné $x_3$, a řešení
    $\vec{h^2}$, odpovídající volné proměnné $x_4$, potom vyjádříme pomocí
    koeficientů volných proměnných:
    \begin{align*}
        \vec{h^1} = \colvec{1\\-2\\1\\0}; \vec{h^2} = \colvec{2\\-3\\0\\1} 
    \end{align*}

\end{proof}

\begin{corollary}
    Každé řešení řešitelné soustavy $\mat{A}\vec{x} = \vec{b}$ lze vyjádřit
    ve tvaru:
    $$ \vec{x} = \vec{x^0} + p_1\vec{h}^1 + \dots + p_{n-r}\vec{h}^{n-r},$$
    kde $r = \rank(\mat{A})$, $\vec{h}^1,\dots,\vec{h}^{n-r}$ jsou řešeními homogenní
    soustavy $\mat{A}\vec{x} = \mathbf{0}$, $p_1,\dots,p_n$ jsou vhodná
    reálná čísla a $\vec{x}^0$ je libovolné řešení $\mat{A}\vec{x} =
    \vec{b}$.
\end{corollary}

\begin{proof}
    Nechť $\bar{\vec{x}}$ je libovolné řešení soustavy $\mat{A}\vec{x} =
    \vec{b}$. Potom $\vec{\bar{x}} - \vec{x}^0$ je řešení $\mat{A}\vec{x} =
    \mathbf{0}$ a lze vyjádřit jako
    $$(\bar{\vec{x}} - \vec{x}^0) = \sum_{i=1}^{n-r}{p_i\vec{h}^i}.$$
\end{proof}

\begin{definition}
    Matice v \newterm{redukovaném odstupňovaném tvaru} má všechny pivoty 
    rovny 1 a ve sloupcích nad pivoty pouze nuly.
\end{definition}

\section{Operace s maticemi, speciální typy matic}

\begin{definition}[Terminologie, značení základních matic]
    \leavevmode
    \begin{itemize}
        \item \newterm{Nulová matice} typu $m \times n$: $\mathbf{0}$, 
                $\forall{i,j} (\mathbf{0})_{ij} = 0$.
        \item \newterm{Jednotková matice} řádu $n$ je čtvercová matice 
            $\identmat{n}$ taková, že:
            \begin{equation*}
                (\identmat{n})_{i,j} = \begin{cases}
                      1, & \text{pokud $i=j$}.\\
                      0, & \text{jinak}.
                        \end{cases}
            \end{equation*}
            Například: $$\identmat{3} = \begin{pmatrix}
                            1 &0  &0\\
                            0 &1 &0\\
                            0 &0  &1
                    \end{pmatrix}$$
        \item \newterm{Hlavní diagonála} čtvercové matice $\mat{A}$ je tvořena 
            prvky $a_{i,i}$.
    \end{itemize}
\end{definition}

\begin{definition}
    \newterm{Transponovaná matice} k matici $\mat{A}$ typu $m \times n$
    je matice $\transmat{A}$ tupu $n \times m$ definovaná vztahem
    $\left(\transmat{A}\right)_{ij} = a_{ji}$. 
    Čtvercová matice se nazývá \newterm{symetrická}, pokud splňuje 
    $\transmat{A} =  \mat{A}$.
\end{definition}

\begin{definition}
    Pro matice $\mat{A}$ a $\mat{B}$ stejného typu definujeme 
    \newterm{součet matic} $\mat{A}+\mat{B}$ předpisem: 
    $$\left(\mat{A} + \mat{B} \right)_{ij} = a_{ij} + b_{ij}.$$
\end{definition}

\begin{definition}
    Pro $\alpha \in \mathbb{R}$ a matici $\mat{A}$ definujeme 
    \newterm{$\alpha$-násobek matice} $\mat{A}$ předpisem:
    $$\left(\alpha\mat{A}\right)_{ij} = \alpha a_{ij}.$$
    Nulový násobek libovolné matice je nulová matice.
\end{definition}

\begin{definition}
    Je-li $\mat{A}$ matice typu $m \times n$ a $\mat{B}$ matice typu $n
    \times p$, potom definujeme \newterm{součin matic} $\mat{A}$ a 
    $\mat{B}$ jako matici $\mat{A}\mat{B}$ typu $m \times p$, kde platí:
    $$\left(\mat{A}\mat{B}\right)_{ij} = \sum_{k=1}^n{a_{ik}b_{kj}}.$$
\end{definition}

\begin{remark}[Užití součinu]
    Maticový součin má mnohá využití:
        \begin{itemize}
        \item Zápis soustav lineárních rovnic: $\mat{A}\vec{x} = \vec{b}$.
        \item Elementární úpravy lze vyjádřit součinem (viz podrobněji 
            Poznámka \ref{remark:elementop} níže).
        \item Lineární zobrazení a výměna báze lze vyjádřit maticovým součinem
            (o tomto budeme mluvit podrobněji v kapitole \ref{ch:linzobr}).
    \end{itemize}
\end{remark}

\begin{proposition}
    Jsou-li výsledky operací definovány, pak platí:
    \begin{multicols}{2}
        \begin{enumerate}[i.]
            \item $(\mat{A}+\mat{B})+\mat{C} = \mat{A}+(\mat{B}+\mat{C})$
            \item $\mat{A} + \mat{0} = \mat{A}$
            \item $\mat{A} + \mat{B} = \mat{B} + \mat{A}$
            \item $\forall \mat{A} \exists ! \mat{B}: \mat{A} + \mat{B} = \mat{0}$
            \item $(\alpha\mat{A})^\top = \alpha\mat{A}^\top$
            \item $(\mat{A}^\top)^\top = \mat{A}$
            \item $\alpha(\beta\mat{A}) = (\alpha\beta)\mat{A}$
            \item $(\alpha + \beta)\mat{A} = \alpha\mat{A} + \beta\mat{A}$
            \item $(\mat{A} + \mat{B})^\top = \mat{A}^\top + \mat{B}^\top$
        \end{enumerate}
    \end{multicols}
\end{proposition}

\begin{proposition}
    Maticový součin není komutativní.
\end{proposition}
\begin{proof}
    Pokud matice nejsou čtvercové, důkaz je jednoduchý: Nechť $\mat{A} \in 
    \mathbb{R}^{a \times b}$, $\mat{B} \in \mathbb{R}^{b \times c}$, a $a
    \neq c$. Potom součin $\mat{A}\mat{B}$ je definován, kdežto součin 
    $\mat{B}\mat{A}$ definován není.

    Pro dvě čtvercové matice $\mat{A}$ a $\mat{B}$ řádu $2$ je
    $$ \mat{A}\mat{B} = \begin{pmatrix}
        a_{11}b_{11} + a_{12}b_{21} &a_{11}b_{12} + a_{12}b_{22} \\
        a_{21}b_{11} + a_{22}b_{21} &a_{21}b_{12} + a_{22}b_{22}
    \end{pmatrix}$$
    a
    $$ \mat{B}\mat{A} = \begin{pmatrix}
        b_{11}a_{11} + b_{12}a_{21} &b_{11}a_{12} + b_{12}a_{22} \\
        b_{21}a_{11} + b_{22}a_{21} &b_{21}a_{12} + b_{22}a_{22}
    \end{pmatrix}.$$
    Je zřejmé, že obecně $\mat{A}\mat{B} \neq \mat{B}\mat{A}$.
\end{proof}


\begin{proposition}
    Matice $\mat{A}\mat{A}^\top$ je vždy symetrická.
\end{proposition}
\begin{proof}
    Nechť $\mat{A} \in \mathbb{R}^{m \times n}$. Dle definice součinu
    je matice $\mat{A}\mat{A}^\top$ čtvercová matice typu $m \times m$.
    Dále: $$(\mat{A}\mat{A}^\top)_{ij} = \sum_{k=1}^{n}{(\mat{A})_{ik}
    (\mat{A}^\top)_{kj}} = \sum_{k=1}^{n}{(\mat{A}^\top)_{ki}(\mat{A})_{jk}} =
    (\mat{A}\mat{A}^\top)_{ji}$$.
\end{proof}

\begin{proposition}
    Nechť $\mat{A} \in \mathbb{R}^{m \times n}$. Potom platí:
    $\mat{I_m}\mat{A} = \mat{A} = \mat{A}\mat{I_n}$.
\end{proposition}

\begin{proposition}
    Pro násobení blokových matic platí:
    $$\begin{pmatrix} \mat{A}_1 &\mat{A}_2 \end{pmatrix} \cdot \begin{pmatrix}
    \mat{B}_1 \\ \mat{B}_2 \end{pmatrix} 
    = \mat{A}_1\mat{B}_1 + \mat{A}_2\mat{B}_2$$
    $$\begin{pmatrix} 
        \mat{A}_1 &\mat{A}_2 \\ \mat{A}_3 &\mat{A}_4
    \end{pmatrix}
    \cdot 
    \begin{pmatrix} 
        \mat{B}_1 &\mat{B}_2 \\ \mat{B}_3 &\mat{B}_4 
    \end{pmatrix} =
    \left(\begin{array}{c|c}
        \mat{A}_1\mat{B}_1 + \mat{A}_2\mat{B}_3 
        &\mat{A}_1\mat{B}_2 + \mat{A}_2\mat{B}_4 \\
        \mat{A}_3\mat{B}_1 + \mat{A}_4\mat{B}_3
        &\mat{A}_3\mat{B}_2 + \mat{A}_4\mat{B}_4
    \end{array}\right)$$
\end{proposition}

\begin{proposition}
    Pro matice $\mat{A}$, $\mat{B}$ a $\mat{C}$ platí:
    \begin{multicols}{2}
        \begin{enumerate}[i.]
            \item $(\mat{A}\mat{B})^\top = \mat{B}^\top\mat{A}^\top$
            \item $(\mat{A}\mat{B})\mat{C} = \mat{A}(\mat{B}\mat{C})$
            \item $(\mat{A} + \mat{B})\mat{C} = \mat{A}\mat{C} + \mat{B}\mat{C}$
            \item $\mat{A}(\mat{B} + \mat{C}) = \mat{A}\mat{B} + \mat{A}\mat{C}$
        \end{enumerate}
    \end{multicols}
    za předpokladu, že všechny výsledky operací jsou definovány.
\end{proposition}

\begin{proof}
    \leavevmode
    \begin{enumerate}
        \item
            $\mattype{A}{m}{n}$, $\mattype{B}{n}{p}$.
            $$((\mat{A}\mat{B})^\top)_{ij} 
                    = (\mat{A}\mat{B})_{ji} %\\
                    = \sum_{k=1}^{n}{a_{jk}b_{ki}} 
                    =
                \sum_{k=1}^{n}{(\mat{A}^\top)_{kj}(\mat{B}^\top)_{ik}} 
                    = 
                \sum_{k=1}^{n}{(\mat{B}^\top)_{ik}(\mat{A}^\top)_{kj}} %\\
                    = (\mat{B}^\top\mat{A}^\top)_{ij}$$
        \item
            $\mattype{A}{m}{n}$, $\mattype{B}{n}{p}$, $\mattype{C}{p}{q}$.
            \begin{multline*}
                ((\mat{A}\mat{B})\mat{C})_{ij} 
                    =\sum_{k=1}^{p}(\mat{A}\mat{B})_{ik}\mat{C}_{kj} %\\
                    =\sum_{k=1}^{p} \left(\sum_{l=1}^{n}
            a_{il}b_{lk}\right) \cdot c_{kj} 
                    =\sum_{k=1}^{p}{\sum_{l=1}^{n}{a_{il}b_{lk}c_{kj}}} \\
                    =\sum_{l=1}^{n}{\sum_{k=1}^{p}{a_{il}b_{lk}c_{kj}}} 
                    =\sum_{l=1}^{n}a_{il}\cdot
                        \left(\sum_{k=1}^{p}{b_{lk}c_{kj}}\right) %\\
                        =\sum_{l=1}^{n}a_{il}\cdot(\mat{B}\mat{C})_{lj} 
                        =(\mat{A}(\mat{B}\mat{C}))_{ij}
                    \end{multline*}
        \item
            $\mattype{A}{m}{n}$, $\mattype{B}{n}{p}$, $\mattype{C}{p}{q}$.
            \begin{multline*}
                ((\mat{A}+\mat{B})\mat{C})_{ij} 
                = \sum_{k=1}^{n}(\mat{A}+\mat{B})_{ik}\cdot c_{kj} 
                = \sum_{k=1}^{n} (a_{ik} + b_{ik}) \cdot c_{kj} 
                = \sum_{k=1}^{n} (a_{ik}c_{kj} + b_{ik}c_{kj}) \\
                = \sum_{k=1}^{n} a_{ik}c_{kj} + \sum_{k=1}^{n} b_{ik}c_{kj}
                = (\mat{A}\mat{C})_{ij} + (\mat{B}\mat{C})_{ij} 
                = (\mat{A}\mat{C} + \mat{B}\mat{C})_{ij}
            \end{multline*}
        \item
            Obdobně.
    \end{enumerate}
\end{proof}

\begin{remark}[Složitost násobení matic]
    Nechť $\mattype{A}{m}{n}$, $\mattype{B}{n}{p}$, $\mattype{C}{p}{q}$.
    Potom:
    \begin{itemize}
        \item Na součin $\mat{A}\mat{B}$ je třeba $mnp$ operací,
            $(\mat{A}\mat{B})\mat{C}$ $mpq$ operací. Celkem: $mp(n+q)$.
        \item Na součin $\mat{B}\mat{C}$ je třeba $npq$ operací,
            $\mat{A}(\mat{B}\mat{C})$ $mnq$ operací. Celkem: $nq(p+m)$.
    \end{itemize}
    Pokud $q << m,n,p$, tak poté $nq(m+p) << mp(n+q)$.
\end{remark}

\begin{remark}[Elementární úpravy jako součin matic]
    \label{remark:elementop}
    Nechť $\mat{B}$ vznikne z $\mattype{A}{m}{n}$ vynásobením $i$-tého řádku
    číslem $t$. Potom platí $\mat{B} = \mat{E}\mat{A}$, kde:
    \begin{equation*}
        (\mat{E})_{kj} = \begin{cases}
            t &\text{pokud $k = j$ a $k = i$;} \\
            1 &\text{pokud $k = j$ a $k \neq i$;} \\
            0 &\text{pokud $k \neq j$.}
        \end{cases}
    \end{equation*}
    Jedná se tedy o jednotkovou matici řádu $m$, kde na $i$-tém řádku
    byla jednička na diagonále nahrazena číslem $t$.

    Nechť naopak $\mat{B}$ vznikne z $\mattype{A}{m}{n}$ přičtením $j$-tého
    řádku k $i$-tému. Potom platí: $\mat{B} = \mat{E}\mat{A}$, kde:
    \begin{equation*}
        (\mat{E})_{kl} = \begin{cases}
            1 &\text{pokud $k = i$ a $l = j$;} \\
            1 &\text{pokud $k = l$;} \\
            0 &\text{jinak.}
        \end{cases}
    \end{equation*}

    Mějme například matici $\mattype{A}{3}{4}$. Přičtení třetího řádku k
            prvnímu lze vyjádřit jako maticový součin 
            $\mat{E}\mat{A}$, kde:
            $$ \mat{E} = \begin{pmatrix}
                1 &0 &1 \\
                0 &1 &0 \\
                0 &0 &1
            \end{pmatrix}.$$
\end{remark}

\begin{definition}
    Nechť $\mat{A}$ je čtvercová matice řádu $n$. Pokud existuje matice
    $\mat{B}$ taková, že $\mat{A}\mat{B} = \mat{I_n}$, potom se $\mat{B}$
    nazývá \newterm{inverzní maticí} k matici $\mat{A}$ a značí se
    $\mat{A^{-1}}$.

    Pokud k matici $\mat{A}$ existuje inverzní matice, potom se $\mat{A}$
    nazývá \newterm{regulární}, v opačném případě se nazývá
    \newterm{singulární}.
\end{definition}

\begin{theorem}
    Pro čtvercovou matici $\mat{A}$ řádu $n$ jsou následující podmínky
    ekvivalentní:
    \begin{enumerate}
        \item Matice $\mat{A}$ je regulární (t.j. $\exists \mat{B}: \mat{A}\mat{B}
            = \mat{I_n}$).
        \item $\rank(\mat{A}) = n$.
        \item Matici $\mat{A}$ lze řádkovými elementárními úpravami převést na
            $\mat{I_n}$.
        \item Homogenní soustava $\mat{A}\vec{x} = \mat{0}$ má pouze
            triviální řešení $\vec{x} = \vec{0}$.
    \end{enumerate}
\end{theorem}


\begin{proof} 
    \leavevmode
    \begin{itemize}
        \item $3 \implies 2$
            
            $\mat{I_n}$ je v odstupňovaném tvaru s $n$ pivoty, tudíž
            $\rank(\mat{I_n}) = 0$.

        \item $2 \implies 3$

            Převedeme $\mat{A}$ na odstupňovaný tvar. Jelikož matice
            $\mat{A}$ je čtvercová a $\rank(\mat{A}) = n$, máme v každém
            sloupci pivot. S pomocí posledního pivotu zeliminuji vše, co je
            nad ním, přejdu k předposlednímu pivotu, opakuji, atp.

        \item $2 \iff 4$

            $4 \iff \text{matice $\mat{A}$ po převedení do odstupňovaného
            tvaru nemá žádné volné proměnné} \iff 2$

        \item $2 \implies 1$

            Označme $\vec{e^1}, \vec{e^2}, \dots, \vec{e^n}$ sloupce
            jednotkové matice. Vyřešíme $n$ soustav tvaru $\mat{A}\vec{x^i} = 
            \vec{e^i}$ pro $i = 1, \dots, n$. 
            Jelikož $\rank(\mat{A}) = n$,
            každá soustava má právě jedno řešení $\vec{x^i}$. Nechť:
            $$ \mat{B} = \begin{pmatrix}
                \vdots &\vdots &\dots &\vdots \\
                \vec{x^1} &\vec{x^2} &\dots &\vec{x^n}\\
                \vdots &\vdots &\dots &\vdots 
            \end{pmatrix}$$
            Potom $\mat{A}\mat{B} = \mat{I_n}$.
            
        \item $1 \implies 2$

            Sporem. Nechť matice $\mat{A}$ je regulární a $\rank(\mat{A})
            < n$. Potom existuje řádek $i$, který lze vynulovat přičtením vhodné
            kombinace ostatních řádků. Matice $(\mat{A}|\vec{e^i})$ je
            rozšířená matice soustavy $\mat{A}\vec{x^i} = 
            \vec{e^i}$. Elementárními úpravami lze $i$-tý řádek této matice
            upravit na $$\begin{pmatrix} 0 &0 &\dots &0 &| &1\end{pmatrix}.$$
            Tato soustava ovšem nemá řešení a tudíž inverzní matice
            $\mat{B}$ neexistuje a matice $\mat{A}$ není regulární.
    \end{itemize}
\end{proof}

\begin{corollary}
    Pokud inverzní matice existuje, je určena jednoznačně.
\end{corollary}

\begin{proposition}
    Pro regulární matici $\mat{A}$ platí:
    $$\mat{A^{-1}}\mat{A} = \mat{A}\mat{A^{-1}} = \mat{I_n}$$
\end{proposition}

\begin{proof}
    Nejprve sporem ukážeme, že $\invmat{A}$ je regulární. Nechť
    $\invmat{A}\vec{x} = \vec{0}$ má netriviální řešení $\vec{x}$. Potom:
    $$\vec{x} = \identmat{n}\vec{x} = \mat{A}\invmat{A}\vec{x} =
    \mat{A}\vec{0} = \vec{0}.$$

    Tudíž existuje inverzní matice k matici $\invmat{A}$; označme ji
    $\invmat{(\invmat{A})}$. Dále platí: $$\invmat{A}\mat{A} =
    \invmat{A}\mat{A}\identmat{n} =
    \invmat{A}\mat{A}\invmat{A}\invmat{(\invmat{A})} =
    \invmat{A}\identmat{n}\invmat{(\invmat{A})} =
    \invmat{A}\invmat{(\invmat{A})} = \identmat{n}.$$
\end{proof}

\begin{remark}[Výpočet inverzní matice k dané čtvercové matici $\mat{A}$]
    \leavevmode
    \begin{enumerate}
        \item Sestavíme $\left(\mat{A}|\identmat{n}\right)$ a elementárními
            řádkovými úpravami ji převedeme na tvar
            $\left(\identmat{n}|\mat{B}\right)$. Pokud tento postup selže,
            matice $\mat{A}$ je singulární.
        \item Označme $\mat{E_1}, \mat{E_2}, \dots, \mat{E_k}$ matice, které
            byly použity v těchto řádkových úpravách:
            $$\mat{E_k}\mat{E_{k-1}}\dots\mat{E_1} 
            \left(\mat{A}|\identmat{n}\right) =
            \left(\identmat{n}|\mat{B}\right).$$ 
            Potom $\mat{E_k}\mat{E_{k-1}}\dots\mat{E_1}\mat{A} =
            \identmat{n}$ a
            $\mat{E_k}\mat{E_{k-1}}\dots\mat{E_1}\identmat{n} = \mat{B}.$ Z
            toho vyplývá, že $\mat{B}\mat{A} = \identmat{n}$ a $\mat{B} =
            \invmat{A}.$
    \end{enumerate}
\end{remark}

\begin{observation}
    Je-li matice $\mat{R}$ regulární, potom: $$ \mat{A} = \mat{B} \iff
    \mat{A}\mat{R} = \mat{B}\mat{R}.$$
\end{observation}

\begin{proof}
    \leavevmode
    \begin{itemize}
        \item $\implies$: Triviální.

        \item $\impliedby$: $\mat{A} = \mat{A}\identmat{n} = 
            \mat{A}\mat{R}\invmat{R} =
            \mat{B}\mat{R}\invmat{R} = \mat{B}\identmat{n} = \mat{B}$
    \end{itemize}
\end{proof}

\begin{proposition}
    Pro regulární matice $\mat{A}$ a $\mat{B}$ stejného řádu platí:
    \begin{multicols}{2}
        \begin{itemize}
            \item $\invmat{(\invmat{A})} = \mat{A}$
            \item $\mat{A}\mat{B}$ je regulární.
            \item $\invmat{(\mat{A}\mat{B})} = \invmat{B}\invmat{A}$
            \item $\invmat{(\transmat{A})} = \transmat{(\invmat{A})}$
        \end{itemize}
    \end{multicols}
\end{proposition}

\begin{remark}[Řešení maticových rovnic]
    \leavevmode
    \begin{itemize}
        \item $\mat{A} + \mat{X} = \mat{B} \implies \mat{X} = \mat{B} -
            \mat{A} = \mat{B} + (-1)\mat{A}$
        \item $\alpha\mat{X} = \mat{B} \implies \mat{X} =
            \frac{1}{\alpha}\mat{B}$
        \item $\mat{A}\mat{X} = \mat{B} \implies \mat{X} =
            \invmat{A}\mat{B}$, je-li $\mat{A}$ regulární.
        \item $\mat{X}\mat{A} = \mat{B} \implies \mat{X} =
            \mat{B}\invmat{A}$, je-li $\mat{A}$ regulární.
    \end{itemize}
\end{remark}

\section{Algebraická tělesa}

\begin{definition}
    \newterm{Binární operací} na množině $K$ rozumíme zorazení
    $K \times K \to K$.
\end{definition}

\begin{remark}
    Příklady binárních operací na $\mathbb{N}$:
    \begin{enumerate}[i.]
        \item $\varphi(a,b) = a + b$
        \item $\varphi(a,b) = \min\{a;b\}$
        \item $\varphi(a,b) = a+ 18$
    \end{enumerate}
    Naopak zobrazení $\varphi(a,b) = a + b - 18$ binární operací na
    $\mathbb{N}$ není, jelikož výsledek této operace může být záporný.

    Binární operaci můžeme definovat i tabulkou, např. na množině $\{0;1\}$:
    \begin{center}
        \begin{tabular}{c | c c}
            $a \backslash b$ & 0 & 1 \\
            \hline
            0 & 0 & 1 \\
            1 & 0 & 0
        \end{tabular}
    \end{center}
    Toto zobrazení odpovídá logické funkci $\varphi(a,b) = \lnot a \land b$.

    Binární zobrazení lze definovat např. i na reálných polynomech jedné
    proměnné: $\varphi(p(x), q(x)) = (p+q)(x)$.
\end{remark}

\begin{definition}
    Nechť $K$ je množina a $+, \cdot$ jsou dvě binární operace na $K$.
    Strukturu $(K, +, \cdot)$ nazveme \newterm{tělesem}, pokud jsou splněny
    následující axiomy:
    \begin{enumerate}
        \item[(SA)] $\forall a,b,c \in K: (a+b)+c = a+(b+c)$ (sčítání je
            asociativní)
        \item[(SK)] $\forall a,b \in K: a+b = b+a$ (sčítání je komutativní)
        \item[(S0)] $\exists 0 \in K: \forall a \in K : a + 0 = a$
            (existence nulového prvku)
        \item[(SI)] $\forall a \in K: \exists {-a} \in K: a + (-a) = 0$
            (existence opačného prvku)
        \item[(NA)] $\forall a,b,c \in K: (a \cdot b) \cdot c = a \cdot (b
            \cdot c)$
        \item[(NK)] $\forall a,b \in K: a \cdot b = b \cdot a$
        \item[(N1)] $\exists 1 \in K: \forall a \in K : a \cdot 1 = a$
        \item[(NI)] $\forall a \in K \setminus \{0\}: \exists a^{-1} \in K: a \cdot
            a^{-1} = 1$ (existence inverzního prvku)
        \item[(D)] $\forall a,b,c \in K: a \cdot (b + c) = a \cdot b + a
            \cdot c$ (distributivita násobení vůči sčítání)
        \item[(01)] $0 \neq 1$ (axiom netriviality)
    \end{enumerate}
\end{definition}

\begin{remark}[Značení]
    \leavevmode
    \begin{multicols}{3}
        \begin{itemize}
            \item $ab \coloneqq a \cdot b$
            \item $a - b \coloneqq a + (-b)$
            \item $\frac{a}{b} \coloneqq a \cdot b^{-1}$
        \end{itemize}
    \end{multicols}
\end{remark}

\begin{remark}[Příklady těles] 
    \leavevmode
    \begin{multicols}{2}
        \begin{enumerate}[i.]
            \item $(\mathbb{Q}, +, \cdot)$
            \item $(\mathbb{R}, +, \cdot)$
            \item $(\mathbb{C}, +, \cdot)$
            \item $(\mathbb{Z}_p,+, \cdot)$, t.j. počítání ve zbytkových
                třídách modulo prvočíslo
            \item $(\text{racionální lomené funkce}, +, \cdot)$
        \end{enumerate}
    \end{multicols}
\end{remark}

\begin{remark}[Příklady struktur, jež nejsou tělesa]
    \leavevmode
    \begin{multicols}{3}
        \begin{enumerate}[i.]
            \item $(\mathbb{N}, +, \cdot)$
            \item $(\mathbb{Z}, +, \cdot)$
            \item $(\mathbb{Z}_4, +, \cdot)$, $(\mathbb{Z}_6, +, 
                \cdot)$\footnote{Důvody viz Tvrzení \ref{proposition:zp}}
            \item $(\mathbb{R}^n, +, \cdot)$
            \item $(\text{polynomy}, +, \cdot)$
        \end{enumerate}
    \end{multicols}
\end{remark}

\begin{metaproposition}
    Všechny definice a věty o řešení soustav a počítání s maticemi nad
    $\mathbb{R}$ platí také pro soustavy a matice nad libovolným tělesem,
    jelikož z $\mathbb{R}$ jsme využili pouze vlastnosti dané axiomy tělesa.
\end{metaproposition}

\begin{observation}
    Prvky $0, {-a}, 1, a^{-1}$ jsou vždy určeny jednoznačně.
\end{observation}

\begin{proof}
    \leavevmode
        Jednoznačnost $0$ dokážeme sporem. Nechť $0, \bar{0}$ jsou dva 
            různé neutrální prvky $0 \neq \bar{0}$. Potom:
            \begin{align*}
                0 &= 0 + \bar{0} \tag{S0} \\
                  &= \bar{0} + 0 \tag{SK} \\
                  &= \bar{0} \tag{S0}
            \end{align*}
        Jednoznačnost ${-a}$ dokážeme taktéž sporem. 
        Nechť $-a$ a $\overline{-a}$ jsou opačné prvky k $a$ a 
            $-a \neq \overline{-a}$:
            \begin{align*}
                -a &= -a + 0 \tag{S0} \\
                   &= -a + (a + (\overline{-a})) \tag{SI} \\
                   &= \overline{-a} + (a + (-a)) \tag {SK, SA} \\
                   &= \overline{-a} + 0 = \overline{-a} \tag{SI, S0}
            \end{align*}
        Zbytek analogicky.
\end{proof}

\begin{observation}
    Nechť $K$ je algebraické těleso. Potom:
        \begin{enumerate}[i.]
            \item $\forall a \in K: -(-a) = a$
            \item $\forall b \in K \setminus \{0\}: (b^{-1})^{-1} = b$
        \end{enumerate}
\end{observation}

\begin{proof}
    \leavevmode
    \begin{enumerate}[i.]
        \item $-(-a) = -(-a) + 0 = -(-a) + (a + (-a)) = a + (-a + -(-a)) 
            = a + 0 = a$
        \item $(b^{-1})^{-1} = (b^{-1})^{-1} \cdot 1 = (b^{-1})^{-1} \cdot 
            (b \cdot b^{-1}) = ((b^{-1})^{-1} \cdot b^{-1}) \cdot b = 
            1 \cdot b = b$
    \end{enumerate}
\end{proof}

\begin{observation}
    Nechť K je algebraické těleso. Potom:
    \begin{enumerate}[i.]
        \item $\forall a \in K: a \cdot 0 = 0$
        \item $\forall a \in K \setminus \{0\}: a \cdot (-1) = -a$
    \end{enumerate}
\end{observation}
   
\begin{proof}
    \leavevmode
    \begin{enumerate}[i.]
        \item $a \cdot 0 = a \cdot 0 + 0 = a \cdot 0 + (a \cdot 0 - a \cdot 
                0) = (a \cdot 0 + a \cdot 0) - a \cdot 0 = a \cdot (0 + 0) 
                - a \cdot 0 = a \cdot 0 - a \cdot 0 = 0$
        \item $a \cdot (-1) = a\cdot(-1) + 0 = a\cdot(-1) + a - a = a \cdot
            (-1) + a \cdot 1 - a = a \cdot (-1 + 1) -a = a \cdot 0 - a = -a$
    \end{enumerate}
\end{proof}

\begin{observation}
    \label{obs:abeq0}
    Pokud $a \cdot b = 0$, potom buď $a = 0$ nebo $b = 0$.
\end{observation}

\begin{proof}
    Sporem. Nechť $a \neq 0$ a $b \neq 0$. Potom existují opačné prvky $a^{-1}$
    a $b^{-1}$. Dále: $0 = \inv{b} \cdot \inv{a} \cdot 0 = \inv{b} \cdot 
    \inv{a} \cdot a \cdot b = \inv{b} \cdot 1 \cdot b = \inv{b} \cdot b = 1$.
\end{proof}

\begin{observation}
    \leavevmode
    $\forall a, b, a', b', a' \neq 0: a + x = b$ a $a' \cdot x = b'$ mají 
    právě jedno řešení.
\end{observation}

\begin{proof}
    Sporem. Nechť $x_1$ a $x_2$ jsou dvě různá řešení rovnice $a + x = b$. 
    Potom: 
    $$ x_1 = x_1 + 0 = x_1 + a - a = (x_1 + a) - a = b - a = (a+x_2) - a = 
    x_2.$$

    Podobně, nechť $x_1$ a $x_2$ jsou dvě různá řešení rovnice $a' \cdot 
    x = b'$. Potom:
    $$x_1 = x_1 \cdot 1 = x_1 \cdot a \cdot a^{-1} = b\cdot a^{-1} = a 
    \cdot x_2 \cdot a^{-1} = x_2.$$
\end{proof}

\begin{observation}
    \leavevmode
    \begin{enumerate}[i.]
        \item $\forall a,b,c: a + b = a + c \iff b = c$
        \item $\forall a \neq 0,b,c: a \cdot b = a \cdot c \iff b = c$
    \end{enumerate}
\end{observation}

\begin{proposition}
    \label{proposition:zp}
    Struktura $\mathbb{Z}_p$ je těleso, právě když $p$ je prvočíslo.
\end{proposition}

\begin{proof}
    \leavevmode
    \begin{itemize}
        \item $\implies$: Nepřímo. $p$ je složené, t.j. $\exists a,b: p = a 
            \cdot b$. Potom v $\mathbb{Z}_p$ neplatí Pozorování 
            \ref{obs:abeq0}.
        \item $\impliedby$: Předpokládáme, že $p$ je prvočíslo. Potom musíme
            ověřit všech 10 axiomů. Jediný obtížný axiom je existence opačného
            prvku (NI): $\forall a \neq 0 \exists \inv{a}: 
            a \cdot \inv{a} = 1$, t.j. $a \cdot \inv{a} = 1 \mod{p}$.

            $\forall a$ definujeme zobrazení $f_a: \{1, 2, \dots, p-1\} 
            \rightarrow \{1, 2, \dots, p-1\}$ takové, že: $$f_a(x) = 
            a \cdot x \mod{p}.$$ 
            Potřebujeme ukázat, že zobrazení $f_a$ je prosté. Poté už vyplyne,
            že je zároveň i na a tedy, že $\exists x: f_a(x) = 1$ čili 
            $x = \inv{a}$.

            Důkaz provedeme sporem. Kdyby $f_a$ nebylo prosté, potom 
            $\exists x' \neq x'': f_a(x') = f_a(x'')$, tedy 
            $ax' \equiv ax'' \mod{p}$, čili $a \cdot (x' - x'') \equiv 0 
            \mod{p}$.
    \end{itemize}
\end{proof}

\begin{theorem}
    Konečné těleso s $n$ prvky existuje, právě když $n$ je mocnina prvočísla.
\end{theorem}

\begin{remark}
    Konečné těleso s $n$ prvky se značí $GF(n)$ (z anglického 
    ``\newterm{Galois Field}'').
\end{remark}

\begin{definition}
    Pokud $\exists k \in \mathbb{N}$ takové, že v tělese $K$ platí:
    $$ \underbrace{1+1+\dots+1}_{k} = 0$$
    tak potom nejmenší takové $k$ se nazývá 
    \newterm{charakteristika tělesa} $K$. Pokud takové $k$ neexistuje, 
    říkáme, že těleso $K$ má charakteristiku $0$.
\end{definition}

\begin{theorem}
    Charakteristika tělesa je vždy $0$ nebo prvočíslo.
\end{theorem}

\begin{proof}
    Sporem. Nechť $k$ je složené, t.j. $\exists a, b: k = a \cdot b$.
    Potom $$ 0 = \underbrace{1 + 1 + \dots + 1}_{k} = 
    \underbrace{(1 + \dots + 1)}_{a} \cdot \underbrace{(1 + \dots + 1)}_{b} =
    x \cdot y,$$
    což je spor, jelikož $x \neq 0$ a $y \neq 0$.
\end{proof}

\section{Vektorové prostory}

\begin{definition}
    Nechť $(k, +, \cdot)$ je těleso. Množinu $V$ spolu s binární operací $+$ na 
    $V$ a zobrazením $\cdot: k \times V \rightarrow V$ se nazývá 
    \newterm{vektorový prostor} $(V, +, \cdot)$ nad $k$, pokud platí následující
    axiomy:
    \begin{enumerate}
        \item[(SA)] $\forall \vu,\vv,\vw \in V: (\vu+\vv)+\vw = \vu+(\vv+\vw)$ 
            (sčítání je asociativní)
        \item[(SK)] $\forall \vu,\vv \in V: \vu+\vv = \vv+\vu$ (sčítání je 
            komutativní)
        \item[(S0)] $\exists \vzero \in V,  \forall \vu \in V: \vu+\vzero = \vu$ 
            (existence nulového prvku)
        \item[(SI)] $\forall \vu \in V, \exists -\vu \in V: \vu+(-\vu) = \vzero$ 
            (existence opačného prvku)
        \item[(NA)] $\forall a,b \in k, \forall \vu \in V a \cdot (b \cdot \vu) = 
            (a \cdot b)\cdot \vu$ (násobení vektoru je asociativní)
        \item[(N1)] $\forall \vn \in V: 1 \cdot \vn = \vn$, kde $1$ je jednotkový 
            prvek tělesa $k$ (invariance vektoru při násobení jednotkovým prvkem
            tělesa)
        \item[(D1)] $\forall a,b \in k, \forall \vu \in V: (a + b)\vu = a\vu + 
            b\vu$ (distributivita násobení vektoru vzhledem ke sčítání prvků 
            tělesa)
        \item[(D2)] $\forall a \in k, \forall \vu,\vv \in V: a(\vu+\vv) = a\vu 
            + a\vv$ (distributivita násobení vektoru vzhledem ke sčítání vektorů)
    \end{enumerate}
\end{definition}

\begin{remark}
    Ingredience vektorového prostoru:
    \begin{enumerate}[i.]
        \item Těleso $k$ s operacemi $+$ a $\cdot$. Jeho prvky se nazývají 
            skaláry.
        \item Prostor $V$ s operací $+$\footnote{Operace $+$ na $V$ je odlišná
            od operace $+$ na $k$. Obě se nicméně značí obvykle stejně.}. 
            Jeho prvky se nazývají vektory.
        \item Operace $\cdot$ ``mezi $k$ a $V$''.
    \end{enumerate}
\end{remark}

\begin{remark}[Příklady vektorových prostorů]
    \leavevmode
    \begin{enumerate}[i.]
        \item $V = \{0\}.$ Triviální vektorový prostor nad libovolným tělesem 
            $k$.
        \item $K^n.$ Aritmetický vektorový prostor dimenze $n$. 
            Vektory jsou uspořádané $n$-tice prvků z $K$. Operace $+$ a $\cdot$
            se provadějí po složkách. Z každého tělesa lze vybudovat vektorový 
            prostor téže velikosti $K^1$.
        \item Matice typu $m \times n$ nad $K$.
        \item Polynomy omezeného stupně.
        \item Spojité funkce, diferencovatelné funkce v $\mathbb{R}$.
        \item Systém podmožin nějaké množiny $X$ jako prostor nad $\mathbb{Z}_2$.
    \end{enumerate}
\end{remark}

\begin{proposition}
    Prvky $\vzero$ a $-\vec{a}$ jsou určeny jednoznačně.
\end{proposition}

\begin{proof}
    Důkaz je stejný jako pro skaláry.
\end{proof}

\begin{proposition}
    $\forall \vu \in V, \forall a \in K: 0 \cdot \vu = a \cdot \vzero = \vzero$
\end{proposition}

\begin{proof}
    $$0\vu = 0\vu + 0 = 0\vu + 0\vu - 0\vu = (0 + 0)\vu - 0\vu = 0\vu - 0\vu = 
    \vzero$$
    $$a\vzero = a\vzero + \vzero = a\vzero + a\vzero - a\vzero = a(\vzero + \vzero) 
    - a\vzero = a\vzero - a\vzero = \vzero$$
\end{proof}

\begin{proposition}
    Pokud $a\cdot\vu = \vzero$, potom $a = 0$ nebo $\vu = \vzero$.
\end{proposition}

\begin{proof}
    Sporem. Nechť $a \neq 0$ a $\vu \neq \vzero$.
    $$\vzero \neq \vu = 1 \cdot \vu = \inv{a} \cdot a \cdot \vu = 
    \inv{a} \cdot \vzero = \vzero$$
\end{proof}

\begin{definition}
    Nechť $(V, +, \cdot)$ je vektorový prostor nad tělesem $K$ a $U$ je 
    neprázdná podmnožinu $V$ taková, že:
    \begin{enumerate}[i.]
        \item $\forall \vu,\vv \in U: \vu + \vv \in U$
        \item $\forall \vu \in V, \forall a \in K: a\cdot \vu \in U.$
    \end{enumerate}
    Potom $(U, +, \cdot)$ nazýváme \newterm{podprostorem} $V$.
\end{definition}

\begin{remark}
    Množinu $U$ splňující výše uvedené podmínky nazýváme \newterm{uzavřenou} 
    na operace $+$ a $\cdot$.
\end{remark}

\begin{observation}
    Podprostor $(U, +, \cdot)$ vektorového prostoru $V$ je vektorový 
    prostor.
\end{observation}

\begin{proof}
    Je třeba ověřit všech 8 axiomů z definice vektorového prostoru. Existence
    nulového a opačného prvku plyne z uzavřenosti $U$ vůči operaci $\cdot$ 
    ($0 \cdot \va = \vzero, -1 \cdot \va = -\va$).
\end{proof}

\begin{remark}[Příklady podprostorů $\mathbb{R}^3$]
    \leavevmode
    \begin{itemize}
        \item
            rovina $\pi$ procházející počátkem
        \item přímka $p$ procházející počátkem
        \item bod $\{0\}$
    \end{itemize}
\end{remark}

\begin{proposition}
    Nechť $(U_i, i \in I)$ je systém podprostorů nějakého vektorového prostoru
    $V$. Potom průnik těchto podprostorů, t.j. $\cap_{i \in I} U_i$, je
    podprostorem $V$.
\end{proposition}

\begin{proof}
    Je třeba ukázat uzavřenost $W$ na $+$ a $\cdot$. 
    
    Označme $W \coloneqq \cap_{i \in I} U_i$. Potom:
    \begin{itemize}
        \item uzavřenost na $+$: $u,v \in W \Rightarrow \vu,\vv \in 
            \cap_{i \in I} U_i \Rightarrow \forall i \in I: \vu,\vv \in U_i 
            \Rightarrow \forall i \in I: \vu+\vv \in U_i \Rightarrow \vu+\vv \in 
            \cap_{i \in I} U_i = W$
        \item uzavřenost na $\cdot$: $a \in K, \vu \in W \Rightarrow \forall i
            \in I: \vu \in U_i \Rightarrow \forall i \in I: a \cdot \vu \in U_i
            \Rightarrow a \cdot \vu \in \cap_{i \in I} U_i = W$
    \end{itemize}
\end{proof}

\begin{definition}
    Nechť $V$ je vektorový prostopr nad $K$ a $X$ je podmnožina $V$. Potom 
    $\obal(X)$ značí \newterm{podprostor generovaný $X$} (či \newterm{lineární 
    obal} množiny X), což je průnik všech podprostorů $V$, které obsahují $X$. 
    Formálně: $$\obal(X) \coloneqq 
    \cap \{U | X \subseteq U, U \text{ podprostor } V\}.$$ 
\end{definition}

\begin{proposition}
    Nechť $V$ je vektorový prostor nad $K$ a $X \subseteq V$. Potom $\obal(X)$
    obsahuje všechny lineární kombinace vektorů z $X$, neboli $$\obal(X) = 
    \{\vu | \vu = \sum_{i=1}^{n}a_i \vec{x_i}, n \geq 0, 
    \forall i = 1, \dots, n: a_i \in K, \vec{x_i} \in X \}$$
\end{proposition}

\begin{proof}
    Definujme:
    $$W_1 \coloneqq \cap_{X \subseteq U \subseteq V} U$$
    $$W_2 \coloneqq \left\{ \sum_{i=1}^{n} a_i \vec{x_i}, a_i \in K, 
    \vec{x_i} \in X \right\}$$

    Nejprve ukážeme, že množina $W_2$ je podprostorem $V$, t.j. že je 
    uzavřená na $+$ a $\cdot$. 
    \begin{itemize}
        \item Uzavřenost na $+$. Nechť $\vu,\vv \in W_2$. Potom:
            $$ \vu = \sum_{i=1}^{k}a_i \vec{x_i}, a_i \in K, \vec{x_i} \in X$$
            $$ \vv = \sum_{i=1}^{l}{a'}_i \vec{{x'}_i}, {a'}_i \in K, 
            \vec{{x'}_i} \in X$$

            Označme $\{\vec{y_1}, \dots, \vec{y_n}\} = \{\vec{x_1}, \dots, 
            \vec{x_k}\} \cup \{\vec{{x'}_1}, \dots, \vec{{x'}_l}\}$. 
            Po přeznačení a doplnění koeficientů lze vyjádřit:
            $$ \vu = \sum_{i=1}^{n} b_i \vec{y_i}$$
            $$ \vv = \sum_{i =1}^{n} {b'}_i \vec{y_i}$$
            Potom
            $$ \vu + \vv = \sum_{i=1}^{n} b_i \vec{y_i} + \sum_{i=1}^{n} 
            {b'}_i \vec{y_i} = \sum_{i=1}^{n} (b_i + {b'}_i)\vec{y_i}$$
            a $\vu + \vv \in W_2$ z definice $W_2$.

        \item Uzavřenost na $\cdot$. Nechť $\vu \in W_2, c \in K$.
            $$c \cdot \vu = c \cdot \sum_{i=1}^{k} a_i \vec{x_i} = 
            \sum_{i = 1}^{k} \underbrace{(ca_i)}_{\in K} 
            \underbrace{\vec{x_i}}_{\in X} \in W_2$$
    \end{itemize}

    Nyní $W_1 \subseteq W_2$, protože $W_2$ lze vzít za nějaké $U_i$, přes
    které děláme průniky, jelikož obsahuje $X$ a je podprostorem $V$. Dále také
    $W_2 \subseteq W_1$, protože každé $U_i$ musí být uzavřené na $+$ a 
    $\cdot$, a tedy $\forall i: W_2 \subseteq U_i \implies W_2 \subseteq \cap U_i = W_1$. 
    Z tohoto plyne $W_1 = W_2$.
\end{proof}

\begin{definition}[Prostory určené maticí]
    Nechť $\mat{A}$ je matice typu $m \times n$ nad tělesem $K$.
    \begin{itemize}
        \item \newterm{Sloupcový prostor} $S(\mat{A})$ je podprostor 
            $\mathbb{K}^m$ generovaný sloupci matice $\mat{A}$. Formálně:
            $S(\mat{A}) = \{\vu \in \mathbb{K}^m, \vu = \mat{A}\vx,
            \vx \in \mathbb{K}^n \}.$
        \item \newterm{Řádkový prostor} $R(\mat{A})$ je podprostor 
            $\mathbb{K}^n$ generovaný řádky matice $\mat{A}$. Formálně:
            $R(\mat{A}) = \{\vv \in \mathbb{K}^n, \vv = \transmat{A}\vy, 
            \vy \in \mathbb{K}^m \}$.
        \item \newterm{Jádro matice} $Ker(\mat{A})$ je podprostor
            $\mathbb{K}^n$ tvořený všemi řešeními homogenní soustavy
            $\mat{A}\vec{x} = \vec{0}$.
    \end{itemize}
\end{definition}

\begin{observation}
    Elementární úpravy nemění $R(A)$ ani $Ker(A)$.
\end{observation}

\begin{observation}
    Nechť $\vx \in Ker(A)$ a $\vv \in R(A)$. Potom $\vv^{\top}\vx = 0$.
\end{observation}

\begin{proof}
    $\vv^{\top}\vx = (\mat{A}^{\top}\vy)^{\top}\vx = \vy^{\top}\mat{A}\vx = 
    \vy^{\top} \cdot \vzero = 0$
\end{proof}

\section{Lineární nezávislost}

\begin{definition}
    Nechť $V$ je vektorový prostor nad tělesem $K$. Daná $n$-tice vektorů
    $\vvone, \vvtwo, \dots, \vvn \in V$ je \newterm{lineárně nezávislá}, 
    pokud rovnice:
    $$a_1\cdot\vvone + a_2\cdot\vvtwo + \dots + a_n\cdot\vvn = \vzero$$
    má pouze triviální řešení $a_1 = a_2 = \dots = a_n = 0$. V opačném 
    případě je daná $n$-tice vektorů \newterm{lineárně závislá}.
\end{definition}

\begin{remark}[Poznámky k definici lineární nezávislosti]
    \leavevmode
    \begin{itemize}
        \item Na pořadí vektorů nezáleží.
        \item Pokud $\exists i \neq j: \vvi = \vvj$, potom je daná $n$-tice
            vektorů lineárně závislá.
        \item Pokud $\exists i: \vvi = \vzero$, potom je daná $n$-tice vektorů 
            lineárně závislá.
        \item Rozšířená definice: Nekonečná množina je lineárně nezávislá,
            pokud všechny její konečné podmnožiny jsou lineárně nezávislé.
        \item Co znamená, že daná $n$-tice vektorů je lineárně závislá?
            Alespoň jeden vektor $\vvi$ lze vyjádřit jako lineární kombinace
            ostatních vektorů (nikoliv nutně všech).
    \end{itemize}
\end{remark}

\begin{remark}[Příklady lineární (ne)závislosti]
    \leavevmode
    \begin{itemize}
        \item Nechť $X \subseteq \mathbb{R}^2$ a $X \neq \emptyset$.
            \begin{itemize}
                \item $X = \{\vx\}$. Lineárně závislá, pouze když $\vx$ je 
                    počátek; jinak lineárně nezávislá.
                \item $X = \{\vx, \vy\}$. Pokud $\vzero \in X$, tak $X$ je 
                    lineárně závislá. Podobně, leží-li $\vx$ a $\vy$ na 
                    přímce procházející počátkem. V ostatních případech 
                    je $X$ lineárně nezávislá.
            \end{itemize}
        \item Řádky nebo sloupce jednotkové nebo regulární matice jsou 
            lineárně nezávislé.
        \item Nenulové řádky matice v odstupňovaném tvaru jsou lineárně
            nezávislé.
        \item Nechť je $V$ prostor polynomů nad $\mathbb{R}$. Potom
            $X = \{x^0, x^1, \dots, x^n, \dots \}$ je lineárně nezávislá.
    \end{itemize}
\end{remark}

\begin{observation}
    \leavevmode
    \begin{enumerate}[i.]
        \item Nechť $X$ je lineárně nezávislá a $Y \subseteq X$. 
            Potom $Y$ je také lineárně nezávislá.
        \item Nechť $X$ je lineárně závislá a $X \subseteq Y$. Potom $Y$ 
            je také lineárně závislá.
    \end{enumerate}
\end{observation}

\begin{observation}
    $X$ je lineárně nezávislá, právě když $\forall u \in X: u \not \in 
    \obal(X \setminus \{u\})$.
\end{observation}

\begin{remark}[Ověřování lineární (ne)závislosti]
    Nechť $X \subseteq K^n, X = \{\vvone, \dots, \vvk \}.$ Pro ověření
    lineární závislosti se nabízí dvě metody:
    \begin{enumerate}[i.]
        \item Řešíme $a_1\cdot\vvone + \dots + a_k\cdot \vvk = \vzero$, 
            t.j. homogenní soustavu s $n$ řádky a $k$ sloupci, a hledáme
            netriviální řešení.
        \item Sestavíme matici, kde $\vvone, \dots, \vvk$ tvoří řádky,
            a tuto matici převedeme do odstupňovaného tvaru. Dostaneme-li
            nulový řádek, je daná $k$-tice lineárně závislá; v opačném
            případě je lineární nezávislá.
    \end{enumerate}
\end{remark}

\begin{definition}
    \newterm{Bazí prostoru} $V$ nazvneme libovolnou množinu $X \subseteq V$,
    která je lineárně nezávislá a navíc generuje celý prostor $V$, t.j. 
    $\obal(X) = V$.
\end{definition}

\begin{remark}[Význam báze]
    Díky tomu, že báze generuje celý prostor $V$, lze
    každý vektor $\vu \in V$ vyjádřit jako lineární kombinaci vektorů z báze.
    Navíc, jak ukazuje následující pozorování, díky lineární nezávislosti
    vektorů báze je toto vyjádření jednoznačné.
\end{remark}

\begin{observation}
    Nechť $X = \{\vvone, \dots, \vvn\}$ je konečná báze prostoru $V$ 
    a~nechť $\vu \in V$. Jelikož $X$ generuje celý prostor $V$, lze
    vektor $\vu$ vyjádřit jako lineární kombinaci vektorů báze: 
    $$\vu = \sum_{i=1}^k a_i \cdot \vvi.$$ Toto vyjádření je jednoznačné.
\end{observation}

\begin{proof}
    Sporem. Nechť existují dvě různá vyjádření vektoru $\vu$ jako lineární
    kombinace vektorů báze, $\sum_{i=1}^k a_i \cdot \vvi$, a 
    $\sum_{i=1}^k {a'}_i \cdot \vvi$. Jelikož jsou tato vyjádření různá, 
    tak $\exists i: a_i - {a'}_i \neq 0$. Dále:
    \begin{align*}
        \vzero &= \vu - \vu\\
               &= \sum_{i=1}^k {a}_i \cdot \vvi 
                - \sum_{i=1}^k {a'}_i \cdot \vvi\\
               &= \sum_{i=1}^k ({a}_i - {a'}_i) \cdot \vvi,
    \end{align*}
    čímž dostáváme spor s lineární nezávislostí vektorů báze.
\end{proof}

\begin{definition}
    Nechť $X = \{\vvone, \dots, \vvn\}$ je konečná uspořádaná báze 
    prostoru $V$ nad $K$. Pro libovolný vektor $\vu \in V$ nazveme 
    koeficienty $(a_1, \dots, a_n)^\top \in K^n$ z jednoznačného 
    vyjádření: $$ \vu = \sum_{i=1}^n a_i \cdot \vvi$$
    \newterm{vektorem souřadnic} vektoru $\vu$ vůči bázi $X$. Značí se
    $[\vu]_X = (a_1, \dots, a_n)$.
\end{definition}

\begin{remark}[Příklad bází a vektorů souřadnic]
    \leavevmode
    \begin{itemize}
        \item Nechť $V = \{\text{kvadratické polynomy}\}$ a báze 
            $X = \{x^2, x^1, x^0\}.$ Potom funkci $f = 2x^2 + 3x - 1$ lze 
            vyjádřit vektorem souřadnic jako $[f]_X = (2; 3; -1)^\top$.
        \item Pro vektorový prostor $V = K^n$ nazveme \newterm{kanonickou
            bází} bázi tvořenou vektory $\{\vec{e_1}, \dots, \vec{e_n}\}$, 
            kde $\vec{e_i}$ je $i$-tý sloupec jednotkové matice.
    \end{itemize}
\end{remark}

\begin{proposition}
    Nechť $X$ je taková množina, že generuje celý vektorový prostor $V$, t.j.
    $\obal(X) = V$, ale $\forall Y \subset X: \obal(Y) \neq V$. Potom $X$ 
    je báze vektorového prostoru $V$.
\end{proposition}

\begin{proof}
    Musíme ověřit obě podmínky báze:
    \begin{enumerate}[i.]
        \item $\obal(X) = V$ víme z předpokladů.
        \item Lineární nezávislost množiny $X$ plyne z toho, že $\forall
            \vu \in X: \vu \not \in \obal(X \setminus \{\vu\})$.
    \end{enumerate}
\end{proof}

\begin{corollary}
    Z každého konečného systému generátorů lze vybrat bázi.
\end{corollary}

\begin{proof}
    Stačí vzít nějakou minimální vzhledem k inkluzi, která generuje $V$.
\end{proof}

\begin{theorem}
    Každý vektorový prostor má bázi.
\end{theorem}

\begin{proof}
    Bez důkazu pro vektorové prostory s nekonečným systémem generátorů,
    jelikož by byl třeba axiom výběru.
\end{proof}

\begin{lemma}[Lemma o výměně]
    \label{lemma:ovymene}
    Nechť $\{\vvone, \dots, \vvn\}$ je systém generátorů $V$ a $\vu \in V$. 
    Potom pro všechna $i$ taková, pro která existuje výjádření 
    $\vu = \sum_{j = 1}^n a_j \cdot \vvj$, kde $a_i \neq 0$, platí, že 
    $\{\vvone, \dots, \vec{v_{i-1}}, \vu, \vec{v_{i+1}}, \dots, \vvn\}$ je 
    opět systém generátorů $V$.
\end{lemma}

\begin{proof}
    Vyjádříme $\vvi$ jako: 
    $$\vvi = \frac{1}{a_i}\left(\vu - \sum_{j=1, j \neq i}^{n} a_j\vvj\right).$$ 
    Potom libovolné $\vw \in V, \vw = \sum_{j=1}^{n} b_j\vvj$ lze zapsat jako:
    \begin{align*}
        \vw &= \sum_{j=1, j \neq i}^{n} b_j\vvj + b_i \cdot
            \frac{1}{a_i}\left(\vu - \sum_{j=1, j \neq i}^{n} a_j\vvj\right) \\
            &= \sum_{j=1, j \neq i}^{n} b_j\vvj + \frac{b_i}{a_i}\vu 
            - \sum_{j=1, j \neq i}^{n} \frac{b_ia_j}{a_i}\vvj \\
            &= \frac{b_i}{a_i}\vu + \sum_{j=1, j \neq i}^{n} 
                \left(b_j - \frac{b_ia_j}{a_i}\right)\vvj
    \end{align*}
\end{proof}

\begin{theorem}[Steinitzova věta o výměně]
    \label{theorem:steinitz}
    Nechť $V$ je vektorový prostor, $X$ je linárně nezávislá ve $V$ a $Y$
    je konečný systém generátorů $V$. Potom existuje $Z$ takové, že
    $X \subseteq Z \subseteq X \cup Y, L(Z) = V$ a $|Z| = |Y|$. Navíc platí
    $|X| \leq |Y|$.
\end{theorem}

\begin{proof} Označme $X \setminus Y = \{u_1, \dots, u_n\}$ a položme $Z_0 
    \coloneqq Y$. Pro $i = 1, \dots, n$ provedeme: $Z_{i-1}$ generuje $V$. 
    Vyjádříme $u_i$ vůči $Z_{i-1}$: $$u_i = \sum_{w_j \in Z_{i-1}} a_jw_j.$$
    $X$ je lineárně nezávislá, a tedy $a_j \neq 0$ pro nějaké 
    $w_j \in Y \setminus X$. Položíme $Z_i \coloneqq Z_{i-1} \cup \{u_i\} 
    \setminus {w_j}.$ Dle lemmatu o výměně $\obal(Z_i) = V$. Nakonec, 
    $Z \coloneqq Z_n, |Z| = |Z_n| = |Z_{n-1}| = \dots = |Z_0| = |Y|.$ 
    
    Pokud by $|X| > |Y|$, potom $\exists i < n: Z_i \subset X$ a 
    $\obal(Z_i) = V$. Dostáváme spor s lineární nezávislostí množiny $X$.
\end{proof}

\begin{corollary}
    Pokud má vektorový prostor $V$ konečnou bázi, potom všechny jeho báze
    mají stejnou mohutnost.
\end{corollary}

\begin{proof}
    Nechť $X$ a $Y$ jsou dvě různé báze vektorového prostoru $V$. Potom:
    \begin{enumerate}[a.]
        \item $X$ je lineárně nezávislá a $\obal(Y) = V$. Potom dle 
            Věty~\ref{theorem:steinitz} je $|X| \leq |Y|$.
        \item Podobně $Y$ je lineárně nezávislá, $\obal(X) = V$ a 
            $|Y| \leq |X|$.
    \end{enumerate}

    Vyplývá, že $|X| = |Y|$.
\end{proof}

\begin{corollary}
    Pokud má vektorový prostor $V$ konečný systém generátorů, potom lze 
    každou lineárně nezávislou množinu $X$ doplnit na bázi.
\end{corollary}

\begin{definition}
    Nechť má vektorový prostor $V$ konečnou bázi. Potom se o $V$ říká, 
    že je \newterm{konečně generovaný} a mohutnost jeho libovolné báze
    nazveme \newterm{dimenzí} prostoru $V$. Značí se $\dm V$.
\end{definition}

\begin{remark}[Příklady bází a jejich dimenzí]
    \leavevmode
    \begin{itemize}
        \item $\dm K^n = n$
        \item Je-li matice $\mat{A}$ v odstupňovaném tvaru, potom 
            $\dm R(\mat{A}) = \rank \mat{A}$.
    \end{itemize}
\end{remark}

\begin{observation}
    Je-li $W$ podprostor vektorového prostoru $V$ konečné dimenze, pak
    $\dm W \leq \dm V$.
\end{observation}

\begin{proof}
    Báze $W$ je lineárně nezávislá v $V$, ale lze ji doplnit na bázi 
    celého prostoru $V$.
\end{proof}

\begin{observation}
    Pro podprostory $U, V \subseteq W$, kde $\dm W < \infty$, platí:
    $$\dm U + \dm V = \dm U \cap V + \dm(\obal(U \cup V)).$$
\end{observation}

\begin{observation}
    \label{obs:dimradima}
    Pro všechna $\mat{A} \in K^{m \times n}$ platí $\dm R(\mat{A}) 
    = \rank \mat{A}.$
\end{observation}

\begin{proof}
    Je-li $\mat{A} \sim \mat{A'}$ v odstupňovaném tvaru: $$\dm R(\mat{A})
    = \dm R(\mat{A'})  = \rank \mat{A'} = \rank \mat{A}.$$
\end{proof}

\begin{remark}
    Pozorování~\ref{obs:dimradima} lze využít k nalezení báze a určení
    dimenze podmnožin $K^n$: Sestavíme matici z vektorů po řádcích a 
    převedeme ji do odstupňovaného tvaru. Výsledné nenulové řádky tvoří 
    bázi.
\end{remark}

\begin{theorem}
    \label{theorem:dimradimsa}
    Nechť $\mat{A} \in K^{m \times n}$. Potom platí:
    $$\dm R(\mat{A}) = \dm S(\mat{A}).$$
\end{theorem}

\begin{proof}
    \leavevmode
    \begin{enumerate}[I.]
        \item Nejdříve ukážeme, že přinásobením matice zleva dimenze
            sloupcového prostoru nevzroste. Matice $\mat{R}$ a $\mat{A}$
            jsou dány. Spočteme $\mat{A'} = \mat{R} \cdot \mat{A}$.
            Označme $\vuone, \cdots, \vun$ sloupce matice $\mat{A}$ a
            $\vec{{u'}_1}, \cdots, \vec{{u'}_n}$ sloupce matice $\mat{A'}$. 
            Platí $\vec{{u'}_i} = \mat{R} \cdot \vui$. 

            Nechť $w' \in S(\mat{A'})$, tedy $w' = \sum_{i=1}^{n} a_i 
            \vec{{u'}_i} =
            \sum_{i=1}^{n} a_i \cdot \mat{R} \cdot \vui = \mat{R} \cdot 
            \sum_{i=1}^{n} a_i \vui = \mat{R} \cdot \vw$ pro nějaké 
            $\vw \in S(A)$.

            Nyní vyjádříme $\vw$ vůči bázi $\vvone, \cdots, \vvd$ prostoru 
            $S(\mat{A})$, čili $\vw = \sum_{i=1}^{d} b_i \vvi$. Potom
            $\vec{w'} = \mat{R} \cdot \vw = \mat{R} \cdot \sum_{i=1}^{n} 
            b_i \vvi = \sum_{i=1}^{d} b_i \mat{R} \vvi = \sum_{i=1}^{d} 
            b_i \vec{{v'}_i}$, kde $\vec{{v'}_i} \in S(\mat{A'})$, 
            čili $\vec{{v'}_1}, \cdots, \vec{{v'}_d}$ tvoří systém 
            generátorů $S(\mat{A'})$.

            Tedy: $\dm S(\mat{A'}) \leq \dm S(\mat{A})$.

        \item Je-li matice $\mat{R}$ regulární, dimenze zůstane zachována:
            $\mat{A} = \invmat{R} \cdot \mat{A'}$, tedy $\dm S(\mat{A}) 
            \leq \dm S(\mat{A'})$.

        \item Pro matici $\mat{A'}$ v odstupňovaném tvaru platí 
            $\dm R(\mat{A'}) = \dm S(\mat{A'})$, jelikož sloupce s pivoty
            jsou lineárně nezávislé a tvoří bázi $S(\mat{A'})$.

        \item Pro danou matici $\mat{A}$ nalezneme $\mat{A'} \sim \mat{A}$,
            $\mat{A'}$ je v odstupňovaném tvaru. Víme, že platí: 
            $\mat{A'} = \mat{R} \cdot \mat{A}$, $\mat{R}$ je regulární.

            $\dim{S(\mat{A})} \stackrel{\text{II}}{=} \dm S(\mat{A'})
            \stackrel{\text{III}}{=} \dm R(\mat{A'}) 
            \underbrace{=}_{\text{Pozorování~\ref{obs:dimradima}}} 
                \dm R(\mat{A})$
    \end{enumerate}
\end{proof}

\begin{corollary}
    $\rank \mat{A} = \rank \transmat{A}$
\end{corollary}

\begin{corollary}
    Nechť $\mat{R}$ je regulární. Potom: 
            $$\rank \mat{A} = \rank \mat{R} \cdot \mat{A}$$
            $$\rank \mat{A} = \rank \mat{A} \cdot \mat{R}$$
\end{corollary}

\begin{corollary}
    $S(\mat{A} \cdot \mat{B}) \subseteq S(\mat{A})$ a 
            $R(\mat{A} \cdot \mat{B}) \subseteq R(\mat{B})$
\end{corollary}

\begin{proof}
    $S(\mat{AB}) = \obal(\{x | x=\mat{A}\cdot u, \text{$u$ je sloupec 
    $\mat{B}$}\})
    = \{x' | x' = \mat{A} \cdot u', u' \in S(\mat{B}) \} \subseteq
    \{x' | x' = \mat{A} \cdot u', u' \in K^n \} = S(\mat{A})$
\end{proof}

\begin{corollary}[Při násobení padáme s hodností]
    $\rank \mat{AB} \leq min \{ \rank \mat{A}; \rank \mat{B} \}$ .
\end{corollary}

\begin{proposition}
    Pro matici $\mat{A}$ řádu $m \times n$ platí:
    $$\dm Ker(\mat{A}) + \rank \mat{A} = n$$
\end{proposition}

\begin{proof}
    \leavevmode
    \begin{enumerate}[i.]
        \item Hodnost matice $\rank \mat{A}$ určuju počet pivotů, tedy
            počet bázových proměnných.
        \item Pokud $\vec{x} \in Ker(\mat{A})$, potom $\vec{x}$ řeší 
            $\mat{A}\vec{x} = \vec{0}$. Vektor $\vec{x}$ lze vyjádřit jako:
            $$\vec{x} = p_1\vec{x^1} + \cdots + p_2\vec{x^{n-r}}$$
            Množina $\left\{\vec{x^i}\right\}^{n-r}_{i=1}$ generuje 
            $Ker(\mA)$. Navíc, $x_i$ jsou lineárně nezávislé, 
            jelikož $\vec{x_i}$ má ve složce odpovídající $i$-té volné proměnné 
            jedničku, zatímco ostatní složky jsou rovny nule. 
            Plyne, že $\left\{\vec{x^i}\right\}^{n-r}_{i=1}$ je báze
            $Ker(\mA)$ a $\dm Ker(\mat{A}) = n-r$.
    \end{enumerate}
\end{proof}

\section{Lineární zobrazení}
\label{ch:linzobr}

\begin{observation}
    Nechť $\mat{A} \in K^{m \times n}$ a $f: K^n \rightarrow K^m$ je
    zobrazení definováno předpisem $f(\vec{u}) = \mat{A} \cdot \vec{u}$.
    Potom platí:
    \begin{enumerate}[i.]
        \item $f(\vec{u} + \vec{v}) = \mat{A}(\vec{u} + \vec{v}) = 
            \mat{A}\vec{u} + \mat{A}\vec{v} = f(\vec{u}) + f(\vec{v})$
        \item $f(a\cdot\vec{u}) = \mat{A}(a \cdot \vec{u}) = 
            a \cdot \mat{A}\vec{u} = a \cdot f(\vec{u})$
    \end{enumerate}
\end{observation}

\begin{definition}
    Nechť $V$ a $W$ jsou vektorové prostory nad stejným tělesem $K$.
    Zobrazení $f: V \rightarrow W$ se nazývá \newterm{lineární zobrazení},
    pokud platí:
    \begin{enumerate}[i.]
        \item $\forall \vu,\vv \in V: f(\vec{u} + \vec{v}) = f(\vec{u}) 
            + f(\vec{v})$
        \item $\forall \vu \in V, \forall a \in K: f(a\cdot\vec{u}) =
            a \cdot f(\vec{u})$
    \end{enumerate}
\end{definition}

\begin{remark}[Příklady lineárních zobrazení]
    \leavevmode
    \begin{itemize}
        \item Pro libovolné $V$ a $W$ můžeme vzít $f(\vec{u}) = \vec{0}$
            pro $\forall \vu \in V$, tzv. nulové zobrazení.
        \item Pro $V \subseteq W$ můžeme vzít $f(\vec{u}) = \vec{u}$, čili
            identita na $V$, neboli vnoření $V$ do $W$.
        \item Pro aritmetické vektorové prostory $V = K^n$, $W = K^1$ 
            definujeme projekci na $i$-tou souřadnici $\pi_i$ předpisem:
            $$\pi_i(\vec{u}) = u_i.$$
        \item Nechť $V$ je vektorový prostor a $X = \{\vvone, \dots, \vvn\}$ 
            jeho konečná báze. Nechť dále $\vu, \vv \in V$, 
            a $\vec{u} = \sum a_i \vec{v_i}$, $\vec{b} = \sum b_i \vec{v_i}.$
            Potom zobrazení $f: V \rightarrow K^n$ na vektor souřadnic 
            je lineární zobrazení:
			\begin{multline*}
				f(\vec{u} + \vec{v}) = [\vec{u} + \vec{v}]_X = 
					\left[\sum a_i\vec{v_i} + \sum b_i \vec{v_i}\right]_X = \\
					\left[\sum (a_i + b_i) \vec{v_i} \right]_X = 
					[\vec{u}]_X + [\vec{v}]_X = f(\vu) + f(\vv)
			\end{multline*}
            Násobení analogicky.
        \item Geometrická zobrazení v rovině:
            \begin{itemize}
                \item Posunutí není lineární zobrazení, jelikož počátek ve
                    $V$ se musí zobrazit na počátek ve $W$.
                \item Osová souměrnost, otočení a stejnolehlost jsou
                    lineární zobrazení, pokud zachovávají počátek.
            \end{itemize}
        \item Obecně v $\mathbb{R}^2$:
            $$f(\vec{u}) = \begin{pmatrix}
                a &c \\
                b &d
            \end{pmatrix} 
            \begin{pmatrix} u_x \\ u_y \end{pmatrix},$$
        \item V prostoru diferencovatelných funkcí je derivace lineární
            zobrazení.
    \end{itemize}
\end{remark}

\begin{observation}
    Nechť $f: U \rightarrow V$ a $g: V \rightarrow W$ jsou lineární 
    zobrazení. Potom je jejich složení $g \circ f: U \rightarrow W$, 
    $(g \circ f)(\vec{u}) = g(f(\vec{u}))$ také lineární zobrazení.
\end{observation}

\begin{proof}
    Ověříme podmínky:
    \begin{enumerate}[i.]
        \item $(g \circ f)(\vec{u} + \vec{v}) = g(f(\vec{u} + \vec{v})) =
            g(f(\vec{u}) + f(\vec{v})) = g(f(\vec{u})) + g(f(\vec{v})) =
            (g \circ f)(\vec{u}) + (g \circ f)(\vec{v}).$
        \item $(g \circ f)(a \cdot \vec{v})$ analogicky.
    \end{enumerate}
\end{proof}

\begin{theorem}
    Nechť $V$ a $W$ jsou vektorové prostory nad společným tělesem a $X$ je
    báze $V$. Potom pro všechna zobrazení $f_0: X \rightarrow W$ existuje
    právě jedno lineární zobrazení $f: V \rightarrow W$, které rozšiřuje
    $f_0$: 
    $$\forall \vec{v} \in X: f(\vec{v}) = f_0(\vec{v}).$$
\end{theorem}

\begin{proof}
    Vyjádříme $\vec{u} \in V$ vůči bázi: $\vu = \sum a_i \vec{v_i}$. Potom
    $f(\vec{u}) = f(\sum a_i \vec{v_i}) = \sum a_i f(\vec{v_i}) = 
    \sum a_i f_0(\vec{v_i}).$
\end{proof}

\begin{corollary}
    Označíme-li $f(V) = \cup_{\vec{u} \in V} f(\vec{u})$, potom $f(V)$ je
    podprostor prostoru $W$ a $\dm f(V) \leq \dm V$, protože obraz
    báze $V$ je systém generátorů $f(V)$. 
\end{corollary}

\begin{definition}
    Nechť $V$ a $W$ jsou vektorové prostory nad $K$ a 
	$X = \{\vvone, \dots, \vvn\}$ a $Y = \{\vec{w_1}, \dots, \vec{w_m})$ 
	jsou jejich báze. Potom pro lineární zobrazení $f: V \rightarrow W$ 
	nazveme matici $[f]_{XY} \in K^{m\times n}$ sestavenou z vektorů 
	souřadnic obrazů vektorů báze $X$ vuči $Y$ 
    \newterm{maticí zobrazení} $f$ vůči bázím $X$ a $Y$:
    $$ [f]_{XY} = \begin{pmatrix}
        \vdots &\vdots &\dots &\vdots\\
        [f(\vec{v_1})]_Y &[f(\vec{v_2})]_Y &\dots &[f(\vec{v_n})]_Y \\
        \vdots &\vdots &\dots &\vdots 
    \end{pmatrix}$$
\end{definition}

\begin{observation}
    \leavevmode
    $$ [f(\vec{u})]_Y = [f]_{XY} [\vec{u}]_X$$
\end{observation}

\begin{proof}
    Vyjádříme $\vec{u}$ vůči bázi: $\vec{u} = \sum_{i=1}^{n} a_i \vec{v_i}$.
    Potom $[\vec{u}]_X = (a_1, \dots, a_n)^\top, f(\vec{u}) = 
    \sum a_i f(\vec{v_i})$ a:
    $$[f(\vec{u})]_Y = \left[\sum a_i f(\vec{v_i})\right]_Y = 
    \sum a_i [f(\vec{v_i})]_Y = [f]_{XY}[\vec{u}]_X$$
\end{proof}

\begin{observation}
    Jsou-li $U, V$, a $W$ prostory nad $K$ s bázemi $X, Y$ a $Z$, a 
    $f: U \rightarrow V$ a $g: V \rightarrow W$ jsou lineární zobrazení, tak
    platí:
    $$[g \circ f]_{XZ} = [g]_{YZ} [f]_{XY}$$
\end{observation}

\begin{proof}
    \leavevmode
    $$[(g \circ f)(\vec{u})]_Z = [g(f(\vec{u}))]_Z = [g]_{YZ}[f(\vec{u})]_Y =
    [g]_{YZ}[f]_{XY}[\vec{u}]_X$$
\end{proof}

\begin{definition}
    Nechť $V$ je prostor nad $K$ a $X, Y$ jsou jeho dvě konečné báze.
    \newterm{Maticí přechodu} od báze $X$ k bázi $Y$ rozumíme matici 
    $[id]_{XY}$, kde $id$ je identita.
\end{definition}

\begin{observation}
    \leavevmode
    \begin{enumerate}[i.]
        \item $[\vec{u}]_Y = [id(u)]_Y = [id]_{XY}[\vec{u}]_X$
        \item $[id]_{XY}[id]_{YX} = [id]_{YY} = \mat{I_n}$
    \end{enumerate}
\end{observation}

\begin{remark}[Výpočet matice přechodu pro $V = K^n$]
    Pro báze $X = \{\vec{v_1}, \dots, \vec{v_n}\}$ a 
    $Y = \{\vec{w_1}, \dots, \vec{w_n}\}$ sestavíme matice:
    $$\mat{A} = \begin{pmatrix}
        \vdots &\dots &\vdots \\
        \vec{v_1} &\dots &\vec{v_n} \\
        \vdots &\dots &\vdots
    \end{pmatrix}; 
    \mat{B} = \begin{pmatrix}
        \vdots &\dots &\vdots \\
        \vec{w_1} &\dots &\vec{w_n} \\
        \vdots &\dots &\vdots
    \end{pmatrix}$$
    Platí $\vec{u} = \sum a_i \vec{v_i} = \mat{A}[\vec{u}]_X$ a 
    $\vec{u} = \sum b_i \vec{w_i} = \mat{B}[\vec{u}]_Y$.
    Potom: 
    $$\mat{A}[\vec{u}]_X = \mat{B}[\vec{u}]_Y$$
    $$[\vec{u}]_Y = \invmat{B}\mat{A}[\vec{u}]_X$$
    Tedy: $[id]_{XY} = \invmat{B}\mat{A}$.
    Prakticky: $(\mat{B}|\mat{A}) \sim (\mat{I_n} | [id]_{XY})$.
\end{remark}

\begin{definition}
    Nechť $V$ a $W$ jsou vektorové prostory nad K. Lineární zobrazení,
    které je prosté a na, nazveme \newterm{isomorfismem} prostorů $V$ a $W$.
\end{definition}

\begin{observation}
    Zobrazení $f^{-1}$ je také isomorfismem.
\end{observation}

\begin{proof}
	Musíme dokázat, že zobrazení $f^{-1}$ je lineární zobrazení:
	\begin{enumerate}[i.]
		\item $f^{-1}(\vec{w} + \vec{w'}) = f^{-1}(f(\vec{u}) + f(\vec{u'})) = 
			f^{-1}(f(\vec{u} + \vec{u'})) = \vec{u} + \vec{u'} 
			= f^{-1}(\vec{w}) + f^{-1}(\vec{w'}).$
	    \item Násobení analogicky.
	\end{enumerate}
\end{proof}

\begin{theorem}
    Nechť $V$ a $W$ jsou vektorové prostory nad $K$ s konečnými bázemi $X$
    a $Y$. Potom platí, že lineární zobrazení $f: V \rightarrow W$ je
    isomorfismus, právě když matice $[f]_{XY}$ je regulární. Navíc platí:
    $$ [f^{-1}]_{YX} = \left([f]_{XY}\right)^{-1}.$$
\end{theorem}

\begin{proof}
    \leavevmode
    \begin{itemize}
        \item $\impliedby$: $[f]_{XY}$ je regulární. Vezmeme zobrazení 
            $g: W \rightarrow V$ definované maticí: $[g]_{YX} = 
            \left([f]_{XY}\right)^{-1}$. Ukážeme, že $g = f^{-1}$ a 
            ověříme vlastnosti isomorfismu.
            \begin{enumerate}[i.]
                \item $[g \circ f]_{XX} = [g]_{YX}[f]_{XY} = \mat{I_n}.$
                    Tedy $g \circ f$ je identita na $V$ a $f$ je prosté.
                \item $[f \circ g]_{YY} = [f]_{XY} [g]_{YX} = \mat{I_n}.$
                    Tedy $f \circ g$ je identita na $W$ a $f$ je na.
            \end{enumerate}
        \item $\implies$: Máme zobrazení $f$ a $f^{-1}$. Pro jejich matice
            platí:
            \begin{enumerate}[i.]
                \item $[f^{-1}]_{YX}[f]_{XY} = [id]_{XX} = \mat{I_n},
                    \dm V = n$
                \item $[f]_{XY}[f^{-1}]_{YX} = [id]_{YY} = \mat{I_m},
                    \dm W = m$
            \end{enumerate}
            Vyplývá, že $n = m$ a $[f]_{XY}$ je regulární.
    \end{itemize}
\end{proof}

\begin{proposition}
    Každý prostor dimense $n$ nad $K$ je isomorfní s $K^n$.
\end{proposition}

\begin{proof}
    Zvolíme bázi $X$, potom zobrazení $f: \underbrace{\vec{u}}_{\in V} 
    \rightarrow \underbrace{[u]_X}_{\in K^n}$ je isomoforfismem. 
    $[f]_{Xk} = \mat{I_n}$ tvoří kanonickou bázi.
\end{proof}

\begin{proposition}
    Nechť $f: V \rightarrow W$ je lineární zobrazení. Potom platí:
    \begin{enumerate}[i.]
        \item $Ker(f) \coloneqq \{\vec{x} | f(\vec{x}) = \vec{0}\}$ je 
            podprostor $V$.
        \item Pokud má rovnice $f(\vec{x}) = \vec{b}$ alespoň 1 řešení 
			$\vec{x_0}$, potom lze každé řešení $\vec{x}$ této rovnice 
			vyjádřit jako $\vec{x} = \vec{x_0} + \vec{x'}$, kde 
			$\vec{x'} \in Ker(f)$.
    \end{enumerate}
\end{proposition}

\begin{proof}
    \leavevmode
    \begin{enumerate}[i.]
        \item Nechť $\vec{x_1}, \vec{x_2} \in Ker(f): f(\vec{x_1} + \vec{x_2})
            = f(\vec{x_1}) + f(\vec{x_2}) = \vec{0} + \vec{0} = \vec{0}$, tedy
            $\vec{x_1} + \vec{x_2} \in Ker(f)$. Násobení analogicky.
        \item $f(\vec{x} - \vec{x_0}) = f(\vec{x}) - f(\vec{x_0}) = \vec{b}
            - \vec{b} = \vec{0}$, tedy $\vec{x} - \vec{x_0} \in Ker(f)$.
    \end{enumerate}
\end{proof}

\section{Skalární součin}

\begin{definition}
    Nechť $V$ je vektorový prostor nad $\mathbb{C}$. Zobrazení, které
    dvojici vektorů $\vec{u}, \vec{v} \in V$ přiřadí 
    $\langle\vec{u}|\vec{v}\rangle \in \mathbb{C}$, se nazývá 
    \newterm{skalární součin}, pokud splňuje následující axiomy:
    \begin{itemize}
        \item[(N)] $\forall \vec{u} \in V: \langle \vec{u} | \vec{u}
            \rangle = 0 \iff \vec{u} = \vzero$
        \item[(L1)] $\forall \vec{u},\vec{v}, \vec{w} \in V:
            \langle \vec{u} + \vec{v} | \vec{w} \rangle = 
            \scp{\vec{u}}{\vec{w}} + \scp{\vec{v}}{\vec{w}}$
        \item[(L2)] $\forall \vec{u}, \vec{v} \in V, \forall a \in
            \mathbb{C}: \scp{a\cdot\vec{u}}{\vec{v}} = a\cdot
            \scp{\vec{u}}{\vec{v}}$
        \item[(KS)] $\forall \vec{u}, \vec{v} \in V: 
            \scp{\vec{u}}{\vec{v}} = \overline{\scp{\vec{v}}{\vec{u}}}$
        \item[(P)] $\forall \vec{u} \in V: \scp{\vec{u}}{\vec{u}} 
            \geq 0$ 
    \end{itemize}
\end{definition}

\begin{remark}[Poznámka k axiomu (P)]
    Jelikož $\forall \vec{u} \in V: \scp{\vec{u}}{\vec{u}} \geq 0$, 
    vyplývá tedy, že $\scp{\vec{u}}{\vec{u}} \in \mathbb{R}$.
\end{remark}

\begin{remark}[Příklady skalárních součinů]
    \leavevmode
    \begin{itemize}
        \item Skalární součin pro aritmetické vektorové prostory:
            \begin{itemize}
                \item $V = \mathbb{C}^n: \scp{\vec{u}}{\vec{v}} = 
                    \sum_{i=1}^n u_i \overline{v_i}$
                \item $V = \mathbb{R}^n: \scp{\vec{u}}{\vec{v}} = 
                    \sum_{i=1}^n u_i v_i$
            \end{itemize}
        \item Skalární součin na $\mathbb{R}^n$ definovaný pomocí
            regulární matice: $\scp{\vec{u}}{\vec{v}} = \vec{u}^\top
            \cdot \transmat{A} \cdot \mat{A} \cdot \vec{v}$
        \item Skalární součin na prostoru reálných spojitých funkcí
            integrovatelných na intervalu $(a;b)$: 
            $\scp{f(x)}{g(x)} \coloneqq \int_a^b f(x)g(x) \,dx$
    \end{itemize}
\end{remark}

\begin{observation}
    $\scp{\vx}{\vzero} = \scp{\vzero}{\vx} = 0$
\end{observation}

\begin{proof}
    $\scp{\vx}{\vzero} = \scp{\vx}{0 \cdot \vx} = 0 \cdot \scp{\vx}{\vx}
    = \scp{0\cdot \vx}{\vx} = \scp{\vzero}{\vx}$
\end{proof}

\begin{definition}
    \label{def:normaskalarnisoucin}
    Nechť $V$ je vektorový prostor se skalárním součinem, potom
    \newterm{norma} odvozená od skalárního součinu je zobrazení
    $\|\bullet\|: V \rightarrow \mathbb{R}$ dané předpisem:
    $$\|\vec{u}\| \coloneqq \sqrt{\scp{\vec{u}}{\vec{u}}}.$$
\end{definition}

\begin{remark}[Geometrická interpretace normy a skalárního součinu
    v $\mathbb{R}^n$]
    \leavevmode
    \begin{itemize}
        \item $\|\vec{u}\|$ určuje délku vektoru $\vec{u}$
        \item $\|\vec{u} - \vec{v}\|$ určuje vzdálenost vektorů
            $\vec{u}$ a $\vec{v}$
        \item $\scp{\vec{u}}{\vec{v}}$ určuje úhel mezi vektory
            $\vec{u}$ a $\vec{v}$
    \end{itemize}
\end{remark}

\begin{observation}
    Pro standardní skalární součin a jím určenou normu na 
    $\mathbb{R}^n$ platí: 
    $$ \scp{\vec{u}}{\vec{v}} = \|\vec{u}\|\cdot\|\vec{v}\|\cdot \cos 
    \varphi,$$
    kde $\varphi$ je úhel sevřený vektory $\vec{u}$ a $\vec{v}$.
\end{observation}

\begin{proof}
    Vektory $\vec{u}$, $\vec{v}$ a $\vec{u}-\vec{v}$ tvoří trojúhelník.
    Podle kosinové věty: $$\|\vec{u} - \vec{v}\|^2 = \|\vec{u}\|^2
    + \|\vec{v}\|^2 - 2 \cdot \|\vec{u}\| \cdot \|\vec{v}\| \cdot 
    \cos \varphi,$$
    tedy:
    $$
        \scp{\vec{u}-\vec{v}}{\vec{u}-\vec{v}} = \scp{\vec{u}}{\vec{u}} +
        \scp{\vec{v}}{\vec{v}} - 2 \cdot \|\vec{u}\| \cdot \|\vec{v}\| \cdot 
        \cos \varphi.$$

    Podle axiomů skalárního součinu ovšem také platí:
    \begin{align*}
        \scp{\vec{u}-\vec{v}}{\vec{u}-\vec{v}} &= \\
        &= \scp{\vec{u}}{\vec{u}-\vec{v}} - \scp{\vec{v}}{\vec{u}-\vec{v}} \\
        &= \scp{\vec{u} - \vec{v}}{\vec{u}} - \scp{\vec{u} - 
            \vec{v}}{\vec{v}} \\
        &= \scp{\vec{u}}{\vec{u}} - \scp{\vec{v}}{\vec{u}} - 
            \scp{\vec{u}}{\vec{v}} + \scp{\vec{v}}{\vec{v}} \\
        &= \scp{\vec{u}}{\vec{u}} - 2\cdot \scp{\vec{u}}{\vec{v}} 
            + \scp{\vec{v}}{\vec{v}}
    \end{align*}

    Odečteme-li tento výsledek od rovnice kosinové věty, dostáváme (po 
    úpravách):
    $$ \scp{\vec{u}}{\vec{v}} = \|\vec{u}\|\cdot\|\vec{v}\|\cdot \cos 
    \varphi $$
\end{proof}

\begin{theorem}[Cauchy-Schwarzova nerovnost]
    Nechť $V$ je vektorový prostor nad $\mathbb{C}$ se skalárním součinem
    a s normou určenou tímto součinem. Potom platí:
    $$ \forall u,v \in V: |\scp{\vec{u}}{\vec{v}}| \leq \|\vec{u}\| \cdot
    \|\vec{v}\|. $$
\end{theorem}

\begin{proof}
    Je-li $\vec{u} = \vzero$ nebo $\vec{v} = \vzero$, nerovnost platí. 
    
    Určitě $\forall a \in \mathbb{C}: \|\vec{u} + a\cdot\vec{v}\|^2 \geq 0$.
    Tedy:
    \begin{align*}
        0 \leq \|\vec{u} + a\vec{v}\|^2 &= \\
        &= \scp{\vec{u} + a\vec{v}}{\vec{u} + a\vec{v}} \\
        &= \scp{\vec{u}}{\vec{u} + a\vec{v}} + \scp{a\vec{v}}{\vec{u} 
            + a\vec{v}} \\
        &= \overline{\scp{\vu + a\vv}{\vu}} + a
            \overline{\scp{\vu + a \vv}{\vv}} \\
        &= \scp{\vu}{\vu} + \overline{a \scp{\vv}{\vu}} + 
            a \overline{\scp{\vu}{\vv}} + a  \overline{a}
            \scp{\vv}{\vv} \\
        &= \overline{a}\scp{\vec{u}}{\vec{v}} + \scp{\vec{u}}{\vec{u}} +
            a\scp{\vec{v}}{\vec{u}} + a\overline{a}\scp{\vec{v}}{\vec{v}}
    \end{align*}

    Dosadíme 
    $a \coloneqq -\frac{\scp{\vec{u}}{\vec{v}}}{\scp{\vec{v}}{\vec{v}}}$,
    čímž se zbavíme prvního a posledního členu. Zbývá:
    \begin{align*}
        0 &\leq \scp{\vec{u}}{\vec{u}} 
        -\frac{\scp{\vec{u}}{\vec{v}}\scp{\vec{v}}{\vec{u}}}
        {\scp{\vec{v}}{\vec{v}}} \\
        \scp{\vec{u}}{\vec{v}}\scp{\vec{v}}{\vec{u}} &\leq
        \scp{\vec{u}}{\vec{u}}\scp{\vec{v}}{\vec{v}}
    \end{align*}

    Upravíme levou stranu nerovnosti:
    $$\scp{\vec{u}}{\vec{v}}\scp{\vec{v}}{\vec{u}} = 
        \scp{\vec{u}}{\vec{v}}\overline{\scp{\vec{u}}{\vec{v}}} =
        |\scp{\vec{u}}{\vec{v}}|^2,$$
    a tedy:
    \begin{align*}
        |\scp{\vec{u}}{\vec{v}}|^2 &\leq
        \scp{\vec{u}}{\vec{u}}\scp{\vec{v}}{\vec{v}} \\
        |\scp{\vec{u}}{\vec{v}}| &\leq \|\vec{u}\|\|\vec{v}\|
    \end{align*}
\end{proof}

\begin{corollary}[Vztah mezi aritmetickým a kvadratickým průměrem]
    Nechť $u_i \in \mathbb{R}$. Potom:
    $$\frac{1}{n}\sum_{i=1}^n u_i \leq \sqrt{\frac{1}{n}\sum_{i=1}^{n} u_i^2}$$
\end{corollary}

\begin{proof}
    Položme 
    \begin{align*} 
        \vec{v} &= \rowvec{1,1,\dots,1}^\top \\
        \vec{u} &= (\text{seřazená čísla})^\top.
    \end{align*}
    Potom 
    $$\scp{\vec{u}}{\vec{v}} = \sum_{i=1}^n u_i, 
    \|\vec{u}\| = \sqrt{\sum_{i=1}^n u_i^2}, \text{ a }
    \|\vec{v}\| = \sqrt{n}.$$
\end{proof}

\begin{corollary}[Trojúhelníková nerovnost]
    Norma odvozená od skalárního součinu splňuje trojúhelníkovou nerovnost:
    $$\|\vec{u} + \vec{v}\| \leq \|\vec{u}\| + \|\vec{v}\|$$
\end{corollary}

\begin{proof}
    \begin{align*}
        \|\vec{u} + \vec{v}\| &= \sqrt{\scp{\vec{u} + \vec{v}}{\vec{u} + 
            \vec{v}}} \\
            &= \sqrt{\scp{\vu}{\vu + \vv} + \scp{\vv}{\vu + \vv}} \\
            &= \sqrt{\scp{\vu}{\vu} + \scp{\vu}{\vv} + \scp{\vv}{\vu} 
                + \scp{\vv}{\vv}} \\
            &= \sqrt{\scp{\vu}{\vu} + \scp{\vu}{\vv} + 
                \overline{\scp{\vu}{\vv}} + \scp{\vv}{\vv}} \\
            &\leq \sqrt{\scp{\vu}{\vu} + 2|\scp{\vu}{\vv}| + \scp{\vv}{\vv}}
                \tag{V $\mathbb{C}$: $a + \overline{a} \leq 2 \cdot |a|$} \\
            &\leq \sqrt{\scp{\vu}{\vu} + 2\|\vu\|\|\vv\| + \scp{\vv}{\vv}} \\
            &= \sqrt{\|\vu\|^2 + 2\|\vu\|\|\vv\| + \|\vv\|^2} \\
            &= \sqrt{\left(\|\vu\| + \|\vv\|\right)^2} = \|\vu\| + \|\vv\|
    \end{align*}
\end{proof}

\begin{definition}
    Obecně \newterm{norma} na vektorovém prostoru $V$ je zobrazení 
    $\|\bullet\|: V \rightarrow \mathbb{R}$, které splňuje následující
    podmínky:
    \begin{enumerate}[i.]
        \item $\|\vec{0}\| = 0$
        \item $\|\vu\| \geq 0$
        \item $\|a \cdot \vu\| = a \cdot \|\vu\|$
        \item $\|\vu + \vv\| \leq \|\vu\| + \|\vv\|$
    \end{enumerate}
\end{definition}

\begin{remark}[Příklady norm]
    \leavevmode
    \begin{itemize}
        \item Norma odvozená od skalárního součinu (viz definici 
            \ref{def:normaskalarnisoucin}).
        \item $L_p$ norma, definovaná: $$ \|\vu\|_p = \sqrt[p]{\sum_{i=1}^n
            |u_i|^p}.$$
            Obvyklá Eukleidovská norma ($\sqrt{\sum_{i=1}^n |u_i|^2}$) je tedy
            speciální případ $L_p$ normy s $p = 2$.
    \end{itemize}
\end{remark}

\section{Ortogonalita}

\begin{definition}
    Vektory $\vu$ a $\vv$ z prostoru se skalárním součinem se nazývají vzájemně
    \newterm{kolmé}, pokud platí: $$\scp{\vu}{\vv} = 0.$$ 
    Značíme $\vu \perp \vv$.
\end{definition}

\begin{observation}
    Každý systém vzájemně kolmých netriviálních vektorů je lineárně nezávislý.
\end{observation}

\begin{proof}
    Sporem. Nechť $\vec{u_0} = \sum_{i=1}^n a_i\vec{u_i}$. Potom:
    $$ 0 \neq \scp{\vec{u_0}}{\vec{u_0}} 
        = \scp{\vec{u_0}}{\sum_{i=1}^n a_i\vec{u_i}}
        = \sum_{i=1}^n\overline{a_i}\underbrace{\scp{\vec{u_0}}{\vec{u_i}}}_{0 \text{ pro } \forall i} = 0$$
\end{proof}

\begin{observation}
    Nechť $V$ je vektorový prostor se skalárním součinem. Potom:
    $$\forall \vv \in V: \vv \perp \vzero.$$
\end{observation}

\begin{definition}
    Nechť $V$ je vektorový prostor se skalárním součinem a $Z$ jeho báze 
    taková, že:
    \begin{itemize}
        \item $\forall \vec{v} \in Z: \|\vec{v}\| = 1$
        \item $\forall \vv \neq \vec{v'} \in Z: \vv \perp \vec{v'}$
    \end{itemize}
    Potom se $Z$ nazývá \newterm{ortonormální báze} prostoru $V$.
\end{definition}

\begin{proposition}
    Nechť $Z = \{\vec{v_1}, \dots, \vec{v_n}\}$ je ortonomální báze prostoru 
    $V$. Potom $\forall \vu \in V$ platí:
    $$\vu = \scp{\vu}{\vec{v_1}}\vvone + \scp{\vu}{\vec{v_2}}\vvtwo + 
        \dots + \scp{\vu}{\vec{v_n}}\vvn$$
\end{proposition}

\begin{proof}
    $\vu = \sum_{i=1}^n a_i\vec{v_i}$, a tedy: 
    $[\vu]_Z = \rowvec{a_1,\dots,a_n}^\top$. Potom:
    $$\scp{\vu}{\vec{v_1}} = \scp{\sum_{i=1}^n a_i\vec{v_i}}{\vec{v_1}}
        = \sum_{i=1}^n a_i \scp{\vec{v_i}}{\vec{v_1}} = a_i,$$
    neboť:
    $$ \scp{\vec{v_i}}{\vec{v_1}} = \begin{cases}
            0 \text{ pro } i \neq 1, \\
            1 \text{ pro } i = 1
        \end{cases}$$
\end{proof}

\begin{definition}
    Nechť $W$ je prostor se skalárním součinem, $V$ je podprostor $W$, a
    $Z = \{\vvone, \dots, \vvn\}$ je nějaká ortonormální báze $V$.

    Zobrazení $p_Z: W \rightarrow V$ definované předpisem:
    $$p_Z(\vu) = \sum_{i=1}^n \scp{\vu}{\vvi}\vvi$$
    se nazývá \newterm{ortogonální projekce} prostoru $W$ na $V$.
\end{definition}

\begin{observation}
    Ortogonální projekce je lineární zobrazení.
\end{observation}

\begin{lemma}
    \label{lemma:gso}
    Nechť $p_Z$ je ortogonální projekce prostoru $W$ na $V$. Potom 
    $\forall \vu \in W$ platí:
    $$(\vu - p_Z(\vu)) \perp \vvi \text{ pro } \forall \vvi \in Z.$$
\end{lemma}

\begin{proof}
    $$\scp{\vu - p_Z(\vu)}{\vvi} = \scp{\vu - \sum_{j=1}^n \scp{\vu}{\vvj}
    \vvj}{\vvi} = \scp{\vu}{\vvi} - \sum_{i=n}^n \scp{\vu}{\vvj}
    \scp{\vvj}{\vvi} = \scp{\vu}{\vvi} - \scp{\vu}{\vvi} = 0$$
    jelikož
    $$\scp{\vvj}{\vvi} = \begin{cases}
        1 \text{ pro } i=j, \\
        0 \text{ pro } i \neq j
    \end{cases}$$
\end{proof}

\begin{observation}
    Projekce $p_Z(\vu)$ je nejbližší vektor k vektoru $\vu$, který leží v 
    prostoru $V$.
\end{observation}

\begin{definition}
    V zápise $\vu = \scp{\vu}{\vvone}\vvone + \dots + \scp{\vu}{\vvn}\vvn$
    se koeficienty $\scp{\vu}{\vvi}$ nazývají 
    \newterm{Fourierovy koeficienty}.
\end{definition}

\begin{algorithm}
    \newterm{Gram-Schmidtova ortonormalizace} je postup, který převede 
    libovolnou
    bázi $\{\vec{u_1}, \dots, \vec{u_n}\}$ na ortonormální bázi $\{\vvone,
    \dots, \vvn\}$.

    Postup. Pro $i$ od $1$ do $n$ opakuj:
    \begin{enumerate}
        \item $\vec{w_i} = \vec{u_i} - \sum_{j=1}^{i-1} \scp{\vui}{\vvj}\vvj$,
            neboli odečtení projekce na doposud spočtený lineární obal 
            $\obal(\vvone, \dots, \vec{v_{i-1}})$.
        \item $\vvi = \frac{1}{\|\vec{w_i}} \vec{w_i}$
    \end{enumerate}
\end{algorithm}

\begin{remark}
    Dokažme korektnost Gram-Schmidtovy ortonormalizace.
    \begin{enumerate}
        \item $\vvi \perp \vvj \forall j < i$, dle Lemmatu \ref{lemma:gso}
        \item $\|\vvi\| = \|\frac{1}{\|\vec{w_i}\|}\vec{w_i}\| 
            = \frac{1}{\|\vec{w_i}\|}\|\vec{w_i}\| = 1$
        \item Zůstáváme ve stejném prostoru, neboli: $$\obal(\vvone, \dots, 
        \vec{v_{i-1}}, \vui, \vec{u_{i+1}}, \dots, \vun) 
        = \obal(\vvone, \dots, \vvi, \vec{u_{i+1}}, \dots, \vun),$$ dle 
        Lemmatu \ref{lemma:ovymene}.
    \end{enumerate}
\end{remark}

\begin{definition}
    Nechť $V$ je množina vektorů ve vektorovém prostoru $W$ se skalárním
    součinem. \newterm{Ortogonální doplněk} $V$ je množina $V^\bot$ 
    definována:
    $$V^\bot \coloneqq \{\vu \in W, \forall \vv \in V: \vu \perp \vv \}$$
\end{definition}

\begin{remark}[Příklady ortogonálních doplňků v $\mathrm{R}^3$]
    Doplněk přímky je kolmá rovina. Doplňek roviny je kolmá přímka.
\end{remark}

\begin{observation}
    Pokud $U \subseteq V$, tak potom $U^\bot \supseteq V^\bot$.
\end{observation}

\begin{proof}
    $\vu \in V^\bot \iff \vu \perp \vv \text{ pro } \forall \vv \in V \implies
    \vu \perp \vv \text{ pro } \forall \vv \in U \iff \vu \in U^\bot$
\end{proof}

\begin{theorem}
    Nechť $V$ je podprostorem prostoru $W$ se skalárním součinem. Potom platí:
    \begin{enumerate}[i.]
        \item $V^\bot$ je podprostorem $W$
        \item $V \cap V^\bot = \{\vec{0}\}$
    \end{enumerate}
    Pokud je navíc $W$ konečné dimenze, tak platí:
    \begin{enumerate}[i.]
        \setcounter{enumi}{2}
        \item $\dm V + \dm V^\bot = \dm W$
        \item $(V^\bot)^\bot = V$
    \end{enumerate}
\end{theorem}

\begin{proof}
    \leavevmode
    \begin{enumerate}
        \item[i.] Je třeba ověřit uzavřenost na sčítání a násobení:
            \begin{itemize}
                \item $\scp{\vu + \vv}{\vw} = \scp{\vu}{\vw} + 
                    \scp{\vv}{\vw} = 0 + 0 = 0$
                \item $\scp{a\vu}{\vw} = a\cdot \scp{\vu}{\vw}
                    = a\cdot0 = 0$
            \end{itemize}
        \item[ii.] Pokud $\vu \in V \cap V^\bot$, potom $\scp{\vu}{\vu} = 0$
            a tedy $\vu = \vec{0}$.
        \item[iii.-iv.] Sestrojíme ortonormální bázi $V$ a doplníme ji
            na ortonormální bázi $W$. To, co jsme přidali, je ortonormální
            báze $V^\bot$.
    \end{enumerate}
\end{proof}


\end{document}
