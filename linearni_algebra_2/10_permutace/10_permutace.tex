\section{Permutace}

\begin{definition}
    \newterm{Permutace} na množině $\{1, \dots, n\}$ je zobrazení 
    $p: \{1, \dots, n\} \rightarrow \{1, \dots, n\}$, které je vzájemně
    jednoznačné. Množinu všech permutací na $\{1, \dots, n\}$ budeme 
    značit $S_n$.
\end{definition}

\begin{remark}[Zápis permutace]
    \label{rem:experm}
    \leavevmode
    \begin{itemize}
        \item Tabulkou
            \begin{center}
                \begin{tabular}{c | c c c c}
                    $i$ &1 &2 &3 &4 \\
                    \hline
                    $p(i)$ &4 &2 &1 &3
                \end{tabular}
            \end{center}
            či zkráceně jako $(4,2,1,3)$.
        \item Grafem
            \begin{center}
                \begin{tikzpicture}[node/.style={circle,draw,font=\Large\bfseries}]
                    \tikzset{edge/.style = {->,> = latex'}}
                    \node (n1) at (1,2) {1};
                    \node (n2) at (2,2) {2};
                    \node (n3) at (1,1) {3};
                    \node (n4) at (2,1) {4};

                    \foreach \from/\to in {n1/n4,n4/n3,n3/n1}
                        \draw[edge] (\from) to (\to);
                    \path
                    (n2) edge [loop above] node {} (n2);

                    \node (na1) at (6,2) {1};
                    \node (na2) at (7,2)  {2};
                    \node (na3) at (8,2)  {3};
                    \node (na4) at (9,2) {4};
                    \node (nb1) at (6,1) {1};
                    \node (nb2) at (7,1)  {2};
                    \node (nb3) at (8,1)  {3};
                    \node (nb4) at (9,1) {4};

                    \foreach \from/\to in {na1/nb4,na2/nb2,na3/nb1, na4/nb3}
                        \draw[edge] (\from) to (\to);
                \end{tikzpicture}
            \end{center}
        \item Permutační maticí $\mat{A_p}$: 
            $$ (\mat{A_p})_{ij} = \begin{cases}
                1 &\text{pokud } j=p(i), \\
                0 &\text{jinak.}
            \end{cases}$$
            V našem případě: 
            $$\mat{A_p} = \begin{pmatrix}
                0 &0 &0 &1 \\
                0 &1 &0 &0 \\
                1 &0 &0 &0 \\
                0 &0 &1 &0
            \end{pmatrix}$$
        \item Cykly: $(1,4,3)(2)$
    \end{itemize}
\end{remark}

\begin{observation}
    Skládání permutací není komutativní.
\end{observation}

\begin{proof}
    Použijme zkrácený zápis tabulkou. Potom:
    $(1, 3, 2) \circ (2,1,3) = (3, 1, 2)$, kdežto $(2,1,3) \circ (1,3,2) 
    = (2,3,1)$
\end{proof}

\begin{definition}
    Mějme permutaci $p$. Permutace $p^{-1}$ taková, že $$p \circ p^{-1} = 
    p^{-1} \circ p = id,$$ se nazývá \newterm{inverzní permutace} k 
    permutaci $p$. Plyne, že $p(i) = j \iff p^{-1}(j) = i$.
\end{definition}

\begin{remark}
    K permutaci $p = (4,2,1,3)$ z Poznámky~\ref{rem:experm} je inverzní
    permutací permutace $q = (3,2,4,1)$.
\end{remark}

\begin{definition}
    \newterm{Transpozice} je taková permutace, která obsahuje jeden cyklus
    délky $2$ a $n-2$ cyklů délky $1$.
\end{definition}

\begin{remark}
    Transpozice je tedy taková permutace, ve které si dva prvky prohodí pozice 
    a ostatní prvky se zobrazí na sebe, např.
    \begin{center}
        \begin{tikzpicture}[node/.style={circle,draw,font=\Large\bfseries}]
            \tikzset{edge/.style = {->,> = latex'}}

            \node (na1) at (6,2) {1};
            \node (na2) at (7,2)  {2};
            \node (na3) at (8,2)  {3};
            \node (na4) at (9,2) {4};
            \node (nb1) at (6,1) {1};
            \node (nb2) at (7,1)  {2};
            \node (nb3) at (8,1)  {3};
            \node (nb4) at (9,1) {4};

            \foreach \from/\to in {na1/nb1,na2/nb4,na3/nb3, na4/nb2}
            \draw[edge] (\from) to (\to);
        \end{tikzpicture}
    \end{center}
\end{remark}

\begin{observation}
    Každou permutaci lze složit z transpozic.
\end{observation}

\begin{proof}
    Indukcí podle součtu délek cyklů délky větší než $2$. Použijeme zápis
    pomocí cyklů. V každém kroku rozdělíme cyklus $(1, \dots, k)$ délky $k$
    na cyklus $(1,k)$ délky $2$ a cyklus $(1, \dots, k-1)$ délky $k-1$. 
    Postupně lze takto každý cyklus $(1, \dots, k)$ délky $k$ lze rozdělit 
    na $k-1$ cyklů délky $2$.
\end{proof}

\begin{definition}
    Nechť $p$ je permutace na množině $\{1, \dots, n\}$. Potom 
    \newterm{inverzí} permutace $p$ rozumíme každou dvojici $i,j \in \{1, 
    \dots, n\}: i < j \land p(i) > p(j)$. Celkový počet inverzí permutace
    $p$ budeme značit $I(p)$.
\end{definition}

\begin{remark}
    Vraťme se znovu k permutaci $p = (4,2,1,3)$ z Poznámky~\ref{rem:experm}. 
    V této permutaci se nacházejí inverze $(1,3), (1,4), (1,2)$ a $(2,3)$. 
    
    Celkový počet inverzí pro permutaci zapsanou ve zkráceném 
    zápise tabulkou, $(4,2,1, 3)$ lze vypočítat jednoduše: Bereme postupně 
    čísla od začátku a ptáme se, kolik je před nimi větších čísel. 
    Před $4$ žádné, před $2$ jedno, před $1$ dvě, před $3$ jedno. Proč tento
    postup funguje je zřejmé ze zápisu permutace grafem:
    \begin{center}
        \begin{tikzpicture}[node/.style={circle,draw,font=\Large\bfseries}]
            \tikzset{edge/.style = {->,> = latex'}}
            \node (na1) at (6,2) {1};
            \node (na2) at (7,2)  {2};
            \node (na3) at (8,2)  {3};
            \node (na4) at (9,2) {4};
            \node (nb1) at (6,1) {1};
            \node (nb2) at (7,1)  {2};
            \node (nb3) at (8,1)  {3};
            \node (nb4) at (9,1) {4};

            \foreach \from/\to in {na1/nb4,na2/nb2,na3/nb1, na4/nb3}
            \draw[edge] (\from) to (\to);
        \end{tikzpicture}
    \end{center}

\end{remark}

\begin{definition}
    \newterm{Znaménkem permutace} $p$ se rozumí číslo: $$\sgn(p) \coloneqq
    (-1)^{I(p)}$$
\end{definition}

\begin{observation}
    $\forall p,q \in S_n: \sgn(p) \cdot \sgn(q) = \sgn(q \circ p)$
\end{observation}

\begin{proof}
    Ukážeme, že $\exists k \in \mathbb{Z}: I(q \circ p) = I(p) + I(q) + 2k$.
    Potom už jednoduše vyplyne, že 
    \begin{multline*} 
        \sgn(q \circ p) = (-1)^{I(q \circ p)} = (-1)^{I(p) + I(q) + 2k} = \\
        (-1)^{I(p)} \cdot (-1)^{I(q)} \cdot (-1)^{2k} = 
        (-1)^{I(p)} \cdot (-1)^{I(q)} = \sgn(p) \cdot \sgn(q)
    \end{multline*}
   
    Nechť $i < j$. Potom může nastat jedna z následujících možností:
    \begin{itemize}
            \centering
        \item[$(--)$:] $p(i) < p(j) \land q(p(i)) < q(p(j)),$
        \item[$(-+)$:] $p(i) < p(j) \land q(p(i)) > q(p(j)),$
        \item[$(+-)$:] $p(i) > p(j) \land q(p(i)) > q(p(j)),$
        \item[$(++)$:] $p(i) > p(j) \land q(p(i)) < q(p(j)),$
    \end{itemize}
    kde $+$ značí přítomnost inverze v zobrazení $p$, resp. $q$, a $-$ její
    nepřítomnost. Takto přiřadíme každou dvojici prvků $i<j$ do příslušné 
    přihrádky; výsledné počty prvků v jednotlivých přihrádkách označíme 
    jako $I_{--}, I_{-+}, I_{+-}, I_{++}$.

    Je zřejmé, že dvojice v přihrádkách $(+-)$ a $(++)$ představují 
    inverze 
    v $p$, dvojice v přihřádkách $(-+)$ a $(++)$ inverze v $q$ a nakonec 
    dvojice v přihrádkách $(-+)$ a $(+-)$ inverze v $q \circ p$. Můžeme 
    tedy psát: $$I(q \circ p) = I_{-+} + I_{+-} 
    = (I_{-+} + I_{++}) + (I_{+-} + I_{++}) - 2\cdot I_{++} 
    = I(q) + I(p) + 2k.$$
\end{proof}

\begin{corollary}
    \label{cor:sgnpsgnp-1}
    Nechť $p \in S_n$. Potom $\sgn(p) = \sgn(p^{-1})$.
\end{corollary}

\begin{proof}
    $1 = \sgn(id) = \sgn(p^{-1} \circ p) = \sgn(p^{-1}) \sgn(p),$ z čehož
    nutně vyplývá, že $\sgn(p) = \sgn(p^{-1}).$
\end{proof}

\begin{observation}
    Nechť $t \in S_n$ je transpozice. Potom $\sgn(t) = -1.$
\end{observation}

\begin{proof}
    V transpozici se vyskytuje lichý počet inverzí: 
    \begin{center}
        \begin{tikzpicture}[node/.style={circle,draw,font=\Large\bfseries}]
            \tikzset{edge/.style = {->,> = latex'}}

            \node (na1) at (6,2) {1};
            \node (na1b) at (7,2) {...};
            \node (na2) at (8,2)  {i};
            \node (na3) at (9,2)  {...};
            \node (na4) at (10,2) {j};
            \node (na5) at (11,2) {...};
            \node (nan) at (12,2) {n};
            \node (nb1) at (6,1) {1};
            \node (nb1b) at (7,1) {...};
            \node (nb2) at (8,1)  {i};
            \node (nb3) at (9,1)  {...};
            \node (nb4) at (10,1) {j};
            \node (nb5) at (11,1) {...};
            \node (nbn) at (12,1) {n};

            \foreach \from/\to in {na1/nb1,na1b/nb1b, na2/nb4,na3/nb3, na4/nb2, na5/nb5, nan/nbn}
            \draw[edge] (\from) to (\to);
        \end{tikzpicture}
    \end{center}
\end{proof}

\begin{corollary}
    $sgn(p) = (-1)^{\text{\# transposic v libovolném rozkladu $p$}} = 
    (-1)^{\text{\# sudých cyklů v $p$}}$
\end{corollary}
