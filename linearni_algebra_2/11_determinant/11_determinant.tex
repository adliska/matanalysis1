\section{Determinant}

\subsection{Definice a základní vlastnosti}

\begin{definition}
    \label{def:det}
    Nechť $\mat{A}$ je čtvercová matice řádu $n$ nad tělesem $K$
    Potom \newterm{determinant} matice $\mA$ je dán výrazem:
    $$\det(\mA) \coloneqq \sum_{p \in S_n} \sgn(p)\cdot 
    \prod_{i=1}^n a_{i, p(i)}.$$
    Formálně jde o zobrazení $\det(\bullet): K^{n \times n} \rightarrow K.$
\end{definition}

\begin{remark}
    Mějme matici $\mA = \begin{pmatrix} a_{11} &a_{12} \\ a_{21} &a_{22}
    \end{pmatrix}$. Množina všech permutací na množině $\{1, 2\}$, t.j.
    $S_2$, obsahuje dva prvky: identitu $(1,2)$ se znaménkem $1$ a 
    transposici $(2,1)$ se znaménkem $-1$. Potom:
    $$\det(\mA) = (+1) \cdot a_{11}a_{22} + (-1) \cdot a_{12}a_{21}.$$
\end{remark}

\begin{observation}
    \label{obs:detaeqdetat}
    $\det(\mA) = \det(\mAt)$
\end{observation}

\begin{proof}
    \begin{align*}
        \det(\mAt) &= \sum_{p \in S_n} \sgn(p) \prod_{i=1}^n 
            (\mAt)_{i,p(i)} \tag{definice determinantu}\\
            &= \sum_{p \in S_n} \sgn(p) \prod_{i=1}^n a_{p(i),i} 
            \tag{transpozice matice $\mA$} \\
            &= \sum_{p \in S_n} \sgn(p) \prod_{i=1}^n a_{i,q(i)} 
            \tag{při volbě $q = p^{-1}$} \\
            &= \sum_{q \in S_n} \sgn(q) \prod_{i=1}^n a_{i,q(i)} 
            \tag{$\sgn(p) = \sgn(q)$, Důsledek~\ref{cor:sgnpsgnp-1}} \\
            &= \det(\mA) \tag{dle Definice \ref{def:det}}
    \end{align*}
\end{proof}

\begin{observation}
    \label{obs:prerovnanisloupcu}
    Přerovnání sloupců matice $\mA$ podle permutace $q$:
    \begin{itemize}
        \item nezmění $\det(\mA)$, je-li $\sgn(q) = 1$,
        \item změní znaménko $\det(\mA)$, je-li $\sgn(q) = -1$.
    \end{itemize}
\end{observation}

\begin{proof}
    Označme $\mat{A'}$ přerovnanou matice. Platí $a'_{ij} = 
    a_{i, q^{-1}(j)}.$ Dále:
    \begin{align*}
        \det(\mat{A'}) &= \sum_{p \in S_n} \sgn(p) \prod_{i=1}^n a'_{i,p(i)} \\
            &= \sum_{p \in S_n} \sgn(p) \prod_{i=1}^n a_{i, q^{-1}(p(i))} \\
            &= \sum_{p \in S_n} 
                \lefteqn{\underbrace{\phantom{\sgn(q)\sgn(q^{-1})}}_{=1}}
                \sgn(q) 
                \overbrace{\sgn(q^{-1})\sgn(p)}^{=\sgn(q^{-1} \circ p)}
                \prod_{i=1}^n a_{i, q^{-1}(p(i))} \\
            &= \sgn(q) \sum_{p \in S_n} \sgn(q^{-1} \circ p) \prod_{i=1}^n
                a_{i, (q^{-1}\circ p)(i)} \\
            &= \sgn(q)\det(\mA)
    \end{align*}
\end{proof}

\begin{remark}[Důsledky Pozorování~\ref{obs:prerovnanisloupcu}]
    \label{rem:prerovnani}
    \leavevmode
    \begin{enumerate}[i.]
        \item Stejné tvrzení platí i pro řádky, díky 
            Pozorování~\ref{obs:detaeqdetat} ($\det(\mA) = \det(\mAt)$).
        \item Záměna dvou řádků (sloupců) změní znaménko determinantu.
        \item Jsou-li dva řádky stejné, je determinant 
            nulový.\footnote{V tělesech charakteristiky $2$ může být i $-1$.}
    \end{enumerate}
\end{remark}

\begin{proposition}
    \label{prop:linearitadet}
    Determinant je lineární funkcí každého řádku i sloupce dané matice.
    \begin{enumerate}[i.]
        \item Linearita vůči skalárnímu násobku:
            $$\begin{vmatrix}
                  a_{1,1} &\dots &a_{1,n} \\
                \vdots &\vdots &\vdots \\
                ta_{i,1} &\dots &ta_{i,n} \\
                \vdots &\vdots &\vdots \\
                a_{n,1} &\dots &a_{n,n} 
            \end{vmatrix} = t \begin{vmatrix}
                  a_{1,1} &\dots &a_{1,n} \\
                \vdots &\vdots &\vdots \\
                a_{i,1} &\dots &a_{i,n} \\
                \vdots &\vdots &\vdots \\
                a_{n,1} &\dots &a_{n,n} 
            \end{vmatrix}$$
         \item Linearita vůči sčítání:
            $$\begin{vmatrix}
                  a_{1,1} &\dots &a_{1,n} \\
                \vdots &\vdots &\vdots \\
                a_{i,1}+b_{i,1} &\dots &a_{i,n}+b_{i,n}\\
                \vdots &\vdots &\vdots \\
                a_{n,1} &\dots &a_{n,n} 
            \end{vmatrix} = \begin{vmatrix}
                  a_{1,1} &\dots &a_{1,n} \\
                \vdots &\vdots &\vdots \\
                a_{i,1} &\dots &a_{i,n} \\
                \vdots &\vdots &\vdots \\
                a_{n,1} &\dots &a_{n,n} 
            \end{vmatrix} + \begin{vmatrix}
                  a_{1,1} &\dots &a_{1,n} \\
                \vdots &\vdots &\vdots \\
                b_{i,1} &\dots &b_{i,n} \\
                \vdots &\vdots &\vdots \\
                a_{n,1} &\dots &a_{n,n} 
            \end{vmatrix}$$
       \end{enumerate}
\end{proposition}

\begin{proof}
    \leavevmode
    \begin{enumerate}[i.]
        \item Linearita vůči skalárnímu násobku:
            \begin{align*}
                \det(\mat{A'}) 
                    &= \sum_{p \in S_n} \sgn(p) \prod_{i=1}^n a'_{i, p(i)} \\
                    &= \sum_{p \in S_n} \sgn(p) (a_{1,p(1)} \cdot ~ \dots ~
                        \cdot ta_{i,p(i)} \cdot ~ \dots ~ \cdot a_{n,p(n)}) \\
                    &= t \det(\mA)
            \end{align*}
        \item Analogicky.
    \end{enumerate}
\end{proof}

\begin{corollary}
    Nechť $\mA \in K^{n\times n}$ a $i,j \leq n, i \neq j$. Potom přičtení 
    $t$-násobku $j$-tého řádku k $i$-tému nezmění determinant.
\end{corollary}

\begin{proof}
    Nechť $\mat{A'}$ je výsledná matice. Potom dle 
    Tvrzení~\ref{prop:linearitadet}:
    \begin{align*}
        \det(\mat{A'}) &= \begin{vmatrix} 
                            a_{1,1} &\dots &a_{1,n} \\
                            \vdots &\vdots &\vdots \\
                            a_{i,1}+ta_{j,1} &\dots &a_{i,n}+ta_{j,n}\\
                            \vdots &\vdots &\vdots \\
                            a_{n,1} &\dots &a_{n,n}
                        \end{vmatrix} \\
                        &= \underbrace{\begin{vmatrix} 
                            a_{1,1} &\dots &a_{1,n} \\
                            \vdots &\vdots &\vdots \\
                            a_{i,1} &\dots &a_{i,n}\\
                            \vdots &\vdots &\vdots \\
                            a_{n,1} &\dots &a_{n,n}
                \end{vmatrix}}_{=\det(\mA)} + t \cdot \underbrace{\begin{vmatrix} 
                            a_{1,1} &\dots &a_{1,n} \\
                            \vdots &\vdots &\vdots \\
                            a_{j,1} &\dots &a_{j,n}\\
                            \vdots &\vdots &\vdots \\
                        a_{n,1} &\dots &a_{n,n}
                        \end{vmatrix}}_{=\det(\mB)}\\
                        &= \det(\mA)
    \end{align*}
    Matice označená jako $\mB$ má na místě $i$-tého řádku kopii řádku $j$-tého.
    Jelikož má dva řádky stejné, její determinant je roven nule (podle bodu iii.
    Poznámky~\ref{rem:prerovnani}).
\end{proof}

\begin{remark}[Výpočet determinantu]
    Matici převedeme na trojúhelníkovou přičítáním $t$-násobků jiných řádku,
    podobně jako při Gaussově eliminaci. Potom $$\det(\mA) = a_{11}a_{22}\dots
    a_{nn}.$$
    
    Pamatujme, že:
    \begin{itemize}
        \item buď neměníme pořadí řádků, anebo si pamatujeme změnu znaménka.
        \item nenásobíme řádky $t$, neboť by se determinant změnil $t$-krát.
        \item lze upravovat i po sloupcích.
    \end{itemize}

    Takto lze vypočítat determinant v polynomiálním čase $\bigo{n^3}$, podobně 
    jako Gaussova eliminace.
\end{remark}

\begin{theorem}
    \label{th:detabdetadetb}
    Nechť $\mA$ a $\mB$ jsou čtvercové matice nad T. Potom platí:
    $$ \det(\mA\cdot\mB) = \det(\mA) \det(\mB).$$
\end{theorem}

\begin{proof}
    Je-li $\mA$ nebo $\mB$ singulární, je i jejich součin singulární. Potom
    $\det(\mA\cdot \mB) = 0 = \det(\mA)\det(\mB)$.

    Předpokládejme, že matice $\mA$ je regulární. Rozložíme $\mA$
    jako součin matic elementárních úprav: 
    $\mA = \mat{E_1}\mat{E_2} \dots \mat{E_k}.$ Potom:
    \begin{align*} 
        \det(\mA\mB) &= \det(\mat{E_1}\mat{E_2}\dots\mat{E_k}\mB) \\
                     &\stackrel{*}{=} \det(\mat{E_1}) 
                        \det(\mat{E_2}\dots\mat{E_k}\mB) \\
                        &= \det(\mat{E_1}) \dots \det(\mat{E_n}) \det(\mB)
                        \tag{iterací předchozího kroku} \\
                        &= \det(\underbrace{\mat{E_1}\dots\mat{E_n}}_{\mA})
                        \det(\mB)
    \end{align*}

    Rovnítko s hvězdičkou platí, jelikož:
    \begin{itemize}
        \item pokud je $\mat{E_1}$ matice záměny dvou řádku, je 
            $\det(\mat{E_1}\mC) = -1\det(\mC)$ a $\det(\mat{E_1})\det(\mC) = 
            -1\det(\mC)$,
        \item pokud je $\mat{E_1}$ matice vynásobení řádku skalárem $t$, potom
            $\det(\mat{E_1}\mC) = t\det(\mC)$ a $\det(\mat{E_1})\det(\mC) = 
            t\det(\mC)$,
        \item pokud je $\mat{E_1}$ matice přičtení $j$-tého řádku k $i$-tému,
            potom
            $\det(\mat{E_1}\mC) = 1\det(\mC)$ a $\det(\mat{E_1})\det(\mC) = 
            1\det(\mC)$.
    \end{itemize}
\end{proof}

\begin{theorem}
    Čtvercová matice $\mA$ je regulární, právě když $\det(\mA) \neq 0$
\end{theorem}

\begin{corollary}
    \label{cor:detai}
    Je-li matice $\mA$ regulární, potom:
    $$\det(\invmat{A}) = \frac{1}{\det(\mA)}.$$
\end{corollary}

\begin{proof}
     $$1 = \det(\identmat{n}) = \det(\mA \mAi) = \det(\mA)\det(\mAi)$$
\end{proof}

\begin{observation}[Rozvoj determinantu podle $i$-tého řádku]
    \label{obs:rozvojdet}
    Nechť $\mat{A^{ij}}$ značí matici, která vznikne z 
    $\mA \in K^{n \times n}$ vypuštěním $i$-tého řádku a $j$-tého sloupce.
    Pak platí pro libovolné $i \in \{1, \dots, n\}$:
    $$\det(\mA) = \sum_{j=1}^n a_{ij} (-1)^{i+j} \det(\mat{A^{ij}}).$$
\end{observation}

\begin{proof}
    \begin{align*}
        \begin{vmatrix}
            \vdots &\vdots &\vdots &\vdots \\
            a_{i1} &a_{i2} &\dots &a_{in} \\
            \vdots &\vdots &\vdots &\vdots
        \end{vmatrix}
        &=  
        \begin{vmatrix}
            \vdots &\vdots &\vdots &\vdots \\
            a_{i1} &0 &\dots &0 \\
            \vdots &\vdots &\vdots &\vdots
        \end{vmatrix}
        +
        \begin{vmatrix}
            \vdots &\vdots &\vdots &\vdots \\
            0 &a_{i2} &\dots &0 \\
            \vdots &\vdots &\vdots &\vdots
        \end{vmatrix}
        +
        \dots
        +
        \begin{vmatrix}
            \vdots &\vdots &\vdots &\vdots \\
            0 &0 &\dots &a_{in} \\
            \vdots &\vdots &\vdots &\vdots
        \end{vmatrix}
        \tag{Tvrzení~\ref{prop:linearitadet}} \\
        &= 
        a_{i1}
        \begin{vmatrix}
            \vdots &\vdots &\vdots &\vdots \\
            1 &0 &\dots &0 \\
            \vdots &\vdots &\vdots &\vdots
        \end{vmatrix}
        +
        a_{i2}
        \begin{vmatrix}
            \vdots &\vdots &\vdots &\vdots \\
            0 &1 &\dots &0 \\
            \vdots &\vdots &\vdots &\vdots
        \end{vmatrix}
        +
        \dots
        +
        a_{in}
        \begin{vmatrix}
            \vdots &\vdots &\vdots &\vdots \\
            0 &0 &\dots &1 \\
            \vdots &\vdots &\vdots &\vdots
        \end{vmatrix}
        \tag{Tvrzení~\ref{prop:linearitadet}} \\
        &=
        \sum_{j=1}^n a_{ij}(-1)^{i+j}\det(\mat{A^{ij}}) 
        \tag{viz výklad níže}
    \end{align*}

    Uvažujme následující matici:
    $$\mC = \begin{pmatrix}
        a_{11} &\dots  &a_{1j} &\dots  &a_{1n} \\
        \vdots &\vdots &\vdots &\vdots &\vdots \\
        0      &\dots  &1      &\dots  &0 \\
        \vdots &\vdots &\vdots &\vdots &\vdots\\
        a_{n1} &\dots  &a_{nj} &\dots  &a_{nn}
      \end{pmatrix}$$
    tj., matici $\mA$, jejíž $i$-tý řádek obsahuje v $j$-tém sloupci jedničku
    a jinak nuly. Postupným vyměňováním 
    řádků $(i,i+1), \dots, (n-1, n)$ se posune $i$-tý řádek na konec matice. 
    Podobně přesuneme $j$-tý sloupec na konec matice a získáme matici:
    $$\mat{C'} = \begin{pmatrix}[ccc|c]
          & & & \\
          &\mat{A^{ij}} & & \\
          &  & & \\
        \hline 
        0 &\dots  &0      &1  
      \end{pmatrix}$$
    jejíž determinant je roven $\det(\mat{A^{ij}})$. Při přesouvání řádků
    a sloupců jsme využili $(n-i) + (n-j)$ transpozic, a tudíž:
    $$\det(\mC) = (-1)^{(n-i) + (n-j)} \det(\mat{A^{ij}}) = 
    (-1)^{(i+j)}\det(\mat{A^{ij}}).$$
\end{proof}


\begin{definition}
    Pro čtvercovou matici $\mA$ definujeme \newterm{adjungovanou matici}
    $\adj(\mA)$:
    $$\adj(\mA)_{ij} = (-1)^{i+j} (\det(\mat{A^{ji}}))$$
    (Poznámka: Všimněte si prohozeného pořadí indexů.)
\end{definition}

\begin{theorem}
    Pro každou regulární matici $\mA$ nad tělesem T platí:
    $$ \mAi = \frac{1}{\det(\mA)} \adj(\mA).$$
\end{theorem}

\begin{proof}
    Z Pozorování~\ref{obs:rozvojdet} plyne, že
    $$\text{$i$-tý řádek $\mA\cdot i$-tý sloupec $\adj(\mA) = \det(\mA)$}$$
    Podobně:
    \begin{multline*}
        \text{$k$-tý řádek $\mA\cdot i$-tý sloupec $\adj(\mA) = $} \\ 
        \text{determinant matice vzniklé z $\mA$ překopírováním $k$-tého 
        řádku na $i$-tý}
    \end{multline*}
    Výraz $\mA \cdot \adj(\mA)$ je tedy roven matici, která má na diagonále 
    $\det(\mA)$ a mimo diagonálu nuly.
\end{proof}

\begin{theorem}[Cramerovo pravidlo]
    Nechť $\mA$ je regulární matice. Řešení soustavy $\mA\vx=\vb$ lze zapsat
    jako:
    $$x_i = \frac{\det(\mat{A_{i\rightarrow\vb}})}{\det(\mA)},$$
    kde $\mat{A_{i\rightarrow\vb}}$ je matice, která vznikne z $\mA$ 
    nahrazením $i$-tého sloupce vektorem $\vb$.
\end{theorem}

\begin{proof}
    \begin{multline*}
        \mA\vx=\vb \implies \vx = \mAi\vb = \frac{1}{\det(\mA)} \adj(\mA)\vb \\
        \implies x_i = \frac{1}{\det(\mA)}(\adj(\mA)\vb)_i = \frac{1}{\det(\mA)}
        \sum_{j=1}^n(\adj(\mA))_{ij}b_j = \frac{1}{\det(\mA)} 
        \det(\mat{A_{i\rightarrow\vb}})
    \end{multline*}
\end{proof}

\subsection{Poznámky na konec}

\begin{remark}[Různé druhy obalů]
    Nechť $X = \{\vxone, \dots, \vxn\}$ je konečná množina v $\Rn$. Definujme
    následující obaly:
    \begin{itemize}
        \item \newterm{Lineární obal}: $\obal(X) = \left\{ \sum a_i\vxi, a_i 
            \in \R \right\}$
        \item \newterm{Afinní obal}: $\aff(X) = \left\{ \sum a_i\vxi, a_i 
            \in \R, \sum a_i = 1 \right\}$ 
        \item \newterm{Konvexní obal}: $\conv(X) = \left\{ \sum a_i\vxi, a_i 
            \in \R, \sum a_i = 1, a_i \in [0,1] \right\}$ 
        \item \newterm{Rovnoběžnostěn}: $\mathop{\mathrm{P}}(X) = \left\{ 
                \sum a_i\vxi, a_i \in \R, a_i \in [0,1] \right\}$
    \end{itemize}
\end{remark}

\begin{remark}[Geometrický význam determinantu]
    $|\det(\mA)|$ udává objem rovnoběžnostěnu určeného řádky nebo sloupci
    matice $\mA.$
\end{remark}

\begin{observation}
    Je-li $f: \Rn \rightarrow \Rn$ lineární zobrazení a $\mA$ je matice tohoto
    zobrazení, potom platí, že objem těles se mění podle předpisu:
    $$ \mathop{\mathrm{vol}}(f(V)) = |\det({\mA})| 
    \underbrace{\mathop{\mathrm{vol}}(V)}_{\text{původní objem}}.$$
\end{observation}

