\section{Vlastní čísla a vlastní vektory}

\subsection{Úvod}

\begin{definition}
    Nechť $V$ je vektorový prostor nad tělesem $T$ a $f: V \rightarrow V$
    je lineární zobrazení. Potom $\lambda \in T$, pro nějž existuje nenulový
    vektor $\vx \in V$ takový, že $f(\vx) = \lambda\vx$, nazveme 
    \newterm{vlastním číslem} zobrazení $f$.

    \newterm{Vlastní vektor} příslušný vlastnímu číslu $\lambda$ je každý
    nenulový vektor $\vx \in V$ splňující $f(\vx) = \lambda\vx.$

    Je-li $\dm(V) = n < \infty$, potom lze $f$ reprezentovat maticí zobrazení
    $\mA \in T^{n \times n}$ a můžeme předchozí definici rozšířit i na matice.
 
    Množina všech vlastních čísel matice či zobrazení se nazývá
    \newterm{spektrum}.
\end{definition}

\begin{observation}
    Množina vlastních vektorů příslušných k pevnému vlastnímu číslu $\lambda$
    tvoří podprostor $V$.
\end{observation}

\begin{proof}
    Nechť $\vx$ a $\vy$ jsou vlastní vektory příslušné k vlastnímu číslu 
    $\lambda$. Potom:
    $f(c\vx) = cf(\vx) = \lambda c \vx$ a $f(\vx + \vy) = f(\vx) + f(\vy) = \lambda
    \vx + \lambda \vy = \lambda (\vx + \vy)$.
\end{proof}

\begin{remark}[Geometrický význam vlastních vektorů]
    Vlastní vektory jsou vektory, které nemění směr při zobrazení $f$.
\end{remark}

\begin{theorem}
    \label{th:indeigenvectors}
    Nechť $f: V \rightarrow V$ je lineární zobrazení, $\lambda_1, \dots, 
    \lambda_k$ jsou navzájem různá vlastní čísla zobrazení $f$ a $\vxone,
    \dots, \vxk$ jsou některé vlastní vektory příslušné k $\lambda_1,
    \dots, \lambda_k$. Potom $\vxone, \dots, \vxk$ jsou lineárně nezávislé.
\end{theorem}

\begin{proof}
    Větu dokážeme indukcí a sporem zároveň. 

    Pro $k=1$ věta platí triviálně. 
    
    Předpokládejme, že věta platí pro $k-1$, a předpokládejme dále, že 
    $\vxone, \dots, \vxk$ jsou lineárně závislé, 
    t.j. $\exists a_1, \dots,
    a_n: \sum_{i=1}^k a_i\vxi = \vzero.$ Potom:
    $$ \vzero = f(\vzero) = f(\sum_{i=1}^k a_i\vxi) = \sum_{i=1}^k a_i f(\vxi) =
    \sum_{i=1}^k a_i \lambda_i x_i$$
    a
    $$\vzero = \lambda_k \vzero = \lambda_k \sum_{i=1}^k a_i\vxi = \sum_{i=1}^k
    a_i\lambda_k\vxi.$$
    Potom:
    $$\vzero = \vzero - \vzero = \sum_{i=1}^k a_i \lambda_i \vxi - \sum_{i=1}^k
    a_i\lambda_k\vxi = \sum_{i=1}^{k} a_i(\lambda_i - \lambda_k)\vxi =
    \sum_{i=1}^{k-1} a_i\lambda_i\vxi.$$
    Tedy, $\vxone, \dots, \vec{x_{k-1}}$ jsou lineárně závislé, což je spor
    s předpoklady.
\end{proof}

\begin{corollary}
    Čtvercová matice řádu $n$ může mít nejvýše $n$ vlastních čísel.
\end{corollary}

\subsection{Podobné a diagonizovatelné matice}

\begin{remark}
    Vztah zobrazení $f$ a matice zobrazení $\mA = [f]_{XX}$ není jednoznačný, 
    jelikož matice zobrazení závisí na volbě báze $X$. Všechny matice
    zobrazení ovšem musejí mít stejná vlastní čísla (jelikož ta jsou
    daná zobrazením). Mějme dvě různé báze $X$ a $Y$. Potom:
    \begin{align*}
        [f(\vu)]_X &= [f]_{XX}[\vu]_X \\
        [f(\vu)]_X &= [id]_{YX}[f]_{YY}[id]_{XY}[\vu]_X \\
    \end{align*}
    a tedy:
    $$[f]_{XX} = [id]_{YX}[f]_{YY}[id]_{XY}.$$
\end{remark}

\begin{definition}
    Čtvercové matice $\mA$ a $\mB$ stejného řádu se nazývají 
    \newterm{podobné}, pokud existuje regulární matice $\mR$ taková, že
    $\mA = \mRi \mB \mR.$
\end{definition}

\begin{theorem}
    Jsou-li matice $\mA$ a $\mB$ podobné, t.j. $\mA = \mR \mB \mRi$, 
    $\lambda$ je vlastní číslo
    matice $\mA$ a $\vx$ je příslušný vlastní vektor, tak potom $\lambda$
    je také vlastní číslo matice $\mB$ a $\vy = \mR \vx$ je příslušný
    vlastní vektor matice $\mB$.
\end{theorem}

\begin{proof}
    $\mB\vy = \underbrace{\mR \mA \mRi}_{\mB} \underbrace{\mR \vx}_{\vy} = 
    \mR \mA \vx = \mR \lambda \vx = \lambda \mR \vx = \lambda \vy$
\end{proof}

\begin{observation}
    Vlastní čísla diagonální matice jsou prvky na diagonále a vlastní 
    vektory jsou příslušné vektory kanonické báze.
\end{observation}

\begin{definition}
    Matice se nazývá \newterm{diagonizovatelná}, pokud je podobná
    nějaké diagonální matici.
\end{definition}

\begin{remark}[Užití diagonalizace]
    \leavevmode
    \begin{itemize}
        \item Snadný popis vlastních čísel a vlastních vektorů: $\lambda_i
            = (\mat{D})_{ii}$, $\vec{e_i}$ je příslušný vlastní vektor.
        \item Snadný výpočet mocnin: Nechť $\mA = \mRi \mD \mR$. Potom:
            $$ \mA^n = (\mRi \mD \mR) (\mRi \mD \mR) \dots (\mRi \mD \mR)
            = \mRi \mD^n \mR,$$
            kde $(\mD^n)_{ii} = ((\mD)_{ii})^n.$
    \end{itemize}
\end{remark}

\begin{theorem}
    Pokud má matice $\mA \in K^{n \times n}$ $n$ různých vlastních čísel, tak potom
    je diagonizovatelná.
\end{theorem}

\begin{proof}
    $\lambda_1, \dots, \lambda_n$ vlastní čísla, $\vx_1, \dots, \vx_n$
    příslušné lineárně nezávislé vektory. Sestavme matici $\mR$, jejíž
    sloupce jsou vektory $\vx_1$ až $\vx_n$. Tato matice je regulární,
    jelikož je sestavena z lineárně nezávislých vektorů 
    (viz Větu~\ref{th:indeigenvectors}). Potom $\mA\mR = \mR \mD,$
    kde:
    $$\mD = \begin{pmatrix}
        \lambda_1 &0         &\dots  &0 \\
                0 &\lambda_2 &\dots  &0 \\
                  &          &\ddots  &  \\
                0 &\dots     &\dots  &\lambda_n
    \end{pmatrix},$$
    a $\mD = \mRi \mA \mR.$
\end{proof}

\begin{theorem}
    Matice $\mA \in T^{n \times n}$ je diagonizovatelná, právě když
    má $n$ lineárně nezávislých vlastních vektorů.
\end{theorem}

\begin{proof}
    \leavevmode
    \begin{itemize}
        \item[$\implies$] Existuje $\mR$ regulární: $\mRi \mA \mR = \mD$,
            neboli $\mA \mR = \mR \mD$. Sloupce $\mR$ jsou vlastní vektory a
            ty jsou lineárně nezávislé, jelikož $\mR$ je regulární.
        \item[$\impliedby$] Z vlastních vektorů sestavíme $\mR$, ta je 
            regulární: $\mA \mR = \mR \mD$ a $\mRi \mA \mR = \mD$.
    \end{itemize}
\end{proof}

\subsection{Charakteristický mnohočlen}

\begin{definition}
    Nechť $\mA$ je čtvercová matice nad tělesem $T$. Potom 
    \newterm{charakteristický mnohočlen} v proměnné $t$ je dán
    předpisem:
    $$p_\mA(t) \coloneqq \det(\mA - t\mIn)$$
\end{definition}

\begin{theorem}
    Pro matici $\mA \in T^{n \times n}$ platí: $\lambda \in T$ je vlastní
    číslo matice $\mA$, právě když je kořenem charakteristického 
    mnohočlenu.
\end{theorem}

\begin{proof}
    $\lambda$ je vlastní číslo $\mA \iff \exists \vx \neq \vzero: \mA\vx
    = \lambda\vx \iff \exists \vx \neq \vzero: \mA\vx - \lambda\vx =
    \mA \vx - \lambda \mIn \vx = \vzero \iff \exists \vx \neq \vzero:
    (\mA - \lambda\mIn) \vx = \vzero \iff$ matice $\mA - \lambda\mIn$ 
    je singulární $\iff \det(\mA - \lambda \mIn) = 0 \iff 
    p_\mA(\lambda) = 0 \iff \lambda$ je kořenem $p_\mA(t)$.
\end{proof}

\begin{proposition}
    Podobné matice mají stejné charakteristické mnohočleny.
\end{proposition}

\begin{proof}
    Uvažujme $\mA = \mRi \mB \mR$. Potom 
    \begin{align*}
        p_\mA(t) &= \det(\mA-t\mIn) \\
                 &= \det(\mRi \mB \mR - t \mRi \mIn \mR) 
                    \tag{Věta~\ref{th:detabdetadetb}}\\
                 &= \det(\mRi (\mB - t\mIn) \mR) \\
                 &= \det(\mRi) \det(\mB - t\mIn) \det(\mR)
                    \tag{Věta~\ref{th:detabdetadetb}, 
                        Důsledek~\ref{cor:detai}}\\
                 &= \det(\mB -t\mIn) \\
                 &= p_\mB(t)
    \end{align*}
\end{proof}

\begin{proposition}
    Pro libovolné čtvercové matice $\mA$ a $\mB$ mají matice $\mA\mB$
    a $\mB\mA$ stejná vlastní čísla.
\end{proposition}

\begin{proof}
    Použijeme pravidlo pro násobení blokových matic:
    $$\begin{pmatrix}
        \mA\mB &\mzero \\
        \mB &\mzero
    \end{pmatrix}
    \begin{pmatrix}
        \mat{I} &\mA \\
        \mzero &\mat{I}
    \end{pmatrix}
    =
    \begin{pmatrix}
        \mA\mB &\mA\mB\mA \\
        \mB &\mB\mA
    \end{pmatrix}
    =
    \begin{pmatrix}
        \mat{I} &\mA \\
        \mzero &\mat{I}
    \end{pmatrix}
    \begin{pmatrix}
        \mzero &\mzero \\
        \mB &\mB\mA
    \end{pmatrix}$$
    Tedy, $\begin{pmatrix}
        \mA\mB &\mzero \\
        \mB &\mzero
    \end{pmatrix}$ je podobná
    $\begin{pmatrix}
        \mzero &\mzero \\
        \mB &\mB\mA
    \end{pmatrix}$, tím pádem mají stejné charakteristické mnohočleny:
    \begin{align*}
    \det\begin{pmatrix}
        \mA\mB-t\mat{I} &\mzero \\
        \mB &-t\mat{I}
    \end{pmatrix} &= \det(\mA\mB - t\mat{I}) \cdot (-t)^n 
    = p_{\mA\mB}(t)\cdot(-t)^n \\
    \det\begin{pmatrix}
        -t\mat{I} &\mzero \\
        \mB &\mB\mA -t\mat{I}
    \end{pmatrix} &= (-t)^n \det(\mB\mA-t\mat{I}) = (-t)^n \cdot p_{\mB\mA}(t)
    \end{align*}
    Vyplývá, že $p_{\mA\mB}(t) = p_{\mB\mA}(t)$.
\end{proof}

\begin{theorem}[Základní věta algebry]
    Každý mnohočlen stupně alespoň 1 v $\mathbb{C}$ má alespoň jeden kořen.
\end{theorem}

\begin{proof}
    Důkaz je poměrně těžký -- větu necháme bez důkazu.
\end{proof}

\begin{corollary}
    Každý komplexní mnohočlen lze rozložit na součin jednočlenů.
\end{corollary}

\begin{proof}
    Nechť $p(t) = a_nt^n + \dots + a_1t + a_0$ je komplexní mnohočlen
    a $\lambda_1$ jeho kořen. Jelikož $p(t)$ je dělitelný $(t-\lambda_1)$ 
    beze zbytku, je $\frac{p_\mA(t)}{t-\lambda_1}$ opět
    mnohočlen, stupně $n-1$. $p_\mA(t)$ lze tedy rozložit: 
    $$p_\mA(t) = a_n(t-\lambda_1)^{r_1} \cdot
    (t-\lambda_2)^{r_2} \dots  (t - \lambda_k)^{r_k},$$ kde $\lambda_1,
    \dots, \lambda_k$ jsou navzájem různá komplexní čísla a $r_1 + \dots + r_k = n$.
    Číslo $r_i$ se nazývá \newterm{násobnost kořene} $\lambda_i$.
\end{proof}

\begin{observation}
    Nechť $p_\mA(t) = \det(\mA -t\mIn).$ Potom lze $p_\mA(t)$ vyjádřit jako:
    $$p_\mA(t) = a_nt^n + a_{n-1}t^{n-1} + \dots + a_1t + a_0.$$
    Platí:
    \begin{enumerate}[i.]
        \item $a_n = (-1)^n$
        \item $a_0 = \lambda_1^{r_1} \cdot \lambda_2^{r_2} \dots \lambda_k^{r_k}
            = \det(\mA)$
        \item $a_{n-1} = (-1)^{n-1}(\lambda_1r_1 + \lambda_2r_2 + \dots + 
            \lambda_kr_k) = (-1)^{n-1}((\mA)_{11} + (\mA)_{22} + \dots +
            (\mA)_{nn})$
    \end{enumerate}
\end{observation}

\begin{proof}
    \leavevmode
    \begin{enumerate}[i.]
        \item Veškerá $(-t)$ se vyskytují na diagonále a existuje pouze jeden
            způsob, jak je vybrat.
        \item Po dosazení $t=0: p_\mA(0) = \det(\mA) = \lambda_1^{r_1}\dots
            \lambda_k^{r_k}$.
        \item $t^{(n-1)}$ lze ze součinu $(t - \lambda_1)^{r_1}\dots(t - 
            \lambda_k)^{r_k}$ získat $n$ způsoby; každý dává $\lambda_i$.
            Navíc, z definice determinantu, $t^{n-1}$ lze získat pouze
            ze součinu odpovídajícímu identitě: $(t - (\mA)_{11})\dots
            (t-(\mA)_{nn})$ (není možné ``uhnout'' z diagonály).
    \end{enumerate}
\end{proof}

\begin{proposition}
    Matice $\mA \in \Cnn$ je diagonizovatelná, právě když pro každé vlastní 
    číslo $\lambda_i$ platí:
    $$\dm(\Ker(\mA - \lambda_i\mIn)) = r_i.$$
\end{proposition}

\begin{proof}
    Matice $\mA$ je diagonizovatelná, právě když existuje báze v $\Cn$ složená
    z vlastních vektorů. Pro vektory v prostoru $\Ker(\mA - \lambda_i\mIn)$ 
    platí: $(\mA - \lambda_i\mIn)\vx = \vzero,$ a tedy $\mA\vx = \lambda_i\vx$.

    Z takto získaných vektorů složíme regulární matici $\mR$ a potom
     $\mA \mR = \mR \mD.$
\end{proof}

\begin{remark}
    Existují matice, které nejsou diagonizovatelné. Například:
    $$\begin{pmatrix}
        1 &1 \\
        0 &1
    \end{pmatrix}$$
    Tato matice má jedno vlastní číslo $\lambda = 1$ násobnosti 2. Pokud by 
    byla diagonizovatelná, musela by být podobná matici $\mD = \begin{pmatrix}
        1 &0 \\
        0 &1
    \end{pmatrix}$. Potom: $\mA = \mRi \mD \mR = \mat{I_2}$ a dostáváme spor.
\end{remark}

\begin{definition}
    Nechť $\lambda \in \C, k \in \N$. \newterm{Jordanova buňka} 
    $J_k(\lambda)$ je čtvercová matice řádu $k$ následujícího tvaru:
    $$\mat{J_k(\lambda)} = \begin{pmatrix}
        \lambda &1       &0      &\dots  &\dots  &0 \\
              0 &\lambda &1      &0      &\dots  &0 \\
         \vdots &\ddots  &\ddots &\ddots &\ddots &\vdots \\
         \vdots &        &\ddots &\ddots &\ddots &0 \\
         \vdots &        &       &\ddots &\ddots &1 \\
              0 &\dots   &\dots  &\dots  &0      &\lambda
    \end{pmatrix}$$
\end{definition}

\begin{definition}
    Matice $\mat{J} \in \Cnn$ je v \newterm{Jordanově normální formě}, pokud
    je v blokově diagonálním tvaru a bloky na diagonále jsou Jordanovy buňky:
    $$\mat{J} = \begin{pmatrix}
        \mat{J_{k_1}(\lambda_1)} &0 &\dots &0 \\
                              0 &\ddots &\ddots &\vdots \\
                         \vdots &\ddots &\ddots &0 \\
                              0 &\dots  &0      &\mat{J_{k_m}(\lambda_m)}
    \end{pmatrix}$$
\end{definition}

\begin{theorem}
    Každá matice $\mA \in \Cnn$ je podobná matici v Jordanově normální formě.
\end{theorem}

\begin{proof}
    Bez důkazu.
\end{proof}

\begin{definition}
    Nechť $\mA \in \Cnn$. Matici $\mAH$, pro níž platí $(\mAH)_{ij} = 
    \overline{a_{ji}}$, nazýváme \newterm{hermitovskou transpozicí} matice 
    $\mA$.
\end{definition}

\begin{observation}
    $(\mA\mB)^\herm = \mB^\herm \mA^\herm$, pokud je operace definována.
\end{observation}

\begin{proof}
    $(\mA\mB)^\herm_{ij} = \overline{(\mA\mB)_{ji}} = \overline{\sum_{k=1}^n 
    \mA_{jk} \cdot \mB_{ki}} = \sum_{k=1}^n \overline{\mA_{jk}} \cdot \overline{\mB_{ki}} = 
    \sum_{k=1}^n \mB^\herm_{ik} \mA^\herm_{kj} = (\mB^\herm \mA^\herm)_{ij}$
\end{proof}

\begin{observation}
    Standardní skalární součin na $\Cn$ je možno zapsat jako:
    $$\scp{\vx}{\vy} = \sum_{i=1}^n x_i \overline{y_i} = \vy^\herm \vx.$$
\end{observation}

\begin{observation}
    Je-li matice $\mA$ složena z vektorů ortonormální báze $\Cn$ (vůči
    standardnímu skalárnímu součinu), tak platí: $$\mA^\herm \mA = \mIn.$$
\end{observation}

\begin{definition}
    Komplexní čtvercová matice $\mA$ se nazývá \newterm{unitární}, pokud
    $\mA^\herm \mA = \mIn.$
\end{definition}

\begin{definition}
    Komplexní čtvercová matice $\mA$ se nazývá \newterm{hermitovská}, 
    pokud $\mA^\herm = \mA.$
\end{definition}

\begin{remark}
    Hermitovské matice představují komplexní analogii reálných 
    symetrických matic. Uvědomme si navíc, že na diagonále hermitovské matice 
    jsou reálná čísla. 
\end{remark}

\begin{theorem}
    \label{th:hermunitr}
    Nechť $\mA$ je hermitovská matice. Potom:
    \begin{enumerate}[i.]
        \item všechna její vlastní čísla jsou reálná, a
        \item existuje unitární matice $\mR$ taková, že $\mRi \mA \mR$ je 
            diagonální.
    \end{enumerate}
\end{theorem}

\begin{proof}
    Větu dokážeme indukcí podle řádu matice $n$. Označme $\mat{A_n} = \mA$.
    Buď $\lambda$ vlastní číslo $\mA$ (jeho existence plyne ze základní
    věty algebry) a $\vx$ příslušný normovaný vlastní vektor,
    $\|\vx\| = 1.$ Doplníme $\vx$ na ortonormální bázi $\Cn$ a z těchto
    vektorů sestavíme unitární matici $\mat{P_n} = (\vx | \dots).$
    Potom platí: $$(\mPn^\herm \mAn \mPn)^\herm = \mPn^\herm 
    \underbrace{\mAn^\herm}_{=\mAn} 
    \underbrace{(\mPn^\herm)^\herm}_{=\mPn} = \mPn^\herm \mAn \mPn,$$
    a tedy matice $\mPn^\herm \mAn \mPn$ je hermitovská.
    Jelikož první sloupec součinu $\mAn \mPn$ je $\lambda$-násobek $\vx$,
    platí, že: $$\mPn^\herm \mAn \mPn = \begin{pmatrix}[c|ccc]
        \lambda &0 &\dots &0 \\
        \hline
        0 & & &\\
        \vdots & &\mat{A_{n-1}} & \\
             0 & & &
    \end{pmatrix},$$
    kde $\mat{A_{n-1}}$ je hermitovská a $\lambda$ je reálné.

    Z indukce, pro $\mat{A_{n-1}}$ existuje unitární matice $\mat{R_{n-1}}$
    taková, že $\invmat{R_{n-1}} \mat{A_{n-1}} \mat{R_{n-1}} = \mat{D_{n-1}}.$
    Položme: $$\mat{S_n} \coloneqq \begin{pmatrix}[c|ccc]
        1 &0 &\dots &0 \\
        \hline
        0 & & &\\
        \vdots & &\mat{R_{n-1}} & \\
             0 & & &
    \end{pmatrix}.$$
    Tato matice je unitární a její sloupce tvoří ortonormální bázi $\Cn$.
    Vezměme dále $\mRn \coloneqq \mPn \mat{S_n}.$ $\mPn$ je unitární (jelikož
    tak byla sestavena). I matice $\mRn$ je unitární, jelikož 
    $\mat{R_n}^\herm \mRn = (\mPn \mat{S_n})^\herm (\mPn \mat{S_n}) = 
    \mat{S_n}^\herm \mPn^\herm \mPn \mat{S_n} = \mIn.$

    Zbývá ověřit, že $\invmat{R_n} \mAn \mRn = \mat{D_n}$:\footnote{Uvědomme 
        si, že pro každou unitární matici $\mR$ platí: 
    $\invmat{R} = \mR^\herm.$}
    \begin{align*}
        \mRn^\herm \mAn \mRn 
        &= \mat{S_n}^\herm \mPn^\herm \mAn \mPn \mat{S_n} \\
        &= \begin{pmatrix}[c|ccc]
            1 &0 &\dots &0 \\
            \hline
            0 & & &\\
            \vdots & &\mat{R_{n-1}}^\herm & \\
                 0 & & &
        \end{pmatrix}
        \begin{pmatrix}[c|ccc]
            \lambda &0 &\dots &0 \\
            \hline
            0 & & &\\
            \vdots & &\mat{A_{n-1}} & \\
                 0 & & &
        \end{pmatrix}
        \begin{pmatrix}[c|ccc]
            1 &0 &\dots &0 \\
            \hline
            0 & & &\\
            \vdots & &\mat{R_{n-1}} & \\
                 0 & & &
        \end{pmatrix} \\
        &=
        \begin{pmatrix}[c|ccc]
            \lambda &0 &\dots &0 \\
            \hline
            0 & & &\\
            \vdots & &\mat{D_{n-1}} & \\
                 0 & & &
        \end{pmatrix}
        =
        \mat{D_n}
    \end{align*}
\end{proof}

\begin{observation}
    \label{obs:scpuv=vhu}
    Nechť $V$ je vektorový prostor se skalárním součinem konečné dimenze
    a $X=\{\vxone, \dots, \vxn\}$ je jeho libovolná ortonormální báze. 
    Potom: $$\forall \vu, \vv \in V: \scp{\vu}{\vv} = \sum_{i=1}^n
    \scp{\vu}{\vxi}\scp{\vxi}{\vv} = [\vv]_X^\herm[\vu]_X.$$
\end{observation}

\begin{proof}
    $\vu = \sum_{i=1}^n \alpha_i\vxi, \alpha_i=\scp{\vu}{\vxi} = ([\vu]_X)_i,
    \vv = \sum_{i=1}^n \beta_i\vxi, \beta_i=\scp{\vv}{\vxi} = ([\vv]_X)_i,
    \overline{\beta_i} = \scp{\vxi}{\vv}.$ Dále:
    \begin{align*}
        \scp{\sum_{i=1}^n \alpha_i\vxi}{\sum_{i=1}^n \beta_i\vxi} 
            &= \sum_{i,j=1}^n \alpha_i\overline{\beta_i} \scp{\vxi}{\vxj} \\
            &= \sum_{i=1}^n \alpha_i \overline{\beta_i} \tag{$X$ je 
                ortonormální báze}
    \end{align*}
\end{proof}

\begin{proposition}
    Nechť $V$ je vektorový prostor konečné dimenze se skalárním součinem,
    $X = \{\vxone, \dots, \vxn\}$ je jeho ortonormální báze a $f: V 
    \rightarrow V$ je lineární zobrazení.

    Pak platí, že $f$ zachovává skalární součin, t.j. $\scp{f(\vu)}{f(\vv)} 
    = \scp{\vu}{\vv}$, právě když $[f]_{XX}$ je unitární.
\end{proposition}

\begin{proof}
    \begin{align*}
        \scp{f(\vu)}{f(\vv)} &= [f(\vv)]_X^\herm[f(\vu)]_X 
        \tag{Pozorování~\ref{obs:scpuv=vhu}} \\
                             &= ([f]_{XX}[\vv]_X)^\herm[f]_{XX}[\vu]_X \\
                             &= [\vv]_X^\herm[f]_{XX}^\herm[f]_{XX}[\vu]_X
    \end{align*}
\end{proof}
