\section{Pozitivně definitní matice}

\begin{observation}
    Nechť $V \cong \Cn$ je prostor se skalárním součinem. Potom
    existuje matice $\mat{E}$ taková, že $\forall \vu, \vv \in V:
    \scp{\vu}{\vv} = \vv^\herm \mat{E} \vu$.
\end{observation}

\begin{proof}
    Vezmeme kanonickou bázi $\Cn$: $\{\veone, \dots, \ven\}$. Potom:
    $$\scp{\vu}{\vv} = \scp{\sum_{i=1}^n u_i\vei}{\sum_{j=1}^n v_j\vej} = 
    \sum_{i,j=1}^n u_i\overline{v_j}\scp{\vei}{\vej}$$
    a tedy lze vzít $(\mat{E})_{ij} = \scp{\vei}{\vej}$.
\end{proof}

\begin{remark}
    Pro standardní skalární součin $\scp{\vu}{\vv} = \sum_{i=1}^n 
    \overline{v_i}u_i$ máme $\mat{E} = \mIn$.
\end{remark}

\begin{observation}
    Matice $\mat{E}$ je hermitovská.
\end{observation}

\begin{definition}
    Splňuje-li hermitovská matice $\mA \in \Cnn$ vlastnost, že
    $\forall \vx \in \Cn \setminus \{0\}: \vx^\herm\mA\vx > 0,$ tak potom
    se nazývá \newterm{pozitivně definitní}.
\end{definition}

\begin{theorem}
    Pro hermitovskou matice $\mA$ jsou následující podmínky ekvivalentní:
    \begin{enumerate}[i.]
        \item $\mA$ je pozitivně definitní.
        \item $\mA$ má všechny vlastní čísla kladná.
        \item Existuje regulární matice $\mU$ taková, že $\mA = \mU^\herm\mU$.
    \end{enumerate}
\end{theorem}

\begin{proof}
    \leavevmode
    \begin{itemize}
        \item[$1 \implies 2$:] Nechť $\vx$ je vlastní vektor k vlastnímu
            číslu $\lambda$. Potom: $$\mA\vx = \lambda\vx \implies
            \vx^\herm \mA \vx = \vx^\herm \lambda \vx = \lambda\vx^\herm\vx$$
            $\mA$ je pozitivně definitní, a tedy $\vx^\herm\mA\vx > 0$.
            Jelikož výsledkem výrazu $\vx^\herm\vx$ je vždy kladné číslo,
            musejí i vlastní čísla matice $\mA$ být kladná.
        \item[$2\implies3$:] Jelikož $\mA$ je hermitovská, existuje unitární
            matice $\mR$ taková, že $\mA = \mR^\herm\mD\mR$, kde $\mD$ je 
            diagonální matice s kladnou diagonálou (viz \ref{th:hermunitr}).
            Vezmu: $$(\mat{\widetilde{D}})_{ij} = \begin{cases}
                \sqrt{(\mD)_{ii}} &i=j \\
                0 &i \neq j
            \end{cases}.$$ Potom $\mat{U} = \mat{\widetilde{D}}\mR$ je
            regulární (jelikož jak $\mat{\widetilde{D}}$, tak $\mR$ jsou
            regulární) a $$\mat{U}^\herm\mat{U} = 
            (\mat{\widetilde{D}}\mR)^\herm\mat{\widetilde{D}}\mR = 
            \mR^\herm \mat{\widetilde{D}}^\herm \mat{\widetilde{D}} \mR = 
            \mR^\herm \mD \mR = \mA.$$
        \item[$3\implies1$:] $\vx^\herm\mA\vx = \vx^\herm \mat{U}^\herm\mat{U}
            \vx = (\mat{U}\vx)^\herm \mat{U}\vx > 0,$ kde vektor $\vx$ je
            netriviální, $\mat{U}$ je regulární a tedy $\mat{U}\vx$ je 
            netriviální.
    \end{itemize}
\end{proof}

\begin{definition}
    Nechť matice $\mA$ je pozitivně definitní a nechť $\mU$ je trojúhelníková
    matice s kladnou diagonálou taková, že $\mU^\herm\mU = \mA$. Tento 
    rozklad pozitivně definitní matice se nazývá \newterm{Choleského 
    rozklad}.
\end{definition}

\begin{proposition}
    Pro pozitivně definitní matici $\mA$ existuje právě jedna trojúhelníková
    matice s kladnou diagonálou $\mU$ taková, že $\mA = \mU^\herm\mU$.
\end{proposition}

\begin{proof}
    Důkaz provedeme sestavením algoritmu, jehož vstupem je hermitovská 
    matice a výstupem její Choleského rozklad nebo tvrzení, že 
    daná matice není positivně definitní.

    Pro $i \coloneqq 1, \dots, n$ proveď:
    $$(\mU)_{ii} \coloneqq \sqrt{a_{ii} - \sum_{k=1}^{i-1}
    \overline{u_{ki}}u_{ki}}.$$
    Pokud $u_{ii} = 0$ nebo $u_{ii} \not \in \R$, stop: $\mA$ není positivně
    definitní.

    Pro $j \coloneqq (i+1), \dots, n$ proveď:
    $$u_{ij} \coloneqq \frac{1}{u_{ii}} (a_{ij} - \sum_{k=1}^{i-1} 
    \overline{u_{ki}}u_{kj}).$$
\end{proof}

\begin{proposition}[Rekurentní podmínka na test pozitivní definitnosti]
    Bloková matice 
    $$\mA = \begin{pmatrix}[c|c]
        \alpha &\vec{a}^\herm \\
        \hline
        \vec{a} &\mAtilde
    \end{pmatrix}$$
    je pozitivně definitní, právě když $\alpha > 0$ a $\mAtilde - 
    \frac{1}{\alpha}\va\va^\herm$ je pozitivně definitní.
\end{proposition}

\begin{proof}
    \leavevmode
    \begin{itemize}
        \item[$\impliedby$:] Nechť $\vx \in \Cn$ je libovolný netriviální
            vektor. Potom
            \begin{align*}
                \vx^\herm \mA \vx &= \left(\overline{x_1}|
                    \widetilde{\vx}^\herm\right)
                    \begin{pmatrix}[c|c]
                        \alpha &\vec{a}^\herm \\
                        \hline
                        \vec{a} &\mAtilde
                    \end{pmatrix}
                    \colvec{x_1\\ \widetilde{\vx}} \\
                &= \left(\alpha\overline{x_1} + \widetilde{\vx}^\herm,
                    \overline{x_1}\va^\herm + \widetilde{\vx}^\herm\mAtilde\right)
                    \colvec{x_1\\\widetilde{\vx}} \\
                &= \alpha\overline{x_1}x_1 + x_1\widetilde{\vx}^\herm\va +
                    \overline{x_1}\va^\herm\widetilde{\vx} 
                    + \widetilde{\vx}^\herm\mAtilde\widetilde{\vx} 
                    - \frac{1}{\alpha}\widetilde{\vx}^\herm\va\va^\herm\widetilde{\vx}
                    + \frac{1}{\alpha}\widetilde{\vx}^\herm\va\va^\herm\widetilde{\vx} \\
                &= \widetilde{\vx}^\herm (\mAtilde - \frac{1}{2}\va\va^\herm)\widetilde{\vx}
                    + (\sqrt{\alpha}\overline{x_1} + \frac{1}{\sqrt{\alpha}}\widetilde{x}^\herm\va)
                    (\sqrt{\alpha}x_1 + \frac{1}{\sqrt{\alpha}}\va^\herm\widetilde{x})
            \end{align*}

            První člen nezáporný; je kladný, je-li vektor $\widetilde{x}$ 
            netriviální. Druhý člen představuje součin komplexně sdružených 
            čísel a je tedy také nezáporný. Alespoň jedna z těchto dvou
            nerovností ovšem musí být ostrá.
        \item[$\implies$:] $\mA$ je pozitivně definitní,
            $\vx^\herm\mA\vx > 0$ pro všechny netriviální $\vx$ a tedy
            speciálně pro $\veone:$ $\veone^\herm\mA\veone = \alpha > 0$
            Nechť $\widetilde{\vx} \in \C^{n-1}$ je libovolný vektor. Zvolíme
            $x_1 \coloneqq -\frac{1}{\alpha}\va^\herm\widetilde{vx}$
            a položíme $\vx \coloneqq \colvec{x_1\\\widetilde{\vx}}.$ Potom
            \begin{align*}
                0 &< \vx^\herm\mA\vx \\
                  &= \widetilde{\vx}^\herm (\mAtilde - \frac{1}{2}\va\va^\herm)\widetilde{\vx}
                    + (\sqrt{\alpha}\overline{x_1} + \frac{1}{\sqrt{\alpha}}\widetilde{x}^\herm\va)
                    (\sqrt{\alpha}x_1 + \frac{1}{\sqrt{\alpha}}\va^\herm\widetilde{x}),
                    \tag{už spočteno dříve}
            \end{align*}
            kde druhý člen po dosazení $x_1$ je roven nule a první člen tedy
            musí být kladný.
    \end{itemize}
\end{proof}

\begin{proposition}[Jakobiho podmínka]
    Hermitovská matice $\mA \in \Cnn$ je pozitivně definitní právě tehdy, když
    mají matice $\mat{A_n}, \mat{A_{n-1}}, \dots, \mat{A_1}$ kladný 
    determinant, kde $\mat{A_i}$ značí matici vzniklou z $\mA$ vymazáním
    posledních $n-i$ řádků a sloupců.
\end{proposition}

\begin{remark}
    Z výpočetního hlediska není tento test efektivní.
\end{remark}
