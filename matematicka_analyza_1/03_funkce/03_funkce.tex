\section{Funkce jedné reálné proměnné}

\subsection{Základní definice}

\begin{definition}
    \newterm{Funkcí jedné reálné proměnné} rozumíme zobrazení
    $$f: M \rightarrow \R,$$
    kde $M \subset \R.$
\end{definition}

\begin{definition}
    Řekneme, že funkce $f: M \rightarrow \R, M \subset \R$ je:
    \begin{itemize} 
        \item \newterm{rostoucí}, pokud $\fa x,y \in M, x < y: f(x) < f(y),$
        \item \newterm{klesající}, pokud $\fa x,y \in M, x < y: f(x) > f(y),$
        \item \newterm{nerostoucí}, pokud $\fa x,y \in M, x < y: f(x) \geq f(y),$
        \item \newterm{neklesající}, pokud $\fa x,y \in M, x < y: f(x) \leq f(y).$
    \end{itemize}
\end{definition}

\begin{definition}
    Řekneme, že funkce $f: M \rightarrow \R, M \subset \R$ je:
    \begin{itemize}
        \item \newterm{sudá}, pokud $\fa x \in M: (-x \in M) \; \& \; (f(x) = f(-x)),$
        \item \newterm{lichá}, pokud $\fa x \in M: (-x \in M) \; \& \; (f(x) =- f(-x)),$
        \item \newterm{periodická}, pokud $\exists p > 0, \fa x \in M: (x+p \in M) \; \& \; 
            (x-p \in M) \; \& \; (f(x) = f(x+p)).$
    \end{itemize}
\end{definition}

\begin{definition}
    Řekneme, že funkce $f: M \rightarrow \R, M \subset \R$ je \newterm{omezená}
    (omezená shora, omezená zdola), jestliže $f(M)$ je omezená (omezená
    shora, omezená zdola) podmnožina $\R.$
\end{definition}

\begin{definition}
    \Necht $\delta > 0$ a $a \in \R.$ \newterm{Prstencové okolí} bodu je:
    \begin{align*}
        P(a, \delta) &= (a - \delta, a+ \delta) \setminus \{a\}, \\
        P(+\infty, \delta) &= (\frac{1}{\delta}, +\infty),\\
        P(-\infty, \delta) &= (-\frac{1}{\delta}, -\infty).
    \end{align*}
    \newterm{Pravé a levé prstencové okolí} bodu $a$ je:
    \begin{align*}
        P_+(a, \delta) &= (a, a+\delta), \\
        P_-(a, \delta) &= (a-\delta, a).
    \end{align*}
    \newterm{Okolí bodu} je: 
    \begin{align*}
        U(a, \delta) &= (a - \delta, a+ \delta), \\
        U(+\infty, \delta) &= (\frac{1}{\delta}, +\infty),\\
        U(-\infty, \delta) &= (-\frac{1}{\delta}, -\infty).
    \end{align*}
    \newterm{Pravé a levé okolí} bodu $a$ je:
    \begin{align*}
        U_+(a, \delta) &= [a, a+\delta), \\
        U_-(a, \delta) &= (a-\delta, a].
    \end{align*}
\end{definition}

\begin{definition}
    \Necht $f:M \rightarrow \R, M \subset \R.$ Řekneme, že $f$ má v bodě $a \in 
    \Rstar$ \newterm{limitu} $A \in \Rstar,$ jestliže platí:
    $$\fa \e > 0 \; \exists \delta > 0 \; \fa x \in P(a,\delta): f(x) \in U(A,\e).$$
    Značíme:
    $$\lim_{x\to a} f(x) = A.$$
\end{definition}

\begin{remark}[Poznámky k definici limity]
    \leavevmode
    \begin{itemize}
        \item Funkce $f$ nemusí být v bodě $a \in \Rstar$ definována, aby v něm
            měla limitu. Z definice limity vyplývá, že pokud $\lim_{x\to a} f(x)$
            existuje, tak je funkce $f$ definována na nějakém prstencovém okolí
            bodu $a.$

            Navíc, je-li $f$ v bodě $a$ definována, na hodnotě $f(a)$ nezáleží.
        \item Pokud $\lim_{\xtoa} f(x)$ existuje a je rovna $A$, tak potom je
            buď \newterm{vlastní} ($A \in \R$) nebo \newterm{nevlastní}
            ($A = \pm \infty$).

        \item $\lim_{\xtoa} f(x)$ je nazývá \newterm{limitou ve vlastním bodě},
            pokud $a \in \R$, nebo \newterm{limitou v nevlastním bodě},
            pokud $a = \pm \infty.$
    \end{itemize}
\end{remark}

\begin{definition}
    \Necht $f:M \rightarrow \R, M \subset \R.$ Řekneme, že $f$ má v bodě $a \in 
    \Rstar$ \newterm{limitu zprava (zleva)} rovnou $A \in \Rstar,$ jestliže platí:
    $$\fa \e > 0 \; \exists \delta > 0 \; \fa x \in P_+(a,\delta): 
    f(x) \in U(A,\e)$$ 
    $$(\fa \e > 0 \; \exists \delta > 0 \; \fa x \in P_-(a,\delta): 
    f(x) \in U(A,\e)).$$
    Značíme:
    $$\lim_{x\to a+} f(x) = A$$
    $$(\lim_{x\to a-} f(x) = A).$$
\end{definition}

\begin{observation}[Vztah limity a jednostranných limit]
    \label{obs:jednostrannelimity}
    \Necht $f:M \rightarrow \R, M \subset \R, a \in \Rstar, A \in \Rstar.$ 
    Potom:
    $$\lim_{\xtoa} f(x) = A \iff \lim_{\xtoa +} f(x) = \lim_{\xtoa -} f(x) = A.$$
\end{observation}

\begin{remark}[Příklady limit]
    \leavevmode
    \begin{itemize}
        \item $f(x) = x$ 
            
            Její limita v bodě $a \in \Rstar:$
            $$\lim_{\xtoa} f(x) = a.$$
            K $\e > 0$ volme $\delta = \e.$ Potom $f(P(a,\delta)) 
            \subseteq U(a,\e).$

        \item $f(x) = k$ 
            
            $\fa a \in \Rstar: \lim_{\xtoa} f(x) = k.$

        \item $f(x) = \sgn(x):$
            \begin{center}
                \begin{tikzpicture}
                    \begin{axis}[
                            axis lines=middle,
                            xlabel=$x$,
                            ylabel={$\sgn(x)$},
                            xmin=-3, xmax=3,
                            ymin=-1.5, ymax=1.5,
                            xtick=\empty,
                            ytick={0, 1},
                            extra y ticks={-1},
                            extra y tick style={
                                tick label style={anchor=west, xshift=3pt},
                            },
                            function line/.style={
                                black,
                                thick,
                                samples=2,
                            },
                            single dot/.style={
                                black,
                                mark=*,
                            },
                            empty point/.style={
                                only marks,
                                mark=*,
                                mark options={fill=white, draw=black},
                            },
                        ]
                        \addplot[function line, domain=\pgfkeysvalueof{/pgfplots/xmin}:0] {-1};
                        \addplot[function line, domain=0:\pgfkeysvalueof{/pgfplots/xmax}] {1};
                        \addplot[single dot] coordinates {(0, 0)};
                        \addplot[empty point] coordinates {(0, -1) (0, 1)};
                    \end{axis}
                \end{tikzpicture}
            \end{center}
            Z grafu je zřejmé, že jednostranné limity se sobě nerovnají:
            $$\lim_{x \to 0 +} \sgn(x) = 1,$$ 
            $$\lim_{x \to 0 -} \sgn(x) = -1,$$ 
            a tedy dle pozorování o vztahu limity a jednostranných limit
            (Pozorování~\ref{obs:jednostrannelimity})
            $\lim_{x \to 0} \sgn(x)$ neexistuje. 
        
        \item Dirichletova funkce:
            $$D(x) = \begin{cases}
                1, & \text{pokud $x \in \Q,$} \\
                0, & \text{pokud $x \in \R \setminus \Q.$}
            \end{cases}$$

            Tato funkce nemá limitu nikde, jelikož dle věty o hustotě $\Q$ 
            a $\R \setminus \Q$ (Věta~\ref{th:hustotaqrq})
            každé prstencové okolí bodu 
            $a\in \Rstar$ obsahuje alespoň jedno racionální a iracionální číslo.
            
        \item Riemannova funkce:
            $$R(x) = \begin{cases}
                \frac{1}{q}, & \text{pokud $x \in \Q,$ tj. $x = \frac{p}{q},$ kde
                $p \in \Z, q\in\N,$ a $p,q$ jsou nesoudělná,}\\
                0, & \text{pokud $x \in \R \setminus \Q.$}
            \end{cases}$$

            Jako domácí cvičení dokažte:
            $$\fa a \in \R: \lim_{\xtoa} R(x) = 0.$$
    \end{itemize}
\end{remark}

\begin{definition}
    \Necht $f:M \rightarrow \R, M \subset \R, a \in M.$ Řekneme, že $f$ je v 
    bodě \newterm{spojitá (spojitá zleva, spojitá zprava)}, jestliže:
    $$\lim_{x \to a} f(x) = f(a) \;\; (\lim_{x \to a+} f(x) = f(a),\;
    \lim_{x \to a-} f(x) = f(a))$$
\end{definition}

\begin{remark}[Příklady spojitých a nespojitých funkcí]
    \leavevmode
    \begin{itemize}
        \item $f(x) = x$

            Spojitá na $\R.$

        \item $f(x) = \sgn(x)$

            Spojitá na $\R \setminus \{0\}.$

        \item $f(x) = D(x),$ Dirichletova funkce

            Není spojitá v žádném bodě $\R.$

        \item $f(x) = R(x),$ Riemannova funkce

            Spojitá v $\R \setminus \Q.$
    \end{itemize}
\end{remark}

\subsection{Věty o limitách}

\begin{theorem}[Heine]
    \label{th:heine}
    Nechť $A \in \Rstar, f: M \rightarrow \R$ a $f$ je definována na prstencovém
    okolí bodu $a \in \Rstar.$ Následující tvrzení jsou ekvivalentní:
    \begin{enumerate}[i.]
        \item $\lim_{x\to a} f(x) = A$
        \item pro každou posloupnost $\seq{x_n}_{n \in \N}$ takovou, že:
            $$\fa n \in \N \; x_n \in M, x_n \neq a, \text{a zároveň} 
            \lim_{n \to \infty} x_n = a$$
            platí:
            $$\lim_{n\to\infty} f(x_n) = A.$$
    \end{enumerate}
\end{theorem}

\begin{proof}
    \leavevmode
    \begin{itemize}
        \item[$\implies$] Z definice limity:
            $$\fa \e > 0 \; \exists \delta > 0 \; \fa x \in P(a,\delta): 
            f(x) \in U(A,\e).$$
            Nechť máme dále posloupnost $\seq{x_n},$ jež splňuje podmínky
            bodu $(ii)$. Jelikož $\lim x_n = a$ a $\fa n \in \N: x_n \neq a$, 
            tak k $\delta > 0:$
            $$\exists n_0 \in \N, \fa n \geq n_0, n\in\N: x_n \in P(a,\delta)$$
            a tedy 
            $$\fa n \geq n_0, n \in \N: f(x_n) \in U(A, \e).$$ 
            Potom
            $$\lim_{\ntoinfty} f(x_n) = A.$$
        \item[$\impliedby$]
            Implikaci dokážeme nepřímo, tj. dokážeme tvrzení 
            $\neg (i.) \implies \neg (ii.),$
            tedy že z tvrzení, že limita funkce $f$ neexistuje nebo není
            rovna $A,$ vyplývá existence 
            alespoň jedné posloupnosti $\seq{x_n}$, která splňuje zadaná 
            kritéria a zároveň $\lim f(x_n) \neq A.$

            Pokud $\lim_{\xtoa} f(x)$ neexistuje nebo není rovna $A,$ potom
            $$\exists \e > 0 \; \fa \delta > 0 \; \exists x \in P(a,\delta):
            f(x) \not \in U(A,\e).$$
            Pro $\delta_n = \frac{1}{n}, n=1,2,3,\dots$ vezmeme takové $x$ a
            označíme ho $x_n.$ 
            
            Zřejmě platí, že $\lim_{\ntoinfty} x_n = a.$ Jelikož dané elementy
            $x_n$ vybíráme z prstencového okolí $a,$ platí, že $\fa n \in \N:
            x_n \neq a.$ Z definice této posloupnosti navíc vyplývá, že 
            $\fa n \in \N: f(x_n) \not \in U(A,\e),$ takže $\lim_{\ntoinfty}
            f(x_n) \neq A.$ Tím je implikace splněna.
    \end{itemize}
\end{proof}

\begin{theorem}[o jednoznačnosti limity]
    Funkce $f$ má v daném bodě nejvýše jednu limitu.
\end{theorem}

\begin{proof}
    Sporem. Nechť $A_1$ a $A_2$ jsou dvě různé limity funkce $f$ daném bodě 
    $a \in \Rstar.$  Mějme dále posloupnost $\seq{x_n}$, $\lim_{\ntoinfty} x_n = a,$
    $\fa n \in \N: x_n \neq a.$ Potom dle Heineho (Věta~\ref{th:heine})
    $\lim_{\ntoinfty} f(x_n) = A_1$ a zároveň $\lim_{\ntoinfty} f(x_n) = A_2.$
    Dostáváme tím spor s větou o jednoznačnosti limity posloupnosti 
    (Věta~\ref{th:jednoznacnostlimity}).
\end{proof}

\begin{theorem}[limita a omezenost]
    \Necht $f$ má vlastní limitu v bodě $a\in\Rstar.$ Pak existuje $\delta > 0$
    tak, že $f$ je na $P(a,\delta)$ omezená.
\end{theorem}

\begin{proof}
    \Necht $\lim_{\xtoa} f(x) = A \in \R.$ Z definice limity vyplývá, že:
    $$\fa \e > 0 \; \exists \delta > 0: f(P(a, \delta)) \subseteq U(A, \e).$$
    Jelikož je limita vlastní, platí dále:
    $$U(A, \e) = (A-\e, A + \e).$$

    Zvolme $\e = 1$. Platí:
    $$\exists \delta > 0 \; \fa x \in P(a,\delta): f(x) \in U(A,1) = (A - 1, A+1),$$
    a tedy $f(x)$ je omezená na $P(a, \delta).$
\end{proof}

\begin{theorem}[o aritmetice limit funkcí]
    \label{th:voalf}
    \Necht $a \in \Rstar,$ $\lim_{\xtoa} f(x) = A \in \Rstar,$ a $\lim_{\xtoa}
    g(x) = B \in \Rstar.$ Pak platí:
    \begin{enumerate}[i.]
        \item $\lim_{\xtoa} (f(x) + g(x)) = A+B,$ pokud je výraz $A+B$ definován;
        \item $\lim_{\xtoa} f(x)g(x) = AB,$ pokud je výraz $AB$ definován;
        \item $\lim_{\xtoa} \frac{f(x)}{g(x)} = \frac{A}{B},$ pokud je výraz 
            $\frac{A}{B}$ definován.
    \end{enumerate}
\end{theorem}

\begin{proof}
    Dokážeme pouze pro bod $(i)$; ostatní případy se řeší analogicky.

    Zvolme libovolnou posloupnost $\seq{x_n},$ splňující
    $$\lim_{\ntoinfty} x_n = a, \fa n \in \N: x_n \neq a.$$
    Potom dle Heineho (Věta~\ref{th:heine}):
    $$\lim_{\ntoinfty} f(x_n) = A,$$
    $$\lim_{\ntoinfty} g(x_n) = B.$$
    a dle věty o aritmetice limit posloupností (Věta~\ref{th:voal}):
    $$\lim_{\ntoinfty} (f(x_n) + g(x_n)) = A + B.$$
    Protože posloupnost $\seq{x_n}$ je libovolná, dle Heineho:
    $$\lim_{\xtoa} (f(x) + g(x)) = A+B.$$
\end{proof}

\begin{corollary}
    \Necht jsou funkce $f$ a $g$ spojité v bodě $a \in \R.$ Pak jsou i 
    funkce $f + g$, $f\cdot g$ spojité v bodě $a.$ Pokud je navíc $g(a) \neq 0$,
    pak je i funkce $\frac{f}{g}$ spojitá v $a.$
\end{corollary}

\begin{theorem}[Limita a uspořádání]
    \label{th:limitaausporadanifce}
    \Necht $a \in \Rstar.$
    \begin{enumerate}[i.]
        \item \Necht $\limxtoa f(x) > \limxtoa g(x).$ Pak existuje prstencové
            okolí $P(a,\delta)$ tak, že:
            $$\fa x \in P(a,\delta): f(x) > g(x).$$

        \item \Necht existuje prstencové okolí $P(a, \delta)$ tak, že:
            $$\fa x \in P(a, \delta): f(x) \leq g(x).$$
            \Necht existují $\limxtoa f(x)$ a $\limxtoa g(x).$ Potom platí:
            $$\limxtoa f(x) \leq \limxtoa g(x).$$

        \item \Necht na nějakém prstencovém okolí $P(a, \delta)$ platí $f(x)
            \leq h(x) \leq g(x).$ \Necht $\limxtoa f(x) = \limxtoa g(x) = A 
            \in \Rstar.$
            Pak existuje $\limxtoa h(x)$ a všechny tři limity se rovnají.
    \end{enumerate}
\end{theorem}

\begin{proof}
    \leavevmode
    \begin{enumerate}[i.]
        \item \Necht $\limxtoa f(x) = A,$ $\limxtoa g(x) = B, A > B.$ Zvolme
            $0 < \e < \frac{A-B}{2}.$ Dle definice limity:
            $$\exists \delta_1: f(P(a,\delta_1)) \subseteq U(A, \e),$$
            $$\exists \delta_2: g(P(a,\delta_2)) \subseteq U(B, \e).$$
            Zvolme $\delta_0 = \min\{\delta_1, \delta_2\}.$ Zřejmě:
            $$\fa x \in P(a,\delta_0): f(x) > g(x).$$

        \item Sporem. Důkaz je analogický k bodu ($ii$) v důkazu věty o 
            limitě a uspořádání posloupností (Věta~\ref{th:limitaausporadani}).

        \item Pro $\e > 0$ existují $\delta_1, \delta_2$ jako v bodě $(i)$. Pro
            $\delta_0 = \min\{\delta, \delta_1, \delta_2\}$ platí:
            $$h(P(a, \delta_0)) \subseteq U(A, \e),$$
            a tedy $\limxtoa h(x) = A.$

    \end{enumerate}
\end{proof}

\begin{definition}
    Mějme funkce $f: M \rightarrow \R, M \subset \R$ a $g: N \rightarrow \R, 
    N \subset \R.$ Pokud $g(N) \subseteq M$, potom funkci $h: N \rightarrow \R,
    h(x)=f(g(x))$ nazveme \newterm{složenou funkcí}. 
    
    Složenou funkci $h$ značíme: $h = f \circ g.$
    Funkci $f$ se říká \newterm{vnější funkce}, funkci $g$ \newterm{vnitřní funkce}.
\end{definition}

\begin{remark}[Vztah limit vnější, vnitřní a složené funkce]
    Nechť:
    $$\lim_{\xtoa} g(x) = A,$$
    $$\lim_{x \to A} f(x) = B.$$
    Platí obecně:
    $$\lim_{\xtoa} f(g(x)) = B?$$
    Neplatí! Uvažujme následující dvě funkce:
    $$g(x) = 3 \; \fa x \in N,$$
    $$f(x) = \begin{cases}
        1 &\text{pro $x = 3 $} \\
        0 &\text{pro $x \neq 3$}
    \end{cases}$$
    Zřejmě:
    $$\lim_{x \to 0} g(x) = \underbrace{3}_{A},$$
    $$\lim_{x \to A=3} f(x) = \underbrace{0}_{B}.$$
    Limita složené funkce:
    $$\lim_{x \to 0} f(g(x)) = \lim_{x \to 0} f(3) = \lim_{x \to 0} 1 = 1 \neq B.$$

    Schéma z počátku poznámky nicméně platí při splnění dodatečných 
    podmínek, které jsou popsány v následující větě.
\end{remark}

\begin{theorem}[Limita složené funkce]
    \label{th:slozenafunkce}
    Nechť funkce $f$ a $g$ splňují:
    \begin{enumerate}[i.]
        \item $\lim_{x \to a} g(x) = A,$
        \item $\lim_{y \to A} f(y) = B.$
    \end{enumerate}
    Je-li navíc splněna alespoň jedna z podmínek:
    \begin{itemize}
        \item[(P1)] $f$ je spojitá v $A,$
        \item[(P2)] $\exists \eta > 0 \; \fa x \in P(a,\eta): g(x) \neq A,$
    \end{itemize}
    pak platí $\lim_{x \to a} f(g(x)) = B.$
\end{theorem}

\begin{remark}
    Funkce $f$ a $g$ z předchozí poznámky nesplňovaly podmínky (P1) a (P2).
    Funkce $f$ nebyla spojitá v $A = 3$ a pro funkci $g$ neexistovalo
    prstencové okolí bodu $a = 0,$ ve kterém nenabývala své limity $A = 3.$
\end{remark}

\begin{proof}
    \leavevmode
    \begin{itemize}
        \item[(P1)] Díky existenci limity a spojitosti funkce $f$ v bodě $A$ 
            platí, že ke zvolenému $\e > 0$:
            $$\exists \varphi > 0: f(U(A,\varphi)) \subseteq U(B,\e).$$
            Dále, k danému $\varphi$:
            $$\exists \chi > 0: g(P(a, \chi)) \subseteq U(A,\varphi).$$
            Nakonec:
            $$f(g(P(a, \chi))) \subseteq f(U(A, \varphi)) \subseteq U(B,\e),$$
            a tedy $\lim_{x \to a} f(g(x)) = B.$
        \item[(P2)]Díky existenci limity funkce $f$ v bodě $A$ 
            platí, že ke zvolenému $\e > 0$:
            $$\exists \varphi > 0: f(P(A,\varphi)) \subseteq U(B,\e).$$
            Dále, k danému $\varphi$:
            $$\exists \chi > 0: g(P(a, \chi)) \subseteq U(A,\varphi).$$
            Pro $\psi = \min(\chi, \eta)$ díky podmínce (P2) dále platí:
            $$g(P(a, \psi)) \subseteq P(A,\varphi).$$
            Nakonec:
            $$f(g(P(a, \psi))) \subseteq f(P(A, \varphi)) \subseteq U(B,\e),$$
            a tedy $\lim_{x \to a} f(g(x)) = B.$
    \end{itemize}
\end{proof}

\begin{theorem}[limita monotónní funkce]
    \label{th:limitamonotonnifce}
    Nechť funkce $f$ je monotónní na intervalu $(a,b), \; a,b \in \Rstar.$
    Potom existuje $\lim_{x \to a+} f(x)$ i $\lim_{x \to b-} f(x).$
\end{theorem}

\begin{proof}
    Větu dokážeme pro $f$ neklesající a pro $\lim_{x \to a+}.$ 
    Ostatní případy se dokáží analogicky.

    Definujme množinu $M = f((a,b)) = \{f(x), x \in (a,b)\}$, položme
    $A \coloneqq \inf M$ a zvolme $\e > 0.$ Z vlastností infima 
    (Definice~\ref{def:inf}) víme, že:
    $$\exists y_0 = f(x_0) \in M: A \leq y_0 < A + \e.$$
    Potom díky monotonii funkce $f$:
    $$\fa x, \; a < x < x_0: f(x) \in U(A,\e).$$
    Nyní zvolíme $\delta > 0$ takové, aby $P_+(a,\delta) \subseteq (a,x_0).$
    Potom:
    $$\fa x \in P_+(a,\delta): f(x) \in U(A,\e),$$
    a tedy $\lim_{x \to a+} f(x) = A.$
\end{proof}

\subsection{Funkce spojité na intervalu}

\begin{definition}
    Množina $M \subseteq \R$ je \newterm{interval}, pokud 
    $\exists a,b \in \Rstar$ tak, že: 
    $$M = \{x \in \R: a \prec x \prec b \},$$ 
    kde relace $\prec$ je buď $\leq$ nebo $<.$

    Body $a,b$ nazýváme \newterm{krajními body} intervalu; ostatní body 
    intervalu $M$ nazýváme \newterm{vnitřními body}.
\end{definition}

\begin{observation}
    \label{obs:intervalconvex}
    Množina $M \subseteq \R$ je interval, právě když
    $$\fa x,y,z\in \R: x\leq z\leq y, x \in M, y\in M \implies z \in M,$$
    tj. právě když je \newterm{konvexní podmnožinou} $\R$.
\end{observation}

\begin{proof}
    \leavevmode
    \begin{itemize}
        \item[$\implies$] Zřejmé: Každý interval je konvexní množina.
        \item[$\impliedby$] Nechť $M \subseteq \R$ je konvexní množina.
            Označme $a \coloneqq \inf M$, $b \coloneqq \sup M.$
            Potom
            $$(a,b) \subseteq M \subseteq [a,b].$$
            Proč? Pokud $x \in (a,b),$ potom z definice suprema a infima
            $\exists \alpha, \beta \in M: \alpha < x < \beta.$ Díky konvexivitě
            je i $x \in M.$ Pokud naopak $x \in M,$ z definice suprema a infima
            vyplývá $a \leq x$ a $x \leq b$, a tedy $x \in [a,b].$

            Množina $M$ se tedy od $(a,b)$ liší jen eventuálním přidáním
            jednoho nebo obou bodů $a,b$, a je tedy intervalem.
    \end{itemize}
\end{proof}

\begin{definition}
    \Necht $f$ je funkce a $J$ interval. Řekneme, že $f$ je 
    \newterm{spojitá na intervalu} $J$, 
    jestliže je spojitá ve všech vnitřních bodech $J$. Je-li počáteční bod $J$
    prvkem $J$, tak požadujeme i spojitost zprava v tomto bodě, a je-li koncový
    bod $J$ prvkem $J$, tak požadujeme i spojitost zleva v tomto bodě.
\end{definition}

\begin{theorem}[Darboux]
    \label{th:darboux}
    \Necht $f$ je spojitá na $[a,b]$ a platí $f(a) < f(b).$ Pak pro každé $y \in 
    (f(a),f(b))$ existuje $x \in (a,b)$ tak, že $f(x) = y.$
\end{theorem}

\begin{proof}
    Definujme množinu $M \coloneqq \{x \in [a,b], f(x) < y \}.$ Označme dále 
    $x_0 = \sup M.$ Tvrdím, že $f(x_0) = y.$ Toto tvrzení nyní dokážeme
    sporem s vlastnostmi suprema.

    Nechť platí $f(x_0) < y.$ Zvolme $\e = \frac{y - f(x_0)}{2} > 0.$ Jelikož
    dle předpokladů je $f$ spojitá v $x_0$, existuje $\delta > 0 \; 
    \fa x \in U(x_0, \delta): f(x) \in U(f(x_0), \e),$ neboli $f(x) < y.$ 
    Zde nicméně dostáváme spor s definicí suprema: $x_0$ nemůže býti 
    supremem množiny $M$, neboť existují $x > x_0,$ pro které také 
    platí $f(x) < y.$

    Nechť naopak platí $f(x_0) > y.$ Zvolme $\e = \frac{f(x_0) - y}{2} > 0.$
    Jelikož je $f$ spojitá, existuje $\delta > 0 \; \fa x \in U(x_0, \delta):
    f(x) \in U(f(x_0), \e),$ neboli $f(x) > y.$ Potom ale $\fa x \in (x_0-\delta,
    x_0): f(x) > y$ a tedy $x_0 - \delta$ je také horní závora množiny $M$ 
    a dostáváme se tak do sporu s druhou vlastností suprema.
\end{proof}

\begin{theorem}[Zobrazení intervalu spojitou funkcí]
    Nechť $J$ je interval. \Necht funkce $f: J \rightarrow \R$ je spojitá.
    Pak je množina $f(J)$ také interval.
\end{theorem}

\begin{proof}
    Nechť $x,y \in f(J), z \in \R$ a $x \leq z \leq y.$ Potom $x = f(\alpha)$
    a $y = f(\beta)$ pro $\alpha, \beta \in J.$ Nechť bez újmy na obecnosti 
    $\alpha \leq \beta.$

    Protože zúžená funkce $f: [\alpha, \beta] \rightarrow \R$ je spojitá
    a $f(\alpha) \leq z \leq f(\beta),$ podle Darbouxovy věty 
    (Věta~\ref{th:darboux}) máme
    i $z = f(\gamma)$ pro nějaké $\gamma \in [\alpha,\beta].$ Množina $f(J)$ je 
    tedy konvexní a dle Pozorování~\ref{obs:intervalconvex} je tedy interval.
\end{proof}

\begin{definition}
    Nechť $f: M \rightarrow \R, M \subseteq \R.$ Řekneme, že funkce $f$ nabývá v
    bodě $a\in M$
    \begin{center}
        \begin{tabular}{lr}
            \newterm{maxima} na $M$, jestliže $\fa x \in M:$ &$f(x) \leq f(a),$ \\
            \newterm{minima} na $M$, jestliže $\fa x \in M:$ &$f(x) \geq f(a),$ \\
            \newterm{ostrého maxima} na $M$, jestliže $\fa x \in M, x \neq a:$ &$f(x) < f(a),$ \\
            \newterm{ostrého minima} na $M$, jestliže $\fa x \in M, x \neq a:$ &$f(x) > f(a),$ \\
        \end{tabular}
    \end{center}
    \newterm{lokálního maxima (ostrého lokálního maxima, ostrého lokálního minima,
    lokálního minima)}, jestliže existuje $\delta > 0$ tak, že $f$ nabývá na $M
    \cap U(a,\delta)$ svého maxima (ostrého maxima, ostrého minima, minima).
\end{definition}

\begin{observation}[Heineho věta pro spojitost]
    \label{obs:heinespojitost}
    Následující tvrzení jsou ekvivalentní:
    \begin{enumerate}[i.]
        \item $f$ je spojitá v $a$, tj. $\lim_{\xtoa} f(x) = f(a);$
        \item pro každou posloupnost $\seq{x_n}_{n \in \N}$ takovou, že:
            $$\fa n \in \N \; x_n \in M, x_n \neq a, \text{a zároveň} 
            \lim_{n \to \infty} x_n = a$$
            platí:
            $$\lim_{n\to\infty} f(x_n) = f(a).$$
    \end{enumerate}
\end{observation}

\begin{theorem}[spojitost funkce a nabývání extrémů]
    \label{th:spojitaextremy}
    Nechť $f$ je spojitá funkce na intervalu $[a,b]$. Pak funkce $f$ nabývá na 
    $[a,b]$ svého maxima a minima.
\end{theorem}

\begin{proof}
    Označme $A \coloneqq \sup \{f(x), x \in [a,b]\}.$ Z vlastností suprema
    existuje posloupnost $\seq{x_n}$ taková, že $\lim_{\ntoinfty} f(x_n) = A$, 
    např. $f(x_1) > A -1, f(x_2) > A - \frac{1}{2}, f(x_3) > A - \frac{1}{3},$ atd.

    Jelikož $\fa n \in \N: x_n \in [a, b]$, posloupnost $\seq{x_n}$ je omezená, 
    a tedy dle Bolzano-Weierstrassovy věty (Věta~\ref{th:bolzanoweierstrass})
    existuje vybraná konvergentní podposloupnost $\seq{x_{n_k}}$:
    $$\lim_{k \to \infty} x_{n_k} = z \in [a,b].$$
    Jelikož je funkce $f$ v bodě $z$ spojitá, platí dle Heineho věty pro
    spojitost (Pozorování~\ref{obs:heinespojitost}):
    $$\lim_{k \to \infty} f(x_{n_k}) = f(z).$$
    Zároveň ale víme, že 
    $$\lim_{\ntoinfty} f(x_n) = A$$
    Dle věty o limitě vybrané posloupnosti (Věta~\ref{th:vybranalimita}) je i
    $$\lim_{k \to \infty} f(x_{n_k}) = A$$
    a dle věty o jednoznačnosti limity posloupnosti 
    (Věta~\ref{th:jednoznacnostlimity}) platí $f(z) = A,$ 
    a tedy funkce $f$ nabývá svého maxima $A$ v bodě $z \in [a,b].$

    Důkaz minima je analogický.
\end{proof}

\begin{theorem}[spojitost funkce a omezenost]
    \Necht $f$ je spojitá funkce na intervalu $[a,b].$ Pak je funkce $f$ na 
    $[a,b]$ omezená.
\end{theorem}

\begin{proof}
    Dle věty o spojitosti funkce a nabývání extrémů (Věta~\ref{th:spojitaextremy})
    nabývá funkce $f$ na $[a,b]$ svého maxima ($A$) i minima ($B$). 
    Potom $\fa x \in [a,b]: B \leq f(x) \leq A$ a funkce $f$ je tedy na intervalu
    $[a,b]$ omezená.
\end{proof}

\begin{definition}
    Nechť $f$ je funkce a $J$ interval. Řekneme, že $f$ je \newterm{prostá} na $J$,
    pokud $\fa x,y \in J: x \neq y \implies f(x) \neq f(y).$

    Pro prostou funkci $f: J \rightarrow \R$ definujeme \newterm{inversní funkci}
    $f^{-1}: f(J) \rightarrow \R$ předpisem: $f^{-1}(y) = x \iff f(x) = y.$
\end{definition}

\begin{theorem}[o inversní funkci]
    \label{th:inversnifce}
    \Necht $f$ je spojitá a rostoucí (klesající) funkce na intervalu $J$. Potom
    je funkce $f^{-1}$ spojitá a rostoucí (klesající) na itervalu $f(J).$
\end{theorem}

\begin{proof}
    Nechť je $f$ například rostoucí. Nejprve dokážeme sporem, že i $\inv{f}$
    je rostoucí. Nechť existují $y_1,y_2 \in f(J), y_1 < y_2$ tak, že $\inv{f}(y_1) 
    > \inv{f}(y_2).$ Potom ovšem dostáváme spor:
    $y_1 = f(\inv{f}(y_1)) > f(\inv{f}(y_2)) = y_2.$

    Dokažme nyní spojitost $\inv{f}.$ Zvolme $y_0 \in f(J), y_0 = f(x_0), \e > 0$ a 
    uvažujme nejprve možnost, kdy $y_0$ je vnitřní bod $f(J)$ (a tedy $x_0$ 
    je vnitřní bod $J$). Potom existují $x_1,x_2$ tak,
    že $x_1 < x_0 < x_2$ a $(x_1,x_2) \subseteq U(x_0, \e).$ Zvolme
    $\delta > 0$ tak, aby $U(y_0,\delta) \subseteq (f(x_1), f(x_2)).$ Potom:
    $$\inv{f}(U(y_0,\e)) \subseteq \inv{f}(f(x_1),f(x_2)) = (x_1,x_2) \subseteq
    U(x_0,\e) = U(\inv{f}(y_0),\e).$$

    Uvažujme dále možnost, že je $J$ uzavřený interval a $x_0$ je jeho levý 
    krajní bod. Zvolme $x_1 \in U_+(x_0, \e)$ a $\delta = f(x_1) - y_0.$ Potom
    díky monotonii funkce $\inv{f}$ platí pro všechny $y \in U_+(y_0, \delta):$
    $$\inv{f}(y) \in U_+(x_0, \e) \subseteq U(x_0, \e).$$
    Pravý krajní bod se řeší analogicky.
\end{proof}

\subsection{Elementární funkce}

\subsubsection{Exponenciála a logaritmus}

\begin{theorem}[Zavedení exponenciály]
    \label{th:exp}
    Existuje právě jedna funkce $\exp : \R \rightarrow \R$ splňující následující
    dvě podmínky:
    \begin{enumerate}[i.]
        \item $\exp(x+y) = \exp(x)\exp(y),$
        \item $\fa x \in \R: \exp(x) \geq 1+x$
    \end{enumerate}
\end{theorem}

\begin{proof}
    Předpokládejme nejprve, že daná funkce existuje, a ukažme si, že v tom případě
    je definována jednoznačně: Postupně odvodíme několik jejích vlastností, až se
    tak dostaneme k jejímu jednoznačnému vyjadření. Poté dokážeme i její 
    existenci\footnote{Důkaz této věty jsem zpracoval na základě 
    poznámek Petra Baudiše z přednášek Luboše Picka: \url{http://math.or.cz/analyza}.}.

    \begin{enumerate}[A.]
        \item Jednoznačnost.

            \begin{enumerate}[1.]
                \item $\fa n \in \N \; \fa x \in \R: \exp(nx) = \exp(x)^n.$

                    Důkaz indukcí:
                    \begin{enumerate}[I.]
                        \item $\exp(1x) = \exp(x)$
                        \item $\exp((n+1)x) = \exp(nx + x) = \exp(nx)\exp(x) = (\exp(x))^n
                            \exp(x) = (\exp(x))^{n+1}$
                    \end{enumerate}

                \item $\exp(0) = 1.$
                    
                    Rozepišme nejprve výraz $\exp(0)$:
                    $$\exp(0) = \exp(0 + 0) \stackrel{i.}{=} \exp(0)\exp(0) = (\exp(0))^2.$$
                    $\exp(0)$ tedy může být buď $0$ nebo $1$. První možnost je 
                    ve sporu s podmínkou $(ii)$, a proto $\exp(0) = 1.$

                \item $\fa x \in \R: \exp(-x) = \frac{1}{\exp(x)} \land \exp(x) \neq 0.$

                    $1 = \exp(0) = \exp(x-x) = \exp(x)\exp(-x).$

                \item $\lim_{x \to \infty} \exp(x) = +\infty.$

                    Plyne z podmínky $(ii)$ a Věty~\ref{th:limitaausporadanifce} (limita a 
                    uspořádání).

                \item $\lim_{x \to -\infty} \exp(x) = 0.$

                    Vyplývá z předchozích dvou bodů.

                \item $\fa x \in \R: \exp(x) > 0.$

                    $\exp(x) = \exp(\frac{x}{2} + \frac{x}{2}) \stackrel{i.)}{=}
                                    \exp(\frac{x}{2})\exp(\frac{x}{2}) \geq 0.$

                \item $\fa x > 0: \exp(x) > 1.$

                    Vyplývá z podmínky $(ii).$

                \item Exponenciála je rostoucí funkce na $\R.$

                    $\fa x, y \in \R, x < y: 1 \stackrel{7.}{<} \exp(y - x) = 
                    \frac{\exp(y)}{\exp(x)}$. Jelikož se jedná o kladnou funkci, platí
                    $\exp(y) > \exp(x).$

                \item $\lim_{x \to 0} \frac{\exp(x) -1}{x} = 1.$

                    $$\frac{1}{\exp(x)} \stackrel{3.}{=} \exp(-x) \stackrel{ii.}{\geq} 1-x,$$          
                    a tedy: $$1 + x \stackrel{(ii.)}{\leq} \exp(x) \leq \frac{1}{1-x}.$$ 
                    Další úpravou získáváme:
                    $$x \leq \exp(x) -1  \leq \frac{1}{1-x} - 1 = \frac{x}{1-x}$$
                    a po vydělení výrazem $x$:
                    $$1 \leq \frac{\exp(x) - 1}{x} \leq \frac{1}{1-x}.$$
                    Výslednou limitu získáme díky Větě~\ref{th:limitaausporadanifce} 
                    (limita a uspořádání).

                \item $\exp(x)$ je spojitá na $\R.$
                    \begin{align*}
                        \lim_{x \to x_0} (\exp(x) - \exp(x_0)) &= \lim_{x \to x_0} 
                        (\exp((x-x_0) + x_0) - \exp(x_0)) \\
                        &= \lim_{x\to x_0} (\exp(x - x_0)\exp(x_0) - \exp(x_0)) \\
                        &= \lim_{x \to x_0} \exp(x_0) \underbrace{\frac{\exp(x-x_0) - 1}{x - x_0}}_{\to 1} 
                        \underbrace{(x - x_0)}_{\to 0} \\
                        &= 0
                    \end{align*}

                \item $\fa x \in \R: \exp(x) = \lim_{\ntoinfty} 
                    \left(1 + \frac{x}{n}\right)^n$

                    Pokud se nám podaří dokázat tuto rovnost, dokážeme
                    díky jednoznačnosti limity posloupnosti (Věta~\ref{th:jednoznacnostlimity})
                    i jednoznačnost definice exponenciály.

                    Dle podmínky $(ii)$ platí:
                    $$\exp\left(-\frac{x}{n}\right) \geq 1 - \frac{x}{n}$$
                    Zvolme $k > |x|.$ Pro $n \geq k, n \in \N$ je i pravá strana 
                    nerovnice kladná a při umocnění obou stran na $n$-tou dostáváme:
                    $$\exp\left(-\frac{x}{n}\right)^n \stackrel{1.}{=} 
                    \exp\left(-n\frac{x}{n}\right) = \exp(-x) 
                        \stackrel{3.}{=} \frac{1}{\exp(x)}
                        \geq \left(1-\frac{x}{n}\right)^n$$
                    a tedy:
                    $$\exp{x} \leq \frac{1}{\left(1 - \frac{x}{n}\right)^n}$$
                    Můžeme tedy psát:
                    $$\left(1 + \frac{x}{n}\right)^n \leq \exp\left(\frac{x}{n}\right)^n
                    = \exp(x) \leq \frac{1}{\left(1 - \frac{x}{n}\right)^n}$$
                    a po vydělení výrazem $\left(1 + \frac{x}{n}\right)^n$:
                    $$1 \leq \frac{\exp(x)}{\left(1 + \frac{x}{n}\right)^n} 
                    \leq \frac{1}{\left(1 - \frac{x^2}{n^2}\right)^n}$$
                    Dle Bernoulliho (Věta~\ref{th:bernoulli}) dále platí:
                    $$\frac{1}{\left(1 - \frac{x^2}{n^2}\right)^n} \leq
                    \frac{1}{\left(1 - \frac{x^2}{n}\right)}$$
                    a proto:
                    $$1 \leq \frac{\exp(x)}{\left(1 + \frac{x}{n}\right)^n} 
                    \leq \underbrace{\frac{1}{\left(1 - \frac{x^2}{n}\right)}}_{\to 1 \text{ při } n \to \infty}$$
                    Nyní již stačí využít strážníků (Věta~\ref{th:limitaausporadani})
                    k dokázání limity z počátku.             
            \end{enumerate}

        \item Existence. 

            V následujících bodech ukážeme, že posloupnost:
                $$a_n = \left(1 + \frac{x}{n}\right)^n$$
            je pro dostatečně velká $n$ neklesající a omezená pro všechna $x \in \R,$
            což zaručí, že tato posloupnost má limitu a že tato limita je vlastní
            (viz také Větu~\ref{th:monotonniposl}). 
            Tím dokončíme formální zavedení exponenciály.

            \begin{enumerate}[I.]
                \item Monotonie.

                    Podobně jako výše zvolme $k > |x|.$ Pro $n \geq k, n \in \N$
                    platí:
                    $$\sqrt[n+1]{\left(1 + \frac{x}{n}\right)^n} \leq 
                    \frac{n\cdot(1+\frac{x}{n})+1}{n+1} = 1 + \frac{x}{n+1}$$

                    Zde jsme využili vztahu geometrického a aritmetického průměru
                    (Věta~\ref{th:agprumer}) pro $z_1 = z_2 = \dots = z_n = 1 + 
                    \frac{x}{n}$ a $z_{n+1} = 1.$ Pokud umocnímě obě strany 
                    nerovnosti na $(n+1)$-tou, dostáváme:                    
                    $$\left(1 + \frac{x}{n}\right)^n \leq 
                    \left(1 + \frac{x}{n+1}\right)^{n+1},$$
                    a tedy platí $\fa n \geq k: a_n \leq a_{n+1}.$ Posloupnost
                    $\seq{a_n}$ je tedy neklesající.

                \item Omezenost.

                    Nyní ukážeme, že posloupnost $\seq{a_n}$ má omezenou 
                    podposloupnost $\seq{a_{n_k}}.$ Z toho pak vyplývá, že 
                    i posloupnost $\seq{a_n}$ je omezená\footnote{Zde
                    využíváme jednoduchého pozorování, které tvrdí, že
                    posloupnost, která je monotónní a která má omezenou
                    podposloupnost, je omezená. Nechť $\seq{a_n}$ je 
                    neklesající posloupnost a nechť pro její podposloupnost 
                    $\seq{a_{n_k}}$ platí: $\fa k \in \N: |a_{n_k}| \leq L.$ 
                    Nechť, pro spor, posloupnost $\seq{a_n}$ není omezená. 
                    Potom $\fa K \; \exists n_0: a_{n_0} > K.$ Díky 
                    monotonii dále platí: $\fa n \geq n_0: a_n > K.$ 
                    Zvolme $K = L.$ Potom $\fa k \geq n_0: n_k \geq k \geq n_0: 
                    a_{n_k} > L,$ čímž dostáváme spor s omezeností
                    podposloupnosti $\seq{a_{n_k}}.$}.

                    Zvolme $k > |x|.$ Dokážeme, že
                    $$\fa n \in \N: \left(1 + \frac{x}{nk}\right)^{nk} 
                    \leq \left(1 - \frac{x}{k}\right)^{-k},$$
                    a tedy, že $\fa n\in \N: a_{nk} \leq
                    \left(1 - \frac{x}{k}\right)^{-k}.$
                    
                    Pro zvolené $k$ a pro $\fa n \in \N$ platí:
                    \begin{align*}
                        \left(1 + \frac{x}{nk}\right)^{-n} 
                            &= \left(\frac{nk+x}{nk}\right)^{-n} \\
                            &= \left(\frac{nk}{nk + x}\right)^n \\
                            &= \left(1 - \frac{x}{nk + x}\right)^n \\
                            &\geq 1 - \frac{nx}{nk + x} \tag{Bernoulliho nerovnost,
                                Věta~\ref{th:bernoulli}} \\
                                &\geq 1 - \frac{x}{k} \tag{$\frac{x}{k} \geq \frac{nx}{nk+x}$} \\
                                &> 0.
                    \end{align*}
                    Po umocnění obou stran nerovnice na $k$-tou dostáváme:
                    $$\left(1 + \frac{x}{nk}\right)^{-nk}
                    \geq \left(1 - \frac{x}{k}\right)^k$$
                    a po úpravě:
                    $$\left(1 + \frac{x}{nk}\right)^{nk} 
                    \leq \left(1 - \frac{x}{k}\right)^{-k}$$
            \end{enumerate}
    \end{enumerate}
\end{proof}

\begin{definition}
    Funkci inversní k exponenciále $\exp$ nazveme \newterm{logaritmus} $\log.$
\end{definition}

\begin{theorem}[Vlastnosti logaritmu]
    Funkce $\log$ splňuje:
    \begin{enumerate}[a)]
        \item $\log: (0, \infty) \rightarrow \R$ je spojitá rostoucí
            funkce.
        \item $\fa x,y > 0: \log(xy) = \log(x) + \log(y).$
        \item $\lim_{x \to 1} \frac{\log(x)}{x - 1} = 1.$
    \end{enumerate}
\end{theorem}

\begin{proof}
    V důkazech budeme využívat vlastnosti exponenciály dokázané v předchozí
    větě (Věta~\ref{th:exp}).
    \begin{enumerate}[a)]
        \item Exponenenciála $\exp: \R \rightarrow (0, \infty)$ je rostoucí
            a spojitá funkce. Podle věty o inversní 
            funkci (Věta~\ref{th:inversnifce}) je tedy i logaritmus jako
            její inversní funkce spojitá a rostoucí funkce.
            
        \item $\log(x) = A, x = \exp(A)$ a $\log(y) = B, y = \exp(B).$
            Potom:
            $$xy = \exp(A)\exp(B) = \exp(A + B)$$
            a tedy
            $$\log(xy) = A + B = \log(x) + \log(y).$$

        \item Definujme funkce
            $$f(y) = \frac{\exp(y) - 1}{y}, \; g(x) = \log(x)$$ 
            pro jejichž limity platí: 
            $$\lim_{y \to 0} f(y) = 1, \; \lim_{x \to 1} g(x) = 0\footnote{Víme, 
                že $\exp(0) = 1$ a tedy $\log(1) = 0.$ Dále víme, že
                logaritmus je spojitá funkce.}.$$
            Potom pro limitu složené funkce $f(g(x))$ v bodě $1$ platí:
            $$\lim_{x \to 1} f(g(x)) = 1.$$
            Zde jsme využili větu o limitě složené funkce 
            (Věta~\ref{th:slozenafunkce}) a její podmínky (P2), 
            tj. že funkce $\log$ nenabývá v prstencovém okolí bodu $1$ hodnoty $0.$
            Můžeme tedy psát:
            $$1 = \lim_{x \to 1} f(g(x)) = \lim_{x \to 1} \frac{\exp(\log(x)) - 1}
            {\log(x)} = \lim_{x \to 1} \frac{x - 1}{\log(x)}.$$
    \end{enumerate}
\end{proof}

\begin{definition}
    Nechť $a > 0$ a $b \in \R.$ Pak definujeme \newterm{obecnou mocninu} jako:
    $$a^b \coloneqq \exp(b\log(a)).$$
    Je-li navíc $b > 0,$ pak definujeme \newterm{logaritmus při základu $b$}
    následovně:
    $$\log_b(a) \coloneqq \frac{\log(a)}{\log(b)}.$$
\end{definition}

\begin{remark}[Korektnost definice obecné mocniny]
    Pro $x > 0$ platí:
    \begin{align*}
        x^n &= \exp(n\log(x)) \tag{nová definice} \\
            &= \exp(\log(x^n)) \tag{matematickou indukcí} \\
            &= x^n \tag{stará definice}
    \end{align*}
\end{remark}

\begin{remark}[Logaritmus při základu $10$]
    Na případu $b = 10$ ukážeme, že $b^{\log_b(x)} = x,$ a tedy že definice 
    logaritmu při základu $b$ je korektní:
    $$10^{\log_{10}(x)} = 10^{\frac{\log(x)}{\log(10)}} = 
    (e^{\log(10)})^{\frac{\log(x)}{\log(10)}} = e^{\log(x)} = x.$$
\end{remark}

\begin{remark}[Odmocnina jako obecná mocnina]
    $$\sqrt[n] x = \begin{cases}
        \exp\left(\frac{1}{n}\log(x)\right) &x>0, \\
        0 &x = 0.
    \end{cases}$$
\end{remark}

\subsubsection{Goniometrické funkce}
\begin{theorem}
    \label{th:goniom}
    Existuje právě jedna funkce $\sin:\R \rightarrow \R$ a právě jedna
    funkce $\cos:\R \rightarrow \R$ splňující:
    \begin{enumerate}[a)]
        \item \begin{tabular}[t]{ll}
                $\fa x,y \in \R:$
                &$\sin(x +y) = \sin x \cos y + \cos x \sin y$ \\
                &$\cos(x + y) = \cos x \cos y - \sin x \sin y$ \\
                &$\cos(-x) = \cos x,$ \\
                &$sin(-x) = -\sin x,$
            \end{tabular}
        \item $\exists \pi > 0$ tak, že $\sin$ je rostoucí na $[0, \frac{\pi}{2}]$
            a $\sin(\frac{\pi}{2}) = 1,$
        \item $$\lim_{x \to 0} \frac{\sin x}{x} = 1.$$
    \end{enumerate}
\end{theorem}

\begin{remark}[Další vlastnosti goniometrických funkcí]
    Ze "základních" vlastností goniometrických funkcí uvedených v předchozí 
    větě lze odvodit jejich další vlastnosti: periodicita, intervaly monotónnosti, 
    ostatní součtové vzorce, atd. V následujícím textu budeme předpokládat,
    že tyto vlastnosti známe ze střední školy, a dokazovat je nebudeme.
\end{remark}

\begin{remark}
    Dokažme tuto limitu:
    $$\lim_{x \to 0} \frac{1 - \cos x}{x^2}.$$
    Nejprve daný výraz v několika krocích upravíme:
    $$\lim_{x \to 0} \frac{1 - \cos x}{x^2} = \lim_{x \to 0}\frac{1 - \cos x}{x^2}
    \frac{1 + \cos x}{1 + \cos x} = \lim_{x \to 0} \frac{1 - \cos^2 x}{x^2}
    \frac{1}{1 + \cos x} = \lim_{x \to 0} \frac{\sin^2 x}{x^2} \frac{1}{1 + \cos x}.$$

    Jelikož
    $$\lim_{x \to 0} \frac{\sin^2 x}{x^2} = \left(\lim_{x\to 0} 
    \frac{\sin x}{x}\right)^2 = 1$$
    a kosinus je spojitá funkce (jak dokážeme později), můžeme využít věty o
    aritmetice limit (Věta~\ref{th:voalf}) a psát:
    $$\lim_{x \to 0} \frac{1 - \cos x}{x^2} = 
    \lim_{x \to 0} \frac{\sin^2 x}{x^2} \frac{1}{1 + \cos x}
    = \left(\lim_{x\to 0}  \frac{\sin x}{x}\right)^2 \lim_{x \to 0}
    \frac{1}{1 + \cos x} = \frac{1}{2}.$$
\end{remark}

\begin{definition}
    Pro $x \in \R \setminus \{\frac{\pi}{2} + k\pi, k \in \Z\}$ a
    $y \in \R \setminus \{k\pi, k \in \Z\}$ definujeme funkce \newterm{tangens}
    a \newterm{kotangens} předpisem:
    $$\tan x = \frac{\sin x}{\cos x} \text{ a } \cotg y = \frac{\cos y}{\sin y}.$$
\end{definition}

\begin{theorem}[spojitost sinu a kosinu]
    Funkce $\sin, \cos, \tan$ a $\cotg$ jsou spojité na svém definičním oboru.
\end{theorem}

\begin{proof}
    Začněme sinem. Nechť $a \in \R.$ Potom:
        \begin{align*}
            \lim_{x \to a} (\sin x - \sin a) 
            &= \lim_{x \to a} 2\cdot\sin\left(\frac{x - a}{2}\right)
                \cos\left(\frac{x-a}{2}\right) \tag{goniometrický vzorec} \\
            &= \lim_{x \to a} 2 \cdot 
                \underbrace{\frac{\sin\left(\frac{x - a}{2}\right)}{\frac{x - a}{2}}}_{\to 1}
                \underbrace{\frac{x - a}{2}}_{\to 0} 
                \underbrace{\cos\left(\frac{x-a}{2}\right)}_{\leq 1}
                \tag{viz níže} \\
            &= 0.
        \end{align*}
    Ve druhém kroku jsme využili věty o limitě složené funkce (Věta~\ref{th:slozenafunkce})
    pro:
    $$f(y) = \frac{\sin y}{y} \text{ a } g(x) = \frac{x-a}{2}$$
    a to za podmínky (P2).

    Nyní dokážeme spojitost pro kosinus. Pokud do vzorce:
    $$\sin(x + y) = \sin x \cos y + \cos x \sin y$$
    dosadíme $y = \frac{\pi}{2},$ získáme následující vztah mezi sinem 
    a kosinem:
    $$\cos x = \sin(x + \frac{\pi}{2}).$$ 
    Dle věty o limitě složené funkce (Věta~\ref{th:slozenafunkce})
    je tedy kosinus spojitý, neboť jak sinus, tak $x \rightarrow x + \frac{\pi}{2}$
    jsou spojité funkce.

    Dle věty o aritmetice limit funkcí (Věta~\ref{th:voalf}) (limita podílu) 
    jsou i $\tan$ a $\cotg$ spojité funkce.
\end{proof}

\begin{definition}
    Nechť
    \begin{align*}
        \sin^* x &= \sin x \text{ pro } x \in [-\frac{\pi}{2}, \frac{\pi}{2}], \\
        \cos^* x &= \cos x \text{ pro } x \in [0, \pi], \\
        \tan^* x &= \tan x \text{ pro } x \in (-\frac{\pi}{2}, \frac{\pi}{2}) \text{ a} \\
        \cotg^* x &= \cotg x \text{ pro } x \in (0, \pi).
    \end{align*}
    Definujeme $\arcsin$ (resp. $\arccos$, $\arctan$, $\arccotg$) jako inversní
    funkce k funkci $\sin^*$ (resp. $\cos^*, \tan^*, \arctan^*$).
\end{definition}

\begin{remark}
    $\arcsin(\sin x) = x$ pouze pro $x \in [-\frac{\pi}{2}, \frac{\pi}{2}].$
\end{remark}

\begin{remark}[Příklady limit]
    \leavevmode
    \begin{itemize}
        \item $\lim_{x \to 0} \frac{\arcsin x}{x}.$

            Definujme následující dvě funkce:
            $$f(y) = \begin{cases}
                \frac{\sin y}{y} &y \neq 0 \\
                1 &y = 0
            \end{cases}, \; g(x) = \arcsin x.$$
            Potom limita jejich složené funkce v bodě $0$ se rovná převrácené
            hodnotě hledané limity:
            $$\lim_{x \to 0} f(g(x)) = \limxo \frac{\sin(\arcsin x)}{\arcsin x} 
            = \limxo \frac{x}{\arcsin x}$$
            Víme dále, že:
            $$\limxo g(x) = 0 \text{ a } \limxo f(y) = 1.$$
            Za použití věty o limitě složené funkce (Věta~\ref{th:slozenafunkce})
            a podmínky (P1) (funkce $f$ je v bodě $0$ spojitá):
            $$\limxo f(g(x)) = 1,$$
            a tedy:
            $$\lim_{x \to 0} \frac{\arcsin x}{x} = 1.$$

        \item $\lim_{\ntoinfty} n\cdot\sin(\frac{1}{n}).$

            $$\lim_{\ntoinfty} n\cdot\sin(\frac{1}{n}) = \lim_{\ntoinfty}
            \frac{\sin(\frac{1}{n})}{\frac{1}{n}} = 1.$$
    \end{itemize}
\end{remark}

\subsection{Derivace funkce}

\begin{definition}
    Nechť $f$ je reálná funkce a $a \in \R.$ Pak \newterm{derivací $f$ v bodě $a$}
    budeme rozumět:
    $$f'(a) = \limho \frac{f(a+h) - f(a)}{h};$$
    \newterm{derivací $f$ v bodě $a$ zprava} budeme rozumět:
    $$f_+'(a) = \limhop \frac{f(a+h) - f(a)}{h};$$
    \newterm{derivací $f$ v bodě $a$ zleva} budeme rozumět:
    $$f_-'(a) = \limhom \frac{f(a+h) - f(a)}{h}.$$
\end{definition}

\begin{remark}
    $$f'(a) = \limho \frac{f(a+h) - f(a)}{h} = \limxa \frac{f(x) - f(a)}{x-a}.$$
\end{remark}

\begin{remark}[Příklady limit]
    \leavevmode
    \begin{itemize}
        \item $f(x) = x^n, a \in \R, n \in \N.$
            \begin{align*}
                f'(a) &= \limho \frac{(a+h)^n - a^n}{h}  \\
                &= \limho \frac{a^n + \binom{n}{1}a^{n-1}h
                + \binom{n}{2}a^{n-2}h^2 + \dots + \binom{n}{n}h^n - a^n}{h} \\
                &=\binom{n}{1}a^{n-1} \\
                &=na^{n-1}.
            \end{align*}

        \item $f(x) = \sgn(x).$

            Vyjádřeme nejdříve jednostranné limity:
            $$f_+'(0) = \limhop \frac{f(h) - f(0)}{h} = \limhop \frac{1-0}{h} = 
            +\infty,$$
            $$f_-'(0) = \limhom \frac{f(h) - f(0)}{h} = \limhom \frac{-1-0}{h} = 
            +\infty.$$
            Díky vztahu limity a jednostranných limit 
            (Pozorování~\ref{obs:jednostrannelimity}) můžeme psát:
            $$\sgn' 0 = +\infty.$$

        \item $f(x) = |x|.$
            $$f_+'(0) = \limhop \frac{f(h) - f(0)}{h} = \limhop \frac{h}{h} = 1,$$
            $$f_-'(0) = \limhom \frac{f(h) - f(0)}{h} = \limhom \frac{-h}{h} = -1$$
            Derivace $f(x)$ v bodě $0$ neexistuje.

        \item $f(x) = \exp x.$

            Vyjádřeme derivaci funkce $\exp$ v bodě $a \in \R:$
            $$\limho \frac{\exp(a+h) - \exp a}{h} 
            = \limho \frac{\exp a \exp h - \exp a}{h} 
            = \limho \frac{\exp a \cdot (\exp h - 1)}{h}.$$
            Jelikož $\limho \exp a = \exp a$ a 
            $\limho \frac{\exp h - 1}{h} = 1,$ získáme za použití
            věty o aritmetice limit funkcí (Věta~\ref{th:voalf}) následující rovnost:
            $$f'(a) = \limho \exp a \cdot \limho \frac{\exp h -1}{h} = \exp a.$$

        \item $f(x) = \log x.$
            $$\limho \frac{\log(a+h) - \log(a)}{h} 
            = \limho \frac{\log\left(1 + \frac{h}{a}\right)}{\frac{h}{a}}\cdot \frac{1}{a}
            = \frac{1}{a}.$$
   \end{itemize}
\end{remark}

\begin{theorem}[vztah derivace a spojitosti]
    \label{th:derivacespojitost}
    \Necht má funkce $f$ v bodě $a \in \R$ vlastní derivaci $f'(a) \in \R.$ Pak
    je $f$ v bodě $a$ spojitá.
\end{theorem}

\begin{proof}
    Chceme dokázat, že $\limxa (f(x) - f(a)) = 0:$
    $$\limxa (f(x) - f(a)) = \limxa \frac{f(x) - f(a)}{x-a}(x-a) =
    \limxa \frac{f(x) - f(a)}{x-a} \limxa (x-a) = f'(a) \cdot 0 = 0.$$
\end{proof}

\begin{remark}
    \label{rm:jednostrannaderivacespojitost}
    Podobná věta platí i pro jednostranné limity. Platí-li $f'_+(a) \in \R,$
    je funkce $f$ spojitá v bodě $a$ zprava.
\end{remark}

\begin{theorem}
    Nechť $f'(a)$ a $g'(a)$ existují. Potom:
    \begin{enumerate}[i.]
        \item $(f+g)'(a) = f'(a) + g'(a),$ pokud má pravá strana smysl.
        \item Nechť je $g$ spojitá v $a,$ pak $(fg)'(a) = f'(a)g(a) + f(a)g'(a),$
            pokud má pravá strana smysl.
        \item \Necht je $g$ spojitá v $a$ a $g(a) \neq 0,$ pak 
            $\left(\frac{f}{g}\right)'(a) = \frac{f'(a)g(a) - f(a)g'(a)}{g^2(a)},$
            pokud má pravá strana smysl.
    \end{enumerate}
\end{theorem}

\begin{proof}
    \leavevmode
    \begin{enumerate}[i.]
        \item 
            \begin{align*}
                (f+g)'(a) &= \\
                          &= \limho \frac{f(a+h) + g(a + h) - f(a) - g(a)}{h}  \\
                &= \limho \frac{f(a+h) - f(a)}{h} + \limho \frac{g(a+h) - g(a)}{h} \\
                &= f'(a) + g'(a).
            \end{align*}

        \item \begin{align*}
                (fg)'(a) &= \\
                &=\limho \frac{f(a+h)g(a+h) - f(a)g(a)}{h} \\
                &=\limho \frac{f(a+h)g(a+h) - f(a)g(a+h) + f(a)g(a+h) - f(a)g(a)}{h} \\
                &=\limho \frac{f(a+h)-f(a)}{h}  \limho g(a+h)  + 
                \limho f(a) \limho \frac{g(a+h)-g(a)}{h} \\
                &= f'(a)g(a) -f(a)g'(a).
            \end{align*}
                Spojitosti funkce $g$ jsme využili při vyjádření limity: 
            $\limho g(a+h) = g(a).$

        \item Funkce $g$ spojitá v $a$ a $g(a) \neq 0,$ tedy $\exists \delta >0 \;
            \fa h \in U(a,\delta): g(a+h) \neq 0.$ Dále:
            \begin{align*}
                \left(\frac{f}{g}\right)'(a) &= \\
                &= \limho \frac{\frac{f(a+h)}{g(a+h)}-\frac{f(a)}{g(a)}}{h} \\
                &= \limho \frac{f(a+h)g(a) - f(a)g(a) + f(a)g(a) -f(a)g(a+h)}{g(a+h)g(a)h} \\
                &= \limho \frac{1}{g(a+h)g(a)}  
                \left( \limho g(a) \limho \frac{f(a+h) - f(a)}{h}
                +\limho (-f(a)) \limho \frac{g(a+h) - g(a)}{h}\right) \\
                &= \frac{1}{g^2(a)} \left(g(a)f'(a) - f(a)g'(a)\right).
            \end{align*}
    \end{enumerate}
\end{proof}

\begin{theorem}[derivace složené funkce]
    \Necht $f$ má derivaci v bodě $y_0, g$ má derivaci v $x_0$ a je v $x_0$ spojitá
    a $y_0 = g(x_0).$ Pak:
    $$(f \circ g)'(x_0) = f'(y_0)g'(x_0) = f'(g(x_0))g'(x_0),$$
    je-li výraz vpravo definován.
\end{theorem}

\begin{example}
    Určete derivaci funkce $e^{x^2}$ v bodě $a$.

    Funkce $e^{x^2}$ je složená funkce $f(g(x)):$
    $$f(y) = e^y, g(x) = x^2, f'(y) = e^y, g'(x) = 2x$$
    Potom:
    $$(e^{x^2})' = e^{x^2} \cdot 2x.$$
\end{example}

\begin{proof}
    Potřebujeme upravit následující limitu:
    $$\limho \frac{f(g(x+h)) - f(g(x))}{h}.$$
    Z existence derivace funkce $f$ v bodě $y_0$ víme, že tato funkce je 
    definována na jistém okolí toho bodu: $U(y_0, \e).$ Funkce $g$ je spojitá
    v $x_0$ a $g(x_0) = y_0.$ Z toho vyplývá, že $\exists \delta >0:
    g(U(x_0, \delta)) \subseteq U(y_0, \e)$ a tedy že složená funkce
    $f \circ g$ je definovaná na $U(x_0, \delta).$
    
    Zabývejme se nejprve případem, kdy derivace funkce $f$ v bodě $y_0$ je vlastní,
    tj. $f'(y_0) \in \R.$ Definujme následující funkci:
    $$F(y) \coloneqq 
        \begin{cases}
            \frac{f(y) - f(y_0)}{y-y_0} &y \neq y_0 \\
            f'(y_0) &y=y_0.
        \end{cases}
    $$
    Funkce $F$ je spojitá v bodě $y_0.$ Díky spojitosti funkce $g$ platí 
    $\lim_{x \to x_0} g(x) = g(x_0) = y_0.$ Dle věty o limitě 
    složené funkce (Věta~\ref{th:slozenafunkce}), podmínka (P1) platí:
    \begin{align*}
        f'(y_0) &= \lim_{y \to y_0} F(y) \\
                &= \lim_{x \to x_0} F(g(x)) \\
                &= \lim_{x \to x_0} \frac{f(g(x)) - f(g(x_0))}{x-x_0} \\
                &= \lim_{x \to x_0} \frac{f(g(x)) - f(g(x_0))}{g(x)-g(x_0)} \cdot \frac{g(x) - g(x_0)}{x-x_0} \\
                &= \lim_{x\to x_0} F(g(x)) \frac{g(x) - g(x_0)}{x-x_0} \\
                &= \lim_{x\to x_0} F(g(x)) \lim_{x \to x_0} \frac{g(x)-g(x_0)}{x-x_0} \\
                &= f'(y_0)g'(x_0).
    \end{align*}

    Uvažujme nyní případ, kdy derivace funkce $f$ v bodě $y_0$ je nevlastní, 
    tj. $f'(y_0) = \pm \infty.$ V tomto případě musí platit, že 
    $g'(x_0) \neq 0,$ jinak by nebyl výraz $f'(y_0)g'(x_0)$ definován.
    Z toho vyplývá, že:
    $$\exists \delta > 0 \; \fa x \in P(x_0, \delta): \frac{g(x) - g(x_0)}{x-x_0}
    \neq 0,$$
    a tedy, že na tomto okolí $g(x) \neq g(x_0).$ Definujme dále funkci 
    $F(y)$ pro $y \neq y_0$:
    $$F(y) = \frac{f(y) - f(y_0)}{y - y_0}.$$
    Potom můžeme psát:
    \begin{align*}
        \lim_{x \to x_0} \frac{f(g(x)) - f(g(x_0))}{x-x_0} 
        &= \lim_{x \to x_0} F(g(x)) \cdot \frac{g(x)-g(x_0)}{x-x_0} \\
        &= f'(y_0)g'(x_0) \tag{viz níže} 
    \end{align*}
    V posledním kroce jsme vuyžili věty o limitě složené funkce 
    (Věta~\ref{th:slozenafunkce}) a podmínky (P2): $g(x)$ nenabývá své limity
    na okolí $x_0.$
\end{proof}

\begin{theorem}[derivace inversní funkce]
    \label{th:derivaceinv}
    \Necht $f$ je na intervalu $(a,b)$ spojitá a rostoucí (klesající). \Necht
    $f$ má v bodě $x_0 \in (a,b)$ derivaci $f'(x_0)$ vlastní a různou od nuly.
    Potom má funkce $f^{-1}$ derivaci v bodě $y_0 = f(x_0)$ a platí:
    $$(f^{-1})'(y_0) = \frac{1}{f'(f^{-1}(y_0))}.$$
\end{theorem}

\begin{proof}
    Definujme funkci $F(x):$
    $$F(x) = 
        \begin{cases}
            \frac{f(x) - f(x_0)}{x-x_0} &x \neq x_0 \\
            f'(x_0) &x=x_0.
        \end{cases}
    $$
    Tato funkce je spojitá v bodě $x_0$. Definujme dále funkci $g(y):$
    $$g(y) = f^{-1}(y).$$
    Funkce inversní k $f$ je spojitá (Věta~\ref{th:inversnifce}), a tedy:
    $$\lim_{y \to y_0} g(y) = \lim_{y \to y_0} f^{-1}(y) = f^{-1}(y_0).$$
    Potom dle věty o limitě složené funkce (Věta~\ref{th:slozenafunkce}, (P1)):
    $$\lim_{y\to y_0} F(g(y)) = f'(x_0).$$
    Zároveň ovšem:
    \begin{align*}
        \lim_{y\to y_0} F(g(y)) 
        &= \lim_{y\to y_0} \frac{f(f^{-1}(y)) - f(f^{-1}(y_0))}{f^{-1}(y) - f^{-1}(y_0)} \\
        &= \lim_{y\to y_0} \frac{y - y_0}{f^{-1}(y) - f^{-1}(y_0)}
    \end{align*}
    Nakonec tedy:
    $$(f^{-1})'(y_0) = \lim_{y\to y_0}\frac{f^{-1}(y) - f^{-1}(y_0)}{y - y_0} = \frac{1}{f'(x_0)}$$
\end{proof}

\begin{remark}[Derivace elementárních funkcí]
    \label{rm:derivaceelfce}
    \leavevmode
    \begin{itemize}
        \item $(k)' = 0,$ $k$ je konstanta.
        \item $(x^n)' = nx^{n-1}, n \in \N, x \in \R.$
        \item $(e^x)' = e^x, x \in \R.$
        \item $x^a, x> 0, a \in \R:$
            $$(x^a)' = (e^{a\log x})' = e^{a \log x} \cdot (a \log x)' = 
            x^a \cdot \frac{a}{x} = ax^{a-1}.$$
        \item $\sin x:$
            \begin{align*}
                (\sin x)' &= \limho \frac{\sin(x+h) - \sin(x)}{h} \\
                          &= \limho \frac{2\sin(\frac{x+h-x}{2})\cos(\frac{x+h+x}{2})}{h} \\
                          &= \limho \frac{\sin(\frac{h}{2})}{\frac{h}{2}} \cdot \limho \cos(x + \frac{h}{2}) \\
                          &= 1 \cdot \cos x = \cos x.
            \end{align*}
        \item $\cos x:$
            \begin{align*}
                (\cos x)' &= \limho \frac{\cos(x+h) - \cos x}{h} \\
                          &= \limho \frac{-2\sin(x+\frac{h}{2})\sin(\frac{h}{2})}{h} \\
                          &= - \limho \frac{\sin(\frac{h}{2})}{\frac{h}{2}} \cdot \limho \sin(x + \frac{h}{2}) \\
                          &= -\sin x.
            \end{align*}
        \item $\tan x:$
            \begin{align*}
                (\tan x)' &= \left(\frac{\sin x}{\cos x}\right)' \\
                         &= \frac{\sin' x \cos x - \sin x \cos' x}{\cos^2 x} \\
                         &= \frac{\cos^2 x + \sin^2 x}{\cos^2 x} \\
                         &= \frac{1}{\cos^2 x}.
            \end{align*}

        \item $(\cotg x)' = - \frac{1}{\sin^2 x},$ analogicky.
        \item $\arcsin x, x \in (-1, 1):$
            \begin{align*}
                (\arcsin x)' &= \frac{1}{\sin' y} \tag{pro $y = \arcsin x$, Věta~\ref{th:derivaceinv}} \\
                             &= \frac{1}{\cos y} \\
                             &= \frac{1}{\sqrt{1 - x^2}}. \tag{$\cos^2 y = 1 - \sin^2 y = 1 - x^2$}
            \end{align*}
        \item $(\arccos x)' = -\frac{1}{\sqrt{1 - x^2}}.$
        \item $(\arctan x)' = \frac{1}{1 + x^2}.$
        \item $(\arccotg x)' = - \frac{1}{1 + x^2}$
    \end{itemize}
\end{remark}

\begin{theorem}[Fermatova]
    \label{th:fermat}
    \Necht $a \in \R$ je bod lokálního extrému funkce $f$ na $M.$ Pak $f'(a)$ 
    neexistuje nebo $f'(a) = 0.$
\end{theorem}

\begin{proof}
    Sporem. \Necht v bodě $a$ existuje nenulová derivace, tj. 
    $f'(a) \neq 0.$ \Necht je \buno $f'(a) > 0.$ Potom:
    $$\exists \delta > 0 \; \fa x \in P(a, \delta): \frac{f(x) - f(a)}{x-a} > 0.$$
    Zvolme $\e = \frac{f'(a)}{2}.$ Potom:
    $$\exists \delta > 0 \; \fa x \in P(a, \delta): 
        |\frac{f(x) - f(a)}{x-a} - f'(a)| < \frac{f'(a)}{2}.$$
    Po úpravě:
    $$\fa x \in P(a, \delta): 0 < \frac{f'(a)}{2} < \frac{f(x) - f(a)}{x-a}.$$
    Potom pro $x \in P_+(a, \delta): x - a> 0,$ a tedy $f(x) - f(a) > 0.$ Naopak,
    pro $x \in P_-(a, \delta): x - a < 0,$ a tedy $f(x) - f(a) < 0.$
    V tom případě ovšem bod $a$ není ani lokálním maximem, ani lokálním minimem.
\end{proof}

\begin{remark}
    V typické úloze máme spojitou funkci na intervalu $[a,b]$ a naším úkolem je
    nalézt její maxima a minima:
    \begin{enumerate}[a.]
        \item Dle věty o spojitosti funkce a nabývání extrémů 
            (Věta~\ref{th:spojitaextremy}) víme, že tato funkce nabývá na daném
            intervalu $[a,b]$ svých extrémů.
        \item Dále dle Fermatovy věty (Věta~\ref{th:fermat}) víme, v jakých 
            bodech se tyto extrémy mohou nacházet, tj. víme, kde hledat:
            $$\{x \in [a,b], f'(x) = 0 \} 
            \cup \{x \in [a,b], f'(x) \text{ neexistuje}\}
            \cup \{a,b\}$$
    \end{enumerate}
\end{remark}

\begin{theorem}[Rolleova]
    \label{th:rolle}
    \Necht $f$ je spojitá na intervalu $[a,b],$ $f'(x)$ existuje pro každé
    $x \in (a,b)$ a $f(a) = f(b).$ Pak existuje $\xi \in (a,b): f'(\xi) = 0.$
\end{theorem}

\begin{proof}
    Pokud pro $\fa x \in [a,b]: f(x) = f(a),$ potom $\fa x \in (a,b): f'(x) = 0.$ 

    \Necht existuje $x \in (a,b)$ \tz $f(x) \neq f(a).$ \Necht \buno $f(x) > f(a).$
    Spojitá funkce na $[a,b]$ nabývá extrémů, a tedy dle Věty~\ref{th:spojitaextremy}:
    $$\exists \xi \in [a,b]: f(\xi) = \max_{x \in [a,b]} f(x).$$
    Jelikož víme, že $\exists x \in (a,b)$ \tz $f(x) > f(a),$ vyplývá, že $\xi \neq a$
    a $\xi \neq b.$ Dále z předpokladů víme, že $f'(x)$ existuje pro všechna 
    $x \in (a,b).$ Z toho nutně vyplývá, že $f'(\xi) = 0.$
\end{proof}

\begin{theorem}[Lagrangeova věta o střední hodnotě]
    \label{th:lagrangemean}
    \Necht je funkce $f$ spojitá na intervalu $[a,b]$ a má derivaci v každém bodě
    intervalu $(a,b).$ Pak existuje $\xi \in (a,b)$ \tz
    $$f'(\xi) = \frac{f(b) - f(a)}{b -a}.$$
\end{theorem}

\begin{proof}
    Definujme následující funkci:
    $$F(x) \coloneqq f(x) - f(a) - \frac{f(b) - f(a)}{b - a}(x-a).$$
    Tato funkce je spojitá na $[a,b]$ a zároveň má derivaci na $(a,b).$ Dále
    $F(a) = F(b) = 0.$ Funkce $F(x)$ tedy na intervalu $[a,b]$ splňuje podmínky
    Rolleovy věty (Věta~\ref{th:rolle}), ze které vyplývá, že existuje $\xi \in
    (a,b)$ \tz $F'(\xi) = 0.$ Potom:
    $$0 = F'(\xi) = f'(\xi) - 0 - \frac{f(b) - f(a)}{b-a} \cdot 1,$$
    a tedy:
    $$f'(\xi) = \frac{f(b) - f(a)}{b-a}.$$
\end{proof}

\begin{theorem}[Cauchyho věta o střední hodnotě]
    \label{th:cauchymean}
    \Necht $f,g$ jsou spojité funkce na intervalu $[a,b]$ takové, že $f$ má v každém
    bodě $(a,b)$ derivaci a $g$ má v každém bodě $(a,b)$ vlastní derivaci různou 
    od nuly. Pak existuje $\xi \in (a,b)$ \tz
    $$\frac{f'(\xi)}{g'(\xi)} = \frac{f(b) - f(a)}{g(b) - g(a)}.$$
\end{theorem}

\begin{proof}
    Z předpokladů vyplývá, že $g(b) \neq g(a), $ protože jinak by dle Rolleovy 
    věty (Věta~\ref{th:rolle}) existovalo $x \in (a,b): g'(x) = 0.$
    Definujme podobně jako výše následující funkci:
    $$H(x) \coloneqq (f(b) - f(a))(g(x)-g(a)) - (f(x) - f(a))(g(b) - g(a)).$$
    Funkce $H(x)$ je spojitá na $[a,b]$ a má na intervalu $(a,b)$ derivaci.
    Dále $H(a) = H(b) = 0$ a tedy dle Rolleovy věty (Věta~\ref{th:rolle})
    $$\exists \xi \in (a,b): H'(\xi) = 0.$$
    Potom:
    $$0 = H'(\xi) = (f(b) - f(a))(g'(\xi) - 0) - (f'(x)-0)(g(b) - g(a))$$
    a tedy:
    $$\frac{f'(\xi)}{g'(\xi)} = \frac{f(b) - f(a)}{g(b) - g(a)}.$$
\end{proof}

\begin{theorem}[l'Hospitalovo pravidlo]
    \label{th:lhospital}
    \leavevmode
    \begin{enumerate}[(i)]
        \item \Necht $a \in \Rstar, \lim_{x \to a+} f(x) = \lim_{x \to a+} g(x) = 0$
            a \necht existuje $\lim_{x \to a+} \frac{f'(x)}{g'(x)}.$ Pak
            $$ \lim_{x \to a+} \frac{f(x)}{g(x)} = \lim_{x \to a+} \frac{f'(x)}{g'(x)}.$$
        \item \Necht $a \in \Rstar, \lim_{x \to a+} |g(x)| = \infty$
            a \necht existuje $\lim_{x \to a+} \frac{f'(x)}{g'(x)}.$ Pak
            $$ \lim_{x \to a+} \frac{f(x)}{g(x)} = \lim_{x \to a+} \frac{f'(x)}{g'(x)}.$$
   \end{enumerate}
\end{theorem}

\begin{remark}[Příklad využití l'Hospitalova pravidla]
    Vypočtěte následující limitu:
    $$\limxo \frac{\sin x - x}{x^3}.$$
    Limita je typu $\frac{0}{0}, $ je zde tedy šance na použití bodu (i) z 
    l'Hospitalova pravidla. Vyjádřeme limitu derivací:
    $$\limxo \frac{(\sin x - x)'}{(x^3)'} = \limxo \frac{\cos x - 1}{3x^2}.$$
    Dostali jsme znovu limitu typu $\frac{0}{0}.$ Pokusme se znovu zderivovat jak
    čitatel, tak jmenovatel:
    \begin{align*}
        \limxo \frac{(\cos x - 1)'}{(3x^2)'} &= \limxo \frac{-\sin x}{6x} \\
                                             &= \limxo \frac{-1}{6} \cdot \limxo \frac{\sin x}{x} \\
                                             &= -\frac{1}{6} \tag{$\limxo \frac{\sin x}{x} = 1$, Věta~\ref{th:goniom}}
    \end{align*}
    Díky l'Hospitalovu pravidlu dostáváme:
    $$\limxo \frac{\sin x - x}{x^3} = \limxo \frac{\cos x - 1}{3x^2} = \limxo \frac{-\sin x}{6x}  = - \frac{1}{6}.$$
    Mimochodem, pro vyjádření poslední limity jsme mohli místo vlastností sina
    použít další iteraci l'Hospitalova pravidla; výsledek bychom dostali stejný.
\end{remark}

\begin{proof}
    \leavevmode
    \begin{enumerate}[(i)]
        \item 
            \begin{itemize}
                \item $a \in \R, \lim_{x \to a+} \frac{f'(x)}{g'(x)} = A \in \Rstar.$

                    Jelikož limita je definována pouze pro reálné funkce, 
                    vyplývá z existence $\lim_{x \to a+} \frac{f'(x)}{g'(x)},$
                    že:
                    $$\exists \delta > 0 \; \fa x \in P_+(a, \delta): f'(x) \in \R
                    \text{ a } g'(x) \in \R \setminus \{0\}.$$

                    Dodefinujme funkce $f$ a $g$ \tz $f(a) = g(a) = 0.$ Vezměme
                    nějaké $x \in P_+(a, \delta).$ Funkce $f$ a $g$ jsou
                    spojité na $[a,x], $ neboť mají vlastní derivaci na 
                    intervalu $(a,x]$ a v bodě $a$ jsme spojitost dodefinovali.
                    Tím jsou splněny předpoklady Cauchyho věty o střední
                    hodnotě (Věta~\ref{th:cauchymean}) a platí, že 
                    $$\fa x \in P_+(a, \delta) \; \exists \xi(x) \in (a,x):
                    \frac{f'(\xi(x))}{g'(\xi(x))} = \frac{f(x) - f(a)}{g(x) - g(a)}
                    \stackrel{\text{při }a=0}{=} \frac{f(x)}{g(x)}$$

                    Zvolme $\e > 0.$ Z existence
                    $\lim_{x \to a+} \frac{f'(x)}{g'(x)}$ víme, že 
                    $$\exists \delta_1 < \delta \; \fa x \in P_+(a, \delta_1):
                    \frac{f'(x)}{g'(x)} \in U(A, \e).$$

                    Nyní, $\fa x \in P_+(a, \delta_1)$ platí, že $a < \xi(x) < x,$
                    a tedy $\frac{f'(\xi(x))}{g'(\xi(x))} \in U(A, \e)$, a tím
                    pádem i $\frac{f(x)}{g(x)} \in U(A, \e).$ Proto:
                    $$\lim_{x \to a+} \frac{f(x)}{g(x)} = A.$$

                \item $a = - \infty, \lim_{x \to a+} \frac{f'(x)}{g'(x)} = A \in \Rstar.$

                    Jelikož platí:
                    $$\lim_{x \to -\infty} h(x) = C \iff \lim_{x \to 0+} h(-\frac{1}{x}) = C,$$
                    můžeme tento případ převést na předchozí pomocí substituce.

                    Definujme následující dvě funkce:
                    $$F(y) = f(-\frac{1}{y}), \; G(y) = g(-\frac{1}{y}).$$
                    Jejich derivace jsou rovny:
                    $$F'(y) = f'(-\frac{1}{y})(\frac{1}{y^2}), 
                    \; G'(y) = g'(-\frac{1}{y})(\frac{1}{y^2}).$$
                    Potom:
                    $$\lim_{x \to -\infty}\frac{f(x)}{g(x)} = 
                    \lim_{y \to 0+} \frac{F(y)}{G(y)} =
                    \lim_{y \to 0+} \frac{F'(y)}{G'(y)} = 
                    \lim_{y \to 0+} \frac{f'(-\frac{1}{y})(\frac{1}{y^2})}{g'(-\frac{1}{y})(\frac{1}{y^2})} =
                    \lim_{x \to -\infty} \frac{f'(x)}{g'(x)}.$$
            \end{itemize}

        \item Tuto variantu si dokážeme pouze pro případ 
            $a \in \R, \lim_{x \to a+} \frac{f'(x)}{g'(x)} = A \in \R, 
            \lim_{x \to a+} g(x) = \infty.$

            Zvolme $0<\e<1.$ Z předpokladů víme, že:
            $$\exists \delta_1 > 0 \; \fa y \in P_+(a, \delta_1): 
            |\frac{f'(y)}{g'(y)} - A| < \e.$$
            Zvolme pevné $y \in P_+(a, \delta_1).$ Z $\lim_{x \to 0+} g(x) = \infty$
            dále vyplývá, že:
            $$\exists \delta_2 < \delta_1 \; \fa x \in P_+(a, \delta_2):
            |\frac{g(y)}{g(x)}| < \e \land |\frac{f(y)}{g(x)}| < \e.$$

            Nechť $x \in P_+(a, \delta_2): a < x < y.$ Potom na $[x,y]$ splňují
            funkce $f,g$ předpoklady Cauchyho věty o střední hodnotě 
            (Věta~\ref{th:cauchymean}), a tedy:
            $$\exists \xi \in (x,y): \frac{f(y) - f(x)}{g(y)-g(x)} = 
            \frac{f'(\xi)}{g'(\xi)}.$$
            Potom:
            $$f(y) - f(x) = \frac{f'(\xi)}{g'(\xi)} g(y) - 
            \frac{f'(\xi)}{g'(\xi)} g(x).$$
            Po vydělení obou stran rovnice výrazem $\frac{1}{g(x)}:$
            $$\frac{f(y)}{g(x)} - \frac{f(x)}{g(x)} = 
            \frac{f'(\xi)}{g'(\xi)}\frac{g(y)}{g(x)} - 
            \frac{f'(\xi)}{g'(\xi)}\frac{g(x)}{g(x)}.$$
            Po úpravách:
            $$\frac{f(x)}{g(x)}  =\frac{f'(\xi)}{g'(\xi)} + 
            \frac{f(y)}{g(x)} -\frac{f'(\xi)}{g'(\xi)}\frac{g(y)}{g(x)}.$$
            Potom:
            \begin{align*}
                |\frac{f(x)}{g(x)} - A| &= |\frac{f'(\xi)}{g'(\xi)} + 
                \frac{f(y)}{g(x)} -\frac{f'(\xi)}{g'(\xi)}\frac{g(y)}{g(x)} - A| \\
                &\leq |\frac{f'(\xi)}{g'(\xi)} - A| + 
                |\frac{f'(\xi)}{g'(\xi)}\frac{g(y)}{g(x)}| +
                |\frac{f(y)}{g(x)}| \tag{trojúhelníková nerovnost} \\
                &< \e + (|A| + \e)\e + \e.
            \end{align*}
            A tedy:
            $$\lim_{x \to a+} \frac{f(x)}{g(x)} = A.$$
    \end{enumerate}
\end{proof}

\begin{theorem}[derivace a limita derivace]
    \label{th:derlimder}
    \Necht je funkce $f$ spojitá zprava v $a$ a \necht existuje $\lim_{x \to a+}
    f'(x) = A \in \Rstar.$ Pak $f'_+(a) = A.$
\end{theorem}

\begin{proof}
    Vyjádřeme $f'_+(a)$ pomocí definice:
    \begin{align*}
        f'_+(a) &= \lim_{x \to a+} \frac{f(x) - f(a)}{x-a} \\
                &\stackrel{?}{=} \lim_{x \to a+} \frac{f'(x)}{1} 
                    \tag{L'Hospitalovo pravidlo, nutno ověřit podmínky} \\
                &= A.
    \end{align*}
    Nyní je třeba ověřit, že použití L'Hospitalova pravidla 
    (Věta~\ref{th:lhospital}) bylo korektní. Jde jednoduše nahlédnout, že
    se jedná o případ (i): $\frac{0}{0}.$ 
\end{proof}

\begin{definition}
    \Necht $J$ je interval. Množinu všech vnitřních bodů $J$ nazýváme 
    \newterm{vnitřek} J a značíme $\vnitrek J.$
\end{definition}

\begin{theorem}[o vztahu derivace a monotonie]
    \label{th:derivacemonotonie}
    \Necht $J \subseteq \R$ je interval a $f$ je spojitá na $J$ a v každém
    vnitřním bodě $J$ má derivaci.
    \begin{enumerate}[(i)]
        \item Je-li $f'(x) > 0$ na $\vnitrek J$, pak je $f$ rostoucí na $J.$
        \item Je-li $f'(x) < 0$ na $\vnitrek J$, pak je $f$ klesající na $J.$
        \item Je-li $f'(x) \geq 0$ na $\vnitrek J$, pak je $f$ neklesající na $J.$
        \item Je-li $f'(x) \leq 0$ na $\vnitrek J$, pak je $f$ nerostoucí na $J.$
    \end{enumerate}
\end{theorem}

\begin{proof}
    Ukážeme si pouze první případ; ostatní se dokazují analogicky.

    Mějme body $a,b \in J, a < b.$ Funkce $f$ je spojitá na $[a,b]$ a má derivaci
    v každém vnitřním bodu intervalu $(a,b).$ Dle Lagrangeovy věty o střední
    hodnotě (Věta~\ref{th:lagrangemean}) platí, že
    $$\exists \xi \in (a,b): f'(\xi) = \frac{f(b) - f(a)}{b-a}.$$ Jelikož
    dle předpokladů $f'(\xi) > 0$ a zároveň $b-a>0$, musí platit i $f(b) - f(a) > 0.$
    Funkce $f$ je tedy na intervalu $J$ rostoucí.
\end{proof}

\begin{remark}
    Implikace v předchozí větě neplatí v opačném směru: Uvažujte například
    funkci $f(x) = x^3$ a případ (i).
\end{remark}

\begin{definition}
    \Necht $n \in \N, a \in \R$ a \necht $f$ má vlastní $n$-tou derivaci na okolí
    bodu $a.$ Pak \newterm{$(n+1)$-ní derivací} funkce $f$ v bodě $a$ budeme rozumět:
    $$f^{(n+1)}(a) = \lim_{h \to 0} \frac{f^{(n)}(a + h) - f^{(n)}(a)}{h}.$$
\end{definition}

\subsection{Konvexní a konkávní funkce}

\begin{definition}
    \Necht $f$ má vlastní derivaci v bodě $a \in \R.$ Označme
    $$T_a = \{[x,y], x \in \R, y = f(a) + f'(a)(x-a)\}.$$
    Řekneme, že bod $[x, f(x)], x \in D_f$ leží \newterm{nad (pod) tečnou} $T_a,$
    jestliže platí:
    $$f(x) > f(a) + f'(a)(x-a) (f(x) < f(a) + f'(a)(x-a)).$$
\end{definition}

\begin{definition}
    Funkce $f$ má v bodě $a$ \newterm{inflexi} ($a$ je \newterm{inflexní bod}),
    jestliže $f'(a) \in \R$ a existuje $\Delta > 0$ \tz
    \begin{enumerate}[(i)]
        \item $\fa x \in (a-\Delta, a): [x,f(x)]$ leží nad tečnou,
        \item $\fa x \in (a, a+ \Delta): [x,f(x)]$ leží pod tečnou,
    \end{enumerate}
    nebo
    \begin{enumerate}[(i)]
        \item $\fa x \in (a-\Delta, a): [x,f(x)]$ leží pod tečnou,
        \item $\fa x \in (a, a+ \Delta): [x,f(x)]$ leží nad tečnou.
    \end{enumerate}
\end{definition}

\begin{theorem}[nutná podmínka pro inflexi]
    \Necht $f''(a) \neq 0.$ Pak $a$ není inflexní bod funkce $f.$
\end{theorem}

\begin{proof}
    \Necht je \buno $f''(a) > 0,$ a tedy
    $$\lim_{x \to a} \frac{f'(x) - f'(a)}{x-a} > 0.$$
    Ukážeme, že body $[x,f(x)]$ leží nad tečnou $T_a$
    jak v levém, tak v pravém okolí bodu $a.$

    Díky předpokladu $f''(a) > 0$ existuje $\delta > 0$ \tz:
    $$\fa x \in P_+(a,\delta): f'(x) > f'(a) \land
    \fa x \in P_-(a, \delta): f'(x) < f'(a).$$

    Zvolme libovolné $y \in P_+(a,\delta).$ Funkce $f$ je spojitá na $[a,y]$ a
    má vlastní derivace ve všech bodech intervalu $(a,y).$ Potom dle
    Langrageovy věty o střední hodnotě (Věta~\ref{th:lagrangemean}):
    $$\exists \xi_1 \in (a,y): f'(a) < f'(\xi_1) = \frac{f(y) - f(a)}{y-a}.$$
    Pro všechna $y \in P_+(a,\delta)$ tedy platí:
    $$f(y) > f(a) + f'(a)(y-a);$$
    jinými slovy, leží nad tečnou $T_a.$

    Zvolme nyní libovolné $z \in P_-(a, \delta).$ Analogicky ukážeme, že
    $$\exists \xi_2 \in (z, a): f'(a) > f'(\xi_2) = \frac{f(a) - f(z)}{a-z},$$
    a tedy, že pro všechna $z \in P_-(a, \delta):$
    $$f(z)>f(a) + f'(a)(z-a)$$
\end{proof}

\begin{theorem}[postačující podmínka pro inflexi]
    \label{th:inflexepostac}
    \Necht existuje $f'(a) \in \R$ a \necht existuje $\delta > 0$ \tz:
    $$\fa x \in P_+(a, \delta): f''(x) > 0 
    \land \fa x \in P_-(a, \delta): f''(x) < 0.$$
    Pak $z$ je inflexní bod $f.$
\end{theorem}

\begin{proof}
    Jelikož $f''(a) > 0$ na $P_+(a, \delta),$ je dle věty o vztahu derivace
    a monotonie (Věta~\ref{th:derivacemonotonie}) $f'(x)$ rostoucí na 
    $P_+(a, \delta),$ a tedy $\fa x \in P_+(a, \delta): f'(x) > f'(a).$

    Zvolme $x \in P_+(a,\delta).$ Dle Lagrangeovy věty o střední hodnotě 
    (Věta~\ref{th:lagrangemean}) platí, že:
    $$\exists \xi_1 \in (a,x): f'(a) < f'(\xi_1) = \frac{f(x) - f(a)}{x-a}.$$
    Z tohoto vztahu vyplývá, že funkce je nad tečnou $T_a$ v pravém okolí
    bodu $a.$
    
    Analogicky ukážeme, že v levém okolí bodu $a$ je funkce pod tečnou $T_a$,
    čímž jsou splněny podmínky pro inflexi.
\end{proof}

\begin{definition}
    Funkci $f$ na intervalu $I$ nazveme \newterm{konvexní (konkávní)}, jestliže:
    $$\fa x_1,x_2,x_3 \in I, x_1 < x_2 < x_3 \implies
    \frac{f(x_2) - f(x_1)}{x_2 - x_1} \leq \frac{f(x_3) - f(x_1)}{x_3 - x_1}$$
    $$\left(\fa x_1,x_2,x_3 \in I, x_1 < x_2 < x_3 \implies
    \frac{f(x_2) - f(x_1)}{x_2 - x_1} \geq \frac{f(x_3) - f(x_1)}{x_3 - x_1}\right).$$
    Funkci nazveme \newterm{ryze konvexní (ryze konkávní)}, je-li příslušná
    nerovnost ostrá.
\end{definition}

\begin{remark}[ekvivalentní definice konvexity]
    Funkce $f$ je na intervalu $J$ konvexní, pokud:
    $$\fa \alpha \in (0,1) \; \fa x,y \in J: f(\alpha x + (1-\alpha)y) \leq
    \alpha f(x) + (1-\alpha)f(y).$$
\end{remark}

\begin{lemma}
    \label{lm:konvexita}
    \Necht je funkce $f$ na intervalu $I$ konvexní, pak:
    $$\fa x_1,x_2,x_3 \in I, x_1 < x_2 < x_3 \implies
    \frac{f(x_2) - f(x_1)}{x_2 - x_1} \leq \frac{f(x_3) - f(x_1)}{x_3 - x_1}
    \leq \frac{f(x_3) - f(x_2)}{x_3 - x_2}.$$
\end{lemma}

\begin{proof}
    Platnost první nerovnosti je dána již z definice konvexnosti. Nás proto
    zajímá druhá nerovnost, tj. chceme dokázat, že platí:
    $$\fa x_1,x_2,x_3 \in I, x_1 < x_2 < x_3 \implies
    \frac{f(x_3) - f(x_1)}{x_3 - x_1}
    \leq \frac{f(x_3) - f(x_2)}{x_3 - x_2}.$$
    Platí:
    \begin{align*}
        \frac{f(x_3) - f(x_2)}{x_3 - x_2} 
        &= \frac{f(x_3) - f(x_1) + f(x_1) - f(x_2)}{x_3 - x_2} \\
        &=    \frac{f(x_3) - f(x_1)}{x_3 - x_2} - \frac{f(x_2) - f(x_1)}{x_2 - x_1} \frac{x_2 - x_1}{x_3 - x_2} \\
        &\geq \frac{f(x_3) - f(x_1)}{x_3 - x_2} - \frac{f(x_3) - f(x_1)}{x_3 - x_1} \frac{x_2 - x_1}{x_3 - x_2} \tag {z definice konvexity} \\
        &\geq \frac{f(x_3) - f(x_1)}{x_3 - x_2} - \frac{f(x_3) - f(x_1)}{x_3 - x_2} \frac{x_2 - x_1}{x_3 - x_1} \\
        &\geq \frac{f(x_3) - f(x_1)}{x_3 - x_2} \left(1 - \frac{x_2 - x_1}{x_3 - x_1}\right) \\
        &\geq \frac{f(x_3) - f(x_1)}{x_3 - x_2} \left(\frac{x_3 - x_1 - x_2 + x_1}{x_3 - x_1}\right) \\
        &= \frac{f(x_3) - f(x_1)}{x_3 - x_1}
    \end{align*}
\end{proof}

\begin{lemma}
    \Necht je funkce $f$ na intervalu $I$ ryze konvexní, pak:
    $$\fa x_1,x_2,x_3 \in I, x_1 < x_2 < x_3 \implies
    \frac{f(x_2) - f(x_1)}{x_2 - x_1} < \frac{f(x_3) - f(x_1)}{x_3 - x_1}
    < \frac{f(x_3) - f(x_2)}{x_3 - x_2}.$$
\end{lemma}

\begin{theorem}[konvexita a jednostranné derivace]
    \label{th:konvexitajednostrannederivace}
    \Necht je funkce $f$ na intervalu $J$ konvexní a $a \in \vnitrek J.$
    Pak $f'_+(a) \in \R$ a $f'_-(a) \in \R.$
\end{theorem}

\begin{proof}
    Omezíme se na důkaz existence $f'_+(a).$ Případ jednostranné derivace zleva
    se dokazuje analogicky.

    Naším úkolem je dokázat, že limita:
    $$\lim_{x \to a+} \frac{f(x) - f(a)}{x-a}$$
    existuje a že je reálná. Z konvexity funkce $f$ na $J$ vyplývá, že
    existuje $\delta > 0$ \tz funkce 
    $$H(x) = \frac{f(x) - f(a)}{x-a}$$
    je na $(a, a+\delta)$ neklesající a tedy dle věty o limitě monotónní funkce
    (Věta~\ref{th:limitamonotonnifce}) má limitu $A$.

    Navíc, nechť $y \in J, y<a.$ Potom dle lemmatu~\ref{lm:konvexita} platí:
    $$\fa x \in (a, a+\delta): \frac{f(a) - f(y)}{a-y} \leq \frac{f(x) - f(a)}{x-a}.$$
    Funkce $H(x)$ je tedy na $(a, a+\delta)$ omezená zdola a $A \in \R.$
\end{proof}

\begin{theorem}
    \Necht $f$ je konvexní na otevřeném intervalu $J.$ Pak je $f$ spojitá na $J.$
\end{theorem}

\begin{proof}
    Uvažujme libovolný bod $a \in \vnitrek J.$ Funkce $f$ je konvexní a tedy 
    dle věty o konvexitě a jednostranných derivacích 
    (Věta~\ref{th:konvexitajednostrannederivace}) existují jednostranné
    derivace funkce $f$ v bodě $a.$ Z věty o vztahu derivace a 
    spojistosti a následující poznámky (Věta~\ref{th:derivacespojitost}, 
    Poznámka~\ref{rm:jednostrannaderivacespojitost}) dále vyplývá, že funkce 
    funkce $f$ je v bodě $a$ spojitá zleva i zprava, a tedy spojitá.
\end{proof}

\begin{theorem}[vztah druhé derivace a konvexity (konkávity)]
    \Necht $f$ má na intervalu $(a,b)$ spojitou první derivaci.
    \begin{enumerate}[(i)]
        \item Jestliže $\fa x \in (a,b): f''(x) \geq 0,$ pak $f$ je konvexní na $(a,b).$
        \item Jestliže $\fa x \in (a,b): f''(x) \leq 0,$ pak $f$ je konkávní na $(a,b).$
        \item Jestliže $\fa x \in (a,b): f''(x) > 0,$ pak $f$ je ryze konvexní na $(a,b).$
        \item Jestliže $\fa x \in (a,b): f''(x) < 0,$ pak $f$ je ryze konkávní na $(a,b).$
    \end{enumerate}
\end{theorem}

\begin{proof}
    Ukážeme si pouze případ (i); ostatní se dokazují analogicky. 
    
    Nechť $x_1,x_2,x_3 \in (a,b), x_1 < x_2 < x_3.$ Dle Lagrangeovy věty
    o střední hodnotě (Věta~\ref{th:lagrangemean}):
    $$\exists \xi_1, \xi_2: f'(\xi_1) = \frac{f(x_2)-f(x_1)}{x_2-x_1} \land
    f'(\xi_2) = \frac{f(x_3)-f(x_2)}{x_3-x_2}.$$
    Dále, jelikož $\fa x \in (a,b): f''(x) \geq 0,$ první derivace funkce $f$ je
    dle věty o vztahu derivace a monotonie (Věta~\ref{th:derivacemonotonie})
    neklesající funkce. Protože $\xi_1 < \xi_2$, platí $f'(\xi_1) \leq f'(\xi_2),$
    a tedy:
    $$\frac{f(x_2)-f(x_1)}{x_2-x_1} \leq \frac{f(x_3)-f(x_2)}{x_3-x_2},$$
    po úpravě:
    $$f(x_3) \geq (f(x_2)-f(x_1))\frac{x_3-x_2}{x_2-x_1} + f(x_2).$$
    Odečtěme od obou stran nerovnice výraz $f(x_1):$
    $$f(x_3) -f(x_1) \geq (f(x_2)-f(x_1))\frac{x_3-x_2}{x_2-x_1} + f(x_2)-f(x_1)$$
    a následně je vynásobme výrazem $\frac{1}{x_3-x_1}$ a upravme:
    \begin{align*}
        \frac{f(x_3) -f(x_1)}{x_3 -x_1} 
        &\geq (f(x_2)-f(x_1)) \left(\frac{x_3-x_2}{(x_2-x_1)(x_3-x_1)} + \frac{1}{x_3-x_1}\right) \\
        &= (f(x_2)-f(x_1)) \frac{x_3-x_2+x_2-x_1}{(x_2-x_1)(x_3-x_1)} \\
        &= \frac{f(x_2)-f(x_1)}{x_2-x_1}.
    \end{align*}
    Tím jsme ukázali, že funkce $f$ je konvexní $(a,b).$
\end{proof}

\subsection{Průběh funkce}

\begin{definition}
    Řekneme, že funkce $ax + b, a,b \in \R,$ je asymptotou funkce $f$ v $+\infty$ 
    ($-\infty$), jestliže:
    $$\lim_{x \to \infty} (f(x)-(ax+b)) = 0$$
    $$(\lim_{x \to -\infty} (f(x)-(ax+b)) = 0)$$
\end{definition}

\begin{theorem}[tvar asymptoty]
    Funkce $f$ má v $\infty$ asymptotu $ax+b,$ právě když
    $$\lim_{x\to\infty}\frac{f(x)}{x} = a\in\R \; \text {a} \; 
    \lim_{x\to\infty}(f(x) - ax) = b \in \R.$$
\end{theorem}

\begin{proof}
    \leavevmode
    \begin{itemize}
        \item[$\implies$] Použijeme větu o aritmetice limit funkcí 
            (Věta~\ref{th:voalf}) a rozepíšeme obě limity. Pro koeficient $a$:
            $$\limxinf \frac{f(x)}{x} 
            \qe \limxinf \frac{f(x) - (ax + b)}{x} + \limxinf \frac{ax+b}{x}
            \qe \frac{0}{\infty} + a = a.$$
            Rozepsání je platné, jen pokud pravé strany mají smysl (proto ty 
            otazníky nad rovnítky). V tomto případě smysl mají a použití věty o 
            aritmetice limit funkcí je tedy korektní.

            Podobně pro koeficient $b:$
            $$\limxinf (f(x) - ax) 
            \qe \limxinf (f(x) - ax + b) + \limxinf b 
            \qe 0 + b = b.$$
        \item[$\impliedby$]
            $$\limxinf (f(x) - (ax + b)) 
            \qe \limxinf (f(x) - ax) - \lim b
            \qe b - b = 0.$$
    \end{itemize}
\end{proof}

\begin{remark}
    \label{rm:prubehfce}
    Při vyšetření průběhu funkce provádíme následující kroky:
    \begin{enumerate}
        \item Určíme definiční obor a obor spojitosti funkce.
        \item Zjistíme průsečíky se souřadnými osami.
        \item Zjistíme symetrii funkce: lichost, sudost, periodicita.
        \item Dopočítáme limity v "krajních bodech definičního oboru."
        \item Spočteme první derivaci, určíme intervaly monotonie a 
            nalezneme lokální a globální extrémy.
        \item Spočteme druhou derivaci a určíme intervaly, kde je $f$ konvexní
            nebo konkávní. Určíme inflexní body.
        \item Vypočteme asymptoty funkce.
        \item Načrtneme graf funkce a určíme obor hodnot.
    \end{enumerate}
\end{remark}

\begin{example}
    Vyšetřete průběh funkce:
    $$f(x) = \sqrt[3]{(x+2)^2} - \sqrt[3]{(x-2)^2}.$$
    \begin{enumerate}
        \item Určíme definiční obor a obor spojitosti funkce. 
            
            $D_f = \R, $ funkce je spojitá na $\R.$

        \item Zjistíme průsečíky se souřadnými osami. 
            
            Položme $x=0:$ 
            $$f(0) = \sqrt[3]{(0+2)^2} - \sqrt[3]{(0 - 2)^2} = 0,$$
            a $y=0:$
            \begin{align*}
                0 &= \sqrt[3]{(x+2)^2} - \sqrt[3]{(x-2)^2} \\
                \sqrt[3]{(x+2)^2} &=  \sqrt[3]{(x-2)^2} \\
                x^2 + 4x + 4 &= x^2 -4x + 4 \\
                x&=0.
            \end{align*}
            Jediný průsečík s osami $x$ a $y$ je bod $P[0,0].$
        
        \item Zjistíme symetrii funkce: lichost, sudost, periodicita.

            Pro určení parity funkce vyjádříme $f(-x):$
            \begin{align*}
                f(-x) 
                &= \sqrt[3]{(-x+2)^2} - \sqrt[3]{(-x-2)^2} \\
                &= \sqrt[3]{(-1)^2(x-2)^2} - \sqrt[3]{(-1)^2(x+2)^2} \\
                &= \sqrt[3]{(x-2)^2} - \sqrt[3]{(x+2)^2} \\
                &= -1 (\sqrt[3]{(x+2)^2} - \sqrt[3]{(x-2)^2}) \\
                &= -f(x).
            \end{align*}
            Funkce $f$ je lichá.
        \item Dopočítáme limity v "krajních bodech definičního oboru."

            \begin{align*}
                \limxinf f(x) 
                &= \limxinf \sqrt[3]{(x+2)^2} - \sqrt[3]{(x - 2)^2} \\
                &= \limxinf \sqrt[3]{(x+2)^2} - \sqrt[3]{(x - 2)^2} 
                \frac{\sqrt[3]{(x+2)^4} + \sqrt[3]{(x+2)^2(x - 2)^2} + \sqrt[3]{(x - 2)^4}}{\sqrt[3]{(x+2)^4} + \sqrt[3]{(x+2)^2(x - 2)^2} + \sqrt[3]{(x - 2)^4}} \\
                &= \limxinf \frac{(x+2)^2 - (x-2)^2}{\sqrt[3]{(x+2)^4} + \sqrt[3]{(x+2)^2(x - 2)^2} + \sqrt[3]{(x - 2)^4}} \tag{vzorec pro $a^3 - b^3$} \\
                &\sim \limxinf \frac{x}{x^{\frac{4}{3}}} 
                = \limxinf \frac{1}{x^{\frac{1}{3}}} = 0.
            \end{align*}

        \item Spočteme první derivaci, určíme intervaly monotonie a 
            nalezneme lokální a globální extrémy.
            
            \begin{enumerate}
                \item Derivace.
                    \begin{align*}
                        f'(x) 
                        &= \frac{2}{3}(x+2)^{\frac{2}{3}-1} \cdot 1 
                            - \frac{2}{3}(x-2)^{\frac{2}{3}-1}\cdot 1 \\
                        &= \frac{2}{3}\left(\frac{1}{\sqrt[3]{x+2}} - \frac{1}{\sqrt[3]{x-2}}\right)
                    \end{align*}
                    Tento výraz platí pro $\fa x \neq \pm2.$ Pro určení derivace v bodech
                    $x = \pm 2$ se pokusíme využít věty o derivaci a limitě derivace
                    (Věta~\ref{th:derlimder}):
                    $$\lim_{x \to 2+} f'(x) = -\infty \implies f'_+(2) = -\infty,$$
                    $$\lim_{x \to 2-} f'(x) = +\infty \implies f'_-(2) = +\infty.$$
                    Díky lichosti funkce platí:
                    $$f'_+(-2) = -f'_+(2) = +\infty,$$
                    $$f'_-(-2) = -f'_-(2) = -\infty.$$
                \item Intervaly monotonie.

                    Určíme intervaly, kde $f'(x) > 0,$ $f'(x) < 0$ a $f'(x) =0:$
                    \begin{align*}
                        f'(x) > 0 
                        &\iff \frac{1}{\sqrt[3]{x+2}} > \frac{1}{\sqrt[3]{x-2}} \\
                        &\iff \frac{1}{x+2} > \frac{1}{x-2} \\
                        &\iff x \in (-2,2).
                    \end{align*}
                    Podobně
                    $$f'(x) < 0 \iff x \in (-\infty,-2) \lor x \in (2,+\infty).$$
                    Dále vyplývá, že $\fa x \in D_f: f'(x) \neq 0.$

                \item Globální extrémy.
                    
                    Dle Fermatovy věta (Věta~\ref{th:fermat}) hledáme globální 
                    extrémy v bodech, kde $f'(x) = 0$ nebo $f'(x)$ neexistuje.
                    V našem případě: $x \in \{-2,2\}.$ 
                    
                    Dle intervalů monotonie určíme,
                    že $f$ má globální minimum v bodě $-2$ a to $-\sqrt[3]{4^2}$
                    a globální maximum v bodě $2$ a to $\sqrt[3]{4^2}.$

                \item První hrubý náčrt grafu funkce.

                    Víme, že v $\limxminf f(x) = 0,$ poté funkce klesá až do bodu
                    $x = -2,$ poté roste do bodu $x=2$ a pak zase klesá, až k nule
                    ($\limxinf f(x) = 0$).
                        \begin{center}
                            \begin{tikzpicture}
                                \begin{axis}[
                                        axis lines=middle,
                                        xlabel=$x$,
                                        ylabel={$f(x)$},
                                        xmin=-4, xmax=4,
                                        ymin=-1.5, ymax=1.5,
                                        xtick={-2, 2},
                                        ytick=\empty,
                                        function line/.style={
                                            black,
                                            thick,
                                            samples=2,
                                        },
                                        single dot/.style={
                                            black,
                                            mark=*,
                                        },
                                        empty point/.style={
                                            only marks,
                                            mark=*,
                                            mark options={fill=white, draw=black},
                                        },
                                    ]
                                    \addplot[function line, domain=\pgfkeysvalueof{/pgfplots/xmin}:-2] {-1-0.1*x};
                                    \addplot[function line, domain=-2:2] {0.4*x};
                                    \addplot[function line, domain=2:\pgfkeysvalueof{/pgfplots/xmax}] {1-0.1*x};
                                    \addplot[single dot] coordinates {(0, 0)};
                                \end{axis}
                            \end{tikzpicture}
                        \end{center}
        \end{enumerate}

        \item Spočteme druhou derivaci a určíme intervaly, kde je $f$ konvexní
            nebo konkávní. Určíme inflexní body.
            \begin{enumerate}
                \item Druhá derivace.

                    $\fa x \in D_f \setminus \{\pm 2\}$ platí:
                    $$f''(x) = -\frac{2}{9}\left((x+2)^{-\frac{4}{3}} - (x-2)^{-\frac{4}{3}}\right).$$
                \item Konvexita, konkávita.

                    Musíme určit intervaly, ve kterých $f''(x) > 0$ nebo
                    $f''(x) < 0.$ Platí:
                    $$f''(x) > 0 \iff (x+2)^{-\frac{4}{3}} < (x-2)^{-\frac{4}{3}},$$
                    $$f''(x) < 0 \iff (x+2)^{-\frac{4}{3}} > (x-2)^{-\frac{4}{3}},$$
                    a tedy funkce $f$ je:
                    \begin{itemize}
                        \item konvexní na $(0,2)$ a $(2, +\infty),$
                        \item konkávní na $(-\infty, -2)$ a $(-2,0).$
                    \end{itemize}
                \item Inflexní bod.

                    Z výrazu pro druhou derivaci víme, že $f''(x) = 0 \iff x =0.$
                    Zároveň jsme již určili, že $f''(x) < 0$ na $(-2,0)$ a 
                    $f''(x) > 0$ na $(0,2).$ Tím jsme splinili postačující podmínku
                    pro inflexi (Věta~\ref{th:inflexepostac}), a proto se v 
                    $x=0$ nachází inflexní bod.
            \end{enumerate}
        \item Vypočteme asymptoty funkce.

            Jelikož $\limxinf f(x) = \limxminf f(x) = 0,$ asymptoty splývají s
            osou $x.$

        \item Načrtneme graf funkce a určíme obor hodnot.
            \begin{center}
                \begin{tikzpicture}
                    \begin{axis}[
                            axis lines=middle,
                            xlabel=$x$,
                            ylabel={$f(x)$},
                            function line/.style={
                                black,
                                thick,
                                samples=200,
                            },
                        ]
                        \addplot[function line, domain=-3:3] {((x+2)^2)^(1/3) - ((x-2)^2)^(1/3)};
                    \end{axis}
                \end{tikzpicture}
            \end{center}

            $H_f = [-\sqrt[3]{4^2},\sqrt[3]{4^2}].$
    \end{enumerate}
\end{example}

\subsection{Taylorův polynom}

\begin{definition}
    \Necht $f$ je reálná funkce, $a \in \R$ a existuje vlastní $n$-tá derivace
    $f$ v bodě $a.$ Pak polynom
    $$\tfan(x) = f(a) + f'(a)(x-a) + \dots + \frac{1}{n!}f^{(n)}(x-a)^n$$
    nazýváme \newterm{Taylorovým polynomem řádu $n$ funkce $f$ v bodě $a$}.
\end{definition}

\begin{remark}[vlastnosti Taylorova polynomu]
    \leavevmode
    \begin{itemize}
        \item $\deg \tfan(x) \leq n.$
        \item Derivace Taylorova polynomu:
            \begin{align*}
                (\tfan)'(x) 
                &= 0 + f'(a) \cdot 1 + \dots + \frac{1}{n!} \cdot f^{(n)}(a)\cdot n \cdot (x-a)^{n-1} \\
                &= T^{f',a}_{n-1}(x).
            \end{align*}
    \end{itemize}
\end{remark}

\begin{lemma}
    \label{lm:taylorqp0}
    \Necht $Q$ je polynom, $a \in \R, \deg Q \leq n$ 
    a $\limxa \frac{Q(x)}{(x-a)^n} = 0.$ Pak $Q \equiv 0.$
\end{lemma}

\begin{proof}
    Lemma dokážeme indukcí. V základním kroce ($n=1$) je polynom $Q$ lineární. 
    Dále:
    \begin{align*}
        Q(a)
        &= \limxa \left(\frac{Q(x)}{x-a}(x-a)\right) \\
        &\qe \limxa \frac{Q(x)}{x-a} \cdot \limxa (x-a)^n 
            \tag{aritmetika limit funkcí, Věta~\ref{th:voalf}}\\
        &= 0 \cdot 0 = 0. \tag{předpoklad}
    \end{align*}
    Polynom $Q$ tedy můžeme vyjádřit jako $Q(x) = c(x-a).$ Nakonec
    odvodíme, že koeficient $c$ je roven nule, jelikož:
    $$ 0 = \limxa \frac{Q(x)}{x-a} = \limxa \frac{c(x-a)}{x-a} = \limxa c = c.$$

    V indukčním kroce ($n-1 \rightarrow n$) platí podobně jako výše, že $a$ je
    kořenem polynomu $Q,$ a proto jej můžeme vyjádřit jako:
    $$Q(x) = (x-a)R(x),$$
    kde $R$ je polynom a $\deg R \leq n-1.$ Dle indukčního předpokladu je 
    $R \equiv 0$, a tedy i $Q \equiv 0.$
\end{proof}

\begin{theorem}[o nejlepší aproximaci Taylorovým polynomem]
    \Necht $a \in \R, f^{(n)}(a) \in \R$ a $P$ je polynom stupně nejvýše $n.$ Pak
    $$\limxa \frac{f(x) - P(x)}{(x-a)^n} = 0 \iff P = \tfan. $$
\end{theorem}

\begin{proof}
    \leavevmode
    \begin{itemize}
        \item[$\impliedby$]
            Využijeme matematickou indukci. Pro případ $n=1:$
            \begin{align*}
                \limxa \frac{f(x) - T^{f,a}_1}{x-a} 
                &= \limxa \frac{f(x) - f(a) - f'(a)(x-a)}{x-a} \\
                &= \limxa \left(\frac{f(a)-f(a)}{x-a} - f'(a)\right) \\
                &= f'(a) - f'(a) = 0.
            \end{align*}
            Indukční krok:
            \begin{align*}
                \limxa \frac{f(a) - \tfan(x)}{(x-a)^n}
                &= \limxa \frac{f'(x) - (\tfan)'(x)}{n(x-a)^{n-1}}
                    \tag{L'Hospital (Věta~\ref{th:lhospital}),
                    $\frac{0}{0}$} \\
                &= \frac{1}{n} \limxa \frac{f'(x) - T^{f',a}_{n-1}(x)}{(x-a)^{n-1}} \\
                &= \frac{1}{n}\cdot0 = 0\tag{indukční předpoklad}
            \end{align*}

        \item[$\implies$]
            Rozepišme limitu na levé straně implikace za použití věty 
            o limitě limit funkcí (Věta~\ref{th:voalf}) jako součet dvou limit:
            $$\limxa \frac{P(x) - \tfan(x)}{(x-a)^n} 
            \qe \underbrace{\limxa \frac{P(x) - f(x)}{(x-a)^n}}_{A} 
                + \underbrace{\limxa \frac{f(x) - \tfan(x)}{(x-a)^n}}_{B}$$
            Výraz $A$ je roven nule dle předpokladů. Výraz $B$ je dle předchozího
            bodu taktéž roven nule, a proto
            $$\limxa \frac{P(x) - \tfan(x)}{(x-a)^n}  = 0$$
            Díky lemmatu~\ref{lm:taylorqp0} platí $P = \tfan.$
    \end{itemize}
\end{proof}

\begin{theorem}[Taylor, či obecný tvar zbytku]
    \Necht funkce $f$ má vlastní $(n+1)$-ní derivaci v intervalu $[a,x]$ a \necht
    $\phi$ je spojitá funkce v $[a,x]$ a má vlastní derivaci v $(a,x),$ která je
    v každém bodě tohoto intervalu různá od nuly. Pak existuje $\xi \in (a,x)$ \tz
    $$f(x) - \tfan(x) = \frac{1}{n!}\frac{\phi(x)-\phi(a)}{\phi'(\xi)}f^{(n+1)}(\xi)(x-\xi)^n.$$
    Speciálně existuje $\xi_1 \in (a,x)$ \tz
    $$f(x) - \tfan(x) = \frac{1}{(n+1)!}f^{(n+1)}(\xi_1)(x-a)^{n+1} \; 
    \text{(Lagrangeův tvar zbytku)}$$
    a existuje $\xi_2 \in (a,x)$ \tz
    $$f(x) - \tfan(x) = \frac{1}{n!}f^{(n+1)}(\xi_2)(x-\xi_2)^n(x-a). \;
    \text{(Cauchyho tvar zbytku)}$$
\end{theorem}

\begin{proof}
    Rozdíl $f(x) - \tfan(x)$ nazýváme zbytek. Pro $t \in [a,x]$ definujme 
    následující funkci:
    $$F(t) = f(x) - T_n^{f,t}(x) = f(x) - \left(f(t) + f'(t)(x-t) + \dots + \frac{f^{(n)}(t)}{n!}(x-t)^n\right)$$
    Tato funkce je:
    \begin{itemize}
        \item spojitá na $[a,x],$
        \item má vlastní derivaci na $(a,x),$
        \item $F(x) = 0,$
        \item $F(a) = f(x) - \tfan(x).$ 
    \end{itemize}
    Dle Cauchyho věty o střední hodnotě (Věta~\ref{th:cauchymean}):
    $$\exists \xi \in (a,x): \frac{F'(\xi)}{\phi'(\xi)} = \frac{F(x) - F(a)}{\phi(x) - \phi(a)} = \frac{0 - (f(x) - \tfan(x))}{\phi(x) - \phi(a)},$$
    z čehož po úpravě dostáváme:
    $$f(x) - \tfan(x) = -\frac{F'(\xi)}{\phi'(\xi)}(\phi(x)-\phi(a)).$$
    Vyjádřeme si nyní derivaci funkce $F:$
    \begin{align*}
        F'(t)
        &= 0 - (f'(t) + f''(t)(x-t) + f'(t)(-1) + \dots + \frac{f^{(n+1)}(t)}{n!}(x-t)^n + \frac{f^{(n)}}{n!}n(x-t)^{n-1}(-1) \\
        &= -\frac{f^{(n+1)}(t)}{n!}(x-t)^n.
    \end{align*}
    Tento výraz nyní můžeme dosadit do vzorce zbytku, který jsme vyjádřili výše,
    a získáme výraz pro obecný tvar zbytku:
    \begin{align*}
        f(x) - \tfan(x) 
        &= -\frac{-\frac{f^{(n+1)}(\xi)}{n!}(x-\xi)^n}{\phi'(\xi)}(\phi(x)-\phi(a)) \\
        &= \frac{1}{n!}\frac{\phi(x)-\phi(a)}{\phi'(\xi)}f^{(n+1)}(\xi)(x-\xi)^n.
    \end{align*}

    Pro Lagrangeův tvar zbytku volíme $\phi(t) = (x-t)^{n+1}.$ Potom:
    \begin{align*}
        \phi(x) - \phi(a) &= 0 - (x-a)^{n+1}, \\
        \phi'(t) &= (n+1)(x-t)^n(-1), 
    \end{align*}
    a po dosazení do obecného tvaru zbytku pro $\xi_1 \in (a,x):$
    \begin{align*}
        f(x) - \tfan(x) 
        &= \frac{1}{n!}\frac{-(x-a)^{n+1}}{(n+1)(x-\xi_1)^n(-1)}f^{(n+1)}(\xi_1)(x-\xi_1)^n \\
        &=  \frac{1}{(n+1)!}f^{(n+1)}(\xi_1)(x-a)^{n+1} 
    \end{align*}
    
    Podobně pro Cauchyho tvar zbytku volíme $\phi(t) = t.$ Potom:
    \begin{align*}
        \phi(x) - \phi(a) &= x-a, \\
        \phi'(t) &= 1, 
    \end{align*}
    a po dosazení do obecného tvaru zbytku pro $\xi_2 \in (a,x):$
    \begin{align*}
        f(x) - \tfan(x) 
        &= \frac{1}{n!}\frac{x-a}{1}f^{(n+1)}(\xi_2)(x-\xi_2)^n \\
        &= \frac{1}{n!}(x-a)f^{(n+1)}(\xi_2)(x-\xi_2)^n 
    \end{align*}

\end{proof}

\begin{remark}
    Taylorova věta platí i pro interval $(x,a).$ 
\end{remark}

\begin{example}[Taylorův polynom pro exponenciálu]
    Z poznámky~\ref{rm:derivaceelfce} víme, že
    $$(e^x)' = (e^x)'' = \dots = (e^x)^{(n)} = e^x.$$
    Jelikož při $x=0$ jsou všechny derivace rovny $1,$ platí:
    $$T^{exp, 0}_n = \sum_{j=0}^n \frac{x^j}{j!}.$$
    Zvolme pevné $x.$ Potom pro dané $n$ existuje $\xi_n \in (0, x)$ (nebo $(x,0),$
    pokud je $x < 0$) \tz pro Lagrangeův tvar zbytku platí:
    $$e^x - T^{exp, 0}_n =e^x - \sum_{j=0}^n \frac{x^j}{j!} 
    = \frac{1}{(n+1)!}e^{\xi_n}x^{n+1}$$
    Jelikož $e^{\xi_n} \leq e^{|x|}$ a $x^{n+1} \leq |x|^{n+1},$ platí:
    $$|e^x - T^{exp, 0}_n| \leq \frac{e^{|x|}|x|^{n+1}}{(n+1)!}.$$
    Za použití věty o dvou strážnících (Věta~\ref{th:dvastraznici}) a 
    $\limninf \frac{x^n}{n!} = 0$ dostáváme:
    $$\limninf e^x - T^{exp, 0}_n = 0$$
    a tedy:
    $$\fa x \in \R: e^x = \sum_{j=0}^\infty \frac{x^j}{j!}.$$
    Speciálně pro $e:$
    $$e = e^1 = \sum_{j=0}^\infty \frac{1^j}{j!}= 1 + \frac{1}{1} + \frac{1}{2!}
    + \dots + \frac{1}{n!} + \dots$$
\end{example}

\begin{example}
    Spočtěte hodnotu $e$ s chybou $0.001.$

    Vyjádřeme $e$ jako součet dvou sum:
    $$e = \sum_{j=0}^\infty \frac{1}{j!} = \sum_{j=0}^n \frac{1}{j!} + \sum_{j=n+1}^\infty \frac{1}{j!}.$$
    Chceme, aby:
    $$\sum_{j=n+1}^\infty \frac{1}{j!} < 0.001,$$
    a naším úkolem je zjistit hodnotu $n$, tj. zjistit, kolik členů Taylorova
    polynomu musíme spočítat pro zajištění dané přesnosti. Platí:
    \begin{align*}
        \sum_{j=n+1}^\infty \frac{1}{j!}
        &= \sum_{j=0}^\infty \frac{1}{(n+1+j)!} \\
        &\leq \frac{1}{(n+1)!}\sum_{j=0}^\infty \frac{1}{(n+1)^j}
            \tag{$(n+1+j)! \geq (n+1)!(n+1)^j$} \\
        &= \frac{1}{(n+1)!} \frac{1}{1 - \frac{1}{n+1}} 
            \tag{součtový vzorec geometrické řady}\\
        &= \frac{1}{(n+1)!} \frac{n+1}{n} \\
        &= \frac{1}{n\cdot n!}
    \end{align*}
    Již při $n = 6$ je $\frac{1}{n\cdot n!} < 0.001,$ a nám tedy stačí 
    spočítat prvních 6+1 členů Taylorova polynomu ($n$ se počítá od nuly):
    $$e \approx 1 + \frac{1}{1} + \frac{1}{2} + \frac{1}{6} + \frac{1}{24} + 
    \frac{1}{120} + \frac{1}{720}.$$
\end{example}

\begin{example}
    Dokažte, že číslo $e$ je iracionální.

    Sporem. \Necht $e = \frac{p}{q},$ kde $p,q \in \N.$ Z předchozího příkladu víme,
    že:
    $$\sum_{n=0}^q \frac{1}{n!} < e = \sum_{n=0}^q \frac{1}{n!} 
    + \sum_{n=q+1}^\infty \frac{1}{q!} < \sum_{n=0}^q\frac{1}{n!} + \frac{1}{q\cdot q!}$$
    Vynásobme nerovnici výrazem $(q\cdot q!)$ a získáme:
    $$\underbrace{(q\cdot q!)\sum_{n=0}^q \frac{1}{n!}}_{A} 
    < pq!
    < \underbrace{(q\cdot q!) \sum_{n=0}^q \frac{1}{n!}}_{A} + 1.$$
    Jelikož $A \in \N$ i $pq! \in \N,$ dostáváme spor, jelikož 
    $pq!$ by muselo být přirozené číslo mezi $A$ a $A+1.$
\end{example}

\begin{example}[Taylorův polynom pro funkci $\sin$]
    \Necht $a = 0.$ Potom platí:
    \begin{align*}
        &\sin'(x) = \cos(x), \; \sin'(a) = 1, \\
        &\sin''(x) = -\sin(x), \; \sin''(a) = 0, \\
        &\sin'''(x) = -\cos(x), \; \sin'''(a) = -1, \\
        &\sin^{(4)}(x) = \sin(x), \; \sin^{(4)}(a) = 0. \\
        &\sin^{(5)}(x) = \cos(x), \; \sin^{(5)}(a) = 1, \; \text{atd.}
    \end{align*}
    Potom:
    $$T^{\sin,0}_n(x) = (0) + \left(\frac{1}{1!}x\right) 
    + \left(\frac{1}{2!}0x^2 \right) 
    + \left(\frac{1}{3!}(-1)x^3\right) 
    + \left(\frac{1}{4!}0x^4 \right) 
    + \left(\frac{1}{5!}1x^5 \right) 
    + \dots$$
    Dále:
    \begin{align*}
        \left|\sin(x) - T^{\sin,0}_n\right| 
        &= \left|\frac{1}{n+1}!f^{(n+1)}(\xi)x^{n+1}\right| \\
        &\leq \left|\frac{x^{n+1}}{(n+1)!}\right|
    \end{align*}
    a tím pádem $\limninf (\sin(x) - T^{\sin,0}_n) = 0.$
    Můžeme tedy psát:
    $$\sin(x) = \sum_{n=0}^{\infty} (-1)^n\frac{x^{2n+1}}{(2n+1)!}.$$
    Podobně pro $\cos(x):$
    $$\cos(x) = \sum_{n=0}^{\infty} (-1)^n\frac{x^{2n}}{(2n)!}.$$
\end{example}
